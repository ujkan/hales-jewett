\documentclass[11pt,a4paper]{article}
\usepackage{isabelle,isabellesym}

% further packages required for unusual symbols (see also
% isabellesym.sty), use only when needed

%\usepackage{amssymb}
  %for \<leadsto>, \<box>, \<diamond>, \<sqsupset>, \<mho>, \<Join>,
  %\<lhd>, \<lesssim>, \<greatersim>, \<lessapprox>, \<greaterapprox>,
  %\<triangleq>, \<yen>, \<lozenge>

%\usepackage{eurosym}
  %for \<euro>

%\usepackage[only,bigsqcap]{stmaryrd}
  %for \<Sqinter>

%\usepackage{eufrak}
  %for \<AA> ... \<ZZ>, \<aa> ... \<zz> (also included in amssymb)

%\usepackage{textcomp}
  %for \<onequarter>, \<onehalf>, \<threequarters>, \<degree>, \<cent>,
  %\<currency>

% this should be the last package used
\usepackage{pdfsetup}

% urls in roman style, theory text in math-similar italics
\urlstyle{rm}
\isabellestyle{it}

% for uniform font size
%\renewcommand{\isastyle}{\isastyleminor}


\begin{document}
\title{The Hales--Jewett Theorem}
\author{Ujkan Sulejmani, Manuel Eberl, Katharina Kreuzer}
\maketitle

\begin{abstract}
    This document is a formalisation of a proof of the Hales--Jewett theorem presented in the textbook \emph{Ramsey Theory} by Graham et al.~\cite{thebook}.
    
    The Hales--Jewett theorem is a result in Ramsey Theory which states that, for any non-negative integers $r$ and $t$, there exists a minimal dimension $N$, such that any $r$-coloured $N'$-dimensional cube over $t$ elements (with $N' \geq N$) contains a monochromatic line. This theorem generalises Van der Waerden's Theorem, which has already been formalised in another AFP entry~\cite{vdw}.
\end{abstract}


\newpage
\tableofcontents

% sane default for proof documents
\newpage
\parindent 0pt\parskip 0.5ex

% generated text of all theories
%
\begin{isabellebody}%
\setisabellecontext{Hales{\isacharminus}{\kern0pt}Jewett}%
%
\isadelimtheory
%
\endisadelimtheory
%
\isatagtheory
\isacommand{theory}\isamarkupfalse%
\ {\isachardoublequoteopen}Hales{\isacharminus}{\kern0pt}Jewett{\isachardoublequoteclose}\isanewline
\ \ \isakeyword{imports}\ Main\ {\isachardoublequoteopen}HOL{\isacharminus}{\kern0pt}Library{\isachardot}{\kern0pt}Disjoint{\isacharunderscore}{\kern0pt}Sets{\isachardoublequoteclose}\ {\isachardoublequoteopen}HOL{\isacharminus}{\kern0pt}Library{\isachardot}{\kern0pt}FuncSet{\isachardoublequoteclose}\isanewline
\isakeyword{begin}%
\endisatagtheory
{\isafoldtheory}%
%
\isadelimtheory
%
\endisadelimtheory
%
\isadelimdocument
%
\endisadelimdocument
%
\isatagdocument
%
\isamarkupsection{Hales-Jewett Theorem%
}
\isamarkuptrue%
%
\endisatagdocument
{\isafolddocument}%
%
\isadelimdocument
%
\endisadelimdocument
%
\begin{isamarkuptext}%
The Hales-Jewett Theorem is at its core a statement about sets of tuples called the n-dimensional cube over t elements; i.e. the set $[t]^n$, where $[t]$ is called the base. 
We use functions $f : [n] \rightarrow [t]$ instead of tuples because they're easier to
 deal with. The set of tuples then becomes the function space $[t]^{[n]}$.  \isa{cube\ n\ t\ {\isasymequiv}\ {\isacharbraceleft}{\kern0pt}{\isachardot}{\kern0pt}{\isachardot}{\kern0pt}{\isacharless}{\kern0pt}n{\isacharbraceright}{\kern0pt}\ {\isasymrightarrow}\isactrlsub E\ {\isacharbraceleft}{\kern0pt}{\isachardot}{\kern0pt}{\isachardot}{\kern0pt}{\isacharless}{\kern0pt}t{\isacharbraceright}{\kern0pt}}. 
Furthermore, $r$-colorings are denoted by mappings from the function space to the set $\{0,\ldots, r-1\}$.%
\end{isamarkuptext}\isamarkuptrue%
%
\isadelimdocument
%
\endisadelimdocument
%
\isatagdocument
%
\isamarkupsubsection{Cubes $C^n_t$%
}
\isamarkuptrue%
%
\endisatagdocument
{\isafolddocument}%
%
\isadelimdocument
%
\endisadelimdocument
%
\begin{isamarkuptext}%
Function spaces in Isabelle are supported by the library construct FuncSet. In essence, \isa{f\ {\isasymin}\ A\ {\isasymrightarrow}\isactrlsub E\ B} means \isa{a\ {\isasymin}\ A\ {\isasymLongrightarrow}\ f\ a\ {\isasymin}\ B} and \isa{a\ {\isasymnotin}\ A\ {\isasymLongrightarrow}\ f\ a\ {\isacharequal}{\kern0pt}\ undefined}%
\end{isamarkuptext}\isamarkuptrue%
%
\begin{isamarkuptext}%
The (canonical) $n$-dimensional cube over $t$ elements is defined in the following using the variables:

\begin{tabular}{lcp{8cm}}
$n$:& \isa{nat}& dimension\\
$t$:&  \isa{nat}& number of elements\\
\end{tabular}%
\end{isamarkuptext}\isamarkuptrue%
\isacommand{definition}\isamarkupfalse%
\ cube\ {\isacharcolon}{\kern0pt}{\isacharcolon}{\kern0pt}\ {\isachardoublequoteopen}nat\ {\isasymRightarrow}\ nat\ {\isasymRightarrow}\ {\isacharparenleft}{\kern0pt}nat\ {\isasymRightarrow}\ nat{\isacharparenright}{\kern0pt}\ set{\isachardoublequoteclose}\isanewline
\ \ \isakeyword{where}\ {\isachardoublequoteopen}cube\ n\ t\ {\isasymequiv}\ {\isacharbraceleft}{\kern0pt}{\isachardot}{\kern0pt}{\isachardot}{\kern0pt}{\isacharless}{\kern0pt}n{\isacharbraceright}{\kern0pt}\ {\isasymrightarrow}\isactrlsub E\ {\isacharbraceleft}{\kern0pt}{\isachardot}{\kern0pt}{\isachardot}{\kern0pt}{\isacharless}{\kern0pt}t{\isacharbraceright}{\kern0pt}{\isachardoublequoteclose}\isanewline
\isanewline
\isacommand{lemma}\isamarkupfalse%
\ ex{\isacharunderscore}{\kern0pt}bij{\isacharunderscore}{\kern0pt}betw{\isacharunderscore}{\kern0pt}nat{\isacharunderscore}{\kern0pt}finite{\isacharunderscore}{\kern0pt}{\isadigit{2}}{\isacharcolon}{\kern0pt}\ \isakeyword{assumes}\ {\isachardoublequoteopen}card\ A\ {\isacharequal}{\kern0pt}\ n{\isachardoublequoteclose}\ \isakeyword{and}\ {\isachardoublequoteopen}n\ {\isachargreater}{\kern0pt}\ {\isadigit{0}}{\isachardoublequoteclose}\ \isakeyword{shows}\ {\isachardoublequoteopen}{\isasymexists}f{\isachardot}{\kern0pt}\ bij{\isacharunderscore}{\kern0pt}betw\ f\ A\ {\isacharbraceleft}{\kern0pt}{\isachardot}{\kern0pt}{\isachardot}{\kern0pt}{\isacharless}{\kern0pt}n{\isacharbraceright}{\kern0pt}{\isachardoublequoteclose}\isanewline
%
\isadelimproof
\ \ %
\endisadelimproof
%
\isatagproof
\isacommand{using}\isamarkupfalse%
\ assms\ ex{\isacharunderscore}{\kern0pt}bij{\isacharunderscore}{\kern0pt}betw{\isacharunderscore}{\kern0pt}finite{\isacharunderscore}{\kern0pt}nat{\isacharbrackleft}{\kern0pt}of\ A{\isacharbrackright}{\kern0pt}\ atLeast{\isadigit{0}}LessThan\ card{\isacharunderscore}{\kern0pt}ge{\isacharunderscore}{\kern0pt}{\isadigit{0}}{\isacharunderscore}{\kern0pt}finite\ \isacommand{by}\isamarkupfalse%
\ auto%
\endisatagproof
{\isafoldproof}%
%
\isadelimproof
%
\endisadelimproof
%
\begin{isamarkuptext}%
For any function $f$ whose image under a set $A$ is a subset of another set $B$, there's a unique function $g$ in the function space $B^A$ that equals $f$ everywhere in $A$.
The function $g$ is usually written as $f|_A$ in the mathematical literature.%
\end{isamarkuptext}\isamarkuptrue%
\isacommand{lemma}\isamarkupfalse%
\ PiE{\isacharunderscore}{\kern0pt}uniqueness{\isacharcolon}{\kern0pt}\ {\isachardoublequoteopen}f\ {\isacharbackquote}{\kern0pt}\ A\ {\isasymsubseteq}\ B\ {\isasymLongrightarrow}\ {\isasymexists}{\isacharbang}{\kern0pt}g\ {\isasymin}\ A\ {\isasymrightarrow}\isactrlsub E\ B{\isachardot}{\kern0pt}\ {\isasymforall}a{\isasymin}A{\isachardot}{\kern0pt}\ g\ a\ {\isacharequal}{\kern0pt}\ f\ a{\isachardoublequoteclose}\isanewline
%
\isadelimproof
\ \ %
\endisadelimproof
%
\isatagproof
\isacommand{using}\isamarkupfalse%
\ exI{\isacharbrackleft}{\kern0pt}of\ {\isachardoublequoteopen}{\isasymlambda}x{\isachardot}{\kern0pt}\ x\ {\isasymin}\ A\ {\isasymrightarrow}\isactrlsub E\ B\ {\isasymand}\ {\isacharparenleft}{\kern0pt}{\isasymforall}a{\isasymin}A{\isachardot}{\kern0pt}\ x\ a\ {\isacharequal}{\kern0pt}\ f\ a{\isacharparenright}{\kern0pt}{\isachardoublequoteclose}\ {\isachardoublequoteopen}restrict\ f\ A{\isachardoublequoteclose}{\isacharbrackright}{\kern0pt}\ PiE{\isacharunderscore}{\kern0pt}ext\ PiE{\isacharunderscore}{\kern0pt}iff\ \isacommand{by}\isamarkupfalse%
\ fastforce%
\endisatagproof
{\isafoldproof}%
%
\isadelimproof
\isanewline
%
\endisadelimproof
\isanewline
\isanewline
\isacommand{lemma}\isamarkupfalse%
\ cube{\isacharunderscore}{\kern0pt}restrict{\isacharcolon}{\kern0pt}\ \isakeyword{assumes}\ {\isachardoublequoteopen}j\ {\isacharless}{\kern0pt}\ n{\isachardoublequoteclose}\ {\isachardoublequoteopen}y\ {\isasymin}\ cube\ n\ t{\isachardoublequoteclose}\ \isakeyword{shows}\ {\isachardoublequoteopen}{\isacharparenleft}{\kern0pt}{\isasymlambda}g\ {\isasymin}\ {\isacharbraceleft}{\kern0pt}{\isachardot}{\kern0pt}{\isachardot}{\kern0pt}{\isacharless}{\kern0pt}j{\isacharbraceright}{\kern0pt}{\isachardot}{\kern0pt}\ y\ g{\isacharparenright}{\kern0pt}\ {\isasymin}\ cube\ j\ t{\isachardoublequoteclose}%
\isadelimproof
\ %
\endisadelimproof
%
\isatagproof
\isacommand{using}\isamarkupfalse%
\ assms\ \isacommand{unfolding}\isamarkupfalse%
\ cube{\isacharunderscore}{\kern0pt}def\ \isacommand{by}\isamarkupfalse%
\ force%
\endisatagproof
{\isafoldproof}%
%
\isadelimproof
%
\endisadelimproof
%
\begin{isamarkuptext}%
A line $L$ in the $n$-dimensional cube

\begin{tabular}{lcp{8cm}}
$n$:& \isa{nat}& dimension\\
$t$:&  \isa{nat}& the size of the base\\
\end{tabular}%
\end{isamarkuptext}\isamarkuptrue%
%
\begin{isamarkuptext}%
Narrowing down the obvious fact $B^A \subseteq C^A$ if $B \subseteq C$ to a specific case for cubes.%
\end{isamarkuptext}\isamarkuptrue%
\isacommand{lemma}\isamarkupfalse%
\ cube{\isacharunderscore}{\kern0pt}subset{\isacharcolon}{\kern0pt}\ {\isachardoublequoteopen}cube\ n\ t\ {\isasymsubseteq}\ cube\ n\ {\isacharparenleft}{\kern0pt}t\ {\isacharplus}{\kern0pt}\ {\isadigit{1}}{\isacharparenright}{\kern0pt}{\isachardoublequoteclose}\isanewline
%
\isadelimproof
\ \ %
\endisadelimproof
%
\isatagproof
\isacommand{unfolding}\isamarkupfalse%
\ cube{\isacharunderscore}{\kern0pt}def\ \isacommand{using}\isamarkupfalse%
\ PiE{\isacharunderscore}{\kern0pt}mono{\isacharbrackleft}{\kern0pt}of\ {\isachardoublequoteopen}{\isacharbraceleft}{\kern0pt}{\isachardot}{\kern0pt}{\isachardot}{\kern0pt}{\isacharless}{\kern0pt}n{\isacharbraceright}{\kern0pt}{\isachardoublequoteclose}\ {\isachardoublequoteopen}{\isasymlambda}x{\isachardot}{\kern0pt}\ {\isacharbraceleft}{\kern0pt}{\isachardot}{\kern0pt}{\isachardot}{\kern0pt}{\isacharless}{\kern0pt}t{\isacharbraceright}{\kern0pt}{\isachardoublequoteclose}\ {\isachardoublequoteopen}{\isasymlambda}x{\isachardot}{\kern0pt}\ {\isacharbraceleft}{\kern0pt}{\isachardot}{\kern0pt}{\isachardot}{\kern0pt}{\isacharless}{\kern0pt}t{\isacharplus}{\kern0pt}{\isadigit{1}}{\isacharbraceright}{\kern0pt}{\isachardoublequoteclose}{\isacharbrackright}{\kern0pt}\isanewline
\ \ \isacommand{by}\isamarkupfalse%
\ simp%
\endisatagproof
{\isafoldproof}%
%
\isadelimproof
%
\endisadelimproof
%
\begin{isamarkuptext}%
A simplifying definition for the 0-dimensional cube.%
\end{isamarkuptext}\isamarkuptrue%
\isacommand{lemma}\isamarkupfalse%
\ cube{\isadigit{0}}{\isacharunderscore}{\kern0pt}alt{\isacharunderscore}{\kern0pt}def{\isacharcolon}{\kern0pt}\ {\isachardoublequoteopen}cube\ {\isadigit{0}}\ t\ {\isacharequal}{\kern0pt}\ {\isacharbraceleft}{\kern0pt}{\isasymlambda}x{\isachardot}{\kern0pt}\ undefined{\isacharbraceright}{\kern0pt}{\isachardoublequoteclose}\isanewline
%
\isadelimproof
\ \ %
\endisadelimproof
%
\isatagproof
\isacommand{unfolding}\isamarkupfalse%
\ cube{\isacharunderscore}{\kern0pt}def\ \isacommand{by}\isamarkupfalse%
\ simp%
\endisatagproof
{\isafoldproof}%
%
\isadelimproof
%
\endisadelimproof
%
\begin{isamarkuptext}%
The cardinality of the n-dimensional over t elements is simply a consequence of the overarching definition of the cardinality of function spaces (over finite sets)%
\end{isamarkuptext}\isamarkuptrue%
\isacommand{lemma}\isamarkupfalse%
\ cube{\isacharunderscore}{\kern0pt}card{\isacharcolon}{\kern0pt}\ {\isachardoublequoteopen}card\ {\isacharparenleft}{\kern0pt}{\isacharbraceleft}{\kern0pt}{\isachardot}{\kern0pt}{\isachardot}{\kern0pt}{\isacharless}{\kern0pt}n{\isacharcolon}{\kern0pt}{\isacharcolon}{\kern0pt}nat{\isacharbraceright}{\kern0pt}\ {\isasymrightarrow}\isactrlsub E\ {\isacharbraceleft}{\kern0pt}{\isachardot}{\kern0pt}{\isachardot}{\kern0pt}{\isacharless}{\kern0pt}t{\isacharcolon}{\kern0pt}{\isacharcolon}{\kern0pt}nat{\isacharbraceright}{\kern0pt}{\isacharparenright}{\kern0pt}\ {\isacharequal}{\kern0pt}\ t\ {\isacharcircum}{\kern0pt}\ n{\isachardoublequoteclose}\isanewline
%
\isadelimproof
\ \ %
\endisadelimproof
%
\isatagproof
\isacommand{by}\isamarkupfalse%
\ {\isacharparenleft}{\kern0pt}simp\ add{\isacharcolon}{\kern0pt}\ card{\isacharunderscore}{\kern0pt}PiE{\isacharparenright}{\kern0pt}%
\endisatagproof
{\isafoldproof}%
%
\isadelimproof
%
\endisadelimproof
%
\begin{isamarkuptext}%
A simplifying definition for the n-dimensional cube over a single element, i.e. the single n-dimensional point (0, 0, ..., 0).%
\end{isamarkuptext}\isamarkuptrue%
\isacommand{lemma}\isamarkupfalse%
\ cube{\isadigit{1}}{\isacharunderscore}{\kern0pt}alt{\isacharunderscore}{\kern0pt}def{\isacharcolon}{\kern0pt}\ {\isachardoublequoteopen}cube\ n\ {\isadigit{1}}\ {\isacharequal}{\kern0pt}\ {\isacharbraceleft}{\kern0pt}{\isasymlambda}x{\isasymin}{\isacharbraceleft}{\kern0pt}{\isachardot}{\kern0pt}{\isachardot}{\kern0pt}{\isacharless}{\kern0pt}n{\isacharbraceright}{\kern0pt}{\isachardot}{\kern0pt}\ {\isadigit{0}}{\isacharbraceright}{\kern0pt}{\isachardoublequoteclose}%
\isadelimproof
\ %
\endisadelimproof
%
\isatagproof
\isacommand{unfolding}\isamarkupfalse%
\ cube{\isacharunderscore}{\kern0pt}def\ \isacommand{by}\isamarkupfalse%
\ {\isacharparenleft}{\kern0pt}simp\ add{\isacharcolon}{\kern0pt}\ lessThan{\isacharunderscore}{\kern0pt}Suc{\isacharparenright}{\kern0pt}%
\endisatagproof
{\isafoldproof}%
%
\isadelimproof
%
\endisadelimproof
%
\isadelimdocument
%
\endisadelimdocument
%
\isatagdocument
%
\isamarkupsubsection{Lines%
}
\isamarkuptrue%
%
\endisatagdocument
{\isafolddocument}%
%
\isadelimdocument
%
\endisadelimdocument
%
\begin{isamarkuptext}%
The property of being a line in the $C^n_t$ is defined in the following using the variables:

\begin{tabular}{llp{8cm}}
$L$:& \isa{nat\ {\isasymRightarrow}\ {\isacharparenleft}{\kern0pt}nat\ {\isasymRightarrow}\ nat{\isacharparenright}{\kern0pt}}& line\\
$n$:& \isa{nat}& dimension of cube\\
$t$:&  \isa{nat}& the size of the cube's base\\
\end{tabular}%
\end{isamarkuptext}\isamarkuptrue%
\isacommand{definition}\isamarkupfalse%
\ is{\isacharunderscore}{\kern0pt}line\ {\isacharcolon}{\kern0pt}{\isacharcolon}{\kern0pt}\ {\isachardoublequoteopen}{\isacharparenleft}{\kern0pt}nat\ {\isasymRightarrow}\ {\isacharparenleft}{\kern0pt}nat\ {\isasymRightarrow}\ nat{\isacharparenright}{\kern0pt}{\isacharparenright}{\kern0pt}\ {\isasymRightarrow}\ nat\ {\isasymRightarrow}\ nat\ {\isasymRightarrow}\ bool{\isachardoublequoteclose}\isanewline
\ \ \isakeyword{where}\ {\isachardoublequoteopen}is{\isacharunderscore}{\kern0pt}line\ L\ n\ t\ {\isasymequiv}\ {\isacharparenleft}{\kern0pt}L\ {\isasymin}\ {\isacharbraceleft}{\kern0pt}{\isachardot}{\kern0pt}{\isachardot}{\kern0pt}{\isacharless}{\kern0pt}t{\isacharbraceright}{\kern0pt}\ {\isasymrightarrow}\isactrlsub E\ cube\ n\ t\ {\isasymand}\ {\isacharparenleft}{\kern0pt}{\isacharparenleft}{\kern0pt}{\isasymforall}j{\isacharless}{\kern0pt}n{\isachardot}{\kern0pt}\ {\isacharparenleft}{\kern0pt}{\isasymforall}x{\isacharless}{\kern0pt}t{\isachardot}{\kern0pt}\ {\isasymforall}y{\isacharless}{\kern0pt}t{\isachardot}{\kern0pt}\ L\ x\ j\ {\isacharequal}{\kern0pt}\ \ L\ y\ j{\isacharparenright}{\kern0pt}\ {\isasymor}\ {\isacharparenleft}{\kern0pt}{\isasymforall}s{\isacharless}{\kern0pt}t{\isachardot}{\kern0pt}\ L\ s\ j\ {\isacharequal}{\kern0pt}\ s{\isacharparenright}{\kern0pt}{\isacharparenright}{\kern0pt}\ {\isasymand}\ {\isacharparenleft}{\kern0pt}{\isasymexists}j\ {\isacharless}{\kern0pt}\ n{\isachardot}{\kern0pt}\ {\isacharparenleft}{\kern0pt}{\isasymforall}s\ {\isacharless}{\kern0pt}\ t{\isachardot}{\kern0pt}\ L\ s\ j\ {\isacharequal}{\kern0pt}\ s{\isacharparenright}{\kern0pt}{\isacharparenright}{\kern0pt}{\isacharparenright}{\kern0pt}{\isacharparenright}{\kern0pt}{\isachardoublequoteclose}%
\begin{isamarkuptext}%
We introduce an elimination rule to relate lines with the more general definition of a subspace (see below).%
\end{isamarkuptext}\isamarkuptrue%
\isacommand{lemma}\isamarkupfalse%
\ is{\isacharunderscore}{\kern0pt}line{\isacharunderscore}{\kern0pt}elim{\isacharunderscore}{\kern0pt}t{\isacharunderscore}{\kern0pt}{\isadigit{1}}{\isacharcolon}{\kern0pt}\isanewline
\ \ \isakeyword{assumes}\ {\isachardoublequoteopen}is{\isacharunderscore}{\kern0pt}line\ L\ n\ t{\isachardoublequoteclose}\ \isakeyword{and}\ {\isachardoublequoteopen}t\ {\isacharequal}{\kern0pt}\ {\isadigit{1}}{\isachardoublequoteclose}\isanewline
\ \ \isakeyword{obtains}\ B\isactrlsub {\isadigit{0}}\ B\isactrlsub {\isadigit{1}}\isanewline
\ \ \isakeyword{where}\ {\isachardoublequoteopen}B\isactrlsub {\isadigit{0}}\ {\isasymunion}\ B\isactrlsub {\isadigit{1}}\ {\isacharequal}{\kern0pt}\ {\isacharbraceleft}{\kern0pt}{\isachardot}{\kern0pt}{\isachardot}{\kern0pt}{\isacharless}{\kern0pt}n{\isacharbraceright}{\kern0pt}\ {\isasymand}\ B\isactrlsub {\isadigit{0}}\ {\isasyminter}\ B\isactrlsub {\isadigit{1}}\ {\isacharequal}{\kern0pt}\ {\isacharbraceleft}{\kern0pt}{\isacharbraceright}{\kern0pt}\ {\isasymand}\ B\isactrlsub {\isadigit{0}}\ {\isasymnoteq}\ {\isacharbraceleft}{\kern0pt}{\isacharbraceright}{\kern0pt}\ {\isasymand}\ {\isacharparenleft}{\kern0pt}{\isasymforall}j\ {\isasymin}\ B\isactrlsub {\isadigit{1}}{\isachardot}{\kern0pt}\ {\isacharparenleft}{\kern0pt}{\isasymforall}x{\isacharless}{\kern0pt}t{\isachardot}{\kern0pt}\ {\isasymforall}y{\isacharless}{\kern0pt}t{\isachardot}{\kern0pt}\ L\ x\ j\ {\isacharequal}{\kern0pt}\ L\ y\ j{\isacharparenright}{\kern0pt}{\isacharparenright}{\kern0pt}\ {\isasymand}\ {\isacharparenleft}{\kern0pt}{\isasymforall}j\ {\isasymin}\ B\isactrlsub {\isadigit{0}}{\isachardot}{\kern0pt}\ {\isacharparenleft}{\kern0pt}{\isasymforall}s{\isacharless}{\kern0pt}t{\isachardot}{\kern0pt}\ L\ s\ j\ {\isacharequal}{\kern0pt}\ s{\isacharparenright}{\kern0pt}{\isacharparenright}{\kern0pt}{\isachardoublequoteclose}\isanewline
%
\isadelimproof
%
\endisadelimproof
%
\isatagproof
\isacommand{proof}\isamarkupfalse%
\ {\isacharminus}{\kern0pt}\isanewline
\ \ \isacommand{define}\isamarkupfalse%
\ B{\isadigit{0}}\ \isakeyword{where}\ {\isachardoublequoteopen}B{\isadigit{0}}\ {\isacharequal}{\kern0pt}\ {\isacharbraceleft}{\kern0pt}{\isachardot}{\kern0pt}{\isachardot}{\kern0pt}{\isacharless}{\kern0pt}n{\isacharbraceright}{\kern0pt}{\isachardoublequoteclose}\isanewline
\ \ \isacommand{define}\isamarkupfalse%
\ B{\isadigit{1}}\ \isakeyword{where}\ {\isachardoublequoteopen}B{\isadigit{1}}\ {\isacharequal}{\kern0pt}\ {\isacharparenleft}{\kern0pt}{\isacharbraceleft}{\kern0pt}{\isacharbraceright}{\kern0pt}{\isacharcolon}{\kern0pt}{\isacharcolon}{\kern0pt}nat\ set{\isacharparenright}{\kern0pt}{\isachardoublequoteclose}\isanewline
\ \ \isacommand{have}\isamarkupfalse%
\ {\isachardoublequoteopen}B{\isadigit{0}}\ {\isasymunion}\ B{\isadigit{1}}\ {\isacharequal}{\kern0pt}\ {\isacharbraceleft}{\kern0pt}{\isachardot}{\kern0pt}{\isachardot}{\kern0pt}{\isacharless}{\kern0pt}n{\isacharbraceright}{\kern0pt}{\isachardoublequoteclose}\ \isacommand{unfolding}\isamarkupfalse%
\ B{\isadigit{0}}{\isacharunderscore}{\kern0pt}def\ B{\isadigit{1}}{\isacharunderscore}{\kern0pt}def\ \isacommand{by}\isamarkupfalse%
\ simp\isanewline
\ \ \isacommand{moreover}\isamarkupfalse%
\ \isacommand{have}\isamarkupfalse%
\ {\isachardoublequoteopen}B{\isadigit{0}}\ {\isasyminter}\ B{\isadigit{1}}\ {\isacharequal}{\kern0pt}\ {\isacharbraceleft}{\kern0pt}{\isacharbraceright}{\kern0pt}{\isachardoublequoteclose}\ \isacommand{unfolding}\isamarkupfalse%
\ B{\isadigit{0}}{\isacharunderscore}{\kern0pt}def\ B{\isadigit{1}}{\isacharunderscore}{\kern0pt}def\ \isacommand{by}\isamarkupfalse%
\ simp\isanewline
\ \ \isacommand{moreover}\isamarkupfalse%
\ \isacommand{have}\isamarkupfalse%
\ {\isachardoublequoteopen}B{\isadigit{0}}\ {\isasymnoteq}\ {\isacharbraceleft}{\kern0pt}{\isacharbraceright}{\kern0pt}{\isachardoublequoteclose}\ \isacommand{using}\isamarkupfalse%
\ assms\ \isacommand{unfolding}\isamarkupfalse%
\ B{\isadigit{0}}{\isacharunderscore}{\kern0pt}def\ is{\isacharunderscore}{\kern0pt}line{\isacharunderscore}{\kern0pt}def\ \isacommand{by}\isamarkupfalse%
\ auto\isanewline
\ \ \isacommand{moreover}\isamarkupfalse%
\ \isacommand{have}\isamarkupfalse%
\ {\isachardoublequoteopen}{\isacharparenleft}{\kern0pt}{\isasymforall}j\ {\isasymin}\ B{\isadigit{1}}{\isachardot}{\kern0pt}\ {\isacharparenleft}{\kern0pt}{\isasymforall}x{\isacharless}{\kern0pt}t{\isachardot}{\kern0pt}\ {\isasymforall}y{\isacharless}{\kern0pt}t{\isachardot}{\kern0pt}\ L\ x\ j\ {\isacharequal}{\kern0pt}\ L\ y\ j{\isacharparenright}{\kern0pt}{\isacharparenright}{\kern0pt}{\isachardoublequoteclose}\ \isacommand{unfolding}\isamarkupfalse%
\ B{\isadigit{1}}{\isacharunderscore}{\kern0pt}def\ \isacommand{by}\isamarkupfalse%
\ simp\isanewline
\ \ \isacommand{moreover}\isamarkupfalse%
\ \isacommand{have}\isamarkupfalse%
\ {\isachardoublequoteopen}{\isacharparenleft}{\kern0pt}{\isasymforall}j\ {\isasymin}\ B{\isadigit{0}}{\isachardot}{\kern0pt}\ {\isacharparenleft}{\kern0pt}{\isasymforall}s{\isacharless}{\kern0pt}t{\isachardot}{\kern0pt}\ L\ s\ j\ {\isacharequal}{\kern0pt}\ s{\isacharparenright}{\kern0pt}{\isacharparenright}{\kern0pt}{\isachardoublequoteclose}\ \isacommand{using}\isamarkupfalse%
\ assms{\isacharparenleft}{\kern0pt}{\isadigit{1}}{\isacharcomma}{\kern0pt}\ {\isadigit{2}}{\isacharparenright}{\kern0pt}\ cube{\isadigit{1}}{\isacharunderscore}{\kern0pt}alt{\isacharunderscore}{\kern0pt}def\ \isacommand{unfolding}\isamarkupfalse%
\ B{\isadigit{0}}{\isacharunderscore}{\kern0pt}def\ is{\isacharunderscore}{\kern0pt}line{\isacharunderscore}{\kern0pt}def\ \isacommand{by}\isamarkupfalse%
\ auto\isanewline
\ \ \isacommand{ultimately}\isamarkupfalse%
\ \isacommand{show}\isamarkupfalse%
\ {\isacharquery}{\kern0pt}thesis\ \isacommand{using}\isamarkupfalse%
\ that\ \isacommand{by}\isamarkupfalse%
\ simp\isanewline
\isacommand{qed}\isamarkupfalse%
%
\endisatagproof
{\isafoldproof}%
%
\isadelimproof
%
\endisadelimproof
%
\begin{isamarkuptext}%
The next two lemmas are used to simplify proofs by enabling us to use the resulting facts directly. This avoids having to unfold the definition of \isa{is{\isacharunderscore}{\kern0pt}line} each time.%
\end{isamarkuptext}\isamarkuptrue%
\isacommand{lemma}\isamarkupfalse%
\ line{\isacharunderscore}{\kern0pt}points{\isacharunderscore}{\kern0pt}in{\isacharunderscore}{\kern0pt}cube{\isacharcolon}{\kern0pt}\ \isakeyword{assumes}\ {\isachardoublequoteopen}is{\isacharunderscore}{\kern0pt}line\ L\ n\ t{\isachardoublequoteclose}\ {\isachardoublequoteopen}s\ {\isacharless}{\kern0pt}\ t{\isachardoublequoteclose}\ \isakeyword{shows}\ {\isachardoublequoteopen}L\ s\ {\isasymin}\ cube\ n\ t{\isachardoublequoteclose}\isanewline
%
\isadelimproof
\ \ %
\endisadelimproof
%
\isatagproof
\isacommand{using}\isamarkupfalse%
\ assms\ \isacommand{unfolding}\isamarkupfalse%
\ cube{\isacharunderscore}{\kern0pt}def\ is{\isacharunderscore}{\kern0pt}line{\isacharunderscore}{\kern0pt}def\isanewline
\ \ \isacommand{by}\isamarkupfalse%
\ auto%
\endisatagproof
{\isafoldproof}%
%
\isadelimproof
\ \ \ \ \ \isanewline
%
\endisadelimproof
\isanewline
\isacommand{lemma}\isamarkupfalse%
\ line{\isacharunderscore}{\kern0pt}points{\isacharunderscore}{\kern0pt}in{\isacharunderscore}{\kern0pt}cube{\isacharunderscore}{\kern0pt}unfolded{\isacharcolon}{\kern0pt}\ \isakeyword{assumes}\ {\isachardoublequoteopen}is{\isacharunderscore}{\kern0pt}line\ L\ n\ t{\isachardoublequoteclose}\ {\isachardoublequoteopen}s\ {\isacharless}{\kern0pt}\ t{\isachardoublequoteclose}\ {\isachardoublequoteopen}j\ {\isacharless}{\kern0pt}\ n{\isachardoublequoteclose}\ \isakeyword{shows}\ {\isachardoublequoteopen}L\ s\ j\ {\isasymin}\ {\isacharbraceleft}{\kern0pt}{\isachardot}{\kern0pt}{\isachardot}{\kern0pt}{\isacharless}{\kern0pt}t{\isacharbraceright}{\kern0pt}{\isachardoublequoteclose}\ \isanewline
%
\isadelimproof
\ \ %
\endisadelimproof
%
\isatagproof
\isacommand{using}\isamarkupfalse%
\ assms\ line{\isacharunderscore}{\kern0pt}points{\isacharunderscore}{\kern0pt}in{\isacharunderscore}{\kern0pt}cube\ \isacommand{unfolding}\isamarkupfalse%
\ cube{\isacharunderscore}{\kern0pt}def\ \isacommand{by}\isamarkupfalse%
\ blast%
\endisatagproof
{\isafoldproof}%
%
\isadelimproof
\isanewline
%
\endisadelimproof
\isanewline
\isanewline
\isacommand{definition}\isamarkupfalse%
\ shiftset\ {\isacharcolon}{\kern0pt}{\isacharcolon}{\kern0pt}\ {\isachardoublequoteopen}nat\ {\isasymRightarrow}\ nat\ set\ {\isasymRightarrow}\ nat\ set{\isachardoublequoteclose}\isanewline
\ \ \isakeyword{where}\isanewline
\ \ \ {\isachardoublequoteopen}shiftset\ n\ S\ {\isasymequiv}\ {\isacharparenleft}{\kern0pt}{\isasymlambda}a{\isachardot}{\kern0pt}\ a\ {\isacharplus}{\kern0pt}\ n{\isacharparenright}{\kern0pt}\ {\isacharbackquote}{\kern0pt}\ S{\isachardoublequoteclose}\isanewline
\isanewline
\isanewline
\isacommand{lemma}\isamarkupfalse%
\ shiftset{\isacharunderscore}{\kern0pt}disjnt{\isacharcolon}{\kern0pt}\ {\isachardoublequoteopen}disjnt\ A\ B\ {\isasymLongrightarrow}\ disjnt\ {\isacharparenleft}{\kern0pt}shiftset\ n\ A{\isacharparenright}{\kern0pt}\ {\isacharparenleft}{\kern0pt}shiftset\ n\ B{\isacharparenright}{\kern0pt}{\isachardoublequoteclose}\ \isanewline
%
\isadelimproof
\ \ %
\endisadelimproof
%
\isatagproof
\isacommand{unfolding}\isamarkupfalse%
\ disjnt{\isacharunderscore}{\kern0pt}def\ shiftset{\isacharunderscore}{\kern0pt}def\ \isacommand{by}\isamarkupfalse%
\ force%
\endisatagproof
{\isafoldproof}%
%
\isadelimproof
\isanewline
%
\endisadelimproof
\isacommand{lemma}\isamarkupfalse%
\ shiftset{\isacharunderscore}{\kern0pt}disjoint{\isacharunderscore}{\kern0pt}family{\isacharcolon}{\kern0pt}\ {\isachardoublequoteopen}disjoint{\isacharunderscore}{\kern0pt}family{\isacharunderscore}{\kern0pt}on\ B\ {\isacharbraceleft}{\kern0pt}{\isachardot}{\kern0pt}{\isachardot}{\kern0pt}k{\isacharbraceright}{\kern0pt}\ {\isasymLongrightarrow}\ disjoint{\isacharunderscore}{\kern0pt}family{\isacharunderscore}{\kern0pt}on\ {\isacharparenleft}{\kern0pt}{\isasymlambda}i{\isachardot}{\kern0pt}\ shiftset\ n\ {\isacharparenleft}{\kern0pt}B\ i{\isacharparenright}{\kern0pt}{\isacharparenright}{\kern0pt}\ {\isacharbraceleft}{\kern0pt}{\isachardot}{\kern0pt}{\isachardot}{\kern0pt}k{\isacharbraceright}{\kern0pt}{\isachardoublequoteclose}%
\isadelimproof
\ %
\endisadelimproof
%
\isatagproof
\isacommand{using}\isamarkupfalse%
\ shiftset{\isacharunderscore}{\kern0pt}disjnt\ \isacommand{unfolding}\isamarkupfalse%
\ disjoint{\isacharunderscore}{\kern0pt}family{\isacharunderscore}{\kern0pt}on{\isacharunderscore}{\kern0pt}def\ \isanewline
\ \ \isacommand{by}\isamarkupfalse%
\ {\isacharparenleft}{\kern0pt}meson\ disjnt{\isacharunderscore}{\kern0pt}def{\isacharparenright}{\kern0pt}%
\endisatagproof
{\isafoldproof}%
%
\isadelimproof
%
\endisadelimproof
\isanewline
\isanewline
\isanewline
\isacommand{lemma}\isamarkupfalse%
\ shiftset{\isacharunderscore}{\kern0pt}altdef{\isacharcolon}{\kern0pt}\ {\isachardoublequoteopen}shiftset\ n\ S\ {\isacharequal}{\kern0pt}\ {\isacharparenleft}{\kern0pt}{\isacharplus}{\kern0pt}{\isacharparenright}{\kern0pt}\ n\ {\isacharbackquote}{\kern0pt}\ S{\isachardoublequoteclose}\isanewline
%
\isadelimproof
\ \ %
\endisadelimproof
%
\isatagproof
\isacommand{by}\isamarkupfalse%
\ {\isacharparenleft}{\kern0pt}auto\ simp{\isacharcolon}{\kern0pt}\ shiftset{\isacharunderscore}{\kern0pt}def{\isacharparenright}{\kern0pt}%
\endisatagproof
{\isafoldproof}%
%
\isadelimproof
\isanewline
%
\endisadelimproof
\isacommand{lemma}\isamarkupfalse%
\ shiftset{\isacharunderscore}{\kern0pt}image{\isacharcolon}{\kern0pt}\isanewline
\ \ \isakeyword{assumes}\ {\isachardoublequoteopen}{\isacharparenleft}{\kern0pt}{\isasymUnion}i{\isasymin}{\isacharbraceleft}{\kern0pt}{\isachardot}{\kern0pt}{\isachardot}{\kern0pt}k{\isacharbraceright}{\kern0pt}{\isachardot}{\kern0pt}\ B\ i{\isacharparenright}{\kern0pt}\ {\isacharequal}{\kern0pt}\ {\isacharbraceleft}{\kern0pt}{\isachardot}{\kern0pt}{\isachardot}{\kern0pt}{\isacharless}{\kern0pt}n{\isacharbraceright}{\kern0pt}{\isachardoublequoteclose}\isanewline
\ \ \isakeyword{shows}\ {\isachardoublequoteopen}{\isacharparenleft}{\kern0pt}{\isasymUnion}i{\isasymin}{\isacharbraceleft}{\kern0pt}{\isachardot}{\kern0pt}{\isachardot}{\kern0pt}k{\isacharbraceright}{\kern0pt}{\isachardot}{\kern0pt}\ shiftset\ m\ {\isacharparenleft}{\kern0pt}B\ i{\isacharparenright}{\kern0pt}{\isacharparenright}{\kern0pt}\ {\isacharequal}{\kern0pt}\ {\isacharbraceleft}{\kern0pt}m{\isachardot}{\kern0pt}{\isachardot}{\kern0pt}{\isacharless}{\kern0pt}m{\isacharplus}{\kern0pt}n{\isacharbraceright}{\kern0pt}{\isachardoublequoteclose}\isanewline
%
\isadelimproof
\ \ %
\endisadelimproof
%
\isatagproof
\isacommand{using}\isamarkupfalse%
\ assms\ \isacommand{by}\isamarkupfalse%
\ {\isacharparenleft}{\kern0pt}simp\ add{\isacharcolon}{\kern0pt}\ shiftset{\isacharunderscore}{\kern0pt}altdef\ add{\isachardot}{\kern0pt}commute\ flip{\isacharcolon}{\kern0pt}\ image{\isacharunderscore}{\kern0pt}UN\ atLeast{\isadigit{0}}LessThan{\isacharparenright}{\kern0pt}%
\endisatagproof
{\isafoldproof}%
%
\isadelimproof
%
\endisadelimproof
%
\begin{isamarkuptext}%
Each tuple of dimension $k+1$ can be split into a tuple of dimension $1$---the first entry---and a tuple of dimension $k$---the remaining entries.%
\end{isamarkuptext}\isamarkuptrue%
\isacommand{lemma}\isamarkupfalse%
\ split{\isacharunderscore}{\kern0pt}cube{\isacharcolon}{\kern0pt}\ \isakeyword{assumes}\ {\isachardoublequoteopen}x\ {\isasymin}\ cube\ {\isacharparenleft}{\kern0pt}k{\isacharplus}{\kern0pt}{\isadigit{1}}{\isacharparenright}{\kern0pt}\ t{\isachardoublequoteclose}\ \isakeyword{shows}\ {\isachardoublequoteopen}{\isacharparenleft}{\kern0pt}{\isasymlambda}y\ {\isasymin}\ {\isacharbraceleft}{\kern0pt}{\isachardot}{\kern0pt}{\isachardot}{\kern0pt}{\isacharless}{\kern0pt}{\isadigit{1}}{\isacharbraceright}{\kern0pt}{\isachardot}{\kern0pt}\ x\ y{\isacharparenright}{\kern0pt}\ {\isasymin}\ cube\ {\isadigit{1}}\ t{\isachardoublequoteclose}\ \isakeyword{and}\ {\isachardoublequoteopen}{\isacharparenleft}{\kern0pt}{\isasymlambda}y\ {\isasymin}\ {\isacharbraceleft}{\kern0pt}{\isachardot}{\kern0pt}{\isachardot}{\kern0pt}{\isacharless}{\kern0pt}k{\isacharbraceright}{\kern0pt}{\isachardot}{\kern0pt}\ x\ {\isacharparenleft}{\kern0pt}y\ {\isacharplus}{\kern0pt}\ {\isadigit{1}}{\isacharparenright}{\kern0pt}{\isacharparenright}{\kern0pt}\ {\isasymin}\ cube\ k\ t{\isachardoublequoteclose}\isanewline
%
\isadelimproof
\ \ %
\endisadelimproof
%
\isatagproof
\isacommand{using}\isamarkupfalse%
\ assms\ \isacommand{unfolding}\isamarkupfalse%
\ cube{\isacharunderscore}{\kern0pt}def\ \isacommand{by}\isamarkupfalse%
\ auto%
\endisatagproof
{\isafoldproof}%
%
\isadelimproof
%
\endisadelimproof
%
\isadelimdocument
%
\endisadelimdocument
%
\isatagdocument
%
\isamarkupsubsection{Subspaces%
}
\isamarkuptrue%
%
\endisatagdocument
{\isafolddocument}%
%
\isadelimdocument
%
\endisadelimdocument
%
\begin{isamarkuptext}%
The property of being a $k$-dimensional subspace of $C^n_t$ is defined in the following using the variables:

\begin{tabular}{llp{8cm}}
$S$:& \isa{{\isacharparenleft}{\kern0pt}nat\ {\isasymRightarrow}\ nat{\isacharparenright}{\kern0pt}\ {\isasymRightarrow}\ {\isacharparenleft}{\kern0pt}nat\ {\isasymRightarrow}\ nat{\isacharparenright}{\kern0pt}}& the subspace\\
$k$:& \isa{nat}& the dimension of the subspace\\
$n$:& \isa{nat}& the dimension of the cube\\
$t$:& \isa{nat}& the size of the cube's base
\end{tabular}%
\end{isamarkuptext}\isamarkuptrue%
\isacommand{definition}\isamarkupfalse%
\ is{\isacharunderscore}{\kern0pt}subspace\isanewline
\ \ \isakeyword{where}\ {\isachardoublequoteopen}is{\isacharunderscore}{\kern0pt}subspace\ S\ k\ n\ t\ {\isasymequiv}\ {\isacharparenleft}{\kern0pt}{\isasymexists}B\ f{\isachardot}{\kern0pt}\ \isanewline
disjoint{\isacharunderscore}{\kern0pt}family{\isacharunderscore}{\kern0pt}on\ B\ {\isacharbraceleft}{\kern0pt}{\isachardot}{\kern0pt}{\isachardot}{\kern0pt}k{\isacharbraceright}{\kern0pt}\ {\isasymand}\ {\isasymUnion}{\isacharparenleft}{\kern0pt}B\ {\isacharbackquote}{\kern0pt}\ {\isacharbraceleft}{\kern0pt}{\isachardot}{\kern0pt}{\isachardot}{\kern0pt}k{\isacharbraceright}{\kern0pt}{\isacharparenright}{\kern0pt}\ {\isacharequal}{\kern0pt}\ {\isacharbraceleft}{\kern0pt}{\isachardot}{\kern0pt}{\isachardot}{\kern0pt}{\isacharless}{\kern0pt}n{\isacharbraceright}{\kern0pt}\ {\isasymand}\ {\isacharparenleft}{\kern0pt}{\isacharbraceleft}{\kern0pt}{\isacharbraceright}{\kern0pt}\ {\isasymnotin}\ B\ {\isacharbackquote}{\kern0pt}\ {\isacharbraceleft}{\kern0pt}{\isachardot}{\kern0pt}{\isachardot}{\kern0pt}{\isacharless}{\kern0pt}k{\isacharbraceright}{\kern0pt}{\isacharparenright}{\kern0pt}\ {\isasymand}\ f\ {\isasymin}\ {\isacharparenleft}{\kern0pt}B\ k{\isacharparenright}{\kern0pt}\ {\isasymrightarrow}\isactrlsub E\ {\isacharbraceleft}{\kern0pt}{\isachardot}{\kern0pt}{\isachardot}{\kern0pt}{\isacharless}{\kern0pt}t{\isacharbraceright}{\kern0pt}\ {\isasymand}\ S\ {\isasymin}\ {\isacharparenleft}{\kern0pt}cube\ k\ t{\isacharparenright}{\kern0pt}\ {\isasymrightarrow}\isactrlsub E\ {\isacharparenleft}{\kern0pt}cube\ n\ t{\isacharparenright}{\kern0pt}\ {\isasymand}\ {\isacharparenleft}{\kern0pt}{\isasymforall}y\ {\isasymin}\ cube\ k\ t{\isachardot}{\kern0pt}\ {\isacharparenleft}{\kern0pt}{\isasymforall}i\ {\isasymin}\ B\ k{\isachardot}{\kern0pt}\ S\ y\ i\ {\isacharequal}{\kern0pt}\ f\ i{\isacharparenright}{\kern0pt}\ {\isasymand}\ {\isacharparenleft}{\kern0pt}{\isasymforall}j{\isacharless}{\kern0pt}k{\isachardot}{\kern0pt}\ {\isasymforall}i\ {\isasymin}\ B\ j{\isachardot}{\kern0pt}\ {\isacharparenleft}{\kern0pt}S\ y{\isacharparenright}{\kern0pt}\ i\ {\isacharequal}{\kern0pt}\ y\ j{\isacharparenright}{\kern0pt}{\isacharparenright}{\kern0pt}{\isacharparenright}{\kern0pt}{\isachardoublequoteclose}%
\begin{isamarkuptext}%
A subspace can be thought of as an embedding of the $k$-dimensional cube
  into $C^n_t$, akin to how a $k$-dimensional vector subspace of $\mathbf{R}^n$ may be thought of as an embedding of $\mathbf{R}^k$ into $\mathbf{R}^n$.%
\end{isamarkuptext}\isamarkuptrue%
\isacommand{lemma}\isamarkupfalse%
\ subspace{\isacharunderscore}{\kern0pt}inj{\isacharunderscore}{\kern0pt}on{\isacharunderscore}{\kern0pt}cube{\isacharcolon}{\kern0pt}\ \isakeyword{assumes}\ {\isachardoublequoteopen}is{\isacharunderscore}{\kern0pt}subspace\ S\ k\ n\ t{\isachardoublequoteclose}\ \isakeyword{shows}\ {\isachardoublequoteopen}inj{\isacharunderscore}{\kern0pt}on\ S\ {\isacharparenleft}{\kern0pt}cube\ k\ t{\isacharparenright}{\kern0pt}{\isachardoublequoteclose}\isanewline
%
\isadelimproof
%
\endisadelimproof
%
\isatagproof
\isacommand{proof}\isamarkupfalse%
\ \isanewline
\ \isacommand{fix}\isamarkupfalse%
\ x\ y\isanewline
\ \isacommand{assume}\isamarkupfalse%
\ a{\isacharcolon}{\kern0pt}\ {\isachardoublequoteopen}x\ {\isasymin}\ cube\ k\ t{\isachardoublequoteclose}\ {\isachardoublequoteopen}y\ {\isasymin}\ cube\ k\ t{\isachardoublequoteclose}\ {\isachardoublequoteopen}S\ x\ {\isacharequal}{\kern0pt}\ S\ y{\isachardoublequoteclose}\isanewline
\ \isacommand{from}\isamarkupfalse%
\ assms\ \isacommand{obtain}\isamarkupfalse%
\ B\ f\ \isakeyword{where}\ Bf{\isacharunderscore}{\kern0pt}props{\isacharcolon}{\kern0pt}\ {\isachardoublequoteopen}disjoint{\isacharunderscore}{\kern0pt}family{\isacharunderscore}{\kern0pt}on\ B\ {\isacharbraceleft}{\kern0pt}{\isachardot}{\kern0pt}{\isachardot}{\kern0pt}k{\isacharbraceright}{\kern0pt}\ {\isasymand}\ {\isasymUnion}{\isacharparenleft}{\kern0pt}B\ {\isacharbackquote}{\kern0pt}\ {\isacharbraceleft}{\kern0pt}{\isachardot}{\kern0pt}{\isachardot}{\kern0pt}k{\isacharbraceright}{\kern0pt}{\isacharparenright}{\kern0pt}\ {\isacharequal}{\kern0pt}\ {\isacharbraceleft}{\kern0pt}{\isachardot}{\kern0pt}{\isachardot}{\kern0pt}{\isacharless}{\kern0pt}n{\isacharbraceright}{\kern0pt}\ {\isasymand}\ {\isacharparenleft}{\kern0pt}{\isacharbraceleft}{\kern0pt}{\isacharbraceright}{\kern0pt}\ {\isasymnotin}\ B\ {\isacharbackquote}{\kern0pt}\ {\isacharbraceleft}{\kern0pt}{\isachardot}{\kern0pt}{\isachardot}{\kern0pt}{\isacharless}{\kern0pt}k{\isacharbraceright}{\kern0pt}{\isacharparenright}{\kern0pt}\ {\isasymand}\ f\ {\isasymin}\ {\isacharparenleft}{\kern0pt}B\ k{\isacharparenright}{\kern0pt}\ {\isasymrightarrow}\isactrlsub E\ {\isacharbraceleft}{\kern0pt}{\isachardot}{\kern0pt}{\isachardot}{\kern0pt}{\isacharless}{\kern0pt}t{\isacharbraceright}{\kern0pt}\ {\isasymand}\ S\ {\isasymin}\ {\isacharparenleft}{\kern0pt}cube\ k\ t{\isacharparenright}{\kern0pt}\ {\isasymrightarrow}\isactrlsub E\ {\isacharparenleft}{\kern0pt}cube\ n\ t{\isacharparenright}{\kern0pt}\ {\isasymand}\ {\isacharparenleft}{\kern0pt}{\isasymforall}y\ {\isasymin}\ cube\ k\ t{\isachardot}{\kern0pt}\ {\isacharparenleft}{\kern0pt}{\isasymforall}i\ {\isasymin}\ B\ k{\isachardot}{\kern0pt}\ S\ y\ i\ {\isacharequal}{\kern0pt}\ f\ i{\isacharparenright}{\kern0pt}\ {\isasymand}\ {\isacharparenleft}{\kern0pt}{\isasymforall}j{\isacharless}{\kern0pt}k{\isachardot}{\kern0pt}\ {\isasymforall}i\ {\isasymin}\ B\ j{\isachardot}{\kern0pt}\ {\isacharparenleft}{\kern0pt}S\ y{\isacharparenright}{\kern0pt}\ i\ {\isacharequal}{\kern0pt}\ y\ j{\isacharparenright}{\kern0pt}{\isacharparenright}{\kern0pt}{\isachardoublequoteclose}\ \isacommand{unfolding}\isamarkupfalse%
\ is{\isacharunderscore}{\kern0pt}subspace{\isacharunderscore}{\kern0pt}def\ \isacommand{by}\isamarkupfalse%
\ auto\isanewline
\ \isacommand{have}\isamarkupfalse%
\ {\isachardoublequoteopen}{\isasymforall}i{\isacharless}{\kern0pt}k{\isachardot}{\kern0pt}\ x\ i\ {\isacharequal}{\kern0pt}\ y\ i{\isachardoublequoteclose}\isanewline
\ \isacommand{proof}\isamarkupfalse%
\ {\isacharparenleft}{\kern0pt}intro\ allI\ impI{\isacharparenright}{\kern0pt}\isanewline
\ \ \isacommand{fix}\isamarkupfalse%
\ j\ \isacommand{assume}\isamarkupfalse%
\ {\isachardoublequoteopen}j\ {\isacharless}{\kern0pt}\ k{\isachardoublequoteclose}\isanewline
\ \ \ \isacommand{then}\isamarkupfalse%
\ \isacommand{have}\isamarkupfalse%
\ {\isachardoublequoteopen}B\ j\ {\isasymnoteq}\ {\isacharbraceleft}{\kern0pt}{\isacharbraceright}{\kern0pt}{\isachardoublequoteclose}\ \isacommand{using}\isamarkupfalse%
\ Bf{\isacharunderscore}{\kern0pt}props\ \isacommand{by}\isamarkupfalse%
\ auto\isanewline
\ \ \ \isacommand{then}\isamarkupfalse%
\ \isacommand{obtain}\isamarkupfalse%
\ i\ \isakeyword{where}\ i{\isacharunderscore}{\kern0pt}prop{\isacharcolon}{\kern0pt}\ {\isachardoublequoteopen}i\ {\isasymin}\ B\ j{\isachardoublequoteclose}\ \isacommand{by}\isamarkupfalse%
\ blast\isanewline
\ \ \ \isacommand{then}\isamarkupfalse%
\ \isacommand{have}\isamarkupfalse%
\ {\isachardoublequoteopen}y\ j\ {\isacharequal}{\kern0pt}\ S\ y\ i{\isachardoublequoteclose}\ \isacommand{using}\isamarkupfalse%
\ Bf{\isacharunderscore}{\kern0pt}props\ a{\isacharparenleft}{\kern0pt}{\isadigit{2}}{\isacharparenright}{\kern0pt}\ {\isacartoucheopen}j\ {\isacharless}{\kern0pt}\ k{\isacartoucheclose}\ \isacommand{by}\isamarkupfalse%
\ auto\isanewline
\ \ \ \isacommand{also}\isamarkupfalse%
\ \isacommand{have}\isamarkupfalse%
\ {\isachardoublequoteopen}\ {\isachardot}{\kern0pt}{\isachardot}{\kern0pt}{\isachardot}{\kern0pt}\ {\isacharequal}{\kern0pt}\ S\ x\ i{\isachardoublequoteclose}\ \isacommand{using}\isamarkupfalse%
\ a\ \isacommand{by}\isamarkupfalse%
\ simp\isanewline
\ \ \ \isacommand{also}\isamarkupfalse%
\ \isacommand{have}\isamarkupfalse%
\ {\isachardoublequoteopen}\ {\isachardot}{\kern0pt}{\isachardot}{\kern0pt}{\isachardot}{\kern0pt}\ {\isacharequal}{\kern0pt}\ x\ j{\isachardoublequoteclose}\ \isacommand{using}\isamarkupfalse%
\ Bf{\isacharunderscore}{\kern0pt}props\ a{\isacharparenleft}{\kern0pt}{\isadigit{1}}{\isacharparenright}{\kern0pt}\ {\isacartoucheopen}j\ {\isacharless}{\kern0pt}\ k{\isacartoucheclose}\ i{\isacharunderscore}{\kern0pt}prop\ \isacommand{by}\isamarkupfalse%
\ blast\isanewline
\ \ \ \isacommand{finally}\isamarkupfalse%
\ \isacommand{show}\isamarkupfalse%
\ {\isachardoublequoteopen}x\ j\ {\isacharequal}{\kern0pt}\ y\ j{\isachardoublequoteclose}\ \isacommand{by}\isamarkupfalse%
\ simp\isanewline
\ \isacommand{qed}\isamarkupfalse%
\isanewline
\ \isacommand{then}\isamarkupfalse%
\ \isacommand{show}\isamarkupfalse%
\ {\isachardoublequoteopen}x\ {\isacharequal}{\kern0pt}\ y{\isachardoublequoteclose}\ \isacommand{using}\isamarkupfalse%
\ a{\isacharparenleft}{\kern0pt}{\isadigit{1}}{\isacharcomma}{\kern0pt}{\isadigit{2}}{\isacharparenright}{\kern0pt}\ \isacommand{unfolding}\isamarkupfalse%
\ cube{\isacharunderscore}{\kern0pt}def\ \isacommand{by}\isamarkupfalse%
\ {\isacharparenleft}{\kern0pt}meson\ PiE{\isacharunderscore}{\kern0pt}ext\ lessThan{\isacharunderscore}{\kern0pt}iff{\isacharparenright}{\kern0pt}\isanewline
\isacommand{qed}\isamarkupfalse%
%
\endisatagproof
{\isafoldproof}%
%
\isadelimproof
%
\endisadelimproof
%
\begin{isamarkuptext}%
Required to handle base cases in the key lemmas.%
\end{isamarkuptext}\isamarkuptrue%
\isacommand{lemma}\isamarkupfalse%
\ dim{\isadigit{0}}{\isacharunderscore}{\kern0pt}subspace{\isacharunderscore}{\kern0pt}ex{\isacharcolon}{\kern0pt}\ \isakeyword{assumes}\ {\isachardoublequoteopen}t\ {\isachargreater}{\kern0pt}\ {\isadigit{0}}{\isachardoublequoteclose}\ \isakeyword{shows}\ {\isachardoublequoteopen}{\isasymexists}S{\isachardot}{\kern0pt}\ is{\isacharunderscore}{\kern0pt}subspace\ S\ {\isadigit{0}}\ n\ t{\isachardoublequoteclose}\isanewline
%
\isadelimproof
%
\endisadelimproof
%
\isatagproof
\isacommand{proof}\isamarkupfalse%
{\isacharminus}{\kern0pt}\isanewline
\ \ \isacommand{define}\isamarkupfalse%
\ B\ \isakeyword{where}\ {\isachardoublequoteopen}B\ {\isasymequiv}\ {\isacharparenleft}{\kern0pt}{\isasymlambda}x{\isacharcolon}{\kern0pt}{\isacharcolon}{\kern0pt}nat{\isachardot}{\kern0pt}\ undefined{\isacharparenright}{\kern0pt}{\isacharparenleft}{\kern0pt}{\isadigit{0}}{\isacharcolon}{\kern0pt}{\isacharequal}{\kern0pt}{\isacharbraceleft}{\kern0pt}{\isachardot}{\kern0pt}{\isachardot}{\kern0pt}{\isacharless}{\kern0pt}n{\isacharbraceright}{\kern0pt}{\isacharparenright}{\kern0pt}{\isachardoublequoteclose}\isanewline
\isanewline
\ \ \isacommand{have}\isamarkupfalse%
\ {\isachardoublequoteopen}{\isacharbraceleft}{\kern0pt}{\isachardot}{\kern0pt}{\isachardot}{\kern0pt}{\isacharless}{\kern0pt}t{\isacharbraceright}{\kern0pt}\ {\isasymnoteq}\ {\isacharbraceleft}{\kern0pt}{\isacharbraceright}{\kern0pt}{\isachardoublequoteclose}\ \isacommand{using}\isamarkupfalse%
\ assms\ \isacommand{by}\isamarkupfalse%
\ auto\isanewline
\ \ \isacommand{then}\isamarkupfalse%
\ \isacommand{have}\isamarkupfalse%
\ {\isachardoublequoteopen}{\isasymexists}f{\isachardot}{\kern0pt}\ f\ {\isasymin}\ {\isacharparenleft}{\kern0pt}B\ {\isadigit{0}}{\isacharparenright}{\kern0pt}\ {\isasymrightarrow}\isactrlsub E\ {\isacharbraceleft}{\kern0pt}{\isachardot}{\kern0pt}{\isachardot}{\kern0pt}{\isacharless}{\kern0pt}t{\isacharbraceright}{\kern0pt}{\isachardoublequoteclose}\ \isanewline
\ \ \ \ \isacommand{by}\isamarkupfalse%
\ {\isacharparenleft}{\kern0pt}meson\ PiE{\isacharunderscore}{\kern0pt}eq{\isacharunderscore}{\kern0pt}empty{\isacharunderscore}{\kern0pt}iff\ all{\isacharunderscore}{\kern0pt}not{\isacharunderscore}{\kern0pt}in{\isacharunderscore}{\kern0pt}conv{\isacharparenright}{\kern0pt}\isanewline
\ \ \isacommand{then}\isamarkupfalse%
\ \isacommand{obtain}\isamarkupfalse%
\ f\ \isakeyword{where}\ f{\isacharunderscore}{\kern0pt}prop{\isacharcolon}{\kern0pt}\ {\isachardoublequoteopen}f\ {\isasymin}\ {\isacharparenleft}{\kern0pt}B\ {\isadigit{0}}{\isacharparenright}{\kern0pt}\ {\isasymrightarrow}\isactrlsub E\ {\isacharbraceleft}{\kern0pt}{\isachardot}{\kern0pt}{\isachardot}{\kern0pt}{\isacharless}{\kern0pt}t{\isacharbraceright}{\kern0pt}{\isachardoublequoteclose}\ \isacommand{by}\isamarkupfalse%
\ blast\isanewline
\ \ \isacommand{define}\isamarkupfalse%
\ S\ \isakeyword{where}\ {\isachardoublequoteopen}S\ {\isasymequiv}\ {\isacharparenleft}{\kern0pt}{\isasymlambda}x{\isacharcolon}{\kern0pt}{\isacharcolon}{\kern0pt}{\isacharparenleft}{\kern0pt}nat\ {\isasymRightarrow}\ nat{\isacharparenright}{\kern0pt}{\isachardot}{\kern0pt}\ undefined{\isacharparenright}{\kern0pt}{\isacharparenleft}{\kern0pt}{\isacharparenleft}{\kern0pt}{\isasymlambda}x{\isachardot}{\kern0pt}\ undefined{\isacharparenright}{\kern0pt}{\isacharcolon}{\kern0pt}{\isacharequal}{\kern0pt}f{\isacharparenright}{\kern0pt}{\isachardoublequoteclose}\isanewline
\isanewline
\ \ \isacommand{have}\isamarkupfalse%
\ {\isachardoublequoteopen}disjoint{\isacharunderscore}{\kern0pt}family{\isacharunderscore}{\kern0pt}on\ B\ {\isacharbraceleft}{\kern0pt}{\isachardot}{\kern0pt}{\isachardot}{\kern0pt}{\isadigit{0}}{\isacharbraceright}{\kern0pt}{\isachardoublequoteclose}\ \isacommand{unfolding}\isamarkupfalse%
\ disjoint{\isacharunderscore}{\kern0pt}family{\isacharunderscore}{\kern0pt}on{\isacharunderscore}{\kern0pt}def\ \isacommand{by}\isamarkupfalse%
\ simp\isanewline
\ \ \isacommand{moreover}\isamarkupfalse%
\ \isacommand{have}\isamarkupfalse%
\ {\isachardoublequoteopen}{\isasymUnion}{\isacharparenleft}{\kern0pt}B\ {\isacharbackquote}{\kern0pt}\ {\isacharbraceleft}{\kern0pt}{\isachardot}{\kern0pt}{\isachardot}{\kern0pt}{\isadigit{0}}{\isacharbraceright}{\kern0pt}{\isacharparenright}{\kern0pt}\ {\isacharequal}{\kern0pt}\ {\isacharbraceleft}{\kern0pt}{\isachardot}{\kern0pt}{\isachardot}{\kern0pt}{\isacharless}{\kern0pt}n{\isacharbraceright}{\kern0pt}{\isachardoublequoteclose}\ \isacommand{unfolding}\isamarkupfalse%
\ B{\isacharunderscore}{\kern0pt}def\ \isacommand{by}\isamarkupfalse%
\ simp\isanewline
\ \ \isacommand{moreover}\isamarkupfalse%
\ \isacommand{have}\isamarkupfalse%
\ {\isachardoublequoteopen}{\isacharparenleft}{\kern0pt}{\isacharbraceleft}{\kern0pt}{\isacharbraceright}{\kern0pt}\ {\isasymnotin}\ B\ {\isacharbackquote}{\kern0pt}\ {\isacharbraceleft}{\kern0pt}{\isachardot}{\kern0pt}{\isachardot}{\kern0pt}{\isacharless}{\kern0pt}{\isadigit{0}}{\isacharbraceright}{\kern0pt}{\isacharparenright}{\kern0pt}{\isachardoublequoteclose}\ \isacommand{by}\isamarkupfalse%
\ simp\isanewline
\ \ \isacommand{moreover}\isamarkupfalse%
\ \isacommand{have}\isamarkupfalse%
\ {\isachardoublequoteopen}S\ {\isasymin}\ {\isacharparenleft}{\kern0pt}cube\ {\isadigit{0}}\ t{\isacharparenright}{\kern0pt}\ {\isasymrightarrow}\isactrlsub E\ {\isacharparenleft}{\kern0pt}cube\ n\ t{\isacharparenright}{\kern0pt}{\isachardoublequoteclose}\isanewline
\ \ \ \ \isacommand{using}\isamarkupfalse%
\ f{\isacharunderscore}{\kern0pt}prop\ PiE{\isacharunderscore}{\kern0pt}I\ \isacommand{unfolding}\isamarkupfalse%
\ B{\isacharunderscore}{\kern0pt}def\ cube{\isacharunderscore}{\kern0pt}def\ S{\isacharunderscore}{\kern0pt}def\ \isacommand{by}\isamarkupfalse%
\ auto\isanewline
\ \ \isacommand{moreover}\isamarkupfalse%
\ \isacommand{have}\isamarkupfalse%
\ {\isachardoublequoteopen}{\isacharparenleft}{\kern0pt}{\isasymforall}y\ {\isasymin}\ cube\ {\isadigit{0}}\ t{\isachardot}{\kern0pt}\ {\isacharparenleft}{\kern0pt}{\isasymforall}i\ {\isasymin}\ B\ {\isadigit{0}}{\isachardot}{\kern0pt}\ S\ y\ i\ {\isacharequal}{\kern0pt}\ f\ i{\isacharparenright}{\kern0pt}\ {\isasymand}\ {\isacharparenleft}{\kern0pt}{\isasymforall}j{\isacharless}{\kern0pt}{\isadigit{0}}{\isachardot}{\kern0pt}\ {\isasymforall}i\ {\isasymin}\ B\ j{\isachardot}{\kern0pt}\ {\isacharparenleft}{\kern0pt}S\ y{\isacharparenright}{\kern0pt}\ i\ {\isacharequal}{\kern0pt}\ y\ j{\isacharparenright}{\kern0pt}{\isacharparenright}{\kern0pt}{\isachardoublequoteclose}\ \isacommand{unfolding}\isamarkupfalse%
\ cube{\isacharunderscore}{\kern0pt}def\ S{\isacharunderscore}{\kern0pt}def\ \isacommand{by}\isamarkupfalse%
\ force\isanewline
\ \ \isacommand{ultimately}\isamarkupfalse%
\ \isacommand{have}\isamarkupfalse%
\ {\isachardoublequoteopen}is{\isacharunderscore}{\kern0pt}subspace\ S\ {\isadigit{0}}\ n\ t{\isachardoublequoteclose}\ \isacommand{using}\isamarkupfalse%
\ f{\isacharunderscore}{\kern0pt}prop\ \isacommand{unfolding}\isamarkupfalse%
\ is{\isacharunderscore}{\kern0pt}subspace{\isacharunderscore}{\kern0pt}def\ \isacommand{by}\isamarkupfalse%
\ blast\isanewline
\ \ \isacommand{then}\isamarkupfalse%
\ \isacommand{show}\isamarkupfalse%
\ {\isachardoublequoteopen}{\isasymexists}S{\isachardot}{\kern0pt}\ is{\isacharunderscore}{\kern0pt}subspace\ S\ {\isadigit{0}}\ n\ t{\isachardoublequoteclose}\ \isacommand{by}\isamarkupfalse%
\ auto\isanewline
\isacommand{qed}\isamarkupfalse%
%
\endisatagproof
{\isafoldproof}%
%
\isadelimproof
%
\endisadelimproof
%
\isadelimdocument
%
\endisadelimdocument
%
\isatagdocument
%
\isamarkupsubsection{Equivalence classes%
}
\isamarkuptrue%
%
\endisatagdocument
{\isafolddocument}%
%
\isadelimdocument
%
\endisadelimdocument
%
\begin{isamarkuptext}%
Defining the equivalence classes of (cube n (t + 1)). \{classes n t 0, ..., classes n t n\}%
\end{isamarkuptext}\isamarkuptrue%
\isacommand{definition}\isamarkupfalse%
\ classes\isanewline
\ \ \isakeyword{where}\ {\isachardoublequoteopen}classes\ n\ t\ {\isasymequiv}\ {\isacharparenleft}{\kern0pt}{\isasymlambda}i{\isachardot}{\kern0pt}\ {\isacharbraceleft}{\kern0pt}x\ {\isachardot}{\kern0pt}\ x\ {\isasymin}\ {\isacharparenleft}{\kern0pt}cube\ n\ {\isacharparenleft}{\kern0pt}t\ {\isacharplus}{\kern0pt}\ {\isadigit{1}}{\isacharparenright}{\kern0pt}{\isacharparenright}{\kern0pt}\ {\isasymand}\ {\isacharparenleft}{\kern0pt}{\isasymforall}u\ {\isasymin}\ {\isacharbraceleft}{\kern0pt}{\isacharparenleft}{\kern0pt}n{\isacharminus}{\kern0pt}i{\isacharparenright}{\kern0pt}{\isachardot}{\kern0pt}{\isachardot}{\kern0pt}{\isacharless}{\kern0pt}n{\isacharbraceright}{\kern0pt}{\isachardot}{\kern0pt}\ x\ u\ {\isacharequal}{\kern0pt}\ t{\isacharparenright}{\kern0pt}\ {\isasymand}\ t\ {\isasymnotin}\ x\ {\isacharbackquote}{\kern0pt}\ {\isacharbraceleft}{\kern0pt}{\isachardot}{\kern0pt}{\isachardot}{\kern0pt}{\isacharless}{\kern0pt}{\isacharparenleft}{\kern0pt}n\ {\isacharminus}{\kern0pt}\ i{\isacharparenright}{\kern0pt}{\isacharbraceright}{\kern0pt}{\isacharbraceright}{\kern0pt}{\isacharparenright}{\kern0pt}{\isachardoublequoteclose}\isanewline
\isanewline
\isacommand{lemma}\isamarkupfalse%
\ classes{\isacharunderscore}{\kern0pt}subset{\isacharunderscore}{\kern0pt}cube{\isacharcolon}{\kern0pt}\ {\isachardoublequoteopen}classes\ n\ t\ i\ {\isasymsubseteq}\ cube\ n\ {\isacharparenleft}{\kern0pt}t{\isacharplus}{\kern0pt}{\isadigit{1}}{\isacharparenright}{\kern0pt}{\isachardoublequoteclose}%
\isadelimproof
\ %
\endisadelimproof
%
\isatagproof
\isacommand{unfolding}\isamarkupfalse%
\ classes{\isacharunderscore}{\kern0pt}def\ \isacommand{by}\isamarkupfalse%
\ blast%
\endisatagproof
{\isafoldproof}%
%
\isadelimproof
%
\endisadelimproof
\isanewline
\isanewline
\isacommand{definition}\isamarkupfalse%
\ layered{\isacharunderscore}{\kern0pt}subspace\isanewline
\ \ \isakeyword{where}\ {\isachardoublequoteopen}layered{\isacharunderscore}{\kern0pt}subspace\ S\ k\ n\ t\ r\ {\isasymchi}\ {\isasymequiv}\ {\isacharparenleft}{\kern0pt}is{\isacharunderscore}{\kern0pt}subspace\ S\ k\ n\ {\isacharparenleft}{\kern0pt}t\ {\isacharplus}{\kern0pt}\ {\isadigit{1}}{\isacharparenright}{\kern0pt}\ \ {\isasymand}\ {\isacharparenleft}{\kern0pt}{\isasymforall}i\ {\isasymin}\ {\isacharbraceleft}{\kern0pt}{\isachardot}{\kern0pt}{\isachardot}{\kern0pt}k{\isacharbraceright}{\kern0pt}{\isachardot}{\kern0pt}\ {\isasymexists}c{\isacharless}{\kern0pt}r{\isachardot}{\kern0pt}\ {\isasymforall}x\ {\isasymin}\ classes\ k\ t\ i{\isachardot}{\kern0pt}\ {\isasymchi}\ {\isacharparenleft}{\kern0pt}S\ x{\isacharparenright}{\kern0pt}\ {\isacharequal}{\kern0pt}\ c{\isacharparenright}{\kern0pt}{\isacharparenright}{\kern0pt}\ {\isasymand}\ {\isasymchi}\ {\isasymin}\ cube\ n\ {\isacharparenleft}{\kern0pt}t\ {\isacharplus}{\kern0pt}\ {\isadigit{1}}{\isacharparenright}{\kern0pt}\ {\isasymrightarrow}\isactrlsub E\ {\isacharbraceleft}{\kern0pt}{\isachardot}{\kern0pt}{\isachardot}{\kern0pt}{\isacharless}{\kern0pt}r{\isacharbraceright}{\kern0pt}{\isachardoublequoteclose}\isanewline
\isanewline
\isanewline
\isacommand{lemma}\isamarkupfalse%
\ layered{\isacharunderscore}{\kern0pt}eq{\isacharunderscore}{\kern0pt}classes{\isacharcolon}{\kern0pt}\ \isakeyword{assumes}{\isachardoublequoteopen}layered{\isacharunderscore}{\kern0pt}subspace\ S\ k\ n\ t\ r\ {\isasymchi}{\isachardoublequoteclose}\ \isakeyword{shows}\ {\isachardoublequoteopen}{\isasymforall}i\ {\isasymin}\ {\isacharbraceleft}{\kern0pt}{\isachardot}{\kern0pt}{\isachardot}{\kern0pt}k{\isacharbraceright}{\kern0pt}{\isachardot}{\kern0pt}\ {\isasymforall}x\ {\isasymin}\ classes\ k\ t\ i{\isachardot}{\kern0pt}\ {\isasymforall}y\ {\isasymin}\ classes\ k\ t\ i{\isachardot}{\kern0pt}\ {\isasymchi}\ {\isacharparenleft}{\kern0pt}S\ x{\isacharparenright}{\kern0pt}\ {\isacharequal}{\kern0pt}\ {\isasymchi}\ {\isacharparenleft}{\kern0pt}S\ y{\isacharparenright}{\kern0pt}{\isachardoublequoteclose}\ \isanewline
%
\isadelimproof
%
\endisadelimproof
%
\isatagproof
\isacommand{proof}\isamarkupfalse%
\ {\isacharparenleft}{\kern0pt}safe{\isacharparenright}{\kern0pt}\isanewline
\ \ \isacommand{fix}\isamarkupfalse%
\ i\ x\ y\isanewline
\ \ \isacommand{assume}\isamarkupfalse%
\ a{\isacharcolon}{\kern0pt}\ {\isachardoublequoteopen}i\ {\isasymle}\ k{\isachardoublequoteclose}\ {\isachardoublequoteopen}x\ {\isasymin}\ classes\ k\ t\ i{\isachardoublequoteclose}\ {\isachardoublequoteopen}y\ {\isasymin}\ classes\ k\ t\ i{\isachardoublequoteclose}\isanewline
\ \ \isacommand{then}\isamarkupfalse%
\ \isacommand{obtain}\isamarkupfalse%
\ c\ \isakeyword{where}\ {\isachardoublequoteopen}c\ {\isacharless}{\kern0pt}\ r\ {\isasymand}\ {\isasymchi}\ {\isacharparenleft}{\kern0pt}S\ x{\isacharparenright}{\kern0pt}\ {\isacharequal}{\kern0pt}\ c\ {\isasymand}\ {\isasymchi}\ {\isacharparenleft}{\kern0pt}S\ y{\isacharparenright}{\kern0pt}\ {\isacharequal}{\kern0pt}\ c{\isachardoublequoteclose}\ \isacommand{using}\isamarkupfalse%
\ assms\ \isacommand{unfolding}\isamarkupfalse%
\ layered{\isacharunderscore}{\kern0pt}subspace{\isacharunderscore}{\kern0pt}def\ \isacommand{by}\isamarkupfalse%
\ fast\isanewline
\ \ \isacommand{then}\isamarkupfalse%
\ \isacommand{show}\isamarkupfalse%
\ {\isachardoublequoteopen}{\isasymchi}\ {\isacharparenleft}{\kern0pt}S\ x{\isacharparenright}{\kern0pt}\ {\isacharequal}{\kern0pt}\ {\isasymchi}\ {\isacharparenleft}{\kern0pt}S\ y{\isacharparenright}{\kern0pt}{\isachardoublequoteclose}\ \isacommand{by}\isamarkupfalse%
\ simp\isanewline
\isacommand{qed}\isamarkupfalse%
%
\endisatagproof
{\isafoldproof}%
%
\isadelimproof
\isanewline
%
\endisadelimproof
\isanewline
\isacommand{lemma}\isamarkupfalse%
\ dim{\isadigit{0}}{\isacharunderscore}{\kern0pt}layered{\isacharunderscore}{\kern0pt}subspace{\isacharunderscore}{\kern0pt}ex{\isacharcolon}{\kern0pt}\ \isakeyword{assumes}\ {\isachardoublequoteopen}{\isasymchi}\ {\isasymin}\ {\isacharparenleft}{\kern0pt}cube\ n\ {\isacharparenleft}{\kern0pt}t\ {\isacharplus}{\kern0pt}\ {\isadigit{1}}{\isacharparenright}{\kern0pt}{\isacharparenright}{\kern0pt}\ {\isasymrightarrow}\isactrlsub E\ {\isacharbraceleft}{\kern0pt}{\isachardot}{\kern0pt}{\isachardot}{\kern0pt}{\isacharless}{\kern0pt}r{\isacharcolon}{\kern0pt}{\isacharcolon}{\kern0pt}nat{\isacharbraceright}{\kern0pt}{\isachardoublequoteclose}\ \isakeyword{shows}\ {\isachardoublequoteopen}{\isasymexists}S{\isachardot}{\kern0pt}\ layered{\isacharunderscore}{\kern0pt}subspace\ S\ {\isacharparenleft}{\kern0pt}{\isadigit{0}}{\isacharcolon}{\kern0pt}{\isacharcolon}{\kern0pt}nat{\isacharparenright}{\kern0pt}\ n\ t\ r\ {\isasymchi}{\isachardoublequoteclose}\isanewline
%
\isadelimproof
%
\endisadelimproof
%
\isatagproof
\isacommand{proof}\isamarkupfalse%
{\isacharminus}{\kern0pt}\isanewline
\ \ \isacommand{obtain}\isamarkupfalse%
\ S\ \isakeyword{where}\ S{\isacharunderscore}{\kern0pt}prop{\isacharcolon}{\kern0pt}\ {\isachardoublequoteopen}is{\isacharunderscore}{\kern0pt}subspace\ S\ {\isacharparenleft}{\kern0pt}{\isadigit{0}}{\isacharcolon}{\kern0pt}{\isacharcolon}{\kern0pt}nat{\isacharparenright}{\kern0pt}\ n\ {\isacharparenleft}{\kern0pt}t{\isacharplus}{\kern0pt}{\isadigit{1}}{\isacharparenright}{\kern0pt}{\isachardoublequoteclose}\ \isacommand{using}\isamarkupfalse%
\ dim{\isadigit{0}}{\isacharunderscore}{\kern0pt}subspace{\isacharunderscore}{\kern0pt}ex\ \isacommand{by}\isamarkupfalse%
\ auto\isanewline
\ \ \isacommand{have}\isamarkupfalse%
\ {\isachardoublequoteopen}classes\ {\isacharparenleft}{\kern0pt}{\isadigit{0}}{\isacharcolon}{\kern0pt}{\isacharcolon}{\kern0pt}nat{\isacharparenright}{\kern0pt}\ t\ {\isadigit{0}}\ {\isacharequal}{\kern0pt}\ cube\ {\isadigit{0}}\ {\isacharparenleft}{\kern0pt}t{\isacharplus}{\kern0pt}{\isadigit{1}}{\isacharparenright}{\kern0pt}{\isachardoublequoteclose}\ \isacommand{unfolding}\isamarkupfalse%
\ classes{\isacharunderscore}{\kern0pt}def\ \isacommand{by}\isamarkupfalse%
\ simp\isanewline
\ \ \isacommand{moreover}\isamarkupfalse%
\ \isacommand{have}\isamarkupfalse%
\ {\isachardoublequoteopen}{\isacharparenleft}{\kern0pt}{\isasymforall}i\ {\isasymin}\ {\isacharbraceleft}{\kern0pt}{\isachardot}{\kern0pt}{\isachardot}{\kern0pt}{\isadigit{0}}{\isacharcolon}{\kern0pt}{\isacharcolon}{\kern0pt}nat{\isacharbraceright}{\kern0pt}{\isachardot}{\kern0pt}\ {\isasymexists}c{\isacharless}{\kern0pt}r{\isachardot}{\kern0pt}\ {\isasymforall}x\ {\isasymin}\ classes\ {\isacharparenleft}{\kern0pt}{\isadigit{0}}{\isacharcolon}{\kern0pt}{\isacharcolon}{\kern0pt}nat{\isacharparenright}{\kern0pt}\ t\ i{\isachardot}{\kern0pt}\ {\isasymchi}\ {\isacharparenleft}{\kern0pt}S\ x{\isacharparenright}{\kern0pt}\ {\isacharequal}{\kern0pt}\ c{\isacharparenright}{\kern0pt}{\isachardoublequoteclose}\isanewline
\ \ \isacommand{proof}\isamarkupfalse%
{\isacharparenleft}{\kern0pt}safe{\isacharparenright}{\kern0pt}\isanewline
\ \ \ \ \isacommand{fix}\isamarkupfalse%
\ i\isanewline
\ \ \ \ \isacommand{have}\isamarkupfalse%
\ {\isachardoublequoteopen}{\isasymforall}x\ {\isasymin}\ classes\ {\isadigit{0}}\ t\ {\isadigit{0}}{\isachardot}{\kern0pt}\ {\isasymchi}\ {\isacharparenleft}{\kern0pt}S\ x{\isacharparenright}{\kern0pt}\ {\isacharequal}{\kern0pt}\ {\isasymchi}\ {\isacharparenleft}{\kern0pt}S\ {\isacharparenleft}{\kern0pt}{\isasymlambda}x{\isachardot}{\kern0pt}\ undefined{\isacharparenright}{\kern0pt}{\isacharparenright}{\kern0pt}{\isachardoublequoteclose}\ \isacommand{using}\isamarkupfalse%
\ cube{\isadigit{0}}{\isacharunderscore}{\kern0pt}alt{\isacharunderscore}{\kern0pt}def\ \isanewline
\ \ \ \ \ \ \isacommand{using}\isamarkupfalse%
\ {\isacartoucheopen}classes\ {\isadigit{0}}\ t\ {\isadigit{0}}\ {\isacharequal}{\kern0pt}\ cube\ {\isadigit{0}}\ {\isacharparenleft}{\kern0pt}t\ {\isacharplus}{\kern0pt}\ {\isadigit{1}}{\isacharparenright}{\kern0pt}{\isacartoucheclose}\ \isacommand{by}\isamarkupfalse%
\ auto\isanewline
\ \ \ \ \isacommand{moreover}\isamarkupfalse%
\ \isacommand{have}\isamarkupfalse%
\ {\isachardoublequoteopen}S\ {\isacharparenleft}{\kern0pt}{\isasymlambda}x{\isachardot}{\kern0pt}\ undefined{\isacharparenright}{\kern0pt}\ {\isasymin}\ cube\ n\ {\isacharparenleft}{\kern0pt}t{\isacharplus}{\kern0pt}{\isadigit{1}}{\isacharparenright}{\kern0pt}{\isachardoublequoteclose}\ \isacommand{using}\isamarkupfalse%
\ S{\isacharunderscore}{\kern0pt}prop\ cube{\isadigit{0}}{\isacharunderscore}{\kern0pt}alt{\isacharunderscore}{\kern0pt}def\ \isacommand{unfolding}\isamarkupfalse%
\ is{\isacharunderscore}{\kern0pt}subspace{\isacharunderscore}{\kern0pt}def\ \isacommand{by}\isamarkupfalse%
\ auto\isanewline
\ \ \ \ \isacommand{moreover}\isamarkupfalse%
\ \isacommand{have}\isamarkupfalse%
\ {\isachardoublequoteopen}{\isasymchi}\ {\isacharparenleft}{\kern0pt}S\ {\isacharparenleft}{\kern0pt}{\isasymlambda}x{\isachardot}{\kern0pt}\ undefined{\isacharparenright}{\kern0pt}{\isacharparenright}{\kern0pt}\ {\isacharless}{\kern0pt}\ r{\isachardoublequoteclose}\ \isacommand{using}\isamarkupfalse%
\ assms\ calculation\ \isacommand{by}\isamarkupfalse%
\ auto\isanewline
\ \ \ \ \isacommand{ultimately}\isamarkupfalse%
\ \isacommand{show}\isamarkupfalse%
\ {\isachardoublequoteopen}{\isasymexists}c{\isacharless}{\kern0pt}r{\isachardot}{\kern0pt}\ {\isasymforall}x{\isasymin}classes\ {\isadigit{0}}\ t\ {\isadigit{0}}{\isachardot}{\kern0pt}\ {\isasymchi}\ {\isacharparenleft}{\kern0pt}S\ x{\isacharparenright}{\kern0pt}\ {\isacharequal}{\kern0pt}\ c{\isachardoublequoteclose}\ \isacommand{by}\isamarkupfalse%
\ auto\isanewline
\ \ \isacommand{qed}\isamarkupfalse%
\isanewline
\ \ \isacommand{ultimately}\isamarkupfalse%
\ \isacommand{have}\isamarkupfalse%
\ {\isachardoublequoteopen}layered{\isacharunderscore}{\kern0pt}subspace\ S\ {\isadigit{0}}\ n\ t\ r\ {\isasymchi}{\isachardoublequoteclose}\ \isacommand{using}\isamarkupfalse%
\ S{\isacharunderscore}{\kern0pt}prop\ assms\ \isacommand{unfolding}\isamarkupfalse%
\ layered{\isacharunderscore}{\kern0pt}subspace{\isacharunderscore}{\kern0pt}def\ \isacommand{by}\isamarkupfalse%
\ blast\isanewline
\ \ \isacommand{then}\isamarkupfalse%
\ \isacommand{show}\isamarkupfalse%
\ {\isachardoublequoteopen}{\isasymexists}S{\isachardot}{\kern0pt}\ layered{\isacharunderscore}{\kern0pt}subspace\ S\ {\isacharparenleft}{\kern0pt}{\isadigit{0}}{\isacharcolon}{\kern0pt}{\isacharcolon}{\kern0pt}nat{\isacharparenright}{\kern0pt}\ n\ t\ r\ {\isasymchi}{\isachardoublequoteclose}\ \isacommand{by}\isamarkupfalse%
\ auto\isanewline
\isacommand{qed}\isamarkupfalse%
%
\endisatagproof
{\isafoldproof}%
%
\isadelimproof
%
\endisadelimproof
%
\begin{isamarkuptext}%
Proving they are equivalence classes.%
\end{isamarkuptext}\isamarkuptrue%
\isacommand{lemma}\isamarkupfalse%
\ disjoint{\isacharunderscore}{\kern0pt}family{\isacharunderscore}{\kern0pt}onI\ {\isacharbrackleft}{\kern0pt}intro{\isacharbrackright}{\kern0pt}{\isacharcolon}{\kern0pt}\isanewline
\ \ \isakeyword{assumes}\ {\isachardoublequoteopen}{\isasymAnd}m\ n{\isachardot}{\kern0pt}\ m\ {\isasymin}\ S\ {\isasymLongrightarrow}\ n\ {\isasymin}\ S\ {\isasymLongrightarrow}\ m\ {\isasymnoteq}\ n\ {\isasymLongrightarrow}\ A\ m\ {\isasyminter}\ A\ n\ {\isacharequal}{\kern0pt}\ {\isacharbraceleft}{\kern0pt}{\isacharbraceright}{\kern0pt}{\isachardoublequoteclose}\isanewline
\ \ \isakeyword{shows}\ \ \ {\isachardoublequoteopen}disjoint{\isacharunderscore}{\kern0pt}family{\isacharunderscore}{\kern0pt}on\ A\ S{\isachardoublequoteclose}\isanewline
%
\isadelimproof
\ \ %
\endisadelimproof
%
\isatagproof
\isacommand{using}\isamarkupfalse%
\ assms\ \isacommand{by}\isamarkupfalse%
\ {\isacharparenleft}{\kern0pt}auto\ simp{\isacharcolon}{\kern0pt}\ disjoint{\isacharunderscore}{\kern0pt}family{\isacharunderscore}{\kern0pt}on{\isacharunderscore}{\kern0pt}def{\isacharparenright}{\kern0pt}%
\endisatagproof
{\isafoldproof}%
%
\isadelimproof
\isanewline
%
\endisadelimproof
\isanewline
\isacommand{lemma}\isamarkupfalse%
\ fun{\isacharunderscore}{\kern0pt}ex{\isacharcolon}{\kern0pt}\ {\isachardoublequoteopen}a\ {\isasymin}\ A\ {\isasymLongrightarrow}\ b\ {\isasymin}\ B\ {\isasymLongrightarrow}\ {\isasymexists}f\ {\isasymin}\ A\ {\isasymrightarrow}\isactrlsub E\ B{\isachardot}{\kern0pt}\ f\ a\ {\isacharequal}{\kern0pt}\ b{\isachardoublequoteclose}\ \isanewline
%
\isadelimproof
%
\endisadelimproof
%
\isatagproof
\isacommand{proof}\isamarkupfalse%
{\isacharminus}{\kern0pt}\isanewline
\ \ \isacommand{assume}\isamarkupfalse%
\ assms{\isacharcolon}{\kern0pt}\ {\isachardoublequoteopen}a\ {\isasymin}\ A{\isachardoublequoteclose}\ {\isachardoublequoteopen}b\ {\isasymin}\ B{\isachardoublequoteclose}\isanewline
\ \ \isacommand{then}\isamarkupfalse%
\ \isacommand{obtain}\isamarkupfalse%
\ g\ \isakeyword{where}\ g{\isacharunderscore}{\kern0pt}def{\isacharcolon}{\kern0pt}\ {\isachardoublequoteopen}g\ {\isasymin}\ A\ {\isasymrightarrow}\ B\ {\isasymand}\ g\ a\ {\isacharequal}{\kern0pt}\ b{\isachardoublequoteclose}\ \isacommand{by}\isamarkupfalse%
\ fast\isanewline
\ \ \isacommand{then}\isamarkupfalse%
\ \isacommand{have}\isamarkupfalse%
\ {\isachardoublequoteopen}restrict\ g\ A\ {\isasymin}\ A\ {\isasymrightarrow}\isactrlsub E\ B\ {\isasymand}\ {\isacharparenleft}{\kern0pt}restrict\ g\ A{\isacharparenright}{\kern0pt}\ a\ {\isacharequal}{\kern0pt}\ b{\isachardoublequoteclose}\ \isacommand{using}\isamarkupfalse%
\ assms{\isacharparenleft}{\kern0pt}{\isadigit{1}}{\isacharparenright}{\kern0pt}\ \isacommand{by}\isamarkupfalse%
\ auto\isanewline
\ \ \isacommand{then}\isamarkupfalse%
\ \isacommand{show}\isamarkupfalse%
\ {\isacharquery}{\kern0pt}thesis\ \isacommand{by}\isamarkupfalse%
\ blast\isanewline
\isacommand{qed}\isamarkupfalse%
%
\endisatagproof
{\isafoldproof}%
%
\isadelimproof
\isanewline
%
\endisadelimproof
\isanewline
\isacommand{lemma}\isamarkupfalse%
\ one{\isacharunderscore}{\kern0pt}dim{\isacharunderscore}{\kern0pt}cube{\isacharunderscore}{\kern0pt}eq{\isacharunderscore}{\kern0pt}nat{\isacharunderscore}{\kern0pt}set{\isacharcolon}{\kern0pt}\ {\isachardoublequoteopen}bij{\isacharunderscore}{\kern0pt}betw\ {\isacharparenleft}{\kern0pt}{\isasymlambda}f{\isachardot}{\kern0pt}\ f\ {\isadigit{0}}{\isacharparenright}{\kern0pt}\ {\isacharparenleft}{\kern0pt}cube\ {\isadigit{1}}\ k{\isacharparenright}{\kern0pt}\ {\isacharbraceleft}{\kern0pt}{\isachardot}{\kern0pt}{\isachardot}{\kern0pt}{\isacharless}{\kern0pt}k{\isacharbraceright}{\kern0pt}{\isachardoublequoteclose}\isanewline
%
\isadelimproof
%
\endisadelimproof
%
\isatagproof
\isacommand{proof}\isamarkupfalse%
\ {\isacharparenleft}{\kern0pt}unfold\ bij{\isacharunderscore}{\kern0pt}betw{\isacharunderscore}{\kern0pt}def{\isacharparenright}{\kern0pt}\isanewline
\ \ \isacommand{have}\isamarkupfalse%
\ {\isacharasterisk}{\kern0pt}{\isacharcolon}{\kern0pt}\ {\isachardoublequoteopen}{\isacharparenleft}{\kern0pt}{\isasymlambda}f{\isachardot}{\kern0pt}\ f\ {\isadigit{0}}{\isacharparenright}{\kern0pt}\ {\isacharbackquote}{\kern0pt}\ cube\ {\isadigit{1}}\ k\ {\isacharequal}{\kern0pt}\ {\isacharbraceleft}{\kern0pt}{\isachardot}{\kern0pt}{\isachardot}{\kern0pt}{\isacharless}{\kern0pt}k{\isacharbraceright}{\kern0pt}{\isachardoublequoteclose}\isanewline
\ \ \isacommand{proof}\isamarkupfalse%
{\isacharparenleft}{\kern0pt}safe{\isacharparenright}{\kern0pt}\isanewline
\ \ \ \ \isacommand{fix}\isamarkupfalse%
\ x\ f\isanewline
\ \ \ \ \isacommand{assume}\isamarkupfalse%
\ {\isachardoublequoteopen}f\ {\isasymin}\ cube\ {\isadigit{1}}\ k{\isachardoublequoteclose}\isanewline
\ \ \ \ \isacommand{then}\isamarkupfalse%
\ \isacommand{show}\isamarkupfalse%
\ {\isachardoublequoteopen}f\ {\isadigit{0}}\ {\isacharless}{\kern0pt}\ k{\isachardoublequoteclose}\ \isacommand{unfolding}\isamarkupfalse%
\ cube{\isacharunderscore}{\kern0pt}def\ \isacommand{by}\isamarkupfalse%
\ blast\isanewline
\ \ \isacommand{next}\isamarkupfalse%
\isanewline
\ \ \ \ \isacommand{fix}\isamarkupfalse%
\ x\isanewline
\ \ \ \ \isacommand{assume}\isamarkupfalse%
\ {\isachardoublequoteopen}x\ {\isacharless}{\kern0pt}\ k{\isachardoublequoteclose}\isanewline
\ \ \ \ \isacommand{then}\isamarkupfalse%
\ \isacommand{have}\isamarkupfalse%
\ {\isachardoublequoteopen}x\ {\isasymin}\ {\isacharbraceleft}{\kern0pt}{\isachardot}{\kern0pt}{\isachardot}{\kern0pt}{\isacharless}{\kern0pt}k{\isacharbraceright}{\kern0pt}{\isachardoublequoteclose}\ \isacommand{by}\isamarkupfalse%
\ simp\isanewline
\ \ \ \ \isacommand{moreover}\isamarkupfalse%
\ \isacommand{have}\isamarkupfalse%
\ {\isachardoublequoteopen}{\isadigit{0}}\ {\isasymin}\ {\isacharbraceleft}{\kern0pt}{\isachardot}{\kern0pt}{\isachardot}{\kern0pt}{\isacharless}{\kern0pt}{\isadigit{1}}{\isacharcolon}{\kern0pt}{\isacharcolon}{\kern0pt}nat{\isacharbraceright}{\kern0pt}{\isachardoublequoteclose}\ \isacommand{by}\isamarkupfalse%
\ simp\isanewline
\ \ \ \ \isacommand{ultimately}\isamarkupfalse%
\ \isacommand{have}\isamarkupfalse%
\ {\isachardoublequoteopen}{\isasymexists}y\ {\isasymin}\ {\isacharbraceleft}{\kern0pt}{\isachardot}{\kern0pt}{\isachardot}{\kern0pt}{\isacharless}{\kern0pt}{\isadigit{1}}{\isacharcolon}{\kern0pt}{\isacharcolon}{\kern0pt}nat{\isacharbraceright}{\kern0pt}\ {\isasymrightarrow}\isactrlsub E\ {\isacharbraceleft}{\kern0pt}{\isachardot}{\kern0pt}{\isachardot}{\kern0pt}{\isacharless}{\kern0pt}k{\isacharbraceright}{\kern0pt}{\isachardot}{\kern0pt}\ y\ {\isadigit{0}}\ {\isacharequal}{\kern0pt}\ x{\isachardoublequoteclose}\ \isacommand{using}\isamarkupfalse%
\ fun{\isacharunderscore}{\kern0pt}ex{\isacharbrackleft}{\kern0pt}of\ {\isachardoublequoteopen}{\isadigit{0}}{\isachardoublequoteclose}\ {\isachardoublequoteopen}{\isacharbraceleft}{\kern0pt}{\isachardot}{\kern0pt}{\isachardot}{\kern0pt}{\isacharless}{\kern0pt}{\isadigit{1}}{\isacharcolon}{\kern0pt}{\isacharcolon}{\kern0pt}nat{\isacharbraceright}{\kern0pt}{\isachardoublequoteclose}\ {\isachardoublequoteopen}x{\isachardoublequoteclose}\ {\isachardoublequoteopen}{\isacharbraceleft}{\kern0pt}{\isachardot}{\kern0pt}{\isachardot}{\kern0pt}{\isacharless}{\kern0pt}k{\isacharbraceright}{\kern0pt}{\isachardoublequoteclose}{\isacharbrackright}{\kern0pt}\ \isacommand{by}\isamarkupfalse%
\ auto\ \isanewline
\ \ \ \ \isacommand{then}\isamarkupfalse%
\ \isacommand{show}\isamarkupfalse%
\ {\isachardoublequoteopen}x\ {\isasymin}\ {\isacharparenleft}{\kern0pt}{\isasymlambda}f{\isachardot}{\kern0pt}\ f\ {\isadigit{0}}{\isacharparenright}{\kern0pt}\ {\isacharbackquote}{\kern0pt}\ cube\ {\isadigit{1}}\ k{\isachardoublequoteclose}\ \isacommand{unfolding}\isamarkupfalse%
\ cube{\isacharunderscore}{\kern0pt}def\ \isacommand{by}\isamarkupfalse%
\ blast\isanewline
\ \ \isacommand{qed}\isamarkupfalse%
\isanewline
\ \ \isacommand{moreover}\isamarkupfalse%
\ \isanewline
\ \ \isacommand{{\isacharbraceleft}{\kern0pt}}\isamarkupfalse%
\isanewline
\ \ \ \ \isacommand{have}\isamarkupfalse%
\ {\isachardoublequoteopen}card\ {\isacharparenleft}{\kern0pt}cube\ {\isadigit{1}}\ k{\isacharparenright}{\kern0pt}\ {\isacharequal}{\kern0pt}\ k{\isachardoublequoteclose}\ \isacommand{using}\isamarkupfalse%
\ cube{\isacharunderscore}{\kern0pt}card\ \isacommand{by}\isamarkupfalse%
\ {\isacharparenleft}{\kern0pt}simp\ add{\isacharcolon}{\kern0pt}\ cube{\isacharunderscore}{\kern0pt}def{\isacharparenright}{\kern0pt}\isanewline
\ \ \ \ \isacommand{moreover}\isamarkupfalse%
\ \isacommand{have}\isamarkupfalse%
\ {\isachardoublequoteopen}card\ {\isacharbraceleft}{\kern0pt}{\isachardot}{\kern0pt}{\isachardot}{\kern0pt}{\isacharless}{\kern0pt}k{\isacharbraceright}{\kern0pt}\ {\isacharequal}{\kern0pt}\ k{\isachardoublequoteclose}\ \isacommand{by}\isamarkupfalse%
\ simp\isanewline
\ \ \ \ \isacommand{ultimately}\isamarkupfalse%
\ \isacommand{have}\isamarkupfalse%
\ {\isachardoublequoteopen}inj{\isacharunderscore}{\kern0pt}on\ {\isacharparenleft}{\kern0pt}{\isasymlambda}f{\isachardot}{\kern0pt}\ f\ {\isadigit{0}}{\isacharparenright}{\kern0pt}\ {\isacharparenleft}{\kern0pt}cube\ {\isadigit{1}}\ k{\isacharparenright}{\kern0pt}{\isachardoublequoteclose}\ \isacommand{using}\isamarkupfalse%
\ {\isacharasterisk}{\kern0pt}\ eq{\isacharunderscore}{\kern0pt}card{\isacharunderscore}{\kern0pt}imp{\isacharunderscore}{\kern0pt}inj{\isacharunderscore}{\kern0pt}on{\isacharbrackleft}{\kern0pt}of\ {\isachardoublequoteopen}cube\ {\isadigit{1}}\ k{\isachardoublequoteclose}\ {\isachardoublequoteopen}{\isasymlambda}f{\isachardot}{\kern0pt}\ f\ {\isadigit{0}}{\isachardoublequoteclose}{\isacharbrackright}{\kern0pt}\ \isacommand{by}\isamarkupfalse%
\ force\isanewline
\ \ \isacommand{{\isacharbraceright}{\kern0pt}}\isamarkupfalse%
\isanewline
\ \ \isacommand{ultimately}\isamarkupfalse%
\ \isacommand{show}\isamarkupfalse%
\ {\isachardoublequoteopen}inj{\isacharunderscore}{\kern0pt}on\ {\isacharparenleft}{\kern0pt}{\isasymlambda}f{\isachardot}{\kern0pt}\ f\ {\isadigit{0}}{\isacharparenright}{\kern0pt}\ {\isacharparenleft}{\kern0pt}cube\ {\isadigit{1}}\ k{\isacharparenright}{\kern0pt}\ {\isasymand}\ {\isacharparenleft}{\kern0pt}{\isasymlambda}f{\isachardot}{\kern0pt}\ f\ {\isadigit{0}}{\isacharparenright}{\kern0pt}\ {\isacharbackquote}{\kern0pt}\ cube\ {\isadigit{1}}\ k\ {\isacharequal}{\kern0pt}\ {\isacharbraceleft}{\kern0pt}{\isachardot}{\kern0pt}{\isachardot}{\kern0pt}{\isacharless}{\kern0pt}k{\isacharbraceright}{\kern0pt}{\isachardoublequoteclose}\ \isacommand{by}\isamarkupfalse%
\ simp\isanewline
\isacommand{qed}\isamarkupfalse%
%
\endisatagproof
{\isafoldproof}%
%
\isadelimproof
%
\endisadelimproof
%
\begin{isamarkuptext}%
An alternative introduction rule for the $\exists!x$ quantifier, which means "there exists exactly one $x$".%
\end{isamarkuptext}\isamarkuptrue%
\isacommand{lemma}\isamarkupfalse%
\ ex{\isadigit{1}}I{\isacharunderscore}{\kern0pt}alt{\isacharcolon}{\kern0pt}\ {\isachardoublequoteopen}{\isacharparenleft}{\kern0pt}{\isasymexists}x{\isachardot}{\kern0pt}\ P\ x\ {\isasymand}\ {\isacharparenleft}{\kern0pt}{\isasymforall}y{\isachardot}{\kern0pt}\ P\ y\ {\isasymlongrightarrow}\ x\ {\isacharequal}{\kern0pt}\ y{\isacharparenright}{\kern0pt}{\isacharparenright}{\kern0pt}\ {\isasymLongrightarrow}\ {\isacharparenleft}{\kern0pt}{\isasymexists}{\isacharbang}{\kern0pt}x{\isachardot}{\kern0pt}\ P\ x{\isacharparenright}{\kern0pt}{\isachardoublequoteclose}\ \isanewline
%
\isadelimproof
\ \ %
\endisadelimproof
%
\isatagproof
\isacommand{by}\isamarkupfalse%
\ blast%
\endisatagproof
{\isafoldproof}%
%
\isadelimproof
\isanewline
%
\endisadelimproof
\isacommand{lemma}\isamarkupfalse%
\ nat{\isacharunderscore}{\kern0pt}set{\isacharunderscore}{\kern0pt}eq{\isacharunderscore}{\kern0pt}one{\isacharunderscore}{\kern0pt}dim{\isacharunderscore}{\kern0pt}cube{\isacharcolon}{\kern0pt}\ {\isachardoublequoteopen}bij{\isacharunderscore}{\kern0pt}betw\ {\isacharparenleft}{\kern0pt}{\isasymlambda}x{\isachardot}{\kern0pt}\ {\isasymlambda}y{\isasymin}{\isacharbraceleft}{\kern0pt}{\isachardot}{\kern0pt}{\isachardot}{\kern0pt}{\isacharless}{\kern0pt}{\isadigit{1}}{\isacharcolon}{\kern0pt}{\isacharcolon}{\kern0pt}nat{\isacharbraceright}{\kern0pt}{\isachardot}{\kern0pt}\ x{\isacharparenright}{\kern0pt}\ {\isacharbraceleft}{\kern0pt}{\isachardot}{\kern0pt}{\isachardot}{\kern0pt}{\isacharless}{\kern0pt}k{\isacharcolon}{\kern0pt}{\isacharcolon}{\kern0pt}nat{\isacharbraceright}{\kern0pt}\ {\isacharparenleft}{\kern0pt}cube\ {\isadigit{1}}\ k{\isacharparenright}{\kern0pt}{\isachardoublequoteclose}\isanewline
%
\isadelimproof
%
\endisadelimproof
%
\isatagproof
\isacommand{proof}\isamarkupfalse%
\ {\isacharparenleft}{\kern0pt}unfold\ bij{\isacharunderscore}{\kern0pt}betw{\isacharunderscore}{\kern0pt}def{\isacharparenright}{\kern0pt}\isanewline
\ \ \isacommand{have}\isamarkupfalse%
\ {\isacharasterisk}{\kern0pt}{\isacharcolon}{\kern0pt}\ {\isachardoublequoteopen}{\isacharparenleft}{\kern0pt}{\isasymlambda}x{\isachardot}{\kern0pt}\ {\isasymlambda}y{\isasymin}{\isacharbraceleft}{\kern0pt}{\isachardot}{\kern0pt}{\isachardot}{\kern0pt}{\isacharless}{\kern0pt}{\isadigit{1}}{\isacharcolon}{\kern0pt}{\isacharcolon}{\kern0pt}nat{\isacharbraceright}{\kern0pt}{\isachardot}{\kern0pt}\ x{\isacharparenright}{\kern0pt}\ {\isacharbackquote}{\kern0pt}\ {\isacharbraceleft}{\kern0pt}{\isachardot}{\kern0pt}{\isachardot}{\kern0pt}{\isacharless}{\kern0pt}k{\isacharbraceright}{\kern0pt}\ {\isacharequal}{\kern0pt}\ cube\ {\isadigit{1}}\ k{\isachardoublequoteclose}\isanewline
\ \ \isacommand{proof}\isamarkupfalse%
\ {\isacharparenleft}{\kern0pt}safe{\isacharparenright}{\kern0pt}\isanewline
\ \ \ \ \isacommand{fix}\isamarkupfalse%
\ x\ y\isanewline
\ \ \ \ \isacommand{assume}\isamarkupfalse%
\ {\isachardoublequoteopen}y\ {\isacharless}{\kern0pt}\ k{\isachardoublequoteclose}\isanewline
\ \ \ \ \isacommand{then}\isamarkupfalse%
\ \isacommand{show}\isamarkupfalse%
\ {\isachardoublequoteopen}{\isacharparenleft}{\kern0pt}{\isasymlambda}z{\isasymin}{\isacharbraceleft}{\kern0pt}{\isachardot}{\kern0pt}{\isachardot}{\kern0pt}{\isacharless}{\kern0pt}{\isadigit{1}}{\isacharbraceright}{\kern0pt}{\isachardot}{\kern0pt}\ y{\isacharparenright}{\kern0pt}\ {\isasymin}\ cube\ {\isadigit{1}}\ k{\isachardoublequoteclose}\ \isacommand{unfolding}\isamarkupfalse%
\ cube{\isacharunderscore}{\kern0pt}def\ \isacommand{by}\isamarkupfalse%
\ simp\isanewline
\ \ \isacommand{next}\isamarkupfalse%
\isanewline
\ \ \ \ \isacommand{fix}\isamarkupfalse%
\ x\isanewline
\ \ \ \ \isacommand{assume}\isamarkupfalse%
\ {\isachardoublequoteopen}x\ {\isasymin}\ cube\ {\isadigit{1}}\ k{\isachardoublequoteclose}\isanewline
\ \ \ \ \isacommand{have}\isamarkupfalse%
\ {\isachardoublequoteopen}x\ {\isacharequal}{\kern0pt}\ {\isacharparenleft}{\kern0pt}{\isasymlambda}z{\isachardot}{\kern0pt}\ {\isasymlambda}y{\isasymin}{\isacharbraceleft}{\kern0pt}{\isachardot}{\kern0pt}{\isachardot}{\kern0pt}{\isacharless}{\kern0pt}{\isadigit{1}}{\isacharcolon}{\kern0pt}{\isacharcolon}{\kern0pt}nat{\isacharbraceright}{\kern0pt}{\isachardot}{\kern0pt}\ z{\isacharparenright}{\kern0pt}\ {\isacharparenleft}{\kern0pt}x\ {\isadigit{0}}{\isacharcolon}{\kern0pt}{\isacharcolon}{\kern0pt}nat{\isacharparenright}{\kern0pt}{\isachardoublequoteclose}\ \isanewline
\ \ \ \ \isacommand{proof}\isamarkupfalse%
\isanewline
\ \ \ \ \ \ \isacommand{fix}\isamarkupfalse%
\ j\ \isanewline
\ \ \ \ \ \ \isacommand{consider}\isamarkupfalse%
\ {\isachardoublequoteopen}j\ {\isasymin}\ {\isacharbraceleft}{\kern0pt}{\isachardot}{\kern0pt}{\isachardot}{\kern0pt}{\isacharless}{\kern0pt}{\isadigit{1}}{\isacharbraceright}{\kern0pt}{\isachardoublequoteclose}\ {\isacharbar}{\kern0pt}\ {\isachardoublequoteopen}j\ {\isasymnotin}\ {\isacharbraceleft}{\kern0pt}{\isachardot}{\kern0pt}{\isachardot}{\kern0pt}{\isacharless}{\kern0pt}{\isadigit{1}}{\isacharcolon}{\kern0pt}{\isacharcolon}{\kern0pt}nat{\isacharbraceright}{\kern0pt}{\isachardoublequoteclose}\ \isacommand{by}\isamarkupfalse%
\ linarith\isanewline
\ \ \ \ \ \ \isacommand{then}\isamarkupfalse%
\ \isacommand{show}\isamarkupfalse%
\ {\isachardoublequoteopen}x\ j\ {\isacharequal}{\kern0pt}\ {\isacharparenleft}{\kern0pt}{\isasymlambda}z{\isachardot}{\kern0pt}\ {\isasymlambda}y{\isasymin}{\isacharbraceleft}{\kern0pt}{\isachardot}{\kern0pt}{\isachardot}{\kern0pt}{\isacharless}{\kern0pt}{\isadigit{1}}{\isacharcolon}{\kern0pt}{\isacharcolon}{\kern0pt}nat{\isacharbraceright}{\kern0pt}{\isachardot}{\kern0pt}\ z{\isacharparenright}{\kern0pt}\ {\isacharparenleft}{\kern0pt}x\ {\isadigit{0}}{\isacharcolon}{\kern0pt}{\isacharcolon}{\kern0pt}nat{\isacharparenright}{\kern0pt}\ j{\isachardoublequoteclose}\ \isacommand{using}\isamarkupfalse%
\ {\isacartoucheopen}x\ {\isasymin}\ cube\ {\isadigit{1}}\ k{\isacartoucheclose}\ \isacommand{unfolding}\isamarkupfalse%
\ cube{\isacharunderscore}{\kern0pt}def\ \isacommand{by}\isamarkupfalse%
\ auto\isanewline
\ \ \ \ \isacommand{qed}\isamarkupfalse%
\isanewline
\ \ \ \ \isacommand{moreover}\isamarkupfalse%
\ \isacommand{have}\isamarkupfalse%
\ {\isachardoublequoteopen}x\ {\isadigit{0}}\ {\isasymin}\ {\isacharbraceleft}{\kern0pt}{\isachardot}{\kern0pt}{\isachardot}{\kern0pt}{\isacharless}{\kern0pt}k{\isacharbraceright}{\kern0pt}{\isachardoublequoteclose}\ \isacommand{using}\isamarkupfalse%
\ {\isacartoucheopen}x\ {\isasymin}\ cube\ {\isadigit{1}}\ k{\isacartoucheclose}\ \isacommand{by}\isamarkupfalse%
\ {\isacharparenleft}{\kern0pt}auto\ simp\ add{\isacharcolon}{\kern0pt}\ cube{\isacharunderscore}{\kern0pt}def{\isacharparenright}{\kern0pt}\isanewline
\ \ \ \ \isacommand{ultimately}\isamarkupfalse%
\ \isacommand{show}\isamarkupfalse%
\ {\isachardoublequoteopen}x\ {\isasymin}\ {\isacharparenleft}{\kern0pt}{\isasymlambda}z{\isachardot}{\kern0pt}\ {\isasymlambda}y{\isasymin}{\isacharbraceleft}{\kern0pt}{\isachardot}{\kern0pt}{\isachardot}{\kern0pt}{\isacharless}{\kern0pt}{\isadigit{1}}{\isacharbraceright}{\kern0pt}{\isachardot}{\kern0pt}\ z{\isacharparenright}{\kern0pt}\ {\isacharbackquote}{\kern0pt}\ {\isacharbraceleft}{\kern0pt}{\isachardot}{\kern0pt}{\isachardot}{\kern0pt}{\isacharless}{\kern0pt}k{\isacharbraceright}{\kern0pt}{\isachardoublequoteclose}\ \ \isacommand{by}\isamarkupfalse%
\ blast\isanewline
\ \ \isacommand{qed}\isamarkupfalse%
\isanewline
\ \ \isacommand{moreover}\isamarkupfalse%
\isanewline
\ \ \isacommand{{\isacharbraceleft}{\kern0pt}}\isamarkupfalse%
\isanewline
\ \ \ \ \isacommand{have}\isamarkupfalse%
\ {\isachardoublequoteopen}card\ {\isacharparenleft}{\kern0pt}cube\ {\isadigit{1}}\ k{\isacharparenright}{\kern0pt}\ {\isacharequal}{\kern0pt}\ k{\isachardoublequoteclose}\ \isacommand{using}\isamarkupfalse%
\ cube{\isacharunderscore}{\kern0pt}card\ \isacommand{by}\isamarkupfalse%
\ {\isacharparenleft}{\kern0pt}simp\ add{\isacharcolon}{\kern0pt}\ cube{\isacharunderscore}{\kern0pt}def{\isacharparenright}{\kern0pt}\isanewline
\ \ \ \ \isacommand{moreover}\isamarkupfalse%
\ \isacommand{have}\isamarkupfalse%
\ {\isachardoublequoteopen}card\ {\isacharbraceleft}{\kern0pt}{\isachardot}{\kern0pt}{\isachardot}{\kern0pt}{\isacharless}{\kern0pt}k{\isacharbraceright}{\kern0pt}\ {\isacharequal}{\kern0pt}\ k{\isachardoublequoteclose}\ \isacommand{by}\isamarkupfalse%
\ simp\isanewline
\ \ \ \ \isacommand{ultimately}\isamarkupfalse%
\ \isacommand{have}\isamarkupfalse%
\ \ {\isachardoublequoteopen}inj{\isacharunderscore}{\kern0pt}on\ {\isacharparenleft}{\kern0pt}{\isasymlambda}x{\isachardot}{\kern0pt}\ {\isasymlambda}y{\isasymin}{\isacharbraceleft}{\kern0pt}{\isachardot}{\kern0pt}{\isachardot}{\kern0pt}{\isacharless}{\kern0pt}{\isadigit{1}}{\isacharcolon}{\kern0pt}{\isacharcolon}{\kern0pt}nat{\isacharbraceright}{\kern0pt}{\isachardot}{\kern0pt}\ x{\isacharparenright}{\kern0pt}\ {\isacharbraceleft}{\kern0pt}{\isachardot}{\kern0pt}{\isachardot}{\kern0pt}{\isacharless}{\kern0pt}k{\isacharbraceright}{\kern0pt}{\isachardoublequoteclose}\ \isacommand{using}\isamarkupfalse%
\ {\isacharasterisk}{\kern0pt}\ eq{\isacharunderscore}{\kern0pt}card{\isacharunderscore}{\kern0pt}imp{\isacharunderscore}{\kern0pt}inj{\isacharunderscore}{\kern0pt}on{\isacharbrackleft}{\kern0pt}of\ {\isachardoublequoteopen}{\isacharbraceleft}{\kern0pt}{\isachardot}{\kern0pt}{\isachardot}{\kern0pt}{\isacharless}{\kern0pt}k{\isacharbraceright}{\kern0pt}{\isachardoublequoteclose}\ {\isachardoublequoteopen}{\isasymlambda}x{\isachardot}{\kern0pt}\ {\isasymlambda}y{\isasymin}{\isacharbraceleft}{\kern0pt}{\isachardot}{\kern0pt}{\isachardot}{\kern0pt}{\isacharless}{\kern0pt}{\isadigit{1}}{\isacharcolon}{\kern0pt}{\isacharcolon}{\kern0pt}nat{\isacharbraceright}{\kern0pt}{\isachardot}{\kern0pt}\ x{\isachardoublequoteclose}{\isacharbrackright}{\kern0pt}\ \isacommand{by}\isamarkupfalse%
\ force\isanewline
\ \ \isacommand{{\isacharbraceright}{\kern0pt}}\isamarkupfalse%
\isanewline
\ \ \isacommand{ultimately}\isamarkupfalse%
\ \isacommand{show}\isamarkupfalse%
\ {\isachardoublequoteopen}inj{\isacharunderscore}{\kern0pt}on\ {\isacharparenleft}{\kern0pt}{\isasymlambda}x{\isachardot}{\kern0pt}\ {\isasymlambda}y{\isasymin}{\isacharbraceleft}{\kern0pt}{\isachardot}{\kern0pt}{\isachardot}{\kern0pt}{\isacharless}{\kern0pt}{\isadigit{1}}{\isacharcolon}{\kern0pt}{\isacharcolon}{\kern0pt}nat{\isacharbraceright}{\kern0pt}{\isachardot}{\kern0pt}\ x{\isacharparenright}{\kern0pt}\ {\isacharbraceleft}{\kern0pt}{\isachardot}{\kern0pt}{\isachardot}{\kern0pt}{\isacharless}{\kern0pt}k{\isacharbraceright}{\kern0pt}\ {\isasymand}\ {\isacharparenleft}{\kern0pt}{\isasymlambda}x{\isachardot}{\kern0pt}\ {\isasymlambda}y{\isasymin}{\isacharbraceleft}{\kern0pt}{\isachardot}{\kern0pt}{\isachardot}{\kern0pt}{\isacharless}{\kern0pt}{\isadigit{1}}{\isacharcolon}{\kern0pt}{\isacharcolon}{\kern0pt}nat{\isacharbraceright}{\kern0pt}{\isachardot}{\kern0pt}\ x{\isacharparenright}{\kern0pt}\ {\isacharbackquote}{\kern0pt}\ {\isacharbraceleft}{\kern0pt}{\isachardot}{\kern0pt}{\isachardot}{\kern0pt}{\isacharless}{\kern0pt}k{\isacharbraceright}{\kern0pt}\ {\isacharequal}{\kern0pt}\ cube\ {\isadigit{1}}\ k{\isachardoublequoteclose}\ \isacommand{by}\isamarkupfalse%
\ blast\isanewline
\isacommand{qed}\isamarkupfalse%
%
\endisatagproof
{\isafoldproof}%
%
\isadelimproof
%
\endisadelimproof
%
\begin{isamarkuptext}%
A bijection $f$ between domains $A_1$ and $A_2$ creates a correspondence between functions in $A_1 \rightarrow B$ and $A_2 \rightarrow B$.%
\end{isamarkuptext}\isamarkuptrue%
\isacommand{lemma}\isamarkupfalse%
\ bij{\isacharunderscore}{\kern0pt}domain{\isacharunderscore}{\kern0pt}PiE{\isacharcolon}{\kern0pt}\isanewline
\ \ \isakeyword{assumes}\ {\isachardoublequoteopen}bij{\isacharunderscore}{\kern0pt}betw\ f\ A{\isadigit{1}}\ A{\isadigit{2}}{\isachardoublequoteclose}\ \isanewline
\ \ \ \ \isakeyword{and}\ {\isachardoublequoteopen}g\ {\isasymin}\ A{\isadigit{2}}\ {\isasymrightarrow}\isactrlsub E\ B{\isachardoublequoteclose}\isanewline
\ \ \isakeyword{shows}\ {\isachardoublequoteopen}{\isacharparenleft}{\kern0pt}restrict\ {\isacharparenleft}{\kern0pt}g\ {\isasymcirc}\ f{\isacharparenright}{\kern0pt}\ A{\isadigit{1}}{\isacharparenright}{\kern0pt}\ {\isasymin}\ A{\isadigit{1}}\ {\isasymrightarrow}\isactrlsub E\ B{\isachardoublequoteclose}\isanewline
%
\isadelimproof
\ \ %
\endisadelimproof
%
\isatagproof
\isacommand{using}\isamarkupfalse%
\ bij{\isacharunderscore}{\kern0pt}betwE\ assms\ \isacommand{by}\isamarkupfalse%
\ fastforce%
\endisatagproof
{\isafoldproof}%
%
\isadelimproof
%
\endisadelimproof
%
\begin{isamarkuptext}%
The following two lemmas relate lines to $1$-dimensional subspaces (in the natural way). This is (almost) a direct consequence of the elimination rule \isa{is{\isacharunderscore}{\kern0pt}line{\isacharunderscore}{\kern0pt}elim} introduced above.%
\end{isamarkuptext}\isamarkuptrue%
\isacommand{lemma}\isamarkupfalse%
\ line{\isacharunderscore}{\kern0pt}is{\isacharunderscore}{\kern0pt}dim{\isadigit{1}}{\isacharunderscore}{\kern0pt}subspace{\isacharunderscore}{\kern0pt}t{\isacharunderscore}{\kern0pt}{\isadigit{1}}{\isacharcolon}{\kern0pt}\ \isakeyword{assumes}\ {\isachardoublequoteopen}n\ {\isachargreater}{\kern0pt}\ {\isadigit{0}}{\isachardoublequoteclose}\ \isakeyword{and}\ {\isachardoublequoteopen}is{\isacharunderscore}{\kern0pt}line\ L\ n\ {\isadigit{1}}{\isachardoublequoteclose}\ \isakeyword{shows}\ {\isachardoublequoteopen}is{\isacharunderscore}{\kern0pt}subspace\ {\isacharparenleft}{\kern0pt}restrict\ {\isacharparenleft}{\kern0pt}{\isasymlambda}y{\isachardot}{\kern0pt}\ L\ {\isacharparenleft}{\kern0pt}y\ {\isadigit{0}}{\isacharparenright}{\kern0pt}{\isacharparenright}{\kern0pt}\ {\isacharparenleft}{\kern0pt}cube\ {\isadigit{1}}\ {\isadigit{1}}{\isacharparenright}{\kern0pt}{\isacharparenright}{\kern0pt}\ {\isadigit{1}}\ n\ {\isadigit{1}}{\isachardoublequoteclose}\isanewline
%
\isadelimproof
%
\endisadelimproof
%
\isatagproof
\isacommand{proof}\isamarkupfalse%
\ {\isacharminus}{\kern0pt}\isanewline
\ \ \isacommand{obtain}\isamarkupfalse%
\ B\isactrlsub {\isadigit{0}}\ B\isactrlsub {\isadigit{1}}\ \isakeyword{where}\ B{\isacharunderscore}{\kern0pt}props{\isacharcolon}{\kern0pt}\ {\isachardoublequoteopen}B\isactrlsub {\isadigit{0}}\ {\isasymunion}\ B\isactrlsub {\isadigit{1}}\ {\isacharequal}{\kern0pt}\ {\isacharbraceleft}{\kern0pt}{\isachardot}{\kern0pt}{\isachardot}{\kern0pt}{\isacharless}{\kern0pt}n{\isacharbraceright}{\kern0pt}\ {\isasymand}\ B\isactrlsub {\isadigit{0}}\ {\isasyminter}\ B\isactrlsub {\isadigit{1}}\ {\isacharequal}{\kern0pt}\ {\isacharbraceleft}{\kern0pt}{\isacharbraceright}{\kern0pt}\ {\isasymand}\ B\isactrlsub {\isadigit{0}}\ {\isasymnoteq}\ {\isacharbraceleft}{\kern0pt}{\isacharbraceright}{\kern0pt}\ {\isasymand}\ {\isacharparenleft}{\kern0pt}{\isasymforall}j\ {\isasymin}\ B\isactrlsub {\isadigit{1}}{\isachardot}{\kern0pt}\ {\isacharparenleft}{\kern0pt}{\isasymforall}x{\isacharless}{\kern0pt}{\isadigit{1}}{\isachardot}{\kern0pt}\ {\isasymforall}y{\isacharless}{\kern0pt}{\isadigit{1}}{\isachardot}{\kern0pt}\ L\ x\ j\ {\isacharequal}{\kern0pt}\ L\ y\ j{\isacharparenright}{\kern0pt}{\isacharparenright}{\kern0pt}\ {\isasymand}\ {\isacharparenleft}{\kern0pt}{\isasymforall}j\ {\isasymin}\ B\isactrlsub {\isadigit{0}}{\isachardot}{\kern0pt}\ {\isacharparenleft}{\kern0pt}{\isasymforall}s{\isacharless}{\kern0pt}{\isadigit{1}}{\isachardot}{\kern0pt}\ L\ s\ j\ {\isacharequal}{\kern0pt}\ s{\isacharparenright}{\kern0pt}{\isacharparenright}{\kern0pt}{\isachardoublequoteclose}\ \isacommand{using}\isamarkupfalse%
\ is{\isacharunderscore}{\kern0pt}line{\isacharunderscore}{\kern0pt}elim{\isacharunderscore}{\kern0pt}t{\isacharunderscore}{\kern0pt}{\isadigit{1}}{\isacharbrackleft}{\kern0pt}of\ L\ n\ {\isadigit{1}}{\isacharbrackright}{\kern0pt}\ assms\ \isacommand{by}\isamarkupfalse%
\ auto\isanewline
\isanewline
\ \ \isacommand{define}\isamarkupfalse%
\ B\ \isakeyword{where}\ {\isachardoublequoteopen}B\ {\isasymequiv}\ {\isacharparenleft}{\kern0pt}{\isasymlambda}i{\isacharcolon}{\kern0pt}{\isacharcolon}{\kern0pt}nat{\isachardot}{\kern0pt}\ {\isacharbraceleft}{\kern0pt}{\isacharbraceright}{\kern0pt}{\isacharcolon}{\kern0pt}{\isacharcolon}{\kern0pt}nat\ set{\isacharparenright}{\kern0pt}{\isacharparenleft}{\kern0pt}{\isadigit{0}}{\isacharcolon}{\kern0pt}{\isacharequal}{\kern0pt}B\isactrlsub {\isadigit{0}}{\isacharcomma}{\kern0pt}\ {\isadigit{1}}{\isacharcolon}{\kern0pt}{\isacharequal}{\kern0pt}B\isactrlsub {\isadigit{1}}{\isacharparenright}{\kern0pt}{\isachardoublequoteclose}\ \isanewline
\ \ \isacommand{define}\isamarkupfalse%
\ f\ \isakeyword{where}\ {\isachardoublequoteopen}f\ {\isasymequiv}\ {\isacharparenleft}{\kern0pt}{\isasymlambda}i\ {\isasymin}\ B\ {\isadigit{1}}{\isachardot}{\kern0pt}\ L\ {\isadigit{0}}\ i{\isacharparenright}{\kern0pt}{\isachardoublequoteclose}\isanewline
\ \ \isacommand{have}\isamarkupfalse%
\ {\isacharasterisk}{\kern0pt}{\isacharcolon}{\kern0pt}\ {\isachardoublequoteopen}L\ {\isadigit{0}}\ {\isasymin}\ {\isacharbraceleft}{\kern0pt}{\isachardot}{\kern0pt}{\isachardot}{\kern0pt}{\isacharless}{\kern0pt}n{\isacharbraceright}{\kern0pt}\ {\isasymrightarrow}\isactrlsub E\ {\isacharbraceleft}{\kern0pt}{\isachardot}{\kern0pt}{\isachardot}{\kern0pt}{\isacharless}{\kern0pt}{\isadigit{1}}{\isacharbraceright}{\kern0pt}{\isachardoublequoteclose}\ \isacommand{using}\isamarkupfalse%
\ assms{\isacharparenleft}{\kern0pt}{\isadigit{2}}{\isacharparenright}{\kern0pt}\ \isacommand{unfolding}\isamarkupfalse%
\ cube{\isacharunderscore}{\kern0pt}def\ is{\isacharunderscore}{\kern0pt}line{\isacharunderscore}{\kern0pt}def\ \isacommand{by}\isamarkupfalse%
\ auto\isanewline
\ \ \isacommand{have}\isamarkupfalse%
\ {\isachardoublequoteopen}disjoint{\isacharunderscore}{\kern0pt}family{\isacharunderscore}{\kern0pt}on\ B\ {\isacharbraceleft}{\kern0pt}{\isachardot}{\kern0pt}{\isachardot}{\kern0pt}{\isadigit{1}}{\isacharbraceright}{\kern0pt}{\isachardoublequoteclose}\ \isacommand{unfolding}\isamarkupfalse%
\ B{\isacharunderscore}{\kern0pt}def\ \isacommand{using}\isamarkupfalse%
\ B{\isacharunderscore}{\kern0pt}props\ \isanewline
\ \ \ \ \isacommand{by}\isamarkupfalse%
\ {\isacharparenleft}{\kern0pt}simp\ add{\isacharcolon}{\kern0pt}\ Int{\isacharunderscore}{\kern0pt}commute\ disjoint{\isacharunderscore}{\kern0pt}family{\isacharunderscore}{\kern0pt}onI{\isacharparenright}{\kern0pt}\isanewline
\ \ \isacommand{moreover}\isamarkupfalse%
\ \isacommand{have}\isamarkupfalse%
\ {\isachardoublequoteopen}{\isasymUnion}\ {\isacharparenleft}{\kern0pt}B\ {\isacharbackquote}{\kern0pt}\ {\isacharbraceleft}{\kern0pt}{\isachardot}{\kern0pt}{\isachardot}{\kern0pt}{\isadigit{1}}{\isacharbraceright}{\kern0pt}{\isacharparenright}{\kern0pt}\ {\isacharequal}{\kern0pt}\ {\isacharbraceleft}{\kern0pt}{\isachardot}{\kern0pt}{\isachardot}{\kern0pt}{\isacharless}{\kern0pt}n{\isacharbraceright}{\kern0pt}{\isachardoublequoteclose}\ \isacommand{unfolding}\isamarkupfalse%
\ B{\isacharunderscore}{\kern0pt}def\ \isacommand{using}\isamarkupfalse%
\ B{\isacharunderscore}{\kern0pt}props\ \isacommand{by}\isamarkupfalse%
\ auto\isanewline
\ \ \isacommand{moreover}\isamarkupfalse%
\ \isacommand{have}\isamarkupfalse%
\ {\isachardoublequoteopen}{\isacharbraceleft}{\kern0pt}{\isacharbraceright}{\kern0pt}\ {\isasymnotin}\ B\ {\isacharbackquote}{\kern0pt}\ {\isacharbraceleft}{\kern0pt}{\isachardot}{\kern0pt}{\isachardot}{\kern0pt}{\isacharless}{\kern0pt}{\isadigit{1}}{\isacharbraceright}{\kern0pt}{\isachardoublequoteclose}\ \isacommand{unfolding}\isamarkupfalse%
\ B{\isacharunderscore}{\kern0pt}def\ \isacommand{using}\isamarkupfalse%
\ B{\isacharunderscore}{\kern0pt}props\ \isacommand{by}\isamarkupfalse%
\ auto\isanewline
\ \ \isacommand{moreover}\isamarkupfalse%
\ \isacommand{have}\isamarkupfalse%
\ {\isachardoublequoteopen}\ f\ {\isasymin}\ B\ {\isadigit{1}}\ {\isasymrightarrow}\isactrlsub E\ {\isacharbraceleft}{\kern0pt}{\isachardot}{\kern0pt}{\isachardot}{\kern0pt}{\isacharless}{\kern0pt}{\isadigit{1}}{\isacharbraceright}{\kern0pt}{\isachardoublequoteclose}\ \isacommand{using}\isamarkupfalse%
\ {\isacharasterisk}{\kern0pt}\ calculation{\isacharparenleft}{\kern0pt}{\isadigit{2}}{\isacharparenright}{\kern0pt}\ \isacommand{unfolding}\isamarkupfalse%
\ f{\isacharunderscore}{\kern0pt}def\ \isacommand{by}\isamarkupfalse%
\ auto\isanewline
\ \ \isacommand{moreover}\isamarkupfalse%
\ \isacommand{have}\isamarkupfalse%
\ {\isachardoublequoteopen}{\isacharparenleft}{\kern0pt}restrict\ {\isacharparenleft}{\kern0pt}{\isasymlambda}y{\isachardot}{\kern0pt}\ L\ {\isacharparenleft}{\kern0pt}y\ {\isadigit{0}}{\isacharparenright}{\kern0pt}{\isacharparenright}{\kern0pt}\ {\isacharparenleft}{\kern0pt}cube\ {\isadigit{1}}\ {\isadigit{1}}{\isacharparenright}{\kern0pt}{\isacharparenright}{\kern0pt}\ {\isasymin}\ cube\ {\isadigit{1}}\ {\isadigit{1}}\ {\isasymrightarrow}\isactrlsub E\ cube\ n\ {\isadigit{1}}{\isachardoublequoteclose}\ \isacommand{using}\isamarkupfalse%
\ assms{\isacharparenleft}{\kern0pt}{\isadigit{2}}{\isacharparenright}{\kern0pt}\ cube{\isadigit{1}}{\isacharunderscore}{\kern0pt}alt{\isacharunderscore}{\kern0pt}def\ \isacommand{unfolding}\isamarkupfalse%
\ is{\isacharunderscore}{\kern0pt}line{\isacharunderscore}{\kern0pt}def\ \isacommand{by}\isamarkupfalse%
\ auto\isanewline
\ \ \isacommand{moreover}\isamarkupfalse%
\ \isacommand{have}\isamarkupfalse%
\ {\isachardoublequoteopen}{\isacharparenleft}{\kern0pt}{\isasymforall}y{\isasymin}cube\ {\isadigit{1}}\ {\isadigit{1}}{\isachardot}{\kern0pt}\ {\isacharparenleft}{\kern0pt}{\isasymforall}i{\isasymin}B\ {\isadigit{1}}{\isachardot}{\kern0pt}\ {\isacharparenleft}{\kern0pt}restrict\ {\isacharparenleft}{\kern0pt}{\isasymlambda}y{\isachardot}{\kern0pt}\ L\ {\isacharparenleft}{\kern0pt}y\ {\isadigit{0}}{\isacharparenright}{\kern0pt}{\isacharparenright}{\kern0pt}\ {\isacharparenleft}{\kern0pt}cube\ {\isadigit{1}}\ {\isadigit{1}}{\isacharparenright}{\kern0pt}{\isacharparenright}{\kern0pt}\ y\ i\ {\isacharequal}{\kern0pt}\ f\ i{\isacharparenright}{\kern0pt}\ {\isasymand}\ {\isacharparenleft}{\kern0pt}{\isasymforall}j{\isacharless}{\kern0pt}{\isadigit{1}}{\isachardot}{\kern0pt}\ {\isasymforall}i{\isasymin}B\ j{\isachardot}{\kern0pt}\ {\isacharparenleft}{\kern0pt}restrict\ {\isacharparenleft}{\kern0pt}{\isasymlambda}y{\isachardot}{\kern0pt}\ L\ {\isacharparenleft}{\kern0pt}y\ {\isadigit{0}}{\isacharparenright}{\kern0pt}{\isacharparenright}{\kern0pt}\ {\isacharparenleft}{\kern0pt}cube\ {\isadigit{1}}\ {\isadigit{1}}{\isacharparenright}{\kern0pt}{\isacharparenright}{\kern0pt}\ y\ i\ {\isacharequal}{\kern0pt}\ y\ j{\isacharparenright}{\kern0pt}{\isacharparenright}{\kern0pt}{\isachardoublequoteclose}\ \isacommand{using}\isamarkupfalse%
\ cube{\isadigit{1}}{\isacharunderscore}{\kern0pt}alt{\isacharunderscore}{\kern0pt}def\ B{\isacharunderscore}{\kern0pt}props\ {\isacharasterisk}{\kern0pt}\ \isacommand{unfolding}\isamarkupfalse%
\ B{\isacharunderscore}{\kern0pt}def\ f{\isacharunderscore}{\kern0pt}def\ \isacommand{by}\isamarkupfalse%
\ auto\isanewline
\ \ \isacommand{ultimately}\isamarkupfalse%
\ \isacommand{show}\isamarkupfalse%
\ {\isacharquery}{\kern0pt}thesis\ \isacommand{unfolding}\isamarkupfalse%
\ is{\isacharunderscore}{\kern0pt}subspace{\isacharunderscore}{\kern0pt}def\ \isacommand{by}\isamarkupfalse%
\ blast\ \isanewline
\isacommand{qed}\isamarkupfalse%
%
\endisatagproof
{\isafoldproof}%
%
\isadelimproof
\isanewline
%
\endisadelimproof
\isanewline
\isacommand{lemma}\isamarkupfalse%
\ line{\isacharunderscore}{\kern0pt}is{\isacharunderscore}{\kern0pt}dim{\isadigit{1}}{\isacharunderscore}{\kern0pt}subspace{\isacharunderscore}{\kern0pt}t{\isacharunderscore}{\kern0pt}ge{\isacharunderscore}{\kern0pt}{\isadigit{1}}{\isacharcolon}{\kern0pt}\ {\isachardoublequoteopen}n\ {\isachargreater}{\kern0pt}\ {\isadigit{0}}\ {\isasymLongrightarrow}\ t\ {\isachargreater}{\kern0pt}\ {\isadigit{1}}\ {\isasymLongrightarrow}\ is{\isacharunderscore}{\kern0pt}line\ L\ n\ t\ {\isasymLongrightarrow}\ is{\isacharunderscore}{\kern0pt}subspace\ {\isacharparenleft}{\kern0pt}restrict\ {\isacharparenleft}{\kern0pt}{\isasymlambda}y{\isachardot}{\kern0pt}\ L\ {\isacharparenleft}{\kern0pt}y\ {\isadigit{0}}{\isacharparenright}{\kern0pt}{\isacharparenright}{\kern0pt}\ {\isacharparenleft}{\kern0pt}cube\ {\isadigit{1}}\ t{\isacharparenright}{\kern0pt}{\isacharparenright}{\kern0pt}\ {\isadigit{1}}\ n\ t{\isachardoublequoteclose}\isanewline
%
\isadelimproof
%
\endisadelimproof
%
\isatagproof
\isacommand{proof}\isamarkupfalse%
\ {\isacharminus}{\kern0pt}\isanewline
\ \ \isacommand{assume}\isamarkupfalse%
\ assms{\isacharcolon}{\kern0pt}\ {\isachardoublequoteopen}n\ {\isachargreater}{\kern0pt}\ {\isadigit{0}}{\isachardoublequoteclose}\ {\isachardoublequoteopen}{\isadigit{1}}\ {\isacharless}{\kern0pt}\ t{\isachardoublequoteclose}\ {\isachardoublequoteopen}is{\isacharunderscore}{\kern0pt}line\ L\ n\ t{\isachardoublequoteclose}\isanewline
\ \ \isacommand{let}\isamarkupfalse%
\ {\isacharquery}{\kern0pt}B{\isadigit{1}}\ {\isacharequal}{\kern0pt}\ {\isachardoublequoteopen}{\isacharbraceleft}{\kern0pt}i{\isacharcolon}{\kern0pt}{\isacharcolon}{\kern0pt}nat\ {\isachardot}{\kern0pt}\ i\ {\isacharless}{\kern0pt}\ n\ {\isasymand}\ {\isacharparenleft}{\kern0pt}{\isasymforall}x{\isacharless}{\kern0pt}t{\isachardot}{\kern0pt}\ {\isasymforall}y{\isacharless}{\kern0pt}t{\isachardot}{\kern0pt}\ L\ x\ i\ {\isacharequal}{\kern0pt}\ \ L\ y\ i{\isacharparenright}{\kern0pt}{\isacharbraceright}{\kern0pt}{\isachardoublequoteclose}\isanewline
\ \ \isacommand{let}\isamarkupfalse%
\ {\isacharquery}{\kern0pt}B{\isadigit{0}}\ {\isacharequal}{\kern0pt}\ {\isachardoublequoteopen}{\isacharbraceleft}{\kern0pt}i{\isacharcolon}{\kern0pt}{\isacharcolon}{\kern0pt}nat\ {\isachardot}{\kern0pt}\ i\ {\isacharless}{\kern0pt}\ n\ {\isasymand}\ {\isacharparenleft}{\kern0pt}{\isasymforall}s\ {\isacharless}{\kern0pt}\ t{\isachardot}{\kern0pt}\ L\ s\ i\ {\isacharequal}{\kern0pt}\ s{\isacharparenright}{\kern0pt}{\isacharbraceright}{\kern0pt}{\isachardoublequoteclose}\isanewline
\ \ \isacommand{define}\isamarkupfalse%
\ B\ \isakeyword{where}\ {\isachardoublequoteopen}B\ {\isasymequiv}\ {\isacharparenleft}{\kern0pt}{\isasymlambda}i{\isacharcolon}{\kern0pt}{\isacharcolon}{\kern0pt}nat{\isachardot}{\kern0pt}\ {\isacharbraceleft}{\kern0pt}{\isacharbraceright}{\kern0pt}{\isacharcolon}{\kern0pt}{\isacharcolon}{\kern0pt}nat\ set{\isacharparenright}{\kern0pt}{\isacharparenleft}{\kern0pt}{\isadigit{0}}{\isacharcolon}{\kern0pt}{\isacharequal}{\kern0pt}{\isacharquery}{\kern0pt}B{\isadigit{0}}{\isacharcomma}{\kern0pt}\ {\isadigit{1}}{\isacharcolon}{\kern0pt}{\isacharequal}{\kern0pt}{\isacharquery}{\kern0pt}B{\isadigit{1}}{\isacharparenright}{\kern0pt}{\isachardoublequoteclose}\isanewline
\ \ \isacommand{let}\isamarkupfalse%
\ {\isacharquery}{\kern0pt}L\ {\isacharequal}{\kern0pt}\ {\isachardoublequoteopen}{\isacharparenleft}{\kern0pt}{\isasymlambda}y\ {\isasymin}\ cube\ {\isadigit{1}}\ t{\isachardot}{\kern0pt}\ L\ {\isacharparenleft}{\kern0pt}y\ {\isadigit{0}}{\isacharparenright}{\kern0pt}{\isacharparenright}{\kern0pt}{\isachardoublequoteclose}\isanewline
\ \ \isacommand{have}\isamarkupfalse%
\ {\isachardoublequoteopen}{\isacharquery}{\kern0pt}B{\isadigit{0}}\ {\isasymnoteq}\ {\isacharbraceleft}{\kern0pt}{\isacharbraceright}{\kern0pt}{\isachardoublequoteclose}\ \isacommand{using}\isamarkupfalse%
\ assms{\isacharparenleft}{\kern0pt}{\isadigit{3}}{\isacharparenright}{\kern0pt}\ \isacommand{unfolding}\isamarkupfalse%
\ is{\isacharunderscore}{\kern0pt}line{\isacharunderscore}{\kern0pt}def\ \isacommand{by}\isamarkupfalse%
\ simp\isanewline
\isanewline
\ \ \isacommand{have}\isamarkupfalse%
\ L{\isadigit{1}}{\isacharcolon}{\kern0pt}\ {\isachardoublequoteopen}{\isacharquery}{\kern0pt}B{\isadigit{0}}\ {\isasymunion}\ {\isacharquery}{\kern0pt}B{\isadigit{1}}\ {\isacharequal}{\kern0pt}\ {\isacharbraceleft}{\kern0pt}{\isachardot}{\kern0pt}{\isachardot}{\kern0pt}{\isacharless}{\kern0pt}n{\isacharbraceright}{\kern0pt}{\isachardoublequoteclose}\ \isacommand{using}\isamarkupfalse%
\ assms{\isacharparenleft}{\kern0pt}{\isadigit{3}}{\isacharparenright}{\kern0pt}\ \isacommand{unfolding}\isamarkupfalse%
\ is{\isacharunderscore}{\kern0pt}line{\isacharunderscore}{\kern0pt}def\ \isacommand{by}\isamarkupfalse%
\ auto\isanewline
\ \ \isacommand{{\isacharbraceleft}{\kern0pt}}\isamarkupfalse%
\isanewline
\ \ \ \ \isacommand{have}\isamarkupfalse%
\ {\isachardoublequoteopen}{\isacharparenleft}{\kern0pt}{\isasymforall}s\ {\isacharless}{\kern0pt}\ t{\isachardot}{\kern0pt}\ L\ s\ i\ {\isacharequal}{\kern0pt}\ s{\isacharparenright}{\kern0pt}\ {\isasymlongrightarrow}\ {\isasymnot}{\isacharparenleft}{\kern0pt}{\isasymforall}x{\isacharless}{\kern0pt}t{\isachardot}{\kern0pt}\ {\isasymforall}y{\isacharless}{\kern0pt}t{\isachardot}{\kern0pt}\ L\ x\ i\ {\isacharequal}{\kern0pt}\ \ L\ y\ i{\isacharparenright}{\kern0pt}{\isachardoublequoteclose}\ \isakeyword{if}\ {\isachardoublequoteopen}i\ {\isacharless}{\kern0pt}\ n{\isachardoublequoteclose}\ \isakeyword{for}\ i\ \isacommand{using}\isamarkupfalse%
\ assms{\isacharparenleft}{\kern0pt}{\isadigit{2}}{\isacharparenright}{\kern0pt}\ \isanewline
\ \ \ \ \ \ \isacommand{using}\isamarkupfalse%
\ less{\isacharunderscore}{\kern0pt}trans\ \isacommand{by}\isamarkupfalse%
\ auto\ \isanewline
\ \ \ \ \isacommand{then}\isamarkupfalse%
\ \isacommand{have}\isamarkupfalse%
\ {\isacharasterisk}{\kern0pt}{\isacharcolon}{\kern0pt}{\isachardoublequoteopen}i\ {\isasymnotin}\ {\isacharquery}{\kern0pt}B{\isadigit{0}}{\isachardoublequoteclose}\ \isakeyword{if}\ {\isachardoublequoteopen}i\ {\isasymin}\ {\isacharquery}{\kern0pt}B{\isadigit{1}}{\isachardoublequoteclose}\ \isakeyword{for}\ i\ \isacommand{using}\isamarkupfalse%
\ that\ \isacommand{by}\isamarkupfalse%
\ blast\isanewline
\ \ \isacommand{{\isacharbraceright}{\kern0pt}}\isamarkupfalse%
\isanewline
\ \ \isacommand{moreover}\isamarkupfalse%
\isanewline
\ \ \isacommand{{\isacharbraceleft}{\kern0pt}}\isamarkupfalse%
\isanewline
\ \ \ \ \isacommand{have}\isamarkupfalse%
\ {\isachardoublequoteopen}{\isacharparenleft}{\kern0pt}{\isasymforall}x{\isacharless}{\kern0pt}t{\isachardot}{\kern0pt}\ {\isasymforall}y{\isacharless}{\kern0pt}t{\isachardot}{\kern0pt}\ L\ x\ i\ {\isacharequal}{\kern0pt}\ \ L\ y\ i{\isacharparenright}{\kern0pt}\ {\isasymlongrightarrow}\ {\isasymnot}{\isacharparenleft}{\kern0pt}{\isasymforall}s\ {\isacharless}{\kern0pt}\ t{\isachardot}{\kern0pt}\ L\ s\ i\ {\isacharequal}{\kern0pt}\ s{\isacharparenright}{\kern0pt}{\isachardoublequoteclose}\ \isakeyword{if}\ {\isachardoublequoteopen}i\ {\isacharless}{\kern0pt}\ n{\isachardoublequoteclose}\ \isakeyword{for}\ i\isanewline
\ \ \ \ \ \ \isacommand{using}\isamarkupfalse%
\ that\ calculation\ \isacommand{by}\isamarkupfalse%
\ blast\isanewline
\ \ \ \ \isacommand{then}\isamarkupfalse%
\ \isacommand{have}\isamarkupfalse%
\ {\isacharasterisk}{\kern0pt}{\isacharasterisk}{\kern0pt}{\isacharcolon}{\kern0pt}\ {\isachardoublequoteopen}{\isasymforall}i\ {\isasymin}\ {\isacharquery}{\kern0pt}B{\isadigit{0}}{\isachardot}{\kern0pt}\ i\ {\isasymnotin}\ {\isacharquery}{\kern0pt}B{\isadigit{1}}{\isachardoublequoteclose}\ \isanewline
\ \ \ \ \ \ \isacommand{by}\isamarkupfalse%
\ blast\isanewline
\ \ \isacommand{{\isacharbraceright}{\kern0pt}}\isamarkupfalse%
\isanewline
\ \ \isacommand{ultimately}\isamarkupfalse%
\ \isacommand{have}\isamarkupfalse%
\ L{\isadigit{2}}{\isacharcolon}{\kern0pt}\ {\isachardoublequoteopen}{\isacharquery}{\kern0pt}B{\isadigit{0}}\ {\isasyminter}\ {\isacharquery}{\kern0pt}B{\isadigit{1}}\ {\isacharequal}{\kern0pt}\ {\isacharbraceleft}{\kern0pt}{\isacharbraceright}{\kern0pt}{\isachardoublequoteclose}\ \isacommand{by}\isamarkupfalse%
\ blast\isanewline
\isanewline
\ \ \isacommand{let}\isamarkupfalse%
\ {\isacharquery}{\kern0pt}f\ {\isacharequal}{\kern0pt}\ {\isachardoublequoteopen}{\isacharparenleft}{\kern0pt}{\isasymlambda}i{\isachardot}{\kern0pt}\ if\ i\ {\isasymin}\ B\ {\isadigit{1}}\ then\ L\ {\isadigit{0}}\ i\ else\ undefined{\isacharparenright}{\kern0pt}{\isachardoublequoteclose}\isanewline
\ \ \isacommand{{\isacharbraceleft}{\kern0pt}}\isamarkupfalse%
\isanewline
\ \ \ \ \isacommand{have}\isamarkupfalse%
\ {\isachardoublequoteopen}{\isacharbraceleft}{\kern0pt}{\isachardot}{\kern0pt}{\isachardot}{\kern0pt}{\isadigit{1}}{\isacharcolon}{\kern0pt}{\isacharcolon}{\kern0pt}nat{\isacharbraceright}{\kern0pt}\ {\isacharequal}{\kern0pt}\ {\isacharbraceleft}{\kern0pt}{\isadigit{0}}{\isacharcomma}{\kern0pt}\ {\isadigit{1}}{\isacharbraceright}{\kern0pt}{\isachardoublequoteclose}\ \isacommand{by}\isamarkupfalse%
\ auto\isanewline
\ \ \ \ \isacommand{then}\isamarkupfalse%
\ \isacommand{have}\isamarkupfalse%
\ {\isachardoublequoteopen}{\isasymUnion}{\isacharparenleft}{\kern0pt}B\ {\isacharbackquote}{\kern0pt}\ {\isacharbraceleft}{\kern0pt}{\isachardot}{\kern0pt}{\isachardot}{\kern0pt}{\isadigit{1}}{\isacharcolon}{\kern0pt}{\isacharcolon}{\kern0pt}nat{\isacharbraceright}{\kern0pt}{\isacharparenright}{\kern0pt}\ {\isacharequal}{\kern0pt}\ B\ {\isadigit{0}}\ {\isasymunion}\ B\ {\isadigit{1}}{\isachardoublequoteclose}\ \isacommand{by}\isamarkupfalse%
\ simp\isanewline
\ \ \ \ \isacommand{then}\isamarkupfalse%
\ \isacommand{have}\isamarkupfalse%
\ {\isachardoublequoteopen}{\isasymUnion}{\isacharparenleft}{\kern0pt}B\ {\isacharbackquote}{\kern0pt}\ {\isacharbraceleft}{\kern0pt}{\isachardot}{\kern0pt}{\isachardot}{\kern0pt}{\isadigit{1}}{\isacharcolon}{\kern0pt}{\isacharcolon}{\kern0pt}nat{\isacharbraceright}{\kern0pt}{\isacharparenright}{\kern0pt}\ {\isacharequal}{\kern0pt}\ {\isacharquery}{\kern0pt}B{\isadigit{0}}\ {\isasymunion}\ {\isacharquery}{\kern0pt}B{\isadigit{1}}{\isachardoublequoteclose}\ \isacommand{unfolding}\isamarkupfalse%
\ B{\isacharunderscore}{\kern0pt}def\ \isacommand{by}\isamarkupfalse%
\ simp\isanewline
\ \ \ \ \isacommand{then}\isamarkupfalse%
\ \isacommand{have}\isamarkupfalse%
\ A{\isadigit{1}}{\isacharcolon}{\kern0pt}\ {\isachardoublequoteopen}disjoint{\isacharunderscore}{\kern0pt}family{\isacharunderscore}{\kern0pt}on\ B\ {\isacharbraceleft}{\kern0pt}{\isachardot}{\kern0pt}{\isachardot}{\kern0pt}{\isadigit{1}}{\isacharcolon}{\kern0pt}{\isacharcolon}{\kern0pt}nat{\isacharbraceright}{\kern0pt}{\isachardoublequoteclose}\ \isacommand{using}\isamarkupfalse%
\ L{\isadigit{2}}\ \isanewline
\ \ \ \ \ \ \isacommand{by}\isamarkupfalse%
\ {\isacharparenleft}{\kern0pt}simp\ add{\isacharcolon}{\kern0pt}\ B{\isacharunderscore}{\kern0pt}def\ Int{\isacharunderscore}{\kern0pt}commute\ disjoint{\isacharunderscore}{\kern0pt}family{\isacharunderscore}{\kern0pt}onI{\isacharparenright}{\kern0pt}\isanewline
\ \ \isacommand{{\isacharbraceright}{\kern0pt}}\isamarkupfalse%
\isanewline
\ \ \isacommand{moreover}\isamarkupfalse%
\isanewline
\ \ \isacommand{{\isacharbraceleft}{\kern0pt}}\isamarkupfalse%
\isanewline
\ \ \ \ \isacommand{have}\isamarkupfalse%
\ {\isachardoublequoteopen}{\isasymUnion}{\isacharparenleft}{\kern0pt}B\ {\isacharbackquote}{\kern0pt}\ {\isacharbraceleft}{\kern0pt}{\isachardot}{\kern0pt}{\isachardot}{\kern0pt}{\isadigit{1}}{\isacharcolon}{\kern0pt}{\isacharcolon}{\kern0pt}nat{\isacharbraceright}{\kern0pt}{\isacharparenright}{\kern0pt}\ {\isacharequal}{\kern0pt}\ B\ {\isadigit{0}}\ {\isasymunion}\ B\ {\isadigit{1}}{\isachardoublequoteclose}\ \isacommand{unfolding}\isamarkupfalse%
\ B{\isacharunderscore}{\kern0pt}def\ \isacommand{by}\isamarkupfalse%
\ auto\isanewline
\ \ \ \ \isacommand{then}\isamarkupfalse%
\ \isacommand{have}\isamarkupfalse%
\ {\isachardoublequoteopen}{\isasymUnion}{\isacharparenleft}{\kern0pt}B\ {\isacharbackquote}{\kern0pt}\ {\isacharbraceleft}{\kern0pt}{\isachardot}{\kern0pt}{\isachardot}{\kern0pt}{\isadigit{1}}{\isacharcolon}{\kern0pt}{\isacharcolon}{\kern0pt}nat{\isacharbraceright}{\kern0pt}{\isacharparenright}{\kern0pt}\ {\isacharequal}{\kern0pt}\ {\isacharbraceleft}{\kern0pt}{\isachardot}{\kern0pt}{\isachardot}{\kern0pt}{\isacharless}{\kern0pt}n{\isacharbraceright}{\kern0pt}{\isachardoublequoteclose}\ \isacommand{using}\isamarkupfalse%
\ L{\isadigit{1}}\ \isacommand{unfolding}\isamarkupfalse%
\ B{\isacharunderscore}{\kern0pt}def\ \isacommand{by}\isamarkupfalse%
\ simp\isanewline
\ \ \isacommand{{\isacharbraceright}{\kern0pt}}\isamarkupfalse%
\isanewline
\ \ \isacommand{moreover}\isamarkupfalse%
\isanewline
\ \ \isacommand{{\isacharbraceleft}{\kern0pt}}\isamarkupfalse%
\isanewline
\ \ \ \ \isacommand{have}\isamarkupfalse%
\ {\isachardoublequoteopen}{\isasymforall}i\ {\isasymin}\ {\isacharbraceleft}{\kern0pt}{\isachardot}{\kern0pt}{\isachardot}{\kern0pt}{\isacharless}{\kern0pt}{\isadigit{1}}{\isacharcolon}{\kern0pt}{\isacharcolon}{\kern0pt}nat{\isacharbraceright}{\kern0pt}{\isachardot}{\kern0pt}\ B\ i\ {\isasymnoteq}\ {\isacharbraceleft}{\kern0pt}{\isacharbraceright}{\kern0pt}{\isachardoublequoteclose}\ \isanewline
\ \ \ \ \ \ \isacommand{using}\isamarkupfalse%
\ {\isacartoucheopen}{\isacharbraceleft}{\kern0pt}i{\isachardot}{\kern0pt}\ i\ {\isacharless}{\kern0pt}\ n\ {\isasymand}\ {\isacharparenleft}{\kern0pt}{\isasymforall}s{\isacharless}{\kern0pt}t{\isachardot}{\kern0pt}\ L\ s\ i\ {\isacharequal}{\kern0pt}\ s{\isacharparenright}{\kern0pt}{\isacharbraceright}{\kern0pt}\ {\isasymnoteq}\ {\isacharbraceleft}{\kern0pt}{\isacharbraceright}{\kern0pt}{\isacartoucheclose}\ fun{\isacharunderscore}{\kern0pt}upd{\isacharunderscore}{\kern0pt}same\ lessThan{\isacharunderscore}{\kern0pt}iff\ less{\isacharunderscore}{\kern0pt}one\ \isacommand{unfolding}\isamarkupfalse%
\ B{\isacharunderscore}{\kern0pt}def\ \isacommand{by}\isamarkupfalse%
\ auto\isanewline
\ \ \ \ \isacommand{then}\isamarkupfalse%
\ \isacommand{have}\isamarkupfalse%
\ {\isachardoublequoteopen}{\isacharbraceleft}{\kern0pt}{\isacharbraceright}{\kern0pt}\ {\isasymnotin}\ B\ {\isacharbackquote}{\kern0pt}\ {\isacharbraceleft}{\kern0pt}{\isachardot}{\kern0pt}{\isachardot}{\kern0pt}{\isacharless}{\kern0pt}{\isadigit{1}}{\isacharcolon}{\kern0pt}{\isacharcolon}{\kern0pt}nat{\isacharbraceright}{\kern0pt}{\isachardoublequoteclose}\ \isacommand{by}\isamarkupfalse%
\ blast\isanewline
\ \ \isacommand{{\isacharbraceright}{\kern0pt}}\isamarkupfalse%
\isanewline
\ \ \isacommand{moreover}\isamarkupfalse%
\ \isanewline
\ \ \isacommand{{\isacharbraceleft}{\kern0pt}}\isamarkupfalse%
\isanewline
\ \ \ \ \isacommand{have}\isamarkupfalse%
\ {\isachardoublequoteopen}{\isacharquery}{\kern0pt}f\ {\isasymin}\ {\isacharparenleft}{\kern0pt}B\ {\isadigit{1}}{\isacharparenright}{\kern0pt}\ {\isasymrightarrow}\isactrlsub E\ {\isacharbraceleft}{\kern0pt}{\isachardot}{\kern0pt}{\isachardot}{\kern0pt}{\isacharless}{\kern0pt}t{\isacharbraceright}{\kern0pt}{\isachardoublequoteclose}\ \isanewline
\ \ \ \ \isacommand{proof}\isamarkupfalse%
\isanewline
\ \ \ \ \ \ \isacommand{fix}\isamarkupfalse%
\ i\isanewline
\ \ \ \ \ \ \isacommand{assume}\isamarkupfalse%
\ asm{\isacharcolon}{\kern0pt}\ {\isachardoublequoteopen}i\ {\isasymin}\ {\isacharparenleft}{\kern0pt}B\ {\isadigit{1}}{\isacharparenright}{\kern0pt}{\isachardoublequoteclose}\isanewline
\ \ \ \ \ \ \isacommand{have}\isamarkupfalse%
\ {\isachardoublequoteopen}L\ a\ b\ {\isasymin}\ {\isacharbraceleft}{\kern0pt}{\isachardot}{\kern0pt}{\isachardot}{\kern0pt}{\isacharless}{\kern0pt}t{\isacharbraceright}{\kern0pt}{\isachardoublequoteclose}\ \isakeyword{if}\ {\isachardoublequoteopen}a\ {\isacharless}{\kern0pt}\ t{\isachardoublequoteclose}\ \isakeyword{and}\ {\isachardoublequoteopen}b\ {\isacharless}{\kern0pt}\ n{\isachardoublequoteclose}\ \isakeyword{for}\ a\ b\ \isacommand{using}\isamarkupfalse%
\ assms{\isacharparenleft}{\kern0pt}{\isadigit{3}}{\isacharparenright}{\kern0pt}\ that\ \isacommand{unfolding}\isamarkupfalse%
\ is{\isacharunderscore}{\kern0pt}line{\isacharunderscore}{\kern0pt}def\ cube{\isacharunderscore}{\kern0pt}def\ \isacommand{by}\isamarkupfalse%
\ auto\isanewline
\ \ \ \ \ \ \isacommand{then}\isamarkupfalse%
\ \isacommand{have}\isamarkupfalse%
\ {\isachardoublequoteopen}L\ {\isadigit{0}}\ i\ {\isasymin}\ {\isacharbraceleft}{\kern0pt}{\isachardot}{\kern0pt}{\isachardot}{\kern0pt}{\isacharless}{\kern0pt}t{\isacharbraceright}{\kern0pt}{\isachardoublequoteclose}\ \isacommand{using}\isamarkupfalse%
\ assms{\isacharparenleft}{\kern0pt}{\isadigit{2}}{\isacharparenright}{\kern0pt}\ asm\ calculation{\isacharparenleft}{\kern0pt}{\isadigit{2}}{\isacharparenright}{\kern0pt}\ \isacommand{by}\isamarkupfalse%
\ blast\isanewline
\ \ \ \ \ \ \isacommand{then}\isamarkupfalse%
\ \isacommand{show}\isamarkupfalse%
\ {\isachardoublequoteopen}{\isacharquery}{\kern0pt}f\ i\ {\isasymin}\ {\isacharbraceleft}{\kern0pt}{\isachardot}{\kern0pt}{\isachardot}{\kern0pt}{\isacharless}{\kern0pt}t{\isacharbraceright}{\kern0pt}{\isachardoublequoteclose}\ \isacommand{using}\isamarkupfalse%
\ asm\ \isacommand{by}\isamarkupfalse%
\ presburger\isanewline
\ \ \ \ \isacommand{qed}\isamarkupfalse%
\ {\isacharparenleft}{\kern0pt}auto{\isacharparenright}{\kern0pt}\isanewline
\ \ \isacommand{{\isacharbraceright}{\kern0pt}}\isamarkupfalse%
\isanewline
\isanewline
\ \ \isacommand{moreover}\isamarkupfalse%
\isanewline
\ \ \isacommand{{\isacharbraceleft}{\kern0pt}}\isamarkupfalse%
\isanewline
\ \ \ \ \isacommand{have}\isamarkupfalse%
\ {\isachardoublequoteopen}L\ {\isasymin}\ {\isacharbraceleft}{\kern0pt}{\isachardot}{\kern0pt}{\isachardot}{\kern0pt}{\isacharless}{\kern0pt}t{\isacharbraceright}{\kern0pt}\ {\isasymrightarrow}\isactrlsub E\ {\isacharparenleft}{\kern0pt}cube\ n\ t{\isacharparenright}{\kern0pt}{\isachardoublequoteclose}\ \isacommand{using}\isamarkupfalse%
\ assms{\isacharparenleft}{\kern0pt}{\isadigit{3}}{\isacharparenright}{\kern0pt}\ \isacommand{by}\isamarkupfalse%
\ {\isacharparenleft}{\kern0pt}simp\ add{\isacharcolon}{\kern0pt}\ is{\isacharunderscore}{\kern0pt}line{\isacharunderscore}{\kern0pt}def{\isacharparenright}{\kern0pt}\isanewline
\ \ \ \ \isacommand{then}\isamarkupfalse%
\ \isacommand{have}\isamarkupfalse%
\ {\isachardoublequoteopen}{\isacharquery}{\kern0pt}L\ {\isasymin}\ {\isacharparenleft}{\kern0pt}cube\ {\isadigit{1}}\ t{\isacharparenright}{\kern0pt}\ {\isasymrightarrow}\isactrlsub E\ {\isacharparenleft}{\kern0pt}cube\ n\ t{\isacharparenright}{\kern0pt}{\isachardoublequoteclose}\isanewline
\ \ \ \ \ \ \isacommand{using}\isamarkupfalse%
\ bij{\isacharunderscore}{\kern0pt}domain{\isacharunderscore}{\kern0pt}PiE{\isacharbrackleft}{\kern0pt}of\ {\isachardoublequoteopen}{\isacharparenleft}{\kern0pt}{\isasymlambda}f{\isachardot}{\kern0pt}\ f\ {\isadigit{0}}{\isacharparenright}{\kern0pt}{\isachardoublequoteclose}\ {\isachardoublequoteopen}{\isacharparenleft}{\kern0pt}cube\ {\isadigit{1}}\ t{\isacharparenright}{\kern0pt}{\isachardoublequoteclose}\ {\isachardoublequoteopen}{\isacharbraceleft}{\kern0pt}{\isachardot}{\kern0pt}{\isachardot}{\kern0pt}{\isacharless}{\kern0pt}t{\isacharbraceright}{\kern0pt}{\isachardoublequoteclose}\ {\isachardoublequoteopen}L{\isachardoublequoteclose}\ {\isachardoublequoteopen}cube\ n\ t{\isachardoublequoteclose}{\isacharbrackright}{\kern0pt}\ one{\isacharunderscore}{\kern0pt}dim{\isacharunderscore}{\kern0pt}cube{\isacharunderscore}{\kern0pt}eq{\isacharunderscore}{\kern0pt}nat{\isacharunderscore}{\kern0pt}set{\isacharbrackleft}{\kern0pt}of\ {\isachardoublequoteopen}t{\isachardoublequoteclose}{\isacharbrackright}{\kern0pt}\ \isacommand{by}\isamarkupfalse%
\ auto\isanewline
\ \ \isacommand{{\isacharbraceright}{\kern0pt}}\isamarkupfalse%
\isanewline
\ \ \isacommand{moreover}\isamarkupfalse%
\isanewline
\ \ \isacommand{{\isacharbraceleft}{\kern0pt}}\isamarkupfalse%
\isanewline
\ \ \ \ \isacommand{have}\isamarkupfalse%
\ {\isachardoublequoteopen}{\isasymforall}y\ {\isasymin}\ cube\ {\isadigit{1}}\ t{\isachardot}{\kern0pt}\ {\isacharparenleft}{\kern0pt}{\isasymforall}i\ {\isasymin}\ B\ {\isadigit{1}}{\isachardot}{\kern0pt}\ {\isacharquery}{\kern0pt}L\ y\ i\ {\isacharequal}{\kern0pt}\ {\isacharquery}{\kern0pt}f\ i{\isacharparenright}{\kern0pt}\ {\isasymand}\ {\isacharparenleft}{\kern0pt}{\isasymforall}j\ {\isacharless}{\kern0pt}\ {\isadigit{1}}{\isachardot}{\kern0pt}\ {\isasymforall}i\ {\isasymin}\ B\ j{\isachardot}{\kern0pt}\ {\isacharparenleft}{\kern0pt}{\isacharquery}{\kern0pt}L\ y{\isacharparenright}{\kern0pt}\ i\ {\isacharequal}{\kern0pt}\ y\ j{\isacharparenright}{\kern0pt}{\isachardoublequoteclose}\isanewline
\ \ \ \ \isacommand{proof}\isamarkupfalse%
\isanewline
\ \ \ \ \ \ \isacommand{fix}\isamarkupfalse%
\ y\ \isanewline
\ \ \ \ \ \ \isacommand{assume}\isamarkupfalse%
\ {\isachardoublequoteopen}y\ {\isasymin}\ cube\ {\isadigit{1}}\ t{\isachardoublequoteclose}\isanewline
\ \ \ \ \ \ \isacommand{then}\isamarkupfalse%
\ \isacommand{have}\isamarkupfalse%
\ {\isachardoublequoteopen}y\ {\isadigit{0}}\ {\isasymin}\ {\isacharbraceleft}{\kern0pt}{\isachardot}{\kern0pt}{\isachardot}{\kern0pt}{\isacharless}{\kern0pt}t{\isacharbraceright}{\kern0pt}{\isachardoublequoteclose}\ \isacommand{unfolding}\isamarkupfalse%
\ cube{\isacharunderscore}{\kern0pt}def\ \isacommand{by}\isamarkupfalse%
\ blast\isanewline
\isanewline
\ \ \ \ \ \ \isacommand{have}\isamarkupfalse%
\ {\isachardoublequoteopen}{\isacharparenleft}{\kern0pt}{\isasymforall}i\ {\isasymin}\ B\ {\isadigit{1}}{\isachardot}{\kern0pt}\ {\isacharquery}{\kern0pt}L\ y\ i\ {\isacharequal}{\kern0pt}\ {\isacharquery}{\kern0pt}f\ i{\isacharparenright}{\kern0pt}{\isachardoublequoteclose}\isanewline
\ \ \ \ \ \ \isacommand{proof}\isamarkupfalse%
\isanewline
\ \ \ \ \ \ \ \ \isacommand{fix}\isamarkupfalse%
\ i\isanewline
\ \ \ \ \ \ \ \ \isacommand{assume}\isamarkupfalse%
\ {\isachardoublequoteopen}i\ {\isasymin}\ B\ {\isadigit{1}}{\isachardoublequoteclose}\isanewline
\ \ \ \ \ \ \ \ \isacommand{then}\isamarkupfalse%
\ \isacommand{have}\isamarkupfalse%
\ {\isachardoublequoteopen}{\isacharquery}{\kern0pt}f\ i\ {\isacharequal}{\kern0pt}\ L\ {\isadigit{0}}\ i{\isachardoublequoteclose}\ \isanewline
\ \ \ \ \ \ \ \ \ \ \isacommand{by}\isamarkupfalse%
\ meson\isanewline
\ \ \ \ \ \ \ \ \isacommand{moreover}\isamarkupfalse%
\ \isacommand{have}\isamarkupfalse%
\ {\isachardoublequoteopen}{\isacharquery}{\kern0pt}L\ y\ i\ {\isacharequal}{\kern0pt}\ L\ {\isacharparenleft}{\kern0pt}y\ {\isadigit{0}}{\isacharparenright}{\kern0pt}\ i{\isachardoublequoteclose}\ \isacommand{using}\isamarkupfalse%
\ {\isacartoucheopen}y\ {\isasymin}\ cube\ {\isadigit{1}}\ t{\isacartoucheclose}\ \isacommand{by}\isamarkupfalse%
\ simp\isanewline
\ \ \ \ \ \ \ \ \isacommand{moreover}\isamarkupfalse%
\ \isacommand{have}\isamarkupfalse%
\ {\isachardoublequoteopen}L\ {\isacharparenleft}{\kern0pt}y\ {\isadigit{0}}{\isacharparenright}{\kern0pt}\ i\ {\isacharequal}{\kern0pt}\ L\ {\isadigit{0}}\ i{\isachardoublequoteclose}\ \isanewline
\ \ \ \ \ \ \ \ \isacommand{proof}\isamarkupfalse%
\ {\isacharminus}{\kern0pt}\isanewline
\ \ \ \ \ \ \ \ \ \ \isacommand{have}\isamarkupfalse%
\ {\isachardoublequoteopen}i\ {\isasymin}\ {\isacharquery}{\kern0pt}B{\isadigit{1}}{\isachardoublequoteclose}\ \isacommand{using}\isamarkupfalse%
\ {\isacartoucheopen}i\ {\isasymin}\ B\ {\isadigit{1}}{\isacartoucheclose}\ \isacommand{unfolding}\isamarkupfalse%
\ B{\isacharunderscore}{\kern0pt}def\ fun{\isacharunderscore}{\kern0pt}upd{\isacharunderscore}{\kern0pt}def\ \isacommand{by}\isamarkupfalse%
\ presburger\isanewline
\ \ \ \ \ \ \ \ \ \ \isacommand{then}\isamarkupfalse%
\ \isacommand{have}\isamarkupfalse%
\ {\isachardoublequoteopen}{\isacharparenleft}{\kern0pt}{\isasymforall}x{\isacharless}{\kern0pt}t{\isachardot}{\kern0pt}\ {\isasymforall}y{\isacharless}{\kern0pt}t{\isachardot}{\kern0pt}\ L\ x\ i\ {\isacharequal}{\kern0pt}\ L\ y\ i{\isacharparenright}{\kern0pt}{\isachardoublequoteclose}\ \isacommand{by}\isamarkupfalse%
\ blast\isanewline
\ \ \ \ \ \ \ \ \ \ \isacommand{then}\isamarkupfalse%
\ \isacommand{show}\isamarkupfalse%
\ {\isachardoublequoteopen}L\ {\isacharparenleft}{\kern0pt}y\ {\isadigit{0}}{\isacharparenright}{\kern0pt}\ i\ {\isacharequal}{\kern0pt}\ L\ {\isadigit{0}}\ i{\isachardoublequoteclose}\ \isacommand{using}\isamarkupfalse%
\ {\isacartoucheopen}y\ {\isadigit{0}}\ {\isasymin}\ {\isacharbraceleft}{\kern0pt}{\isachardot}{\kern0pt}{\isachardot}{\kern0pt}{\isacharless}{\kern0pt}t{\isacharbraceright}{\kern0pt}{\isacartoucheclose}\ \isacommand{by}\isamarkupfalse%
\ blast\isanewline
\ \ \ \ \ \ \ \ \isacommand{qed}\isamarkupfalse%
\isanewline
\ \ \ \ \ \ \ \ \isacommand{ultimately}\isamarkupfalse%
\ \isacommand{show}\isamarkupfalse%
\ {\isachardoublequoteopen}{\isacharquery}{\kern0pt}L\ y\ i\ {\isacharequal}{\kern0pt}\ {\isacharquery}{\kern0pt}f\ i{\isachardoublequoteclose}\ \isacommand{by}\isamarkupfalse%
\ simp\isanewline
\ \ \ \ \ \ \isacommand{qed}\isamarkupfalse%
\isanewline
\isanewline
\ \ \ \ \ \ \isacommand{moreover}\isamarkupfalse%
\ \isacommand{have}\isamarkupfalse%
\ {\isachardoublequoteopen}{\isacharparenleft}{\kern0pt}{\isacharquery}{\kern0pt}L\ y{\isacharparenright}{\kern0pt}\ i\ {\isacharequal}{\kern0pt}\ y\ j{\isachardoublequoteclose}\ \isakeyword{if}\ {\isachardoublequoteopen}j\ {\isacharless}{\kern0pt}\ {\isadigit{1}}{\isachardoublequoteclose}\ \isakeyword{and}\ {\isachardoublequoteopen}i\ {\isasymin}\ B\ j{\isachardoublequoteclose}\ \isakeyword{for}\ i\ j\isanewline
\ \ \ \ \ \ \isacommand{proof}\isamarkupfalse%
{\isacharminus}{\kern0pt}\isanewline
\ \ \ \ \ \ \ \ \isacommand{have}\isamarkupfalse%
\ {\isachardoublequoteopen}i\ {\isasymin}\ B\ {\isadigit{0}}{\isachardoublequoteclose}\ \isacommand{using}\isamarkupfalse%
\ that\ \isacommand{by}\isamarkupfalse%
\ blast\isanewline
\ \ \ \ \ \ \ \ \isacommand{then}\isamarkupfalse%
\ \isacommand{have}\isamarkupfalse%
\ {\isachardoublequoteopen}i\ {\isasymin}\ {\isacharquery}{\kern0pt}B{\isadigit{0}}{\isachardoublequoteclose}\ \isacommand{unfolding}\isamarkupfalse%
\ B{\isacharunderscore}{\kern0pt}def\ \isacommand{by}\isamarkupfalse%
\ auto\ \isanewline
\ \ \ \ \ \ \ \ \isacommand{then}\isamarkupfalse%
\ \isacommand{have}\isamarkupfalse%
\ {\isachardoublequoteopen}{\isacharparenleft}{\kern0pt}{\isasymforall}s\ {\isacharless}{\kern0pt}\ t{\isachardot}{\kern0pt}\ L\ s\ i\ {\isacharequal}{\kern0pt}\ s{\isacharparenright}{\kern0pt}{\isachardoublequoteclose}\ \isacommand{by}\isamarkupfalse%
\ blast\isanewline
\ \ \ \ \ \ \ \ \isacommand{moreover}\isamarkupfalse%
\ \isacommand{have}\isamarkupfalse%
\ {\isachardoublequoteopen}y\ {\isadigit{0}}\ {\isacharless}{\kern0pt}\ t{\isachardoublequoteclose}\ \isacommand{using}\isamarkupfalse%
\ {\isacartoucheopen}y\ {\isasymin}\ cube\ {\isadigit{1}}\ t{\isacartoucheclose}\ \isacommand{unfolding}\isamarkupfalse%
\ cube{\isacharunderscore}{\kern0pt}def\ \isacommand{by}\isamarkupfalse%
\ auto\isanewline
\ \ \ \ \ \ \ \ \isacommand{ultimately}\isamarkupfalse%
\ \isacommand{have}\isamarkupfalse%
\ {\isachardoublequoteopen}L\ {\isacharparenleft}{\kern0pt}y\ {\isadigit{0}}{\isacharparenright}{\kern0pt}\ i\ {\isacharequal}{\kern0pt}\ y\ {\isadigit{0}}{\isachardoublequoteclose}\ \isacommand{by}\isamarkupfalse%
\ simp\isanewline
\ \ \ \ \ \ \ \ \isacommand{then}\isamarkupfalse%
\ \isacommand{show}\isamarkupfalse%
\ {\isachardoublequoteopen}{\isacharquery}{\kern0pt}L\ y\ i\ {\isacharequal}{\kern0pt}\ y\ j{\isachardoublequoteclose}\ \isacommand{using}\isamarkupfalse%
\ that\ \isacommand{using}\isamarkupfalse%
\ {\isacartoucheopen}y\ {\isasymin}\ cube\ {\isadigit{1}}\ t{\isacartoucheclose}\ \isacommand{by}\isamarkupfalse%
\ force\isanewline
\ \ \ \ \ \ \isacommand{qed}\isamarkupfalse%
\isanewline
\isanewline
\ \ \ \ \ \ \isacommand{ultimately}\isamarkupfalse%
\ \isacommand{show}\isamarkupfalse%
\ {\isachardoublequoteopen}{\isacharparenleft}{\kern0pt}{\isasymforall}i\ {\isasymin}\ B\ {\isadigit{1}}{\isachardot}{\kern0pt}\ {\isacharquery}{\kern0pt}L\ y\ i\ {\isacharequal}{\kern0pt}\ {\isacharquery}{\kern0pt}f\ i{\isacharparenright}{\kern0pt}\ {\isasymand}\ {\isacharparenleft}{\kern0pt}{\isasymforall}j\ {\isacharless}{\kern0pt}\ {\isadigit{1}}{\isachardot}{\kern0pt}\ {\isasymforall}i\ {\isasymin}\ B\ j{\isachardot}{\kern0pt}\ {\isacharparenleft}{\kern0pt}{\isacharquery}{\kern0pt}L\ y{\isacharparenright}{\kern0pt}\ i\ {\isacharequal}{\kern0pt}\ y\ j{\isacharparenright}{\kern0pt}{\isachardoublequoteclose}\ \isanewline
\ \ \ \ \ \ \ \ \isacommand{by}\isamarkupfalse%
\ blast\isanewline
\ \ \ \ \isacommand{qed}\isamarkupfalse%
\isanewline
\ \ \isacommand{{\isacharbraceright}{\kern0pt}}\isamarkupfalse%
\isanewline
\ \ \isacommand{ultimately}\isamarkupfalse%
\ \isacommand{show}\isamarkupfalse%
\ {\isachardoublequoteopen}is{\isacharunderscore}{\kern0pt}subspace\ {\isacharquery}{\kern0pt}L\ {\isadigit{1}}\ n\ t{\isachardoublequoteclose}\ \isacommand{unfolding}\isamarkupfalse%
\ is{\isacharunderscore}{\kern0pt}subspace{\isacharunderscore}{\kern0pt}def\ \isacommand{by}\isamarkupfalse%
\ blast\isanewline
\isacommand{qed}\isamarkupfalse%
%
\endisatagproof
{\isafoldproof}%
%
\isadelimproof
\isanewline
%
\endisadelimproof
\isanewline
\isacommand{lemma}\isamarkupfalse%
\ line{\isacharunderscore}{\kern0pt}is{\isacharunderscore}{\kern0pt}dim{\isadigit{1}}{\isacharunderscore}{\kern0pt}subspace{\isacharcolon}{\kern0pt}\ \isakeyword{assumes}\ {\isachardoublequoteopen}n\ {\isachargreater}{\kern0pt}\ {\isadigit{0}}{\isachardoublequoteclose}\ {\isachardoublequoteopen}t\ {\isachargreater}{\kern0pt}\ {\isadigit{0}}{\isachardoublequoteclose}\ {\isachardoublequoteopen}is{\isacharunderscore}{\kern0pt}line\ L\ n\ t{\isachardoublequoteclose}\ \isakeyword{shows}\ {\isachardoublequoteopen}is{\isacharunderscore}{\kern0pt}subspace\ {\isacharparenleft}{\kern0pt}restrict\ {\isacharparenleft}{\kern0pt}{\isasymlambda}y{\isachardot}{\kern0pt}\ L\ {\isacharparenleft}{\kern0pt}y\ {\isadigit{0}}{\isacharparenright}{\kern0pt}{\isacharparenright}{\kern0pt}\ {\isacharparenleft}{\kern0pt}cube\ {\isadigit{1}}\ t{\isacharparenright}{\kern0pt}{\isacharparenright}{\kern0pt}\ {\isadigit{1}}\ n\ t{\isachardoublequoteclose}\isanewline
%
\isadelimproof
\ \ %
\endisadelimproof
%
\isatagproof
\isacommand{using}\isamarkupfalse%
\ line{\isacharunderscore}{\kern0pt}is{\isacharunderscore}{\kern0pt}dim{\isadigit{1}}{\isacharunderscore}{\kern0pt}subspace{\isacharunderscore}{\kern0pt}t{\isacharunderscore}{\kern0pt}{\isadigit{1}}{\isacharbrackleft}{\kern0pt}of\ n\ L{\isacharbrackright}{\kern0pt}\ line{\isacharunderscore}{\kern0pt}is{\isacharunderscore}{\kern0pt}dim{\isadigit{1}}{\isacharunderscore}{\kern0pt}subspace{\isacharunderscore}{\kern0pt}t{\isacharunderscore}{\kern0pt}ge{\isacharunderscore}{\kern0pt}{\isadigit{1}}{\isacharbrackleft}{\kern0pt}of\ n\ t\ L{\isacharbrackright}{\kern0pt}\ assms\ not{\isacharunderscore}{\kern0pt}less{\isacharunderscore}{\kern0pt}iff{\isacharunderscore}{\kern0pt}gr{\isacharunderscore}{\kern0pt}or{\isacharunderscore}{\kern0pt}eq\ \isacommand{by}\isamarkupfalse%
\ blast%
\endisatagproof
{\isafoldproof}%
%
\isadelimproof
\isanewline
%
\endisadelimproof
\isanewline
\isacommand{definition}\isamarkupfalse%
\ hj\ \isanewline
\ \ \isakeyword{where}\ {\isachardoublequoteopen}hj\ r\ t\ {\isasymequiv}\ {\isacharparenleft}{\kern0pt}{\isasymexists}N{\isachargreater}{\kern0pt}{\isadigit{0}}{\isachardot}{\kern0pt}\ {\isasymforall}N{\isacharprime}{\kern0pt}\ {\isasymge}\ N{\isachardot}{\kern0pt}\ {\isasymforall}{\isasymchi}{\isachardot}{\kern0pt}\ {\isasymchi}\ {\isasymin}\ {\isacharparenleft}{\kern0pt}cube\ N{\isacharprime}{\kern0pt}\ t{\isacharparenright}{\kern0pt}\ {\isasymrightarrow}\isactrlsub E\ {\isacharbraceleft}{\kern0pt}{\isachardot}{\kern0pt}{\isachardot}{\kern0pt}{\isacharless}{\kern0pt}r{\isacharcolon}{\kern0pt}{\isacharcolon}{\kern0pt}nat{\isacharbraceright}{\kern0pt}\ {\isasymlongrightarrow}\ {\isacharparenleft}{\kern0pt}{\isasymexists}L{\isachardot}{\kern0pt}\ {\isasymexists}c{\isacharless}{\kern0pt}r{\isachardot}{\kern0pt}\ is{\isacharunderscore}{\kern0pt}line\ L\ N{\isacharprime}{\kern0pt}\ t\ {\isasymand}\ {\isacharparenleft}{\kern0pt}{\isasymforall}y\ {\isasymin}\ L\ {\isacharbackquote}{\kern0pt}\ {\isacharbraceleft}{\kern0pt}{\isachardot}{\kern0pt}{\isachardot}{\kern0pt}{\isacharless}{\kern0pt}t{\isacharbraceright}{\kern0pt}{\isachardot}{\kern0pt}\ {\isasymchi}\ y\ {\isacharequal}{\kern0pt}\ c{\isacharparenright}{\kern0pt}{\isacharparenright}{\kern0pt}{\isacharparenright}{\kern0pt}{\isachardoublequoteclose}\isanewline
\isanewline
\isacommand{definition}\isamarkupfalse%
\ lhj\isanewline
\ \ \isakeyword{where}\ {\isachardoublequoteopen}lhj\ r\ t\ k\ {\isasymequiv}\ {\isacharparenleft}{\kern0pt}{\isasymexists}M\ {\isachargreater}{\kern0pt}\ {\isadigit{0}}{\isachardot}{\kern0pt}\ {\isasymforall}M{\isacharprime}{\kern0pt}\ {\isasymge}\ M{\isachardot}{\kern0pt}\ {\isasymforall}{\isasymchi}{\isachardot}{\kern0pt}\ {\isasymchi}\ {\isasymin}\ {\isacharparenleft}{\kern0pt}cube\ M{\isacharprime}{\kern0pt}\ {\isacharparenleft}{\kern0pt}t\ {\isacharplus}{\kern0pt}\ {\isadigit{1}}{\isacharparenright}{\kern0pt}{\isacharparenright}{\kern0pt}\ {\isasymrightarrow}\isactrlsub E\ {\isacharbraceleft}{\kern0pt}{\isachardot}{\kern0pt}{\isachardot}{\kern0pt}{\isacharless}{\kern0pt}r{\isacharcolon}{\kern0pt}{\isacharcolon}{\kern0pt}nat{\isacharbraceright}{\kern0pt}\ {\isasymlongrightarrow}\ {\isacharparenleft}{\kern0pt}{\isasymexists}S{\isachardot}{\kern0pt}\ layered{\isacharunderscore}{\kern0pt}subspace\ S\ k\ M{\isacharprime}{\kern0pt}\ t\ r\ {\isasymchi}{\isacharparenright}{\kern0pt}{\isacharparenright}{\kern0pt}{\isachardoublequoteclose}%
\begin{isamarkuptext}%
Base case of Theorem 4%
\end{isamarkuptext}\isamarkuptrue%
\isacommand{lemma}\isamarkupfalse%
\ thm{\isadigit{4}}{\isacharunderscore}{\kern0pt}k{\isacharunderscore}{\kern0pt}{\isadigit{1}}{\isacharcolon}{\kern0pt}\ \isanewline
\ \ \isakeyword{fixes}\ \ \ r\ t\isanewline
\ \ \isakeyword{assumes}\ {\isachardoublequoteopen}t\ {\isachargreater}{\kern0pt}\ {\isadigit{0}}{\isachardoublequoteclose}\isanewline
\ \ \ \ \isakeyword{and}\ {\isachardoublequoteopen}{\isasymAnd}r{\isacharprime}{\kern0pt}{\isachardot}{\kern0pt}\ hj\ r{\isacharprime}{\kern0pt}\ t{\isachardoublequoteclose}\ \isanewline
\ \ \isakeyword{shows}\ {\isachardoublequoteopen}lhj\ r\ t\ {\isadigit{1}}{\isachardoublequoteclose}\isanewline
%
\isadelimproof
\isanewline
%
\endisadelimproof
%
\isatagproof
\isacommand{proof}\isamarkupfalse%
{\isacharminus}{\kern0pt}\isanewline
\ \ \isacommand{obtain}\isamarkupfalse%
\ N\ \isakeyword{where}\ N{\isacharunderscore}{\kern0pt}def{\isacharcolon}{\kern0pt}\ {\isachardoublequoteopen}N\ {\isachargreater}{\kern0pt}\ {\isadigit{0}}\ {\isasymand}\ {\isacharparenleft}{\kern0pt}{\isasymforall}N{\isacharprime}{\kern0pt}\ {\isasymge}\ N{\isachardot}{\kern0pt}\ {\isasymforall}{\isasymchi}{\isachardot}{\kern0pt}\ {\isasymchi}\ {\isasymin}\ {\isacharparenleft}{\kern0pt}cube\ N{\isacharprime}{\kern0pt}\ t{\isacharparenright}{\kern0pt}\ {\isasymrightarrow}\isactrlsub E\ {\isacharbraceleft}{\kern0pt}{\isachardot}{\kern0pt}{\isachardot}{\kern0pt}{\isacharless}{\kern0pt}r{\isacharcolon}{\kern0pt}{\isacharcolon}{\kern0pt}nat{\isacharbraceright}{\kern0pt}\ {\isasymlongrightarrow}\ {\isacharparenleft}{\kern0pt}{\isasymexists}L{\isachardot}{\kern0pt}\ {\isasymexists}c{\isacharless}{\kern0pt}r{\isachardot}{\kern0pt}\ is{\isacharunderscore}{\kern0pt}line\ L\ N{\isacharprime}{\kern0pt}\ t\ {\isasymand}\ {\isacharparenleft}{\kern0pt}{\isasymforall}y\ {\isasymin}\ L\ {\isacharbackquote}{\kern0pt}\ {\isacharbraceleft}{\kern0pt}{\isachardot}{\kern0pt}{\isachardot}{\kern0pt}{\isacharless}{\kern0pt}t{\isacharbraceright}{\kern0pt}{\isachardot}{\kern0pt}\ {\isasymchi}\ y\ {\isacharequal}{\kern0pt}\ c{\isacharparenright}{\kern0pt}{\isacharparenright}{\kern0pt}{\isacharparenright}{\kern0pt}{\isachardoublequoteclose}\ \isacommand{using}\isamarkupfalse%
\ assms{\isacharparenleft}{\kern0pt}{\isadigit{2}}{\isacharparenright}{\kern0pt}\ \isacommand{unfolding}\isamarkupfalse%
\ hj{\isacharunderscore}{\kern0pt}def\ \isacommand{by}\isamarkupfalse%
\ blast\isanewline
\isanewline
\ \ \isacommand{have}\isamarkupfalse%
\ {\isachardoublequoteopen}{\isasymforall}N{\isacharprime}{\kern0pt}\ {\isasymge}\ N{\isachardot}{\kern0pt}\ {\isasymforall}{\isasymchi}{\isachardot}{\kern0pt}\ {\isasymchi}\ {\isasymin}\ {\isacharparenleft}{\kern0pt}cube\ N{\isacharprime}{\kern0pt}\ {\isacharparenleft}{\kern0pt}t\ {\isacharplus}{\kern0pt}\ {\isadigit{1}}{\isacharparenright}{\kern0pt}{\isacharparenright}{\kern0pt}\ {\isasymrightarrow}\isactrlsub E\ {\isacharbraceleft}{\kern0pt}{\isachardot}{\kern0pt}{\isachardot}{\kern0pt}{\isacharless}{\kern0pt}r{\isacharcolon}{\kern0pt}{\isacharcolon}{\kern0pt}nat{\isacharbraceright}{\kern0pt}\ {\isasymlongrightarrow}\ {\isacharparenleft}{\kern0pt}{\isasymexists}S{\isachardot}{\kern0pt}\ is{\isacharunderscore}{\kern0pt}subspace\ S\ {\isadigit{1}}\ N{\isacharprime}{\kern0pt}\ {\isacharparenleft}{\kern0pt}t\ {\isacharplus}{\kern0pt}\ {\isadigit{1}}{\isacharparenright}{\kern0pt}\ {\isasymand}\ {\isacharparenleft}{\kern0pt}{\isasymforall}i\ {\isasymin}\ {\isacharbraceleft}{\kern0pt}{\isachardot}{\kern0pt}{\isachardot}{\kern0pt}{\isadigit{1}}{\isacharbraceright}{\kern0pt}{\isachardot}{\kern0pt}\ {\isasymexists}c\ {\isacharless}{\kern0pt}\ r{\isachardot}{\kern0pt}\ {\isacharparenleft}{\kern0pt}{\isasymforall}x\ {\isasymin}\ classes\ {\isadigit{1}}\ t\ i{\isachardot}{\kern0pt}\ {\isasymchi}\ {\isacharparenleft}{\kern0pt}S\ x{\isacharparenright}{\kern0pt}\ {\isacharequal}{\kern0pt}\ c{\isacharparenright}{\kern0pt}{\isacharparenright}{\kern0pt}{\isacharparenright}{\kern0pt}{\isachardoublequoteclose}\isanewline
\ \ \isacommand{proof}\isamarkupfalse%
{\isacharparenleft}{\kern0pt}safe{\isacharparenright}{\kern0pt}\isanewline
\ \ \ \ \isacommand{fix}\isamarkupfalse%
\ N{\isacharprime}{\kern0pt}\ {\isasymchi}\isanewline
\ \ \ \ \isacommand{assume}\isamarkupfalse%
\ asm{\isacharcolon}{\kern0pt}\ {\isachardoublequoteopen}N{\isacharprime}{\kern0pt}\ {\isasymge}\ N{\isachardoublequoteclose}\ {\isachardoublequoteopen}{\isasymchi}\ {\isasymin}\ cube\ N{\isacharprime}{\kern0pt}\ {\isacharparenleft}{\kern0pt}t\ {\isacharplus}{\kern0pt}\ {\isadigit{1}}{\isacharparenright}{\kern0pt}\ {\isasymrightarrow}\isactrlsub E\ {\isacharbraceleft}{\kern0pt}{\isachardot}{\kern0pt}{\isachardot}{\kern0pt}{\isacharless}{\kern0pt}r{\isacharcolon}{\kern0pt}{\isacharcolon}{\kern0pt}nat{\isacharbraceright}{\kern0pt}{\isachardoublequoteclose}\isanewline
\ \ \ \ \isacommand{then}\isamarkupfalse%
\ \isacommand{have}\isamarkupfalse%
\ N{\isacharprime}{\kern0pt}{\isacharunderscore}{\kern0pt}props{\isacharcolon}{\kern0pt}\ {\isachardoublequoteopen}N{\isacharprime}{\kern0pt}\ {\isachargreater}{\kern0pt}\ {\isadigit{0}}\ {\isasymand}\ {\isacharparenleft}{\kern0pt}{\isasymforall}{\isasymchi}{\isachardot}{\kern0pt}\ {\isasymchi}\ {\isasymin}\ {\isacharparenleft}{\kern0pt}cube\ N{\isacharprime}{\kern0pt}\ t{\isacharparenright}{\kern0pt}\ {\isasymrightarrow}\isactrlsub E\ {\isacharbraceleft}{\kern0pt}{\isachardot}{\kern0pt}{\isachardot}{\kern0pt}{\isacharless}{\kern0pt}r{\isacharcolon}{\kern0pt}{\isacharcolon}{\kern0pt}nat{\isacharbraceright}{\kern0pt}\ {\isasymlongrightarrow}\ {\isacharparenleft}{\kern0pt}{\isasymexists}L{\isachardot}{\kern0pt}\ {\isasymexists}c{\isacharless}{\kern0pt}r{\isachardot}{\kern0pt}\ is{\isacharunderscore}{\kern0pt}line\ L\ N{\isacharprime}{\kern0pt}\ t\ {\isasymand}\ {\isacharparenleft}{\kern0pt}{\isasymforall}y\ {\isasymin}\ L\ {\isacharbackquote}{\kern0pt}\ {\isacharbraceleft}{\kern0pt}{\isachardot}{\kern0pt}{\isachardot}{\kern0pt}{\isacharless}{\kern0pt}t{\isacharbraceright}{\kern0pt}{\isachardot}{\kern0pt}\ {\isasymchi}\ y\ {\isacharequal}{\kern0pt}\ c{\isacharparenright}{\kern0pt}{\isacharparenright}{\kern0pt}{\isacharparenright}{\kern0pt}{\isachardoublequoteclose}\ \isacommand{using}\isamarkupfalse%
\ N{\isacharunderscore}{\kern0pt}def\ \isacommand{by}\isamarkupfalse%
\ simp\isanewline
\ \ \ \ \isacommand{let}\isamarkupfalse%
\ {\isacharquery}{\kern0pt}chi{\isacharunderscore}{\kern0pt}t\ {\isacharequal}{\kern0pt}\ {\isachardoublequoteopen}{\isacharparenleft}{\kern0pt}{\isasymlambda}x\ {\isasymin}\ cube\ N{\isacharprime}{\kern0pt}\ t{\isachardot}{\kern0pt}\ {\isasymchi}\ x{\isacharparenright}{\kern0pt}{\isachardoublequoteclose}\isanewline
\ \ \ \ \isacommand{have}\isamarkupfalse%
\ {\isachardoublequoteopen}{\isacharquery}{\kern0pt}chi{\isacharunderscore}{\kern0pt}t\ {\isasymin}\ cube\ N{\isacharprime}{\kern0pt}\ t\ {\isasymrightarrow}\isactrlsub E\ {\isacharbraceleft}{\kern0pt}{\isachardot}{\kern0pt}{\isachardot}{\kern0pt}{\isacharless}{\kern0pt}r{\isacharcolon}{\kern0pt}{\isacharcolon}{\kern0pt}nat{\isacharbraceright}{\kern0pt}{\isachardoublequoteclose}\ \isacommand{using}\isamarkupfalse%
\ cube{\isacharunderscore}{\kern0pt}subset\ asm\ \isacommand{by}\isamarkupfalse%
\ auto\isanewline
\ \ \ \ \isacommand{then}\isamarkupfalse%
\ \isacommand{obtain}\isamarkupfalse%
\ L\ \isakeyword{where}\ L{\isacharunderscore}{\kern0pt}def{\isacharcolon}{\kern0pt}\ {\isachardoublequoteopen}is{\isacharunderscore}{\kern0pt}line\ L\ N{\isacharprime}{\kern0pt}\ t\ {\isasymand}\ {\isacharparenleft}{\kern0pt}{\isasymexists}c{\isacharless}{\kern0pt}r{\isachardot}{\kern0pt}\ \ {\isacharparenleft}{\kern0pt}{\isasymforall}y\ {\isasymin}\ L\ {\isacharbackquote}{\kern0pt}\ {\isacharbraceleft}{\kern0pt}{\isachardot}{\kern0pt}{\isachardot}{\kern0pt}{\isacharless}{\kern0pt}t{\isacharbraceright}{\kern0pt}{\isachardot}{\kern0pt}\ {\isacharquery}{\kern0pt}chi{\isacharunderscore}{\kern0pt}t\ y\ {\isacharequal}{\kern0pt}\ c{\isacharparenright}{\kern0pt}{\isacharparenright}{\kern0pt}{\isachardoublequoteclose}\ \isacommand{using}\isamarkupfalse%
\ N{\isacharprime}{\kern0pt}{\isacharunderscore}{\kern0pt}props\ \isacommand{by}\isamarkupfalse%
\ blast\isanewline
\isanewline
\ \ \ \ \isacommand{have}\isamarkupfalse%
\ {\isachardoublequoteopen}is{\isacharunderscore}{\kern0pt}subspace\ {\isacharparenleft}{\kern0pt}restrict\ {\isacharparenleft}{\kern0pt}{\isasymlambda}y{\isachardot}{\kern0pt}\ L\ {\isacharparenleft}{\kern0pt}y\ {\isadigit{0}}{\isacharparenright}{\kern0pt}{\isacharparenright}{\kern0pt}\ {\isacharparenleft}{\kern0pt}cube\ {\isadigit{1}}\ t{\isacharparenright}{\kern0pt}{\isacharparenright}{\kern0pt}\ {\isadigit{1}}\ N{\isacharprime}{\kern0pt}\ t{\isachardoublequoteclose}\ \isacommand{using}\isamarkupfalse%
\ line{\isacharunderscore}{\kern0pt}is{\isacharunderscore}{\kern0pt}dim{\isadigit{1}}{\isacharunderscore}{\kern0pt}subspace\ N{\isacharprime}{\kern0pt}{\isacharunderscore}{\kern0pt}props\ L{\isacharunderscore}{\kern0pt}def\ \isanewline
\ \ \ \ \ \ \isacommand{using}\isamarkupfalse%
\ assms{\isacharparenleft}{\kern0pt}{\isadigit{1}}{\isacharparenright}{\kern0pt}\ \isacommand{by}\isamarkupfalse%
\ auto\ \isanewline
\ \ \ \ \isacommand{then}\isamarkupfalse%
\ \isacommand{obtain}\isamarkupfalse%
\ B\ f\ \isakeyword{where}\ Bf{\isacharunderscore}{\kern0pt}defs{\isacharcolon}{\kern0pt}\ {\isachardoublequoteopen}disjoint{\isacharunderscore}{\kern0pt}family{\isacharunderscore}{\kern0pt}on\ B\ {\isacharbraceleft}{\kern0pt}{\isachardot}{\kern0pt}{\isachardot}{\kern0pt}{\isadigit{1}}{\isacharbraceright}{\kern0pt}\ {\isasymand}\ {\isasymUnion}{\isacharparenleft}{\kern0pt}B\ {\isacharbackquote}{\kern0pt}\ {\isacharbraceleft}{\kern0pt}{\isachardot}{\kern0pt}{\isachardot}{\kern0pt}{\isadigit{1}}{\isacharbraceright}{\kern0pt}{\isacharparenright}{\kern0pt}\ {\isacharequal}{\kern0pt}\ {\isacharbraceleft}{\kern0pt}{\isachardot}{\kern0pt}{\isachardot}{\kern0pt}{\isacharless}{\kern0pt}N{\isacharprime}{\kern0pt}{\isacharbraceright}{\kern0pt}\ {\isasymand}\ {\isacharparenleft}{\kern0pt}{\isacharbraceleft}{\kern0pt}{\isacharbraceright}{\kern0pt}\ {\isasymnotin}\ B\ {\isacharbackquote}{\kern0pt}\ {\isacharbraceleft}{\kern0pt}{\isachardot}{\kern0pt}{\isachardot}{\kern0pt}{\isacharless}{\kern0pt}{\isadigit{1}}{\isacharbraceright}{\kern0pt}{\isacharparenright}{\kern0pt}\ {\isasymand}\ f\ {\isasymin}\ {\isacharparenleft}{\kern0pt}B\ {\isadigit{1}}{\isacharparenright}{\kern0pt}\ {\isasymrightarrow}\isactrlsub E\ {\isacharbraceleft}{\kern0pt}{\isachardot}{\kern0pt}{\isachardot}{\kern0pt}{\isacharless}{\kern0pt}t{\isacharbraceright}{\kern0pt}\ {\isasymand}\ {\isacharparenleft}{\kern0pt}restrict\ {\isacharparenleft}{\kern0pt}{\isasymlambda}y{\isachardot}{\kern0pt}\ L\ {\isacharparenleft}{\kern0pt}y\ {\isadigit{0}}{\isacharparenright}{\kern0pt}{\isacharparenright}{\kern0pt}\ {\isacharparenleft}{\kern0pt}cube\ {\isadigit{1}}\ t{\isacharparenright}{\kern0pt}{\isacharparenright}{\kern0pt}\ {\isasymin}\ {\isacharparenleft}{\kern0pt}cube\ {\isadigit{1}}\ t{\isacharparenright}{\kern0pt}\ {\isasymrightarrow}\isactrlsub E\ {\isacharparenleft}{\kern0pt}cube\ N{\isacharprime}{\kern0pt}\ t{\isacharparenright}{\kern0pt}\ {\isasymand}\ {\isacharparenleft}{\kern0pt}{\isasymforall}y\ {\isasymin}\ cube\ {\isadigit{1}}\ t{\isachardot}{\kern0pt}\ {\isacharparenleft}{\kern0pt}{\isasymforall}i\ {\isasymin}\ B\ {\isadigit{1}}{\isachardot}{\kern0pt}\ {\isacharparenleft}{\kern0pt}restrict\ {\isacharparenleft}{\kern0pt}{\isasymlambda}y{\isachardot}{\kern0pt}\ L\ {\isacharparenleft}{\kern0pt}y\ {\isadigit{0}}{\isacharparenright}{\kern0pt}{\isacharparenright}{\kern0pt}\ {\isacharparenleft}{\kern0pt}cube\ {\isadigit{1}}\ t{\isacharparenright}{\kern0pt}{\isacharparenright}{\kern0pt}\ y\ i\ {\isacharequal}{\kern0pt}\ f\ i{\isacharparenright}{\kern0pt}\ {\isasymand}\ {\isacharparenleft}{\kern0pt}{\isasymforall}j{\isacharless}{\kern0pt}{\isadigit{1}}{\isachardot}{\kern0pt}\ {\isasymforall}i\ {\isasymin}\ B\ j{\isachardot}{\kern0pt}\ {\isacharparenleft}{\kern0pt}{\isacharparenleft}{\kern0pt}restrict\ {\isacharparenleft}{\kern0pt}{\isasymlambda}y{\isachardot}{\kern0pt}\ L\ {\isacharparenleft}{\kern0pt}y\ {\isadigit{0}}{\isacharparenright}{\kern0pt}{\isacharparenright}{\kern0pt}\ {\isacharparenleft}{\kern0pt}cube\ {\isadigit{1}}\ t{\isacharparenright}{\kern0pt}{\isacharparenright}{\kern0pt}\ y{\isacharparenright}{\kern0pt}\ i\ {\isacharequal}{\kern0pt}\ y\ j{\isacharparenright}{\kern0pt}{\isacharparenright}{\kern0pt}{\isachardoublequoteclose}\ \isacommand{unfolding}\isamarkupfalse%
\ is{\isacharunderscore}{\kern0pt}subspace{\isacharunderscore}{\kern0pt}def\ \isacommand{by}\isamarkupfalse%
\ auto\ \isanewline
\isanewline
\ \ \ \ \isacommand{have}\isamarkupfalse%
\ {\isachardoublequoteopen}{\isacharbraceleft}{\kern0pt}{\isachardot}{\kern0pt}{\isachardot}{\kern0pt}{\isadigit{1}}{\isacharcolon}{\kern0pt}{\isacharcolon}{\kern0pt}nat{\isacharbraceright}{\kern0pt}\ {\isacharequal}{\kern0pt}\ {\isacharbraceleft}{\kern0pt}{\isadigit{0}}{\isacharcomma}{\kern0pt}\ {\isadigit{1}}{\isacharbraceright}{\kern0pt}{\isachardoublequoteclose}\ \isacommand{by}\isamarkupfalse%
\ auto\isanewline
\ \ \ \ \isacommand{then}\isamarkupfalse%
\ \isacommand{have}\isamarkupfalse%
\ B{\isacharunderscore}{\kern0pt}props{\isacharcolon}{\kern0pt}\ {\isachardoublequoteopen}B\ {\isadigit{0}}\ {\isasymunion}\ B\ {\isadigit{1}}\ {\isacharequal}{\kern0pt}\ {\isacharbraceleft}{\kern0pt}{\isachardot}{\kern0pt}{\isachardot}{\kern0pt}{\isacharless}{\kern0pt}N{\isacharprime}{\kern0pt}{\isacharbraceright}{\kern0pt}\ {\isasymand}\ {\isacharparenleft}{\kern0pt}B\ {\isadigit{0}}\ {\isasyminter}\ B\ {\isadigit{1}}\ {\isacharequal}{\kern0pt}\ {\isacharbraceleft}{\kern0pt}{\isacharbraceright}{\kern0pt}{\isacharparenright}{\kern0pt}{\isachardoublequoteclose}\ \isacommand{using}\isamarkupfalse%
\ Bf{\isacharunderscore}{\kern0pt}defs\ \isacommand{unfolding}\isamarkupfalse%
\ disjoint{\isacharunderscore}{\kern0pt}family{\isacharunderscore}{\kern0pt}on{\isacharunderscore}{\kern0pt}def\ \isacommand{by}\isamarkupfalse%
\ auto\isanewline
\ \ \ \ \isacommand{define}\isamarkupfalse%
\ L{\isacharprime}{\kern0pt}\ \isakeyword{where}\ \ {\isachardoublequoteopen}L{\isacharprime}{\kern0pt}\ {\isasymequiv}\ L{\isacharparenleft}{\kern0pt}t{\isacharcolon}{\kern0pt}{\isacharequal}{\kern0pt}{\isacharparenleft}{\kern0pt}{\isasymlambda}j{\isachardot}{\kern0pt}\ if\ j\ {\isasymin}\ B\ {\isadigit{1}}\ then\ L\ {\isacharparenleft}{\kern0pt}t\ {\isacharminus}{\kern0pt}\ {\isadigit{1}}{\isacharparenright}{\kern0pt}\ j\ else\ {\isacharparenleft}{\kern0pt}if\ j\ {\isasymin}\ B\ {\isadigit{0}}\ then\ t\ else\ undefined{\isacharparenright}{\kern0pt}{\isacharparenright}{\kern0pt}{\isacharparenright}{\kern0pt}{\isachardoublequoteclose}\ \isanewline
\ \ \ \ \isacommand{have}\isamarkupfalse%
\ line{\isacharunderscore}{\kern0pt}prop{\isacharcolon}{\kern0pt}\ {\isachardoublequoteopen}is{\isacharunderscore}{\kern0pt}line\ L{\isacharprime}{\kern0pt}\ N{\isacharprime}{\kern0pt}\ {\isacharparenleft}{\kern0pt}t\ {\isacharplus}{\kern0pt}\ {\isadigit{1}}{\isacharparenright}{\kern0pt}{\isachardoublequoteclose}\isanewline
\ \ \ \ \isacommand{proof}\isamarkupfalse%
{\isacharminus}{\kern0pt}\isanewline
\ \ \ \ \ \ \isacommand{have}\isamarkupfalse%
\ A{\isadigit{1}}{\isacharcolon}{\kern0pt}\ {\isachardoublequoteopen}L{\isacharprime}{\kern0pt}\ {\isasymin}\ {\isacharbraceleft}{\kern0pt}{\isachardot}{\kern0pt}{\isachardot}{\kern0pt}{\isacharless}{\kern0pt}t{\isacharplus}{\kern0pt}{\isadigit{1}}{\isacharbraceright}{\kern0pt}\ {\isasymrightarrow}\isactrlsub E\ cube\ N{\isacharprime}{\kern0pt}\ {\isacharparenleft}{\kern0pt}t\ {\isacharplus}{\kern0pt}\ {\isadigit{1}}{\isacharparenright}{\kern0pt}{\isachardoublequoteclose}\ \isanewline
\ \ \ \ \ \ \isacommand{proof}\isamarkupfalse%
\isanewline
\ \ \ \ \ \ \ \ \isacommand{fix}\isamarkupfalse%
\ x\isanewline
\ \ \ \ \ \ \ \ \isacommand{assume}\isamarkupfalse%
\ asm{\isacharcolon}{\kern0pt}\ {\isachardoublequoteopen}x\ {\isasymin}\ {\isacharbraceleft}{\kern0pt}{\isachardot}{\kern0pt}{\isachardot}{\kern0pt}{\isacharless}{\kern0pt}t\ {\isacharplus}{\kern0pt}\ {\isadigit{1}}{\isacharbraceright}{\kern0pt}{\isachardoublequoteclose}\isanewline
\ \ \ \ \ \ \ \ \isacommand{then}\isamarkupfalse%
\ \isacommand{show}\isamarkupfalse%
\ {\isachardoublequoteopen}L{\isacharprime}{\kern0pt}\ x\ {\isasymin}\ cube\ N{\isacharprime}{\kern0pt}\ {\isacharparenleft}{\kern0pt}t\ {\isacharplus}{\kern0pt}\ {\isadigit{1}}{\isacharparenright}{\kern0pt}{\isachardoublequoteclose}\isanewline
\ \ \ \ \ \ \ \ \isacommand{proof}\isamarkupfalse%
\ {\isacharparenleft}{\kern0pt}cases\ {\isachardoublequoteopen}x\ {\isacharless}{\kern0pt}\ t{\isachardoublequoteclose}{\isacharparenright}{\kern0pt}\isanewline
\ \ \ \ \ \ \ \ \ \ \isacommand{case}\isamarkupfalse%
\ True\isanewline
\ \ \ \ \ \ \ \ \ \ \isacommand{then}\isamarkupfalse%
\ \isacommand{have}\isamarkupfalse%
\ {\isachardoublequoteopen}L{\isacharprime}{\kern0pt}\ x\ {\isacharequal}{\kern0pt}\ L\ x{\isachardoublequoteclose}\ \isacommand{by}\isamarkupfalse%
\ {\isacharparenleft}{\kern0pt}simp\ add{\isacharcolon}{\kern0pt}\ L{\isacharprime}{\kern0pt}{\isacharunderscore}{\kern0pt}def{\isacharparenright}{\kern0pt}\isanewline
\ \ \ \ \ \ \ \ \ \ \isacommand{then}\isamarkupfalse%
\ \isacommand{have}\isamarkupfalse%
\ {\isachardoublequoteopen}L{\isacharprime}{\kern0pt}\ x\ {\isasymin}\ cube\ N{\isacharprime}{\kern0pt}\ t{\isachardoublequoteclose}\ \isacommand{using}\isamarkupfalse%
\ L{\isacharunderscore}{\kern0pt}def\ True\ \isacommand{unfolding}\isamarkupfalse%
\ is{\isacharunderscore}{\kern0pt}line{\isacharunderscore}{\kern0pt}def\ \isacommand{by}\isamarkupfalse%
\ auto\isanewline
\ \ \ \ \ \ \ \ \ \ \isacommand{then}\isamarkupfalse%
\ \isacommand{show}\isamarkupfalse%
\ {\isachardoublequoteopen}L{\isacharprime}{\kern0pt}\ x\ {\isasymin}\ cube\ N{\isacharprime}{\kern0pt}\ {\isacharparenleft}{\kern0pt}t\ {\isacharplus}{\kern0pt}\ {\isadigit{1}}{\isacharparenright}{\kern0pt}{\isachardoublequoteclose}\ \isacommand{using}\isamarkupfalse%
\ cube{\isacharunderscore}{\kern0pt}subset\ \isacommand{by}\isamarkupfalse%
\ blast\isanewline
\ \ \ \ \ \ \ \ \isacommand{next}\isamarkupfalse%
\isanewline
\ \ \ \ \ \ \ \ \ \ \isacommand{case}\isamarkupfalse%
\ False\isanewline
\ \ \ \ \ \ \ \ \ \ \isacommand{then}\isamarkupfalse%
\ \isacommand{have}\isamarkupfalse%
\ {\isachardoublequoteopen}x\ {\isacharequal}{\kern0pt}\ t{\isachardoublequoteclose}\ \isacommand{using}\isamarkupfalse%
\ asm\ \isacommand{by}\isamarkupfalse%
\ simp\isanewline
\ \ \ \ \ \ \ \ \ \ \isacommand{show}\isamarkupfalse%
\ {\isachardoublequoteopen}L{\isacharprime}{\kern0pt}\ x\ {\isasymin}\ cube\ N{\isacharprime}{\kern0pt}\ {\isacharparenleft}{\kern0pt}t\ {\isacharplus}{\kern0pt}\ {\isadigit{1}}{\isacharparenright}{\kern0pt}{\isachardoublequoteclose}\isanewline
\ \ \ \ \ \ \ \ \ \ \isacommand{proof}\isamarkupfalse%
{\isacharparenleft}{\kern0pt}unfold\ cube{\isacharunderscore}{\kern0pt}def{\isacharcomma}{\kern0pt}\ intro\ PiE{\isacharunderscore}{\kern0pt}I{\isacharparenright}{\kern0pt}\isanewline
\ \ \ \ \ \ \ \ \ \ \ \ \isacommand{fix}\isamarkupfalse%
\ j\isanewline
\ \ \ \ \ \ \ \ \ \ \ \ \isacommand{assume}\isamarkupfalse%
\ {\isachardoublequoteopen}j\ {\isasymin}\ {\isacharbraceleft}{\kern0pt}{\isachardot}{\kern0pt}{\isachardot}{\kern0pt}{\isacharless}{\kern0pt}N{\isacharprime}{\kern0pt}{\isacharbraceright}{\kern0pt}{\isachardoublequoteclose}\isanewline
\ \ \ \ \ \ \ \ \ \ \ \ \isacommand{have}\isamarkupfalse%
\ {\isachardoublequoteopen}j\ {\isasymin}\ B\ {\isadigit{1}}\ {\isasymor}\ j\ {\isasymin}\ B\ {\isadigit{0}}\ {\isasymor}\ j\ {\isasymnotin}\ {\isacharparenleft}{\kern0pt}B\ {\isadigit{0}}\ {\isasymunion}\ B\ {\isadigit{1}}{\isacharparenright}{\kern0pt}{\isachardoublequoteclose}\ \isacommand{by}\isamarkupfalse%
\ blast\isanewline
\ \ \ \ \ \ \ \ \ \ \ \ \isacommand{then}\isamarkupfalse%
\ \isacommand{show}\isamarkupfalse%
\ {\isachardoublequoteopen}L{\isacharprime}{\kern0pt}\ x\ j\ {\isasymin}\ {\isacharbraceleft}{\kern0pt}{\isachardot}{\kern0pt}{\isachardot}{\kern0pt}{\isacharless}{\kern0pt}t\ {\isacharplus}{\kern0pt}\ {\isadigit{1}}{\isacharbraceright}{\kern0pt}{\isachardoublequoteclose}\isanewline
\ \ \ \ \ \ \ \ \ \ \ \ \isacommand{proof}\isamarkupfalse%
\ {\isacharparenleft}{\kern0pt}elim\ disjE{\isacharparenright}{\kern0pt}\isanewline
\ \ \ \ \ \ \ \ \ \ \ \ \ \ \isacommand{assume}\isamarkupfalse%
\ {\isachardoublequoteopen}j\ {\isasymin}\ B\ {\isadigit{1}}{\isachardoublequoteclose}\isanewline
\ \ \ \ \ \ \ \ \ \ \ \ \ \ \isacommand{then}\isamarkupfalse%
\ \isacommand{have}\isamarkupfalse%
\ {\isachardoublequoteopen}L{\isacharprime}{\kern0pt}\ x\ j\ {\isacharequal}{\kern0pt}\ L\ {\isacharparenleft}{\kern0pt}t\ {\isacharminus}{\kern0pt}\ {\isadigit{1}}{\isacharparenright}{\kern0pt}\ j{\isachardoublequoteclose}\ \isanewline
\ \ \ \ \ \ \ \ \ \ \ \ \ \ \ \ \isacommand{by}\isamarkupfalse%
\ {\isacharparenleft}{\kern0pt}simp\ add{\isacharcolon}{\kern0pt}\ {\isacartoucheopen}x\ {\isacharequal}{\kern0pt}\ t{\isacartoucheclose}\ L{\isacharprime}{\kern0pt}{\isacharunderscore}{\kern0pt}def{\isacharparenright}{\kern0pt}\isanewline
\ \ \ \ \ \ \ \ \ \ \ \ \ \ \isacommand{have}\isamarkupfalse%
\ {\isachardoublequoteopen}L\ {\isacharparenleft}{\kern0pt}t\ {\isacharminus}{\kern0pt}\ {\isadigit{1}}{\isacharparenright}{\kern0pt}\ {\isasymin}\ cube\ N{\isacharprime}{\kern0pt}\ t{\isachardoublequoteclose}\ \isacommand{using}\isamarkupfalse%
\ line{\isacharunderscore}{\kern0pt}points{\isacharunderscore}{\kern0pt}in{\isacharunderscore}{\kern0pt}cube\ L{\isacharunderscore}{\kern0pt}def\ \isanewline
\ \ \ \ \ \ \ \ \ \ \ \ \ \ \ \ \isacommand{by}\isamarkupfalse%
\ {\isacharparenleft}{\kern0pt}meson\ assms{\isacharparenleft}{\kern0pt}{\isadigit{1}}{\isacharparenright}{\kern0pt}\ diff{\isacharunderscore}{\kern0pt}less\ less{\isacharunderscore}{\kern0pt}numeral{\isacharunderscore}{\kern0pt}extra{\isacharparenleft}{\kern0pt}{\isadigit{1}}{\isacharparenright}{\kern0pt}{\isacharparenright}{\kern0pt}\isanewline
\ \ \ \ \ \ \ \ \ \ \ \ \ \ \isacommand{then}\isamarkupfalse%
\ \isacommand{have}\isamarkupfalse%
\ {\isachardoublequoteopen}L\ {\isacharparenleft}{\kern0pt}t\ {\isacharminus}{\kern0pt}\ {\isadigit{1}}{\isacharparenright}{\kern0pt}\ j\ {\isacharless}{\kern0pt}\ t{\isachardoublequoteclose}\ \isacommand{using}\isamarkupfalse%
\ {\isacartoucheopen}j\ {\isasymin}\ {\isacharbraceleft}{\kern0pt}{\isachardot}{\kern0pt}{\isachardot}{\kern0pt}{\isacharless}{\kern0pt}N{\isacharprime}{\kern0pt}{\isacharbraceright}{\kern0pt}{\isacartoucheclose}\ \isacommand{unfolding}\isamarkupfalse%
\ cube{\isacharunderscore}{\kern0pt}def\ \isacommand{by}\isamarkupfalse%
\ auto\ \isanewline
\ \ \ \ \ \ \ \ \ \ \ \ \ \ \isacommand{then}\isamarkupfalse%
\ \isacommand{show}\isamarkupfalse%
\ {\isachardoublequoteopen}L{\isacharprime}{\kern0pt}\ x\ j\ {\isasymin}\ {\isacharbraceleft}{\kern0pt}{\isachardot}{\kern0pt}{\isachardot}{\kern0pt}{\isacharless}{\kern0pt}t\ {\isacharplus}{\kern0pt}\ {\isadigit{1}}{\isacharbraceright}{\kern0pt}{\isachardoublequoteclose}\ \isacommand{using}\isamarkupfalse%
\ {\isacartoucheopen}L{\isacharprime}{\kern0pt}\ x\ j\ {\isacharequal}{\kern0pt}\ L\ {\isacharparenleft}{\kern0pt}t\ {\isacharminus}{\kern0pt}\ {\isadigit{1}}{\isacharparenright}{\kern0pt}\ j{\isacartoucheclose}\ \isacommand{by}\isamarkupfalse%
\ simp\isanewline
\ \ \ \ \ \ \ \ \ \ \ \ \isacommand{next}\isamarkupfalse%
\isanewline
\ \ \ \ \ \ \ \ \ \ \ \ \ \ \isacommand{assume}\isamarkupfalse%
\ {\isachardoublequoteopen}j\ {\isasymin}\ B\ {\isadigit{0}}{\isachardoublequoteclose}\isanewline
\ \ \ \ \ \ \ \ \ \ \ \ \ \ \isacommand{then}\isamarkupfalse%
\ \isacommand{have}\isamarkupfalse%
\ {\isachardoublequoteopen}j\ {\isasymnotin}\ B\ {\isadigit{1}}{\isachardoublequoteclose}\ \isacommand{using}\isamarkupfalse%
\ Bf{\isacharunderscore}{\kern0pt}defs\ \isacommand{unfolding}\isamarkupfalse%
\ disjoint{\isacharunderscore}{\kern0pt}family{\isacharunderscore}{\kern0pt}on{\isacharunderscore}{\kern0pt}def\ \isacommand{by}\isamarkupfalse%
\ auto\isanewline
\ \ \ \ \ \ \ \ \ \ \ \ \ \ \isacommand{then}\isamarkupfalse%
\ \isacommand{have}\isamarkupfalse%
\ {\isachardoublequoteopen}L{\isacharprime}{\kern0pt}\ x\ j\ {\isacharequal}{\kern0pt}\ t{\isachardoublequoteclose}\ \ \isacommand{by}\isamarkupfalse%
\ {\isacharparenleft}{\kern0pt}simp\ add{\isacharcolon}{\kern0pt}\ {\isacartoucheopen}j\ {\isasymin}\ B\ {\isadigit{0}}{\isacartoucheclose}\ {\isacartoucheopen}x\ {\isacharequal}{\kern0pt}\ t{\isacartoucheclose}\ L{\isacharprime}{\kern0pt}{\isacharunderscore}{\kern0pt}def{\isacharparenright}{\kern0pt}\isanewline
\ \ \ \ \ \ \ \ \ \ \ \ \ \ \isacommand{then}\isamarkupfalse%
\ \isacommand{show}\isamarkupfalse%
\ {\isachardoublequoteopen}L{\isacharprime}{\kern0pt}\ x\ j\ {\isasymin}\ {\isacharbraceleft}{\kern0pt}{\isachardot}{\kern0pt}{\isachardot}{\kern0pt}{\isacharless}{\kern0pt}t\ {\isacharplus}{\kern0pt}\ {\isadigit{1}}{\isacharbraceright}{\kern0pt}{\isachardoublequoteclose}\ \isacommand{by}\isamarkupfalse%
\ simp\isanewline
\ \ \ \ \ \ \ \ \ \ \ \ \isacommand{next}\isamarkupfalse%
\isanewline
\ \ \ \ \ \ \ \ \ \ \ \ \ \ \isacommand{assume}\isamarkupfalse%
\ a{\isacharcolon}{\kern0pt}\ {\isachardoublequoteopen}j\ {\isasymnotin}\ {\isacharparenleft}{\kern0pt}B\ {\isadigit{0}}\ {\isasymunion}\ B\ {\isadigit{1}}{\isacharparenright}{\kern0pt}{\isachardoublequoteclose}\isanewline
\ \ \ \ \ \ \ \ \ \ \ \ \ \ \isacommand{have}\isamarkupfalse%
\ {\isachardoublequoteopen}{\isacharbraceleft}{\kern0pt}{\isachardot}{\kern0pt}{\isachardot}{\kern0pt}{\isadigit{1}}{\isacharcolon}{\kern0pt}{\isacharcolon}{\kern0pt}nat{\isacharbraceright}{\kern0pt}\ {\isacharequal}{\kern0pt}\ {\isacharbraceleft}{\kern0pt}{\isadigit{0}}{\isacharcomma}{\kern0pt}\ {\isadigit{1}}{\isacharbraceright}{\kern0pt}{\isachardoublequoteclose}\ \isacommand{by}\isamarkupfalse%
\ auto\isanewline
\ \ \ \ \ \ \ \ \ \ \ \ \ \ \isacommand{then}\isamarkupfalse%
\ \isacommand{have}\isamarkupfalse%
\ {\isachardoublequoteopen}B\ {\isadigit{0}}\ {\isasymunion}\ B\ {\isadigit{1}}\ {\isacharequal}{\kern0pt}\ {\isacharparenleft}{\kern0pt}{\isasymUnion}{\isacharparenleft}{\kern0pt}B\ {\isacharbackquote}{\kern0pt}\ {\isacharbraceleft}{\kern0pt}{\isachardot}{\kern0pt}{\isachardot}{\kern0pt}{\isadigit{1}}{\isacharcolon}{\kern0pt}{\isacharcolon}{\kern0pt}nat{\isacharbraceright}{\kern0pt}{\isacharparenright}{\kern0pt}{\isacharparenright}{\kern0pt}{\isachardoublequoteclose}\ \isacommand{by}\isamarkupfalse%
\ simp\isanewline
\ \ \ \ \ \ \ \ \ \ \ \ \ \ \isacommand{then}\isamarkupfalse%
\ \isacommand{have}\isamarkupfalse%
\ {\isachardoublequoteopen}B\ {\isadigit{0}}\ {\isasymunion}\ B\ {\isadigit{1}}\ {\isacharequal}{\kern0pt}\ {\isacharbraceleft}{\kern0pt}{\isachardot}{\kern0pt}{\isachardot}{\kern0pt}{\isacharless}{\kern0pt}N{\isacharprime}{\kern0pt}{\isacharbraceright}{\kern0pt}{\isachardoublequoteclose}\ \isacommand{using}\isamarkupfalse%
\ Bf{\isacharunderscore}{\kern0pt}defs\ \isacommand{unfolding}\isamarkupfalse%
\ partition{\isacharunderscore}{\kern0pt}on{\isacharunderscore}{\kern0pt}def\ \isacommand{by}\isamarkupfalse%
\ simp\isanewline
\ \ \ \ \ \ \ \ \ \ \ \ \ \ \isacommand{then}\isamarkupfalse%
\ \isacommand{have}\isamarkupfalse%
\ {\isachardoublequoteopen}{\isasymnot}{\isacharparenleft}{\kern0pt}j\ {\isasymin}\ {\isacharbraceleft}{\kern0pt}{\isachardot}{\kern0pt}{\isachardot}{\kern0pt}{\isacharless}{\kern0pt}N{\isacharprime}{\kern0pt}{\isacharbraceright}{\kern0pt}{\isacharparenright}{\kern0pt}{\isachardoublequoteclose}\ \isacommand{using}\isamarkupfalse%
\ a\ \isacommand{by}\isamarkupfalse%
\ simp\isanewline
\ \ \ \ \ \ \ \ \ \ \ \ \ \ \isacommand{then}\isamarkupfalse%
\ \isacommand{have}\isamarkupfalse%
\ False\ \isacommand{using}\isamarkupfalse%
\ {\isacartoucheopen}j\ {\isasymin}\ {\isacharbraceleft}{\kern0pt}{\isachardot}{\kern0pt}{\isachardot}{\kern0pt}{\isacharless}{\kern0pt}N{\isacharprime}{\kern0pt}{\isacharbraceright}{\kern0pt}{\isacartoucheclose}\ \isacommand{by}\isamarkupfalse%
\ simp\isanewline
\ \ \ \ \ \ \ \ \ \ \ \ \ \ \isacommand{then}\isamarkupfalse%
\ \isacommand{show}\isamarkupfalse%
\ {\isacharquery}{\kern0pt}thesis\ \isacommand{by}\isamarkupfalse%
\ simp\isanewline
\ \ \ \ \ \ \ \ \ \ \ \ \isacommand{qed}\isamarkupfalse%
\isanewline
\ \ \ \ \ \ \ \ \ \ \isacommand{next}\isamarkupfalse%
\isanewline
\ \ \ \ \ \ \ \ \ \ \ \ \isacommand{fix}\isamarkupfalse%
\ j\ \isanewline
\ \ \ \ \ \ \ \ \ \ \ \ \isacommand{assume}\isamarkupfalse%
\ {\isachardoublequoteopen}j\ {\isasymnotin}\ {\isacharbraceleft}{\kern0pt}{\isachardot}{\kern0pt}{\isachardot}{\kern0pt}{\isacharless}{\kern0pt}N{\isacharprime}{\kern0pt}{\isacharbraceright}{\kern0pt}{\isachardoublequoteclose}\isanewline
\ \ \ \ \ \ \ \ \ \ \ \ \isacommand{then}\isamarkupfalse%
\ \isacommand{have}\isamarkupfalse%
\ {\isachardoublequoteopen}j\ {\isasymnotin}\ {\isacharparenleft}{\kern0pt}B\ {\isadigit{0}}{\isacharparenright}{\kern0pt}\ {\isasymand}\ j\ {\isasymnotin}\ B\ {\isadigit{1}}{\isachardoublequoteclose}\ \isacommand{using}\isamarkupfalse%
\ Bf{\isacharunderscore}{\kern0pt}defs\ \isacommand{unfolding}\isamarkupfalse%
\ partition{\isacharunderscore}{\kern0pt}on{\isacharunderscore}{\kern0pt}def\ \isacommand{by}\isamarkupfalse%
\ auto\isanewline
\ \ \ \ \ \ \ \ \ \ \ \ \isacommand{then}\isamarkupfalse%
\ \isacommand{show}\isamarkupfalse%
\ {\isachardoublequoteopen}L{\isacharprime}{\kern0pt}\ x\ j\ {\isacharequal}{\kern0pt}\ undefined{\isachardoublequoteclose}\ \isacommand{using}\isamarkupfalse%
\ {\isacartoucheopen}x\ {\isacharequal}{\kern0pt}\ t{\isacartoucheclose}\ \isacommand{by}\isamarkupfalse%
\ {\isacharparenleft}{\kern0pt}simp\ add{\isacharcolon}{\kern0pt}\ L{\isacharprime}{\kern0pt}{\isacharunderscore}{\kern0pt}def{\isacharparenright}{\kern0pt}\isanewline
\ \ \ \ \ \ \ \ \ \ \isacommand{qed}\isamarkupfalse%
\isanewline
\ \ \ \ \ \ \ \ \isacommand{qed}\isamarkupfalse%
\isanewline
\ \ \ \ \ \ \isacommand{next}\isamarkupfalse%
\isanewline
\ \ \ \ \ \ \ \ \isacommand{fix}\isamarkupfalse%
\ x\isanewline
\ \ \ \ \ \ \ \ \isacommand{assume}\isamarkupfalse%
\ asm{\isacharcolon}{\kern0pt}\ {\isachardoublequoteopen}x\ {\isasymnotin}\ {\isacharbraceleft}{\kern0pt}{\isachardot}{\kern0pt}{\isachardot}{\kern0pt}{\isacharless}{\kern0pt}t{\isacharplus}{\kern0pt}{\isadigit{1}}{\isacharbraceright}{\kern0pt}{\isachardoublequoteclose}\ \isanewline
\ \ \ \ \ \ \ \ \isacommand{then}\isamarkupfalse%
\ \isacommand{have}\isamarkupfalse%
\ {\isachardoublequoteopen}x\ {\isasymnotin}\ {\isacharbraceleft}{\kern0pt}{\isachardot}{\kern0pt}{\isachardot}{\kern0pt}{\isacharless}{\kern0pt}t{\isacharbraceright}{\kern0pt}\ {\isasymand}\ x\ {\isasymnoteq}\ t{\isachardoublequoteclose}\ \isacommand{by}\isamarkupfalse%
\ simp\isanewline
\ \ \ \ \ \ \ \ \isacommand{then}\isamarkupfalse%
\ \isacommand{show}\isamarkupfalse%
\ {\isachardoublequoteopen}L{\isacharprime}{\kern0pt}\ x\ {\isacharequal}{\kern0pt}\ undefined{\isachardoublequoteclose}\ \isacommand{using}\isamarkupfalse%
\ L{\isacharunderscore}{\kern0pt}def\ \isacommand{unfolding}\isamarkupfalse%
\ L{\isacharprime}{\kern0pt}{\isacharunderscore}{\kern0pt}def\ is{\isacharunderscore}{\kern0pt}line{\isacharunderscore}{\kern0pt}def\ \isacommand{by}\isamarkupfalse%
\ auto\isanewline
\ \ \ \ \ \ \isacommand{qed}\isamarkupfalse%
\isanewline
\isanewline
\isanewline
\ \ \ \ \ \ \isacommand{have}\isamarkupfalse%
\ A{\isadigit{2}}{\isacharcolon}{\kern0pt}\ {\isachardoublequoteopen}{\isacharparenleft}{\kern0pt}{\isasymexists}j{\isacharless}{\kern0pt}N{\isacharprime}{\kern0pt}{\isachardot}{\kern0pt}\ {\isacharparenleft}{\kern0pt}{\isasymforall}s\ {\isacharless}{\kern0pt}\ {\isacharparenleft}{\kern0pt}t\ {\isacharplus}{\kern0pt}\ {\isadigit{1}}{\isacharparenright}{\kern0pt}{\isachardot}{\kern0pt}\ L{\isacharprime}{\kern0pt}\ s\ j\ {\isacharequal}{\kern0pt}\ s{\isacharparenright}{\kern0pt}{\isacharparenright}{\kern0pt}{\isachardoublequoteclose}\isanewline
\ \ \ \ \ \ \isacommand{proof}\isamarkupfalse%
\ {\isacharparenleft}{\kern0pt}cases\ {\isachardoublequoteopen}t\ {\isacharequal}{\kern0pt}\ {\isadigit{1}}{\isachardoublequoteclose}{\isacharparenright}{\kern0pt}\isanewline
\ \ \ \ \ \ \ \ \isacommand{case}\isamarkupfalse%
\ True\isanewline
\ \ \ \ \ \ \ \ \isacommand{obtain}\isamarkupfalse%
\ j\ \isakeyword{where}\ j{\isacharunderscore}{\kern0pt}prop{\isacharcolon}{\kern0pt}\ {\isachardoublequoteopen}j\ {\isasymin}\ B\ {\isadigit{0}}\ {\isasymand}\ j\ {\isacharless}{\kern0pt}\ N{\isacharprime}{\kern0pt}{\isachardoublequoteclose}\ \isacommand{using}\isamarkupfalse%
\ Bf{\isacharunderscore}{\kern0pt}defs\ \isacommand{by}\isamarkupfalse%
\ blast\isanewline
\ \ \ \ \ \ \ \ \isacommand{then}\isamarkupfalse%
\ \isacommand{have}\isamarkupfalse%
\ {\isachardoublequoteopen}L{\isacharprime}{\kern0pt}\ s\ j\ {\isacharequal}{\kern0pt}\ L\ s\ j{\isachardoublequoteclose}\ \isakeyword{if}\ {\isachardoublequoteopen}s\ {\isacharless}{\kern0pt}\ t{\isachardoublequoteclose}\ \isakeyword{for}\ s\ \isacommand{using}\isamarkupfalse%
\ that\ \isacommand{by}\isamarkupfalse%
\ {\isacharparenleft}{\kern0pt}auto\ simp{\isacharcolon}{\kern0pt}\ L{\isacharprime}{\kern0pt}{\isacharunderscore}{\kern0pt}def{\isacharparenright}{\kern0pt}\isanewline
\ \ \ \ \ \ \ \ \isacommand{moreover}\isamarkupfalse%
\ \isacommand{have}\isamarkupfalse%
\ {\isachardoublequoteopen}L\ s\ j\ {\isacharequal}{\kern0pt}\ {\isadigit{0}}{\isachardoublequoteclose}\ \isakeyword{if}\ {\isachardoublequoteopen}s\ {\isacharless}{\kern0pt}\ t{\isachardoublequoteclose}\ \isakeyword{for}\ s\ \ \isacommand{using}\isamarkupfalse%
\ that\ True\ L{\isacharunderscore}{\kern0pt}def\ j{\isacharunderscore}{\kern0pt}prop\ line{\isacharunderscore}{\kern0pt}points{\isacharunderscore}{\kern0pt}in{\isacharunderscore}{\kern0pt}cube{\isacharunderscore}{\kern0pt}unfolded{\isacharbrackleft}{\kern0pt}of\ L\ N{\isacharprime}{\kern0pt}\ t{\isacharbrackright}{\kern0pt}\ \isacommand{by}\isamarkupfalse%
\ simp\isanewline
\ \ \ \ \ \ \ \ \isacommand{moreover}\isamarkupfalse%
\ \isacommand{have}\isamarkupfalse%
\ {\isachardoublequoteopen}L{\isacharprime}{\kern0pt}\ s\ j\ {\isacharequal}{\kern0pt}\ s{\isachardoublequoteclose}\ \isakeyword{if}\ {\isachardoublequoteopen}s\ {\isacharless}{\kern0pt}\ t{\isachardoublequoteclose}\ \isakeyword{for}\ s\ \isacommand{using}\isamarkupfalse%
\ True\ calculation\ that\ \isacommand{by}\isamarkupfalse%
\ simp\isanewline
\ \ \ \ \ \ \ \ \isacommand{moreover}\isamarkupfalse%
\ \isacommand{have}\isamarkupfalse%
\ {\isachardoublequoteopen}L{\isacharprime}{\kern0pt}\ t\ j\ {\isacharequal}{\kern0pt}\ t{\isachardoublequoteclose}\ \isacommand{using}\isamarkupfalse%
\ j{\isacharunderscore}{\kern0pt}prop\ B{\isacharunderscore}{\kern0pt}props\ \isacommand{by}\isamarkupfalse%
\ {\isacharparenleft}{\kern0pt}auto\ simp{\isacharcolon}{\kern0pt}\ L{\isacharprime}{\kern0pt}{\isacharunderscore}{\kern0pt}def{\isacharparenright}{\kern0pt}\isanewline
\ \ \ \ \ \ \ \ \isacommand{ultimately}\isamarkupfalse%
\ \isacommand{show}\isamarkupfalse%
\ {\isacharquery}{\kern0pt}thesis\ \isacommand{unfolding}\isamarkupfalse%
\ L{\isacharprime}{\kern0pt}{\isacharunderscore}{\kern0pt}def\ \isacommand{using}\isamarkupfalse%
\ j{\isacharunderscore}{\kern0pt}prop\ \isacommand{by}\isamarkupfalse%
\ auto\isanewline
\ \ \ \ \ \ \isacommand{next}\isamarkupfalse%
\isanewline
\ \ \ \ \ \ \ \ \isacommand{case}\isamarkupfalse%
\ False\isanewline
\ \ \ \ \ \ \ \ \isacommand{then}\isamarkupfalse%
\ \isacommand{show}\isamarkupfalse%
\ {\isacharquery}{\kern0pt}thesis\isanewline
\ \ \ \ \ \ \ \ \isacommand{proof}\isamarkupfalse%
{\isacharminus}{\kern0pt}\isanewline
\ \ \ \ \ \ \ \ \ \ \isacommand{have}\isamarkupfalse%
\ {\isachardoublequoteopen}{\isacharparenleft}{\kern0pt}{\isasymexists}j{\isacharless}{\kern0pt}N{\isacharprime}{\kern0pt}{\isachardot}{\kern0pt}\ {\isacharparenleft}{\kern0pt}{\isasymforall}s\ {\isacharless}{\kern0pt}\ t{\isachardot}{\kern0pt}\ L{\isacharprime}{\kern0pt}\ s\ j\ {\isacharequal}{\kern0pt}\ s{\isacharparenright}{\kern0pt}{\isacharparenright}{\kern0pt}{\isachardoublequoteclose}\ \isacommand{using}\isamarkupfalse%
\ L{\isacharunderscore}{\kern0pt}def\ \isacommand{unfolding}\isamarkupfalse%
\ is{\isacharunderscore}{\kern0pt}line{\isacharunderscore}{\kern0pt}def\ \isacommand{by}\isamarkupfalse%
\ {\isacharparenleft}{\kern0pt}auto\ simp{\isacharcolon}{\kern0pt}\ L{\isacharprime}{\kern0pt}{\isacharunderscore}{\kern0pt}def{\isacharparenright}{\kern0pt}\isanewline
\ \ \ \ \ \ \ \ \ \ \isacommand{then}\isamarkupfalse%
\ \isacommand{obtain}\isamarkupfalse%
\ j\ \isakeyword{where}\ j{\isacharunderscore}{\kern0pt}def{\isacharcolon}{\kern0pt}\ {\isachardoublequoteopen}j\ {\isacharless}{\kern0pt}\ N{\isacharprime}{\kern0pt}\ {\isasymand}\ {\isacharparenleft}{\kern0pt}{\isasymforall}s\ {\isacharless}{\kern0pt}\ t{\isachardot}{\kern0pt}\ L{\isacharprime}{\kern0pt}\ s\ j\ {\isacharequal}{\kern0pt}\ s{\isacharparenright}{\kern0pt}{\isachardoublequoteclose}\ \isacommand{by}\isamarkupfalse%
\ blast\isanewline
\ \ \ \ \ \ \ \ \ \ \isacommand{have}\isamarkupfalse%
\ {\isachardoublequoteopen}j\ {\isasymnotin}\ B\ {\isadigit{1}}{\isachardoublequoteclose}\isanewline
\ \ \ \ \ \ \ \ \ \ \isacommand{proof}\isamarkupfalse%
\ \isanewline
\ \ \ \ \ \ \ \ \ \ \ \ \isacommand{assume}\isamarkupfalse%
\ a{\isacharcolon}{\kern0pt}{\isachardoublequoteopen}j\ {\isasymin}\ B\ {\isadigit{1}}{\isachardoublequoteclose}\isanewline
\ \ \ \ \ \ \ \ \ \ \ \ \isacommand{then}\isamarkupfalse%
\ \isacommand{have}\isamarkupfalse%
\ {\isachardoublequoteopen}{\isacharparenleft}{\kern0pt}restrict\ {\isacharparenleft}{\kern0pt}{\isasymlambda}y{\isachardot}{\kern0pt}\ L\ {\isacharparenleft}{\kern0pt}y\ {\isadigit{0}}{\isacharparenright}{\kern0pt}{\isacharparenright}{\kern0pt}\ {\isacharparenleft}{\kern0pt}cube\ {\isadigit{1}}\ t{\isacharparenright}{\kern0pt}{\isacharparenright}{\kern0pt}\ y\ j\ {\isacharequal}{\kern0pt}\ f\ j{\isachardoublequoteclose}\ \isakeyword{if}\ {\isachardoublequoteopen}y\ {\isasymin}\ cube\ {\isadigit{1}}\ t{\isachardoublequoteclose}\ \isakeyword{for}\ y\ \isacommand{using}\isamarkupfalse%
\ Bf{\isacharunderscore}{\kern0pt}defs\ that\ \isacommand{by}\isamarkupfalse%
\ simp\isanewline
\ \ \ \ \ \ \ \ \ \ \ \ \isacommand{then}\isamarkupfalse%
\ \isacommand{have}\isamarkupfalse%
\ {\isachardoublequoteopen}L\ {\isacharparenleft}{\kern0pt}y\ {\isadigit{0}}{\isacharparenright}{\kern0pt}\ j\ {\isacharequal}{\kern0pt}\ f\ j{\isachardoublequoteclose}\ \isakeyword{if}\ {\isachardoublequoteopen}y\ {\isasymin}\ cube\ {\isadigit{1}}\ t{\isachardoublequoteclose}\ \isakeyword{for}\ y\ \isacommand{using}\isamarkupfalse%
\ that\ \isacommand{by}\isamarkupfalse%
\ simp\isanewline
\ \ \ \ \ \ \ \ \ \ \ \ \isacommand{moreover}\isamarkupfalse%
\ \isacommand{have}\isamarkupfalse%
\ {\isachardoublequoteopen}{\isasymexists}{\isacharbang}{\kern0pt}i{\isachardot}{\kern0pt}\ i\ {\isacharless}{\kern0pt}\ t\ {\isasymand}\ y\ {\isadigit{0}}\ {\isacharequal}{\kern0pt}\ i{\isachardoublequoteclose}\ \isakeyword{if}\ {\isachardoublequoteopen}y\ {\isasymin}\ cube\ {\isadigit{1}}\ t{\isachardoublequoteclose}\ \isakeyword{for}\ y\ \isacommand{using}\isamarkupfalse%
\ that\ one{\isacharunderscore}{\kern0pt}dim{\isacharunderscore}{\kern0pt}cube{\isacharunderscore}{\kern0pt}eq{\isacharunderscore}{\kern0pt}nat{\isacharunderscore}{\kern0pt}set{\isacharbrackleft}{\kern0pt}of\ {\isachardoublequoteopen}t{\isachardoublequoteclose}{\isacharbrackright}{\kern0pt}\ \isacommand{unfolding}\isamarkupfalse%
\ bij{\isacharunderscore}{\kern0pt}betw{\isacharunderscore}{\kern0pt}def\ \isacommand{by}\isamarkupfalse%
\ blast\isanewline
\ \ \ \ \ \ \ \ \ \ \ \ \isacommand{moreover}\isamarkupfalse%
\ \isacommand{have}\isamarkupfalse%
\ {\isachardoublequoteopen}{\isasymexists}{\isacharbang}{\kern0pt}y{\isachardot}{\kern0pt}\ y\ {\isasymin}\ cube\ {\isadigit{1}}\ t\ {\isasymand}\ y\ {\isadigit{0}}\ {\isacharequal}{\kern0pt}\ i{\isachardoublequoteclose}\ \isakeyword{if}\ {\isachardoublequoteopen}i\ {\isacharless}{\kern0pt}\ t{\isachardoublequoteclose}\ \isakeyword{for}\ i\ \isanewline
\ \ \ \ \ \ \ \ \ \ \ \ \isacommand{proof}\isamarkupfalse%
\ {\isacharparenleft}{\kern0pt}intro\ ex{\isadigit{1}}I{\isacharunderscore}{\kern0pt}alt{\isacharparenright}{\kern0pt}\isanewline
\ \ \ \ \ \ \ \ \ \ \ \ \ \ \isacommand{define}\isamarkupfalse%
\ y\ \isakeyword{where}\ {\isachardoublequoteopen}y\ {\isasymequiv}\ {\isacharparenleft}{\kern0pt}{\isasymlambda}x{\isacharcolon}{\kern0pt}{\isacharcolon}{\kern0pt}nat{\isachardot}{\kern0pt}\ {\isasymlambda}y{\isasymin}{\isacharbraceleft}{\kern0pt}{\isachardot}{\kern0pt}{\isachardot}{\kern0pt}{\isacharless}{\kern0pt}{\isadigit{1}}{\isacharcolon}{\kern0pt}{\isacharcolon}{\kern0pt}nat{\isacharbraceright}{\kern0pt}{\isachardot}{\kern0pt}\ x{\isacharparenright}{\kern0pt}{\isachardoublequoteclose}\ \isanewline
\ \ \ \ \ \ \ \ \ \ \ \ \ \ \isacommand{have}\isamarkupfalse%
\ {\isachardoublequoteopen}y\ i\ {\isasymin}\ {\isacharparenleft}{\kern0pt}cube\ {\isadigit{1}}\ t{\isacharparenright}{\kern0pt}{\isachardoublequoteclose}\ \isacommand{using}\isamarkupfalse%
\ that\ \isacommand{unfolding}\isamarkupfalse%
\ cube{\isacharunderscore}{\kern0pt}def\ y{\isacharunderscore}{\kern0pt}def\ \isacommand{by}\isamarkupfalse%
\ simp\isanewline
\ \ \ \ \ \ \ \ \ \ \ \ \ \ \isacommand{moreover}\isamarkupfalse%
\ \isacommand{have}\isamarkupfalse%
\ {\isachardoublequoteopen}y\ i\ {\isadigit{0}}\ {\isacharequal}{\kern0pt}\ i{\isachardoublequoteclose}\ \isacommand{unfolding}\isamarkupfalse%
\ y{\isacharunderscore}{\kern0pt}def\ \isacommand{by}\isamarkupfalse%
\ simp\isanewline
\ \ \ \ \ \ \ \ \ \ \ \ \ \ \isacommand{moreover}\isamarkupfalse%
\ \isacommand{have}\isamarkupfalse%
\ {\isachardoublequoteopen}z\ {\isacharequal}{\kern0pt}\ y\ i{\isachardoublequoteclose}\ \isakeyword{if}\ {\isachardoublequoteopen}z\ {\isasymin}\ cube\ {\isadigit{1}}\ t{\isachardoublequoteclose}\ \isakeyword{and}\ {\isachardoublequoteopen}z\ {\isadigit{0}}\ {\isacharequal}{\kern0pt}\ i{\isachardoublequoteclose}\ \isakeyword{for}\ z\isanewline
\ \ \ \ \ \ \ \ \ \ \ \ \ \ \isacommand{proof}\isamarkupfalse%
\ {\isacharparenleft}{\kern0pt}rule\ ccontr{\isacharparenright}{\kern0pt}\isanewline
\ \ \ \ \ \ \ \ \ \ \ \ \ \ \ \ \isacommand{assume}\isamarkupfalse%
\ {\isachardoublequoteopen}z\ {\isasymnoteq}\ y\ i{\isachardoublequoteclose}\ \isanewline
\ \ \ \ \ \ \ \ \ \ \ \ \ \ \ \ \isacommand{then}\isamarkupfalse%
\ \isacommand{obtain}\isamarkupfalse%
\ l\ \isakeyword{where}\ l{\isacharunderscore}{\kern0pt}prop{\isacharcolon}{\kern0pt}\ {\isachardoublequoteopen}z\ l\ {\isasymnoteq}\ y\ i\ l{\isachardoublequoteclose}\ \isacommand{by}\isamarkupfalse%
\ blast\isanewline
\ \ \ \ \ \ \ \ \ \ \ \ \ \ \ \ \isacommand{consider}\isamarkupfalse%
\ {\isachardoublequoteopen}l\ {\isasymin}\ {\isacharbraceleft}{\kern0pt}{\isachardot}{\kern0pt}{\isachardot}{\kern0pt}{\isacharless}{\kern0pt}{\isadigit{1}}{\isacharcolon}{\kern0pt}{\isacharcolon}{\kern0pt}nat{\isacharbraceright}{\kern0pt}{\isachardoublequoteclose}\ {\isacharbar}{\kern0pt}\ {\isachardoublequoteopen}l\ {\isasymnotin}\ {\isacharbraceleft}{\kern0pt}{\isachardot}{\kern0pt}{\isachardot}{\kern0pt}{\isacharless}{\kern0pt}{\isadigit{1}}{\isacharcolon}{\kern0pt}{\isacharcolon}{\kern0pt}nat{\isacharbraceright}{\kern0pt}{\isachardoublequoteclose}\ \isacommand{by}\isamarkupfalse%
\ blast\isanewline
\ \ \ \ \ \ \ \ \ \ \ \ \ \ \ \ \isacommand{then}\isamarkupfalse%
\ \isacommand{show}\isamarkupfalse%
\ False\isanewline
\ \ \ \ \ \ \ \ \ \ \ \ \ \ \ \ \isacommand{proof}\isamarkupfalse%
\ cases\isanewline
\ \ \ \ \ \ \ \ \ \ \ \ \ \ \ \ \ \ \isacommand{case}\isamarkupfalse%
\ {\isadigit{1}}\isanewline
\ \ \ \ \ \ \ \ \ \ \ \ \ \ \ \ \ \ \isacommand{then}\isamarkupfalse%
\ \isacommand{show}\isamarkupfalse%
\ {\isacharquery}{\kern0pt}thesis\ \isacommand{using}\isamarkupfalse%
\ l{\isacharunderscore}{\kern0pt}prop\ that{\isacharparenleft}{\kern0pt}{\isadigit{2}}{\isacharparenright}{\kern0pt}\ \isacommand{unfolding}\isamarkupfalse%
\ y{\isacharunderscore}{\kern0pt}def\ \isacommand{by}\isamarkupfalse%
\ auto\isanewline
\ \ \ \ \ \ \ \ \ \ \ \ \ \ \ \ \isacommand{next}\isamarkupfalse%
\isanewline
\ \ \ \ \ \ \ \ \ \ \ \ \ \ \ \ \ \ \isacommand{case}\isamarkupfalse%
\ {\isadigit{2}}\isanewline
\ \ \ \ \ \ \ \ \ \ \ \ \ \ \ \ \ \ \isacommand{then}\isamarkupfalse%
\ \isacommand{have}\isamarkupfalse%
\ {\isachardoublequoteopen}z\ l\ {\isacharequal}{\kern0pt}\ undefined{\isachardoublequoteclose}\ \isacommand{using}\isamarkupfalse%
\ that\ \isacommand{unfolding}\isamarkupfalse%
\ cube{\isacharunderscore}{\kern0pt}def\ \isacommand{by}\isamarkupfalse%
\ blast\isanewline
\ \ \ \ \ \ \ \ \ \ \ \ \ \ \ \ \ \ \isacommand{moreover}\isamarkupfalse%
\ \isacommand{have}\isamarkupfalse%
\ {\isachardoublequoteopen}y\ i\ l\ {\isacharequal}{\kern0pt}\ undefined{\isachardoublequoteclose}\ \isacommand{unfolding}\isamarkupfalse%
\ y{\isacharunderscore}{\kern0pt}def\ \isacommand{using}\isamarkupfalse%
\ {\isadigit{2}}\ \isacommand{by}\isamarkupfalse%
\ auto\isanewline
\ \ \ \ \ \ \ \ \ \ \ \ \ \ \ \ \ \ \isacommand{ultimately}\isamarkupfalse%
\ \isacommand{show}\isamarkupfalse%
\ {\isacharquery}{\kern0pt}thesis\ \isacommand{using}\isamarkupfalse%
\ l{\isacharunderscore}{\kern0pt}prop\ \isacommand{by}\isamarkupfalse%
\ presburger\isanewline
\ \ \ \ \ \ \ \ \ \ \ \ \ \ \ \ \isacommand{qed}\isamarkupfalse%
\isanewline
\ \ \ \ \ \ \ \ \ \ \ \ \ \ \isacommand{qed}\isamarkupfalse%
\isanewline
\ \ \ \ \ \ \ \ \ \ \ \ \ \ \isacommand{ultimately}\isamarkupfalse%
\ \isacommand{show}\isamarkupfalse%
\ {\isachardoublequoteopen}{\isasymexists}y{\isachardot}{\kern0pt}\ {\isacharparenleft}{\kern0pt}y\ {\isasymin}\ cube\ {\isadigit{1}}\ t\ {\isasymand}\ y\ {\isadigit{0}}\ {\isacharequal}{\kern0pt}\ i{\isacharparenright}{\kern0pt}\ {\isasymand}\ {\isacharparenleft}{\kern0pt}{\isasymforall}ya{\isachardot}{\kern0pt}\ ya\ {\isasymin}\ cube\ {\isadigit{1}}\ t\ {\isasymand}\ ya\ {\isadigit{0}}\ {\isacharequal}{\kern0pt}\ i\ {\isasymlongrightarrow}\ y\ {\isacharequal}{\kern0pt}\ ya{\isacharparenright}{\kern0pt}{\isachardoublequoteclose}\ \isacommand{by}\isamarkupfalse%
\ blast\isanewline
\ \ \ \ \ \ \ \ \ \ \ \ \isacommand{qed}\isamarkupfalse%
\isanewline
\isanewline
\ \ \ \ \ \ \ \ \ \ \ \ \isacommand{moreover}\isamarkupfalse%
\ \isacommand{have}\isamarkupfalse%
\ {\isachardoublequoteopen}L\ i\ j\ {\isacharequal}{\kern0pt}\ f\ j{\isachardoublequoteclose}\ \isakeyword{if}\ {\isachardoublequoteopen}i\ {\isacharless}{\kern0pt}\ t{\isachardoublequoteclose}\ \isakeyword{for}\ i\ \isacommand{using}\isamarkupfalse%
\ that\ calculation\ \isacommand{by}\isamarkupfalse%
\ blast\isanewline
\ \ \ \ \ \ \ \ \ \ \ \ \isacommand{moreover}\isamarkupfalse%
\ \isacommand{have}\isamarkupfalse%
\ {\isachardoublequoteopen}{\isacharparenleft}{\kern0pt}{\isasymexists}j{\isacharless}{\kern0pt}N{\isacharprime}{\kern0pt}{\isachardot}{\kern0pt}\ {\isacharparenleft}{\kern0pt}{\isasymforall}s\ {\isacharless}{\kern0pt}\ t{\isachardot}{\kern0pt}\ L\ s\ j\ {\isacharequal}{\kern0pt}\ s{\isacharparenright}{\kern0pt}{\isacharparenright}{\kern0pt}{\isachardoublequoteclose}\ \isacommand{using}\isamarkupfalse%
\ {\isacartoucheopen}{\isacharparenleft}{\kern0pt}{\isasymexists}j{\isacharless}{\kern0pt}N{\isacharprime}{\kern0pt}{\isachardot}{\kern0pt}\ {\isacharparenleft}{\kern0pt}{\isasymforall}s\ {\isacharless}{\kern0pt}\ t{\isachardot}{\kern0pt}\ L{\isacharprime}{\kern0pt}\ s\ j\ {\isacharequal}{\kern0pt}\ s{\isacharparenright}{\kern0pt}{\isacharparenright}{\kern0pt}{\isacartoucheclose}\ \isacommand{by}\isamarkupfalse%
\ {\isacharparenleft}{\kern0pt}auto\ simp{\isacharcolon}{\kern0pt}\ L{\isacharprime}{\kern0pt}{\isacharunderscore}{\kern0pt}def{\isacharparenright}{\kern0pt}\isanewline
\ \ \ \ \ \ \ \ \ \ \ \ \isacommand{ultimately}\isamarkupfalse%
\ \isacommand{show}\isamarkupfalse%
\ False\ \isacommand{using}\isamarkupfalse%
\ False\isanewline
\ \ \ \ \ \ \ \ \ \ \ \ \ \ \isacommand{by}\isamarkupfalse%
\ {\isacharparenleft}{\kern0pt}metis\ {\isacharparenleft}{\kern0pt}no{\isacharunderscore}{\kern0pt}types{\isacharcomma}{\kern0pt}\ lifting{\isacharparenright}{\kern0pt}\ L{\isacharprime}{\kern0pt}{\isacharunderscore}{\kern0pt}def\ assms{\isacharparenleft}{\kern0pt}{\isadigit{1}}{\isacharparenright}{\kern0pt}\ fun{\isacharunderscore}{\kern0pt}upd{\isacharunderscore}{\kern0pt}apply\ j{\isacharunderscore}{\kern0pt}def\ less{\isacharunderscore}{\kern0pt}one\ nat{\isacharunderscore}{\kern0pt}neq{\isacharunderscore}{\kern0pt}iff{\isacharparenright}{\kern0pt}\isanewline
\ \ \ \ \ \ \ \ \ \ \isacommand{qed}\isamarkupfalse%
\isanewline
\ \ \ \ \ \ \ \ \ \ \isacommand{then}\isamarkupfalse%
\ \isacommand{have}\isamarkupfalse%
\ {\isachardoublequoteopen}j\ {\isasymin}\ B\ {\isadigit{0}}{\isachardoublequoteclose}\ \isacommand{using}\isamarkupfalse%
\ {\isacartoucheopen}j\ {\isasymnotin}\ B\ {\isadigit{1}}{\isacartoucheclose}\ j{\isacharunderscore}{\kern0pt}def\ B{\isacharunderscore}{\kern0pt}props\ \isacommand{by}\isamarkupfalse%
\ auto\isanewline
\isanewline
\ \ \ \ \ \ \ \ \ \ \isacommand{then}\isamarkupfalse%
\ \isacommand{have}\isamarkupfalse%
\ {\isachardoublequoteopen}L{\isacharprime}{\kern0pt}\ t\ j\ {\isacharequal}{\kern0pt}\ t{\isachardoublequoteclose}\ \isacommand{using}\isamarkupfalse%
\ {\isacartoucheopen}j\ {\isasymnotin}\ B\ {\isadigit{1}}{\isacartoucheclose}\ \isacommand{by}\isamarkupfalse%
\ {\isacharparenleft}{\kern0pt}auto\ simp{\isacharcolon}{\kern0pt}\ L{\isacharprime}{\kern0pt}{\isacharunderscore}{\kern0pt}def{\isacharparenright}{\kern0pt}\isanewline
\ \ \ \ \ \ \ \ \ \ \isacommand{then}\isamarkupfalse%
\ \isacommand{have}\isamarkupfalse%
\ {\isachardoublequoteopen}L{\isacharprime}{\kern0pt}\ s\ j\ {\isacharequal}{\kern0pt}\ s{\isachardoublequoteclose}\ \isakeyword{if}\ {\isachardoublequoteopen}s\ {\isacharless}{\kern0pt}\ t\ {\isacharplus}{\kern0pt}\ {\isadigit{1}}{\isachardoublequoteclose}\ \isakeyword{for}\ s\ \isacommand{using}\isamarkupfalse%
\ j{\isacharunderscore}{\kern0pt}def\ that\ \isacommand{by}\isamarkupfalse%
\ {\isacharparenleft}{\kern0pt}auto\ simp{\isacharcolon}{\kern0pt}\ L{\isacharprime}{\kern0pt}{\isacharunderscore}{\kern0pt}def{\isacharparenright}{\kern0pt}\isanewline
\ \ \ \ \ \ \ \ \ \ \isacommand{then}\isamarkupfalse%
\ \isacommand{show}\isamarkupfalse%
\ {\isacharquery}{\kern0pt}thesis\ \isacommand{using}\isamarkupfalse%
\ j{\isacharunderscore}{\kern0pt}def\ \isacommand{by}\isamarkupfalse%
\ blast\isanewline
\ \ \ \ \ \ \ \ \isacommand{qed}\isamarkupfalse%
\isanewline
\ \ \ \ \ \ \isacommand{qed}\isamarkupfalse%
\isanewline
\isanewline
\isanewline
\ \ \ \ \ \ \isacommand{have}\isamarkupfalse%
\ A{\isadigit{3}}{\isacharcolon}{\kern0pt}\ {\isachardoublequoteopen}{\isacharparenleft}{\kern0pt}{\isasymforall}x{\isacharless}{\kern0pt}t{\isacharplus}{\kern0pt}{\isadigit{1}}{\isachardot}{\kern0pt}\ {\isasymforall}y{\isacharless}{\kern0pt}t{\isacharplus}{\kern0pt}{\isadigit{1}}{\isachardot}{\kern0pt}\ L{\isacharprime}{\kern0pt}\ x\ j\ {\isacharequal}{\kern0pt}\ \ L{\isacharprime}{\kern0pt}\ y\ j{\isacharparenright}{\kern0pt}\ {\isasymor}\ {\isacharparenleft}{\kern0pt}{\isasymforall}s{\isacharless}{\kern0pt}t{\isacharplus}{\kern0pt}{\isadigit{1}}{\isachardot}{\kern0pt}\ L{\isacharprime}{\kern0pt}\ s\ j\ {\isacharequal}{\kern0pt}\isanewline
\ \ \ \ \ \ \ \ s{\isacharparenright}{\kern0pt}{\isachardoublequoteclose}\ \isakeyword{if}\ {\isachardoublequoteopen}j\ {\isacharless}{\kern0pt}\ N{\isacharprime}{\kern0pt}{\isachardoublequoteclose}\ \isakeyword{for}\ j\ \isanewline
\ \ \ \ \ \ \isacommand{proof}\isamarkupfalse%
{\isacharminus}{\kern0pt}\isanewline
\ \ \ \ \ \ \ \ \isacommand{show}\isamarkupfalse%
\ {\isachardoublequoteopen}{\isacharparenleft}{\kern0pt}{\isasymforall}x{\isacharless}{\kern0pt}t{\isacharplus}{\kern0pt}{\isadigit{1}}{\isachardot}{\kern0pt}\ {\isasymforall}y{\isacharless}{\kern0pt}t{\isacharplus}{\kern0pt}{\isadigit{1}}{\isachardot}{\kern0pt}\ L{\isacharprime}{\kern0pt}\ x\ j\ {\isacharequal}{\kern0pt}\ \ L{\isacharprime}{\kern0pt}\ y\ j{\isacharparenright}{\kern0pt}\ {\isasymor}\ {\isacharparenleft}{\kern0pt}{\isasymforall}s{\isacharless}{\kern0pt}t{\isacharplus}{\kern0pt}{\isadigit{1}}{\isachardot}{\kern0pt}\ L{\isacharprime}{\kern0pt}\ s\ j\ {\isacharequal}{\kern0pt}\ s{\isacharparenright}{\kern0pt}{\isachardoublequoteclose}\isanewline
\ \ \ \ \ \ \ \ \isacommand{proof}\isamarkupfalse%
\ {\isacharparenleft}{\kern0pt}cases\ {\isachardoublequoteopen}j\ {\isasymin}\ B\ {\isadigit{1}}{\isachardoublequoteclose}{\isacharparenright}{\kern0pt}\isanewline
\ \ \ \ \ \ \ \ \ \ \isacommand{case}\isamarkupfalse%
\ True\isanewline
\ \ \ \ \ \ \ \ \ \ \isacommand{then}\isamarkupfalse%
\ \isacommand{have}\isamarkupfalse%
\ {\isachardoublequoteopen}{\isacharparenleft}{\kern0pt}restrict\ {\isacharparenleft}{\kern0pt}{\isasymlambda}y{\isachardot}{\kern0pt}\ L\ {\isacharparenleft}{\kern0pt}y\ {\isadigit{0}}{\isacharparenright}{\kern0pt}{\isacharparenright}{\kern0pt}\ {\isacharparenleft}{\kern0pt}cube\ {\isadigit{1}}\ t{\isacharparenright}{\kern0pt}{\isacharparenright}{\kern0pt}\ y\ j\ {\isacharequal}{\kern0pt}\ f\ j{\isachardoublequoteclose}\ \isakeyword{if}\ {\isachardoublequoteopen}y\ {\isasymin}\ cube\ {\isadigit{1}}\ t{\isachardoublequoteclose}\ \isakeyword{for}\ y\ \isacommand{using}\isamarkupfalse%
\ that\ Bf{\isacharunderscore}{\kern0pt}defs\ \isacommand{by}\isamarkupfalse%
\ simp\isanewline
\ \ \ \ \ \ \ \ \ \ \isacommand{moreover}\isamarkupfalse%
\ \isacommand{have}\isamarkupfalse%
\ {\isachardoublequoteopen}{\isasymexists}{\isacharbang}{\kern0pt}i{\isachardot}{\kern0pt}\ i\ {\isacharless}{\kern0pt}\ t\ {\isasymand}\ y\ {\isadigit{0}}\ {\isacharequal}{\kern0pt}\ i{\isachardoublequoteclose}\ \isakeyword{if}\ {\isachardoublequoteopen}y\ {\isasymin}\ cube\ {\isadigit{1}}\ t{\isachardoublequoteclose}\ \isakeyword{for}\ y\ \isacommand{using}\isamarkupfalse%
\ that\ one{\isacharunderscore}{\kern0pt}dim{\isacharunderscore}{\kern0pt}cube{\isacharunderscore}{\kern0pt}eq{\isacharunderscore}{\kern0pt}nat{\isacharunderscore}{\kern0pt}set{\isacharbrackleft}{\kern0pt}of\ {\isachardoublequoteopen}t{\isachardoublequoteclose}{\isacharbrackright}{\kern0pt}\ \isacommand{unfolding}\isamarkupfalse%
\ bij{\isacharunderscore}{\kern0pt}betw{\isacharunderscore}{\kern0pt}def\ \isacommand{by}\isamarkupfalse%
\ blast\isanewline
\ \ \ \ \ \ \ \ \ \ \isacommand{moreover}\isamarkupfalse%
\ \isacommand{have}\isamarkupfalse%
\ {\isachardoublequoteopen}{\isasymexists}{\isacharbang}{\kern0pt}y{\isachardot}{\kern0pt}\ y\ {\isasymin}\ cube\ {\isadigit{1}}\ t\ {\isasymand}\ y\ {\isadigit{0}}\ {\isacharequal}{\kern0pt}\ i{\isachardoublequoteclose}\ \isakeyword{if}\ {\isachardoublequoteopen}i\ {\isacharless}{\kern0pt}\ t{\isachardoublequoteclose}\ \isakeyword{for}\ i\ \isanewline
\ \ \ \ \ \ \ \ \ \ \isacommand{proof}\isamarkupfalse%
\ {\isacharparenleft}{\kern0pt}intro\ ex{\isadigit{1}}I{\isacharunderscore}{\kern0pt}alt{\isacharparenright}{\kern0pt}\isanewline
\ \ \ \ \ \ \ \ \ \ \ \ \isacommand{define}\isamarkupfalse%
\ y\ \isakeyword{where}\ {\isachardoublequoteopen}y\ {\isasymequiv}\ {\isacharparenleft}{\kern0pt}{\isasymlambda}x{\isacharcolon}{\kern0pt}{\isacharcolon}{\kern0pt}nat{\isachardot}{\kern0pt}\ {\isasymlambda}y{\isasymin}{\isacharbraceleft}{\kern0pt}{\isachardot}{\kern0pt}{\isachardot}{\kern0pt}{\isacharless}{\kern0pt}{\isadigit{1}}{\isacharcolon}{\kern0pt}{\isacharcolon}{\kern0pt}nat{\isacharbraceright}{\kern0pt}{\isachardot}{\kern0pt}\ x{\isacharparenright}{\kern0pt}{\isachardoublequoteclose}\ \isanewline
\ \ \ \ \ \ \ \ \ \ \ \ \isacommand{have}\isamarkupfalse%
\ {\isachardoublequoteopen}y\ i\ {\isasymin}\ {\isacharparenleft}{\kern0pt}cube\ {\isadigit{1}}\ t{\isacharparenright}{\kern0pt}{\isachardoublequoteclose}\ \isacommand{using}\isamarkupfalse%
\ that\ \isacommand{unfolding}\isamarkupfalse%
\ cube{\isacharunderscore}{\kern0pt}def\ y{\isacharunderscore}{\kern0pt}def\ \isacommand{by}\isamarkupfalse%
\ simp\isanewline
\ \ \ \ \ \ \ \ \ \ \ \ \isacommand{moreover}\isamarkupfalse%
\ \isacommand{have}\isamarkupfalse%
\ {\isachardoublequoteopen}y\ i\ {\isadigit{0}}\ {\isacharequal}{\kern0pt}\ i{\isachardoublequoteclose}\ \isacommand{unfolding}\isamarkupfalse%
\ y{\isacharunderscore}{\kern0pt}def\ \isacommand{by}\isamarkupfalse%
\ auto\isanewline
\ \ \ \ \ \ \ \ \ \ \ \ \isacommand{moreover}\isamarkupfalse%
\ \isacommand{have}\isamarkupfalse%
\ {\isachardoublequoteopen}z\ {\isacharequal}{\kern0pt}\ y\ i{\isachardoublequoteclose}\ \isakeyword{if}\ {\isachardoublequoteopen}z\ {\isasymin}\ cube\ {\isadigit{1}}\ t{\isachardoublequoteclose}\ \isakeyword{and}\ {\isachardoublequoteopen}z\ {\isadigit{0}}\ {\isacharequal}{\kern0pt}\ i{\isachardoublequoteclose}\ \isakeyword{for}\ z\isanewline
\ \ \ \ \ \ \ \ \ \ \ \ \isacommand{proof}\isamarkupfalse%
\ {\isacharparenleft}{\kern0pt}rule\ ccontr{\isacharparenright}{\kern0pt}\isanewline
\ \ \ \ \ \ \ \ \ \ \ \ \ \ \isacommand{assume}\isamarkupfalse%
\ {\isachardoublequoteopen}z\ {\isasymnoteq}\ y\ i{\isachardoublequoteclose}\ \isanewline
\ \ \ \ \ \ \ \ \ \ \ \ \ \ \isacommand{then}\isamarkupfalse%
\ \isacommand{obtain}\isamarkupfalse%
\ l\ \isakeyword{where}\ l{\isacharunderscore}{\kern0pt}prop{\isacharcolon}{\kern0pt}\ {\isachardoublequoteopen}z\ l\ {\isasymnoteq}\ y\ i\ l{\isachardoublequoteclose}\ \isacommand{by}\isamarkupfalse%
\ blast\isanewline
\ \ \ \ \ \ \ \ \ \ \ \ \ \ \isacommand{consider}\isamarkupfalse%
\ {\isachardoublequoteopen}l\ {\isasymin}\ {\isacharbraceleft}{\kern0pt}{\isachardot}{\kern0pt}{\isachardot}{\kern0pt}{\isacharless}{\kern0pt}{\isadigit{1}}{\isacharcolon}{\kern0pt}{\isacharcolon}{\kern0pt}nat{\isacharbraceright}{\kern0pt}{\isachardoublequoteclose}\ {\isacharbar}{\kern0pt}\ {\isachardoublequoteopen}l\ {\isasymnotin}\ {\isacharbraceleft}{\kern0pt}{\isachardot}{\kern0pt}{\isachardot}{\kern0pt}{\isacharless}{\kern0pt}{\isadigit{1}}{\isacharcolon}{\kern0pt}{\isacharcolon}{\kern0pt}nat{\isacharbraceright}{\kern0pt}{\isachardoublequoteclose}\ \isacommand{by}\isamarkupfalse%
\ blast\isanewline
\ \ \ \ \ \ \ \ \ \ \ \ \ \ \isacommand{then}\isamarkupfalse%
\ \isacommand{show}\isamarkupfalse%
\ False\isanewline
\ \ \ \ \ \ \ \ \ \ \ \ \ \ \isacommand{proof}\isamarkupfalse%
\ cases\isanewline
\ \ \ \ \ \ \ \ \ \ \ \ \ \ \ \ \isacommand{case}\isamarkupfalse%
\ {\isadigit{1}}\isanewline
\ \ \ \ \ \ \ \ \ \ \ \ \ \ \ \ \isacommand{then}\isamarkupfalse%
\ \isacommand{show}\isamarkupfalse%
\ {\isacharquery}{\kern0pt}thesis\ \isacommand{using}\isamarkupfalse%
\ l{\isacharunderscore}{\kern0pt}prop\ that{\isacharparenleft}{\kern0pt}{\isadigit{2}}{\isacharparenright}{\kern0pt}\ \isacommand{unfolding}\isamarkupfalse%
\ y{\isacharunderscore}{\kern0pt}def\ \isacommand{by}\isamarkupfalse%
\ auto\isanewline
\ \ \ \ \ \ \ \ \ \ \ \ \ \ \isacommand{next}\isamarkupfalse%
\isanewline
\ \ \ \ \ \ \ \ \ \ \ \ \ \ \ \ \isacommand{case}\isamarkupfalse%
\ {\isadigit{2}}\isanewline
\ \ \ \ \ \ \ \ \ \ \ \ \ \ \ \ \isacommand{then}\isamarkupfalse%
\ \isacommand{have}\isamarkupfalse%
\ {\isachardoublequoteopen}z\ l\ {\isacharequal}{\kern0pt}\ undefined{\isachardoublequoteclose}\ \isacommand{using}\isamarkupfalse%
\ that\ \isacommand{unfolding}\isamarkupfalse%
\ cube{\isacharunderscore}{\kern0pt}def\ \isacommand{by}\isamarkupfalse%
\ blast\isanewline
\ \ \ \ \ \ \ \ \ \ \ \ \ \ \ \ \isacommand{moreover}\isamarkupfalse%
\ \isacommand{have}\isamarkupfalse%
\ {\isachardoublequoteopen}y\ i\ l\ {\isacharequal}{\kern0pt}\ undefined{\isachardoublequoteclose}\ \isacommand{unfolding}\isamarkupfalse%
\ y{\isacharunderscore}{\kern0pt}def\ \isacommand{using}\isamarkupfalse%
\ {\isadigit{2}}\ \isacommand{by}\isamarkupfalse%
\ auto\isanewline
\ \ \ \ \ \ \ \ \ \ \ \ \ \ \ \ \isacommand{ultimately}\isamarkupfalse%
\ \isacommand{show}\isamarkupfalse%
\ {\isacharquery}{\kern0pt}thesis\ \isacommand{using}\isamarkupfalse%
\ l{\isacharunderscore}{\kern0pt}prop\ \isacommand{by}\isamarkupfalse%
\ presburger\isanewline
\ \ \ \ \ \ \ \ \ \ \ \ \ \ \isacommand{qed}\isamarkupfalse%
\isanewline
\ \ \ \ \ \ \ \ \ \ \ \ \isacommand{qed}\isamarkupfalse%
\isanewline
\ \ \ \ \ \ \ \ \ \ \ \ \isacommand{ultimately}\isamarkupfalse%
\ \isacommand{show}\isamarkupfalse%
\ {\isachardoublequoteopen}{\isasymexists}y{\isachardot}{\kern0pt}\ {\isacharparenleft}{\kern0pt}y\ {\isasymin}\ cube\ {\isadigit{1}}\ t\ {\isasymand}\ y\ {\isadigit{0}}\ {\isacharequal}{\kern0pt}\ i{\isacharparenright}{\kern0pt}\ {\isasymand}\ {\isacharparenleft}{\kern0pt}{\isasymforall}ya{\isachardot}{\kern0pt}\ ya\ {\isasymin}\ cube\ {\isadigit{1}}\ t\ {\isasymand}\ ya\ {\isadigit{0}}\ {\isacharequal}{\kern0pt}\ i\ {\isasymlongrightarrow}\ y\ {\isacharequal}{\kern0pt}\ ya{\isacharparenright}{\kern0pt}{\isachardoublequoteclose}\ \isacommand{by}\isamarkupfalse%
\ blast\isanewline
\isanewline
\ \ \ \ \ \ \ \ \ \ \isacommand{qed}\isamarkupfalse%
\isanewline
\ \ \ \ \ \ \ \ \ \ \isacommand{moreover}\isamarkupfalse%
\ \isacommand{have}\isamarkupfalse%
\ {\isachardoublequoteopen}L\ i\ j\ {\isacharequal}{\kern0pt}\ f\ j{\isachardoublequoteclose}\ \isakeyword{if}\ {\isachardoublequoteopen}i\ {\isacharless}{\kern0pt}\ t{\isachardoublequoteclose}\ \isakeyword{for}\ i\ \isacommand{using}\isamarkupfalse%
\ calculation\ that\ \isacommand{by}\isamarkupfalse%
\ force\isanewline
\ \ \ \ \ \ \ \ \ \ \isacommand{moreover}\isamarkupfalse%
\ \isacommand{have}\isamarkupfalse%
\ \ {\isachardoublequoteopen}L\ i\ j\ {\isacharequal}{\kern0pt}\ L\ x\ j{\isachardoublequoteclose}\ \isakeyword{if}\ {\isachardoublequoteopen}x\ {\isacharless}{\kern0pt}\ t{\isachardoublequoteclose}\ {\isachardoublequoteopen}i\ {\isacharless}{\kern0pt}\ t{\isachardoublequoteclose}\ \isakeyword{for}\ x\ i\ \isacommand{using}\isamarkupfalse%
\ that\ calculation\ \isacommand{by}\isamarkupfalse%
\ simp\isanewline
\ \ \ \ \ \ \ \ \ \ \isacommand{moreover}\isamarkupfalse%
\ \isacommand{have}\isamarkupfalse%
\ {\isachardoublequoteopen}L{\isacharprime}{\kern0pt}\ x\ j\ {\isacharequal}{\kern0pt}\ L\ x\ j{\isachardoublequoteclose}\ \isakeyword{if}\ {\isachardoublequoteopen}x\ {\isacharless}{\kern0pt}\ t{\isachardoublequoteclose}\ \isakeyword{for}\ x\ \isacommand{using}\isamarkupfalse%
\ that\ fun{\isacharunderscore}{\kern0pt}upd{\isacharunderscore}{\kern0pt}other{\isacharbrackleft}{\kern0pt}of\ x\ t\ L\ {\isachardoublequoteopen}{\isasymlambda}j{\isachardot}{\kern0pt}\ if\ j\ {\isasymin}\ B\ {\isadigit{1}}\ then\ L\ {\isacharparenleft}{\kern0pt}t\ {\isacharminus}{\kern0pt}\ {\isadigit{1}}{\isacharparenright}{\kern0pt}\ j\ else\ if\ j\ {\isasymin}\ B\ {\isadigit{0}}\ then\ t\ else\ undefined{\isachardoublequoteclose}{\isacharbrackright}{\kern0pt}\ \isacommand{unfolding}\isamarkupfalse%
\ L{\isacharprime}{\kern0pt}{\isacharunderscore}{\kern0pt}def\ \isacommand{by}\isamarkupfalse%
\ simp\isanewline
\ \ \ \ \ \ \ \ \ \ \isacommand{ultimately}\isamarkupfalse%
\ \isacommand{have}\isamarkupfalse%
\ {\isacharasterisk}{\kern0pt}{\isacharcolon}{\kern0pt}\ {\isachardoublequoteopen}L{\isacharprime}{\kern0pt}\ x\ j\ {\isacharequal}{\kern0pt}\ L{\isacharprime}{\kern0pt}\ y\ j{\isachardoublequoteclose}\ \isakeyword{if}\ {\isachardoublequoteopen}x\ {\isacharless}{\kern0pt}\ t{\isachardoublequoteclose}\ {\isachardoublequoteopen}y\ {\isacharless}{\kern0pt}\ t{\isachardoublequoteclose}\ \isakeyword{for}\ x\ y\ \isacommand{using}\isamarkupfalse%
\ that\ \isacommand{by}\isamarkupfalse%
\ presburger\isanewline
\isanewline
\ \ \ \ \ \ \ \ \ \ \isacommand{have}\isamarkupfalse%
\ {\isachardoublequoteopen}L{\isacharprime}{\kern0pt}\ t\ j\ {\isacharequal}{\kern0pt}\ L{\isacharprime}{\kern0pt}\ {\isacharparenleft}{\kern0pt}t\ {\isacharminus}{\kern0pt}\ {\isadigit{1}}{\isacharparenright}{\kern0pt}\ j{\isachardoublequoteclose}\ \isacommand{using}\isamarkupfalse%
\ {\isacartoucheopen}j\ {\isasymin}\ B\ {\isadigit{1}}{\isacartoucheclose}\ \isacommand{by}\isamarkupfalse%
\ {\isacharparenleft}{\kern0pt}auto\ simp{\isacharcolon}{\kern0pt}\ L{\isacharprime}{\kern0pt}{\isacharunderscore}{\kern0pt}def{\isacharparenright}{\kern0pt}\isanewline
\ \ \ \ \ \ \ \ \ \ \isacommand{also}\isamarkupfalse%
\ \isacommand{have}\isamarkupfalse%
\ {\isachardoublequoteopen}{\isachardot}{\kern0pt}{\isachardot}{\kern0pt}{\isachardot}{\kern0pt}\ {\isacharequal}{\kern0pt}\ L{\isacharprime}{\kern0pt}\ x\ j{\isachardoublequoteclose}\ \isakeyword{if}\ {\isachardoublequoteopen}x\ {\isacharless}{\kern0pt}\ t{\isachardoublequoteclose}\ \isakeyword{for}\ x\ \isacommand{using}\isamarkupfalse%
\ {\isacharasterisk}{\kern0pt}\ \isacommand{by}\isamarkupfalse%
\ {\isacharparenleft}{\kern0pt}simp\ add{\isacharcolon}{\kern0pt}\ assms{\isacharparenleft}{\kern0pt}{\isadigit{1}}{\isacharparenright}{\kern0pt}\ that{\isacharparenright}{\kern0pt}\isanewline
\ \ \ \ \ \ \ \ \ \ \isacommand{finally}\isamarkupfalse%
\ \isacommand{have}\isamarkupfalse%
\ {\isacharasterisk}{\kern0pt}{\isacharasterisk}{\kern0pt}{\isacharcolon}{\kern0pt}\ {\isachardoublequoteopen}L{\isacharprime}{\kern0pt}\ t\ j\ {\isacharequal}{\kern0pt}\ L{\isacharprime}{\kern0pt}\ x\ j{\isachardoublequoteclose}\ \isakeyword{if}\ {\isachardoublequoteopen}x\ {\isacharless}{\kern0pt}\ t{\isachardoublequoteclose}\ \isakeyword{for}\ x\ \isacommand{using}\isamarkupfalse%
\ that\ \isacommand{by}\isamarkupfalse%
\ auto\isanewline
\ \ \ \ \ \ \ \ \ \ \isacommand{have}\isamarkupfalse%
\ {\isachardoublequoteopen}L{\isacharprime}{\kern0pt}\ x\ j\ {\isacharequal}{\kern0pt}\ L{\isacharprime}{\kern0pt}\ y\ j{\isachardoublequoteclose}\ \isakeyword{if}\ {\isachardoublequoteopen}x\ {\isacharless}{\kern0pt}\ t\ {\isacharplus}{\kern0pt}\ {\isadigit{1}}{\isachardoublequoteclose}\ {\isachardoublequoteopen}y\ {\isacharless}{\kern0pt}\ t\ {\isacharplus}{\kern0pt}\ {\isadigit{1}}{\isachardoublequoteclose}\ \isakeyword{for}\ x\ y\ \isanewline
\ \ \ \ \ \ \ \ \ \ \isacommand{proof}\isamarkupfalse%
{\isacharminus}{\kern0pt}\isanewline
\ \ \ \ \ \ \ \ \ \ \ \ \isacommand{consider}\isamarkupfalse%
\ {\isachardoublequoteopen}x\ {\isacharless}{\kern0pt}\ t\ {\isasymand}\ y\ {\isacharequal}{\kern0pt}\ t{\isachardoublequoteclose}\ {\isacharbar}{\kern0pt}\ {\isachardoublequoteopen}y\ {\isacharless}{\kern0pt}\ t\ {\isasymand}\ x\ {\isacharequal}{\kern0pt}\ t{\isachardoublequoteclose}\ {\isacharbar}{\kern0pt}\ {\isachardoublequoteopen}x\ {\isacharequal}{\kern0pt}\ t\ {\isasymand}\ y\ {\isacharequal}{\kern0pt}\ t{\isachardoublequoteclose}\ {\isacharbar}{\kern0pt}\ {\isachardoublequoteopen}x\ {\isacharless}{\kern0pt}\ t\ {\isasymand}\ y\ {\isacharless}{\kern0pt}\ t{\isachardoublequoteclose}\ \isacommand{using}\isamarkupfalse%
\ {\isacartoucheopen}x\ {\isacharless}{\kern0pt}\ t\ {\isacharplus}{\kern0pt}\ {\isadigit{1}}{\isacartoucheclose}\ {\isacartoucheopen}y\ {\isacharless}{\kern0pt}\ t\ {\isacharplus}{\kern0pt}\ {\isadigit{1}}{\isacartoucheclose}\ \isacommand{by}\isamarkupfalse%
\ linarith\isanewline
\ \ \ \ \ \ \ \ \ \ \ \ \isacommand{then}\isamarkupfalse%
\ \isacommand{show}\isamarkupfalse%
\ {\isachardoublequoteopen}L{\isacharprime}{\kern0pt}\ x\ j\ {\isacharequal}{\kern0pt}\ L{\isacharprime}{\kern0pt}\ y\ j{\isachardoublequoteclose}\ \isanewline
\ \ \ \ \ \ \ \ \ \ \ \ \isacommand{proof}\isamarkupfalse%
\ cases\isanewline
\ \ \ \ \ \ \ \ \ \ \ \ \ \ \isacommand{case}\isamarkupfalse%
\ {\isadigit{1}}\isanewline
\ \ \ \ \ \ \ \ \ \ \ \ \ \ \isacommand{then}\isamarkupfalse%
\ \isacommand{show}\isamarkupfalse%
\ {\isacharquery}{\kern0pt}thesis\ \isacommand{using}\isamarkupfalse%
\ {\isacharasterisk}{\kern0pt}{\isacharasterisk}{\kern0pt}\ \isacommand{by}\isamarkupfalse%
\ auto\isanewline
\ \ \ \ \ \ \ \ \ \ \ \ \isacommand{next}\isamarkupfalse%
\isanewline
\ \ \ \ \ \ \ \ \ \ \ \ \ \ \isacommand{case}\isamarkupfalse%
\ {\isadigit{2}}\isanewline
\ \ \ \ \ \ \ \ \ \ \ \ \ \ \isacommand{then}\isamarkupfalse%
\ \isacommand{show}\isamarkupfalse%
\ {\isacharquery}{\kern0pt}thesis\ \isacommand{using}\isamarkupfalse%
\ {\isacharasterisk}{\kern0pt}{\isacharasterisk}{\kern0pt}\ \isacommand{by}\isamarkupfalse%
\ auto\isanewline
\ \ \ \ \ \ \ \ \ \ \ \ \isacommand{next}\isamarkupfalse%
\isanewline
\ \ \ \ \ \ \ \ \ \ \ \ \ \ \isacommand{case}\isamarkupfalse%
\ {\isadigit{3}}\isanewline
\ \ \ \ \ \ \ \ \ \ \ \ \ \ \isacommand{then}\isamarkupfalse%
\ \isacommand{show}\isamarkupfalse%
\ {\isacharquery}{\kern0pt}thesis\ \isacommand{by}\isamarkupfalse%
\ simp\isanewline
\ \ \ \ \ \ \ \ \ \ \ \ \isacommand{next}\isamarkupfalse%
\isanewline
\ \ \ \ \ \ \ \ \ \ \ \ \ \ \isacommand{case}\isamarkupfalse%
\ {\isadigit{4}}\isanewline
\ \ \ \ \ \ \ \ \ \ \ \ \ \ \isacommand{then}\isamarkupfalse%
\ \isacommand{show}\isamarkupfalse%
\ {\isacharquery}{\kern0pt}thesis\ \isacommand{using}\isamarkupfalse%
\ {\isacharasterisk}{\kern0pt}\ \isacommand{by}\isamarkupfalse%
\ auto\isanewline
\ \ \ \ \ \ \ \ \ \ \ \ \isacommand{qed}\isamarkupfalse%
\isanewline
\ \ \ \ \ \ \ \ \ \ \isacommand{qed}\isamarkupfalse%
\isanewline
\ \ \ \ \ \ \ \ \ \ \isacommand{then}\isamarkupfalse%
\ \isacommand{show}\isamarkupfalse%
\ {\isacharquery}{\kern0pt}thesis\ \isacommand{by}\isamarkupfalse%
\ blast\isanewline
\ \ \ \ \ \ \ \ \isacommand{next}\isamarkupfalse%
\isanewline
\ \ \ \ \ \ \ \ \ \ \isacommand{case}\isamarkupfalse%
\ False\isanewline
\ \ \ \ \ \ \ \ \ \ \isacommand{then}\isamarkupfalse%
\ \isacommand{have}\isamarkupfalse%
\ {\isachardoublequoteopen}j\ {\isasymin}\ B\ {\isadigit{0}}{\isachardoublequoteclose}\ \isacommand{using}\isamarkupfalse%
\ B{\isacharunderscore}{\kern0pt}props\ {\isacartoucheopen}j\ {\isacharless}{\kern0pt}N{\isacharprime}{\kern0pt}{\isacartoucheclose}\ \isacommand{by}\isamarkupfalse%
\ auto\isanewline
\ \ \ \ \ \ \ \ \ \ \isacommand{then}\isamarkupfalse%
\ \isacommand{have}\isamarkupfalse%
\ {\isachardoublequoteopen}{\isasymforall}y\ {\isasymin}\ cube\ {\isadigit{1}}\ t{\isachardot}{\kern0pt}\ {\isacharparenleft}{\kern0pt}{\isacharparenleft}{\kern0pt}restrict\ {\isacharparenleft}{\kern0pt}{\isasymlambda}y{\isachardot}{\kern0pt}\ L\ {\isacharparenleft}{\kern0pt}y\ {\isadigit{0}}{\isacharparenright}{\kern0pt}{\isacharparenright}{\kern0pt}\ {\isacharparenleft}{\kern0pt}cube\ {\isadigit{1}}\ t{\isacharparenright}{\kern0pt}{\isacharparenright}{\kern0pt}\ y{\isacharparenright}{\kern0pt}\ j\ {\isacharequal}{\kern0pt}\ y\ {\isadigit{0}}{\isachardoublequoteclose}\ \isacommand{using}\isamarkupfalse%
\ {\isacartoucheopen}j\ {\isasymin}\ B\ {\isadigit{0}}{\isacartoucheclose}\ Bf{\isacharunderscore}{\kern0pt}defs\ \isacommand{by}\isamarkupfalse%
\ auto\isanewline
\ \ \ \ \ \ \ \ \ \ \isacommand{then}\isamarkupfalse%
\ \isacommand{have}\isamarkupfalse%
\ {\isachardoublequoteopen}{\isasymforall}y\ {\isasymin}\ cube\ {\isadigit{1}}\ t{\isachardot}{\kern0pt}\ L\ {\isacharparenleft}{\kern0pt}y\ {\isadigit{0}}{\isacharparenright}{\kern0pt}\ j\ {\isacharequal}{\kern0pt}\ y\ {\isadigit{0}}{\isachardoublequoteclose}\ \ \isacommand{by}\isamarkupfalse%
\ auto\isanewline
\ \ \ \ \ \ \ \ \ \ \isacommand{moreover}\isamarkupfalse%
\ \isacommand{have}\isamarkupfalse%
\ {\isachardoublequoteopen}{\isasymexists}{\isacharbang}{\kern0pt}y{\isachardot}{\kern0pt}\ y\ {\isasymin}\ cube\ {\isadigit{1}}\ t\ {\isasymand}\ y\ {\isadigit{0}}\ {\isacharequal}{\kern0pt}\ i{\isachardoublequoteclose}\ \isakeyword{if}\ {\isachardoublequoteopen}i\ {\isacharless}{\kern0pt}\ t{\isachardoublequoteclose}\ \isakeyword{for}\ i\ \isanewline
\ \ \ \ \ \ \ \ \ \ \isacommand{proof}\isamarkupfalse%
\ {\isacharparenleft}{\kern0pt}intro\ ex{\isadigit{1}}I{\isacharunderscore}{\kern0pt}alt{\isacharparenright}{\kern0pt}\isanewline
\ \ \ \ \ \ \ \ \ \ \ \ \isacommand{define}\isamarkupfalse%
\ y\ \isakeyword{where}\ {\isachardoublequoteopen}y\ {\isasymequiv}\ {\isacharparenleft}{\kern0pt}{\isasymlambda}x{\isacharcolon}{\kern0pt}{\isacharcolon}{\kern0pt}nat{\isachardot}{\kern0pt}\ {\isasymlambda}y{\isasymin}{\isacharbraceleft}{\kern0pt}{\isachardot}{\kern0pt}{\isachardot}{\kern0pt}{\isacharless}{\kern0pt}{\isadigit{1}}{\isacharcolon}{\kern0pt}{\isacharcolon}{\kern0pt}nat{\isacharbraceright}{\kern0pt}{\isachardot}{\kern0pt}\ x{\isacharparenright}{\kern0pt}{\isachardoublequoteclose}\ \isanewline
\ \ \ \ \ \ \ \ \ \ \ \ \isacommand{have}\isamarkupfalse%
\ {\isachardoublequoteopen}y\ i\ {\isasymin}\ {\isacharparenleft}{\kern0pt}cube\ {\isadigit{1}}\ t{\isacharparenright}{\kern0pt}{\isachardoublequoteclose}\ \isacommand{using}\isamarkupfalse%
\ that\ \isacommand{unfolding}\isamarkupfalse%
\ cube{\isacharunderscore}{\kern0pt}def\ y{\isacharunderscore}{\kern0pt}def\ \isacommand{by}\isamarkupfalse%
\ simp\isanewline
\ \ \ \ \ \ \ \ \ \ \ \ \isacommand{moreover}\isamarkupfalse%
\ \isacommand{have}\isamarkupfalse%
\ {\isachardoublequoteopen}y\ i\ {\isadigit{0}}\ {\isacharequal}{\kern0pt}\ i{\isachardoublequoteclose}\ \isacommand{unfolding}\isamarkupfalse%
\ y{\isacharunderscore}{\kern0pt}def\ \isacommand{by}\isamarkupfalse%
\ auto\isanewline
\ \ \ \ \ \ \ \ \ \ \ \ \isacommand{moreover}\isamarkupfalse%
\ \isacommand{have}\isamarkupfalse%
\ {\isachardoublequoteopen}z\ {\isacharequal}{\kern0pt}\ y\ i{\isachardoublequoteclose}\ \isakeyword{if}\ {\isachardoublequoteopen}z\ {\isasymin}\ cube\ {\isadigit{1}}\ t{\isachardoublequoteclose}\ \isakeyword{and}\ {\isachardoublequoteopen}z\ {\isadigit{0}}\ {\isacharequal}{\kern0pt}\ i{\isachardoublequoteclose}\ \isakeyword{for}\ z\isanewline
\ \ \ \ \ \ \ \ \ \ \ \ \isacommand{proof}\isamarkupfalse%
\ {\isacharparenleft}{\kern0pt}rule\ ccontr{\isacharparenright}{\kern0pt}\isanewline
\ \ \ \ \ \ \ \ \ \ \ \ \ \ \isacommand{assume}\isamarkupfalse%
\ {\isachardoublequoteopen}z\ {\isasymnoteq}\ y\ i{\isachardoublequoteclose}\ \isanewline
\ \ \ \ \ \ \ \ \ \ \ \ \ \ \isacommand{then}\isamarkupfalse%
\ \isacommand{obtain}\isamarkupfalse%
\ l\ \isakeyword{where}\ l{\isacharunderscore}{\kern0pt}prop{\isacharcolon}{\kern0pt}\ {\isachardoublequoteopen}z\ l\ {\isasymnoteq}\ y\ i\ l{\isachardoublequoteclose}\ \isacommand{by}\isamarkupfalse%
\ blast\isanewline
\ \ \ \ \ \ \ \ \ \ \ \ \ \ \isacommand{consider}\isamarkupfalse%
\ {\isachardoublequoteopen}l\ {\isasymin}\ {\isacharbraceleft}{\kern0pt}{\isachardot}{\kern0pt}{\isachardot}{\kern0pt}{\isacharless}{\kern0pt}{\isadigit{1}}{\isacharcolon}{\kern0pt}{\isacharcolon}{\kern0pt}nat{\isacharbraceright}{\kern0pt}{\isachardoublequoteclose}\ {\isacharbar}{\kern0pt}\ {\isachardoublequoteopen}l\ {\isasymnotin}\ {\isacharbraceleft}{\kern0pt}{\isachardot}{\kern0pt}{\isachardot}{\kern0pt}{\isacharless}{\kern0pt}{\isadigit{1}}{\isacharcolon}{\kern0pt}{\isacharcolon}{\kern0pt}nat{\isacharbraceright}{\kern0pt}{\isachardoublequoteclose}\ \isacommand{by}\isamarkupfalse%
\ blast\isanewline
\ \ \ \ \ \ \ \ \ \ \ \ \ \ \isacommand{then}\isamarkupfalse%
\ \isacommand{show}\isamarkupfalse%
\ False\isanewline
\ \ \ \ \ \ \ \ \ \ \ \ \ \ \isacommand{proof}\isamarkupfalse%
\ cases\isanewline
\ \ \ \ \ \ \ \ \ \ \ \ \ \ \ \ \isacommand{case}\isamarkupfalse%
\ {\isadigit{1}}\isanewline
\ \ \ \ \ \ \ \ \ \ \ \ \ \ \ \ \isacommand{then}\isamarkupfalse%
\ \isacommand{show}\isamarkupfalse%
\ {\isacharquery}{\kern0pt}thesis\ \isacommand{using}\isamarkupfalse%
\ l{\isacharunderscore}{\kern0pt}prop\ that{\isacharparenleft}{\kern0pt}{\isadigit{2}}{\isacharparenright}{\kern0pt}\ \isacommand{unfolding}\isamarkupfalse%
\ y{\isacharunderscore}{\kern0pt}def\ \isacommand{by}\isamarkupfalse%
\ auto\isanewline
\ \ \ \ \ \ \ \ \ \ \ \ \ \ \isacommand{next}\isamarkupfalse%
\isanewline
\ \ \ \ \ \ \ \ \ \ \ \ \ \ \ \ \isacommand{case}\isamarkupfalse%
\ {\isadigit{2}}\isanewline
\ \ \ \ \ \ \ \ \ \ \ \ \ \ \ \ \isacommand{then}\isamarkupfalse%
\ \isacommand{have}\isamarkupfalse%
\ {\isachardoublequoteopen}z\ l\ {\isacharequal}{\kern0pt}\ undefined{\isachardoublequoteclose}\ \isacommand{using}\isamarkupfalse%
\ that\ \isacommand{unfolding}\isamarkupfalse%
\ cube{\isacharunderscore}{\kern0pt}def\ \isacommand{by}\isamarkupfalse%
\ blast\isanewline
\ \ \ \ \ \ \ \ \ \ \ \ \ \ \ \ \isacommand{moreover}\isamarkupfalse%
\ \isacommand{have}\isamarkupfalse%
\ {\isachardoublequoteopen}y\ i\ l\ {\isacharequal}{\kern0pt}\ undefined{\isachardoublequoteclose}\ \isacommand{unfolding}\isamarkupfalse%
\ y{\isacharunderscore}{\kern0pt}def\ \isacommand{using}\isamarkupfalse%
\ {\isadigit{2}}\ \isacommand{by}\isamarkupfalse%
\ auto\isanewline
\ \ \ \ \ \ \ \ \ \ \ \ \ \ \ \ \isacommand{ultimately}\isamarkupfalse%
\ \isacommand{show}\isamarkupfalse%
\ {\isacharquery}{\kern0pt}thesis\ \isacommand{using}\isamarkupfalse%
\ l{\isacharunderscore}{\kern0pt}prop\ \isacommand{by}\isamarkupfalse%
\ presburger\isanewline
\ \ \ \ \ \ \ \ \ \ \ \ \ \ \isacommand{qed}\isamarkupfalse%
\isanewline
\ \ \ \ \ \ \ \ \ \ \ \ \isacommand{qed}\isamarkupfalse%
\isanewline
\ \ \ \ \ \ \ \ \ \ \ \ \isacommand{ultimately}\isamarkupfalse%
\ \isacommand{show}\isamarkupfalse%
\ {\isachardoublequoteopen}{\isasymexists}y{\isachardot}{\kern0pt}\ {\isacharparenleft}{\kern0pt}y\ {\isasymin}\ cube\ {\isadigit{1}}\ t\ {\isasymand}\ y\ {\isadigit{0}}\ {\isacharequal}{\kern0pt}\ i{\isacharparenright}{\kern0pt}\ {\isasymand}\ {\isacharparenleft}{\kern0pt}{\isasymforall}ya{\isachardot}{\kern0pt}\ ya\ {\isasymin}\ cube\ {\isadigit{1}}\ t\ {\isasymand}\ ya\ {\isadigit{0}}\ {\isacharequal}{\kern0pt}\ i\ {\isasymlongrightarrow}\ y\ {\isacharequal}{\kern0pt}\ ya{\isacharparenright}{\kern0pt}{\isachardoublequoteclose}\ \isacommand{by}\isamarkupfalse%
\ blast\isanewline
\isanewline
\ \ \ \ \ \ \ \ \ \ \isacommand{qed}\isamarkupfalse%
\isanewline
\ \ \ \ \ \ \ \ \ \ \isacommand{ultimately}\isamarkupfalse%
\ \isacommand{have}\isamarkupfalse%
\ {\isachardoublequoteopen}L\ s\ j\ {\isacharequal}{\kern0pt}\ s{\isachardoublequoteclose}\ \isakeyword{if}\ {\isachardoublequoteopen}s\ {\isacharless}{\kern0pt}\ t{\isachardoublequoteclose}\ \isakeyword{for}\ s\ \isacommand{using}\isamarkupfalse%
\ that\ \isacommand{by}\isamarkupfalse%
\ blast\isanewline
\ \ \ \ \ \ \ \ \ \ \isacommand{then}\isamarkupfalse%
\ \isacommand{have}\isamarkupfalse%
\ {\isachardoublequoteopen}L{\isacharprime}{\kern0pt}\ s\ j\ {\isacharequal}{\kern0pt}\ s{\isachardoublequoteclose}\ \isakeyword{if}\ {\isachardoublequoteopen}s\ {\isacharless}{\kern0pt}\ t{\isachardoublequoteclose}\ \isakeyword{for}\ s\ \isacommand{using}\isamarkupfalse%
\ that\ \isacommand{by}\isamarkupfalse%
\ {\isacharparenleft}{\kern0pt}auto\ simp{\isacharcolon}{\kern0pt}\ L{\isacharprime}{\kern0pt}{\isacharunderscore}{\kern0pt}def{\isacharparenright}{\kern0pt}\isanewline
\ \ \ \ \ \ \ \ \ \ \isacommand{moreover}\isamarkupfalse%
\ \isacommand{have}\isamarkupfalse%
\ {\isachardoublequoteopen}L{\isacharprime}{\kern0pt}\ t\ j\ {\isacharequal}{\kern0pt}\ t{\isachardoublequoteclose}\ \isacommand{using}\isamarkupfalse%
\ False\ {\isacartoucheopen}j\ {\isasymin}\ B\ {\isadigit{0}}{\isacartoucheclose}\ \isacommand{by}\isamarkupfalse%
\ {\isacharparenleft}{\kern0pt}auto\ simp{\isacharcolon}{\kern0pt}\ L{\isacharprime}{\kern0pt}{\isacharunderscore}{\kern0pt}def{\isacharparenright}{\kern0pt}\isanewline
\ \ \ \ \ \ \ \ \ \ \isacommand{ultimately}\isamarkupfalse%
\ \isacommand{have}\isamarkupfalse%
\ {\isachardoublequoteopen}L{\isacharprime}{\kern0pt}\ s\ j\ {\isacharequal}{\kern0pt}\ s{\isachardoublequoteclose}\ \isakeyword{if}\ {\isachardoublequoteopen}s\ {\isacharless}{\kern0pt}\ t{\isacharplus}{\kern0pt}{\isadigit{1}}{\isachardoublequoteclose}\ \isakeyword{for}\ s\ \isacommand{using}\isamarkupfalse%
\ that\ \isacommand{by}\isamarkupfalse%
\ {\isacharparenleft}{\kern0pt}auto\ simp{\isacharcolon}{\kern0pt}\ L{\isacharprime}{\kern0pt}{\isacharunderscore}{\kern0pt}def{\isacharparenright}{\kern0pt}\isanewline
\ \ \ \ \ \ \ \ \ \ \isacommand{then}\isamarkupfalse%
\ \isacommand{show}\isamarkupfalse%
\ {\isacharquery}{\kern0pt}thesis\ \isacommand{by}\isamarkupfalse%
\ blast\isanewline
\ \ \ \ \ \ \ \ \isacommand{qed}\isamarkupfalse%
\isanewline
\isanewline
\isanewline
\isanewline
\isanewline
\ \ \ \ \ \ \isacommand{qed}\isamarkupfalse%
\isanewline
\ \ \ \ \ \ \isacommand{from}\isamarkupfalse%
\ A{\isadigit{1}}\ A{\isadigit{2}}\ A{\isadigit{3}}\ \isacommand{show}\isamarkupfalse%
\ {\isacharquery}{\kern0pt}thesis\ \isacommand{unfolding}\isamarkupfalse%
\ is{\isacharunderscore}{\kern0pt}line{\isacharunderscore}{\kern0pt}def\ \isacommand{by}\isamarkupfalse%
\ simp\isanewline
\isanewline
\isanewline
\ \ \ \ \isacommand{qed}\isamarkupfalse%
\isanewline
\ \ \ \ \isacommand{then}\isamarkupfalse%
\ \isacommand{have}\isamarkupfalse%
\ F{\isadigit{1}}{\isacharcolon}{\kern0pt}\ {\isachardoublequoteopen}is{\isacharunderscore}{\kern0pt}subspace\ {\isacharparenleft}{\kern0pt}restrict\ {\isacharparenleft}{\kern0pt}{\isasymlambda}y{\isachardot}{\kern0pt}\ L{\isacharprime}{\kern0pt}\ {\isacharparenleft}{\kern0pt}y\ {\isadigit{0}}{\isacharparenright}{\kern0pt}{\isacharparenright}{\kern0pt}\ {\isacharparenleft}{\kern0pt}cube\ {\isadigit{1}}\ {\isacharparenleft}{\kern0pt}t\ {\isacharplus}{\kern0pt}\ {\isadigit{1}}{\isacharparenright}{\kern0pt}{\isacharparenright}{\kern0pt}{\isacharparenright}{\kern0pt}\ {\isadigit{1}}\ N{\isacharprime}{\kern0pt}\ {\isacharparenleft}{\kern0pt}t\ {\isacharplus}{\kern0pt}\ {\isadigit{1}}{\isacharparenright}{\kern0pt}{\isachardoublequoteclose}\ \isacommand{using}\isamarkupfalse%
\ line{\isacharunderscore}{\kern0pt}is{\isacharunderscore}{\kern0pt}dim{\isadigit{1}}{\isacharunderscore}{\kern0pt}subspace{\isacharbrackleft}{\kern0pt}of\ {\isachardoublequoteopen}N{\isacharprime}{\kern0pt}{\isachardoublequoteclose}\ {\isachardoublequoteopen}t{\isacharplus}{\kern0pt}{\isadigit{1}}{\isachardoublequoteclose}{\isacharbrackright}{\kern0pt}\ N{\isacharprime}{\kern0pt}{\isacharunderscore}{\kern0pt}props\ assms{\isacharparenleft}{\kern0pt}{\isadigit{1}}{\isacharparenright}{\kern0pt}\ \isacommand{by}\isamarkupfalse%
\ force\isanewline
\isanewline
\ \ \ \ \isacommand{define}\isamarkupfalse%
\ S{\isadigit{1}}\ \isakeyword{where}\ {\isachardoublequoteopen}S{\isadigit{1}}\ {\isasymequiv}\ {\isacharparenleft}{\kern0pt}restrict\ {\isacharparenleft}{\kern0pt}{\isasymlambda}y{\isachardot}{\kern0pt}\ L{\isacharprime}{\kern0pt}\ {\isacharparenleft}{\kern0pt}y\ {\isacharparenleft}{\kern0pt}{\isadigit{0}}{\isacharcolon}{\kern0pt}{\isacharcolon}{\kern0pt}nat{\isacharparenright}{\kern0pt}{\isacharparenright}{\kern0pt}{\isacharparenright}{\kern0pt}\ {\isacharparenleft}{\kern0pt}cube\ {\isadigit{1}}\ {\isacharparenleft}{\kern0pt}t{\isacharplus}{\kern0pt}{\isadigit{1}}{\isacharparenright}{\kern0pt}{\isacharparenright}{\kern0pt}{\isacharparenright}{\kern0pt}{\isachardoublequoteclose}\isanewline
\ \ \ \ \isacommand{have}\isamarkupfalse%
\ F{\isadigit{2}}{\isacharcolon}{\kern0pt}\ {\isachardoublequoteopen}{\isasymexists}c\ {\isacharless}{\kern0pt}\ r{\isachardot}{\kern0pt}\ {\isacharparenleft}{\kern0pt}{\isasymforall}x\ {\isasymin}\ classes\ {\isadigit{1}}\ t\ i{\isachardot}{\kern0pt}\ {\isasymchi}\ {\isacharparenleft}{\kern0pt}S{\isadigit{1}}\ x{\isacharparenright}{\kern0pt}\ {\isacharequal}{\kern0pt}\ c{\isacharparenright}{\kern0pt}{\isachardoublequoteclose}\ \isakeyword{if}\ {\isachardoublequoteopen}i\ {\isasymle}\ {\isadigit{1}}{\isachardoublequoteclose}\ \isakeyword{for}\ i\isanewline
\ \ \ \ \isacommand{proof}\isamarkupfalse%
{\isacharminus}{\kern0pt}\isanewline
\ \ \ \ \ \ \isacommand{have}\isamarkupfalse%
\ {\isachardoublequoteopen}{\isasymexists}c\ {\isacharless}{\kern0pt}\ r{\isachardot}{\kern0pt}\ {\isacharparenleft}{\kern0pt}{\isasymforall}y\ {\isasymin}\ L{\isacharprime}{\kern0pt}\ {\isacharbackquote}{\kern0pt}\ {\isacharbraceleft}{\kern0pt}{\isachardot}{\kern0pt}{\isachardot}{\kern0pt}{\isacharless}{\kern0pt}t{\isacharbraceright}{\kern0pt}{\isachardot}{\kern0pt}\ {\isacharquery}{\kern0pt}chi{\isacharunderscore}{\kern0pt}t\ y\ {\isacharequal}{\kern0pt}\ c{\isacharparenright}{\kern0pt}{\isachardoublequoteclose}\ \isacommand{unfolding}\isamarkupfalse%
\ L{\isacharprime}{\kern0pt}{\isacharunderscore}{\kern0pt}def\ \isacommand{using}\isamarkupfalse%
\ L{\isacharunderscore}{\kern0pt}def\ \isacommand{by}\isamarkupfalse%
\ fastforce\isanewline
\ \ \ \ \ \ \isacommand{have}\isamarkupfalse%
\ {\isachardoublequoteopen}{\isasymforall}x\ {\isasymin}\ {\isacharparenleft}{\kern0pt}L\ {\isacharbackquote}{\kern0pt}\ {\isacharbraceleft}{\kern0pt}{\isachardot}{\kern0pt}{\isachardot}{\kern0pt}{\isacharless}{\kern0pt}t{\isacharbraceright}{\kern0pt}{\isacharparenright}{\kern0pt}{\isachardot}{\kern0pt}\ x\ {\isasymin}\ cube\ N{\isacharprime}{\kern0pt}\ t{\isachardoublequoteclose}\ \isacommand{using}\isamarkupfalse%
\ L{\isacharunderscore}{\kern0pt}def\ \isanewline
\ \ \ \ \ \ \ \ \isacommand{using}\isamarkupfalse%
\ line{\isacharunderscore}{\kern0pt}points{\isacharunderscore}{\kern0pt}in{\isacharunderscore}{\kern0pt}cube\ \isacommand{by}\isamarkupfalse%
\ blast\isanewline
\ \ \ \ \ \ \isacommand{then}\isamarkupfalse%
\ \isacommand{have}\isamarkupfalse%
\ {\isachardoublequoteopen}{\isasymforall}x\ {\isasymin}\ {\isacharparenleft}{\kern0pt}L{\isacharprime}{\kern0pt}\ {\isacharbackquote}{\kern0pt}\ {\isacharbraceleft}{\kern0pt}{\isachardot}{\kern0pt}{\isachardot}{\kern0pt}{\isacharless}{\kern0pt}t{\isacharbraceright}{\kern0pt}{\isacharparenright}{\kern0pt}{\isachardot}{\kern0pt}\ x\ {\isasymin}\ cube\ N{\isacharprime}{\kern0pt}\ t{\isachardoublequoteclose}\ \isacommand{by}\isamarkupfalse%
\ {\isacharparenleft}{\kern0pt}auto\ simp{\isacharcolon}{\kern0pt}\ L{\isacharprime}{\kern0pt}{\isacharunderscore}{\kern0pt}def{\isacharparenright}{\kern0pt}\isanewline
\ \ \ \ \ \ \isacommand{then}\isamarkupfalse%
\ \isacommand{have}\isamarkupfalse%
\ {\isacharasterisk}{\kern0pt}{\isacharcolon}{\kern0pt}{\isachardoublequoteopen}{\isasymforall}x\ {\isasymin}\ {\isacharparenleft}{\kern0pt}L{\isacharprime}{\kern0pt}\ {\isacharbackquote}{\kern0pt}\ {\isacharbraceleft}{\kern0pt}{\isachardot}{\kern0pt}{\isachardot}{\kern0pt}{\isacharless}{\kern0pt}t{\isacharbraceright}{\kern0pt}{\isacharparenright}{\kern0pt}{\isachardot}{\kern0pt}\ {\isasymchi}\ x\ {\isacharequal}{\kern0pt}\ {\isacharquery}{\kern0pt}chi{\isacharunderscore}{\kern0pt}t\ x{\isachardoublequoteclose}\ \isacommand{by}\isamarkupfalse%
\ simp\isanewline
\ \ \ \ \ \ \isacommand{then}\isamarkupfalse%
\ \isacommand{have}\isamarkupfalse%
\ {\isachardoublequoteopen}{\isacharquery}{\kern0pt}chi{\isacharunderscore}{\kern0pt}t\ {\isacharbackquote}{\kern0pt}\ {\isacharparenleft}{\kern0pt}L{\isacharprime}{\kern0pt}\ {\isacharbackquote}{\kern0pt}\ {\isacharbraceleft}{\kern0pt}{\isachardot}{\kern0pt}{\isachardot}{\kern0pt}{\isacharless}{\kern0pt}t{\isacharbraceright}{\kern0pt}{\isacharparenright}{\kern0pt}\ {\isacharequal}{\kern0pt}\ {\isasymchi}\ {\isacharbackquote}{\kern0pt}\ {\isacharparenleft}{\kern0pt}L{\isacharprime}{\kern0pt}\ {\isacharbackquote}{\kern0pt}\ {\isacharbraceleft}{\kern0pt}{\isachardot}{\kern0pt}{\isachardot}{\kern0pt}{\isacharless}{\kern0pt}t{\isacharbraceright}{\kern0pt}{\isacharparenright}{\kern0pt}{\isachardoublequoteclose}\ \isacommand{by}\isamarkupfalse%
\ force\isanewline
\ \ \ \ \ \ \isacommand{then}\isamarkupfalse%
\ \isacommand{have}\isamarkupfalse%
\ {\isachardoublequoteopen}{\isasymexists}c\ {\isacharless}{\kern0pt}\ r{\isachardot}{\kern0pt}\ {\isacharparenleft}{\kern0pt}{\isasymforall}y\ {\isasymin}\ L{\isacharprime}{\kern0pt}\ {\isacharbackquote}{\kern0pt}\ {\isacharbraceleft}{\kern0pt}{\isachardot}{\kern0pt}{\isachardot}{\kern0pt}{\isacharless}{\kern0pt}t{\isacharbraceright}{\kern0pt}{\isachardot}{\kern0pt}\ {\isasymchi}\ y\ {\isacharequal}{\kern0pt}\ c{\isacharparenright}{\kern0pt}{\isachardoublequoteclose}\ \isacommand{using}\isamarkupfalse%
\ {\isacartoucheopen}{\isasymexists}c\ {\isacharless}{\kern0pt}\ r{\isachardot}{\kern0pt}\ {\isacharparenleft}{\kern0pt}{\isasymforall}y\ {\isasymin}\ L{\isacharprime}{\kern0pt}\ {\isacharbackquote}{\kern0pt}\ {\isacharbraceleft}{\kern0pt}{\isachardot}{\kern0pt}{\isachardot}{\kern0pt}{\isacharless}{\kern0pt}t{\isacharbraceright}{\kern0pt}{\isachardot}{\kern0pt}\ {\isacharquery}{\kern0pt}chi{\isacharunderscore}{\kern0pt}t\ y\ {\isacharequal}{\kern0pt}\ c{\isacharparenright}{\kern0pt}{\isacartoucheclose}\ \isacommand{by}\isamarkupfalse%
\ fastforce\isanewline
\ \ \ \ \ \ \isacommand{then}\isamarkupfalse%
\ \isacommand{obtain}\isamarkupfalse%
\ linecol\ \isakeyword{where}\ lc{\isacharunderscore}{\kern0pt}def{\isacharcolon}{\kern0pt}\ {\isachardoublequoteopen}linecol\ {\isacharless}{\kern0pt}\ r\ {\isasymand}\ {\isacharparenleft}{\kern0pt}{\isasymforall}y\ {\isasymin}\ L{\isacharprime}{\kern0pt}\ {\isacharbackquote}{\kern0pt}\ {\isacharbraceleft}{\kern0pt}{\isachardot}{\kern0pt}{\isachardot}{\kern0pt}{\isacharless}{\kern0pt}t{\isacharbraceright}{\kern0pt}{\isachardot}{\kern0pt}\ {\isasymchi}\ y\ {\isacharequal}{\kern0pt}\ linecol{\isacharparenright}{\kern0pt}{\isachardoublequoteclose}\ \isacommand{by}\isamarkupfalse%
\ blast\isanewline
\ \ \ \ \ \ \isacommand{consider}\isamarkupfalse%
\ {\isachardoublequoteopen}i\ {\isacharequal}{\kern0pt}\ {\isadigit{0}}{\isachardoublequoteclose}\ {\isacharbar}{\kern0pt}\ {\isachardoublequoteopen}i\ {\isacharequal}{\kern0pt}\ {\isadigit{1}}{\isachardoublequoteclose}\ \isacommand{using}\isamarkupfalse%
\ {\isacartoucheopen}i\ {\isasymle}\ {\isadigit{1}}{\isacartoucheclose}\ \isacommand{by}\isamarkupfalse%
\ linarith\isanewline
\ \ \ \ \ \ \isacommand{then}\isamarkupfalse%
\ \isacommand{show}\isamarkupfalse%
\ {\isachardoublequoteopen}{\isasymexists}c\ {\isacharless}{\kern0pt}\ r{\isachardot}{\kern0pt}\ {\isacharparenleft}{\kern0pt}{\isasymforall}x\ {\isasymin}\ classes\ {\isadigit{1}}\ t\ i{\isachardot}{\kern0pt}\ {\isasymchi}\ {\isacharparenleft}{\kern0pt}S{\isadigit{1}}\ x{\isacharparenright}{\kern0pt}\ {\isacharequal}{\kern0pt}\ c{\isacharparenright}{\kern0pt}{\isachardoublequoteclose}\isanewline
\ \ \ \ \ \ \isacommand{proof}\isamarkupfalse%
\ {\isacharparenleft}{\kern0pt}cases{\isacharparenright}{\kern0pt}\isanewline
\ \ \ \ \ \ \ \ \isacommand{case}\isamarkupfalse%
\ {\isadigit{1}}\isanewline
\ \ \ \ \ \ \ \ \isacommand{assume}\isamarkupfalse%
\ {\isachardoublequoteopen}i\ {\isacharequal}{\kern0pt}\ {\isadigit{0}}{\isachardoublequoteclose}\isanewline
\ \ \ \ \ \ \ \ \isacommand{have}\isamarkupfalse%
\ {\isacharasterisk}{\kern0pt}{\isacharcolon}{\kern0pt}\ {\isachardoublequoteopen}{\isasymforall}a\ t{\isachardot}{\kern0pt}\ a\ {\isasymin}\ {\isacharbraceleft}{\kern0pt}{\isachardot}{\kern0pt}{\isachardot}{\kern0pt}{\isacharless}{\kern0pt}t{\isacharplus}{\kern0pt}{\isadigit{1}}{\isacharbraceright}{\kern0pt}\ {\isasymand}\ a\ {\isasymnoteq}\ t\ {\isasymlongleftrightarrow}\ a\ {\isasymin}\ {\isacharbraceleft}{\kern0pt}{\isachardot}{\kern0pt}{\isachardot}{\kern0pt}{\isacharless}{\kern0pt}{\isacharparenleft}{\kern0pt}t{\isacharcolon}{\kern0pt}{\isacharcolon}{\kern0pt}nat{\isacharparenright}{\kern0pt}{\isacharbraceright}{\kern0pt}{\isachardoublequoteclose}\ \isacommand{by}\isamarkupfalse%
\ auto\isanewline
\ \ \ \ \ \ \ \ \isacommand{from}\isamarkupfalse%
\ {\isacartoucheopen}i\ {\isacharequal}{\kern0pt}\ {\isadigit{0}}{\isacartoucheclose}\ \isacommand{have}\isamarkupfalse%
\ {\isachardoublequoteopen}classes\ {\isadigit{1}}\ t\ {\isadigit{0}}\ {\isacharequal}{\kern0pt}\ {\isacharbraceleft}{\kern0pt}x\ {\isachardot}{\kern0pt}\ x\ {\isasymin}\ {\isacharparenleft}{\kern0pt}cube\ {\isadigit{1}}\ {\isacharparenleft}{\kern0pt}t\ {\isacharplus}{\kern0pt}\ {\isadigit{1}}{\isacharparenright}{\kern0pt}{\isacharparenright}{\kern0pt}\ {\isasymand}\ {\isacharparenleft}{\kern0pt}{\isasymforall}u\ {\isasymin}\ {\isacharbraceleft}{\kern0pt}{\isacharparenleft}{\kern0pt}{\isacharparenleft}{\kern0pt}{\isadigit{1}}{\isacharcolon}{\kern0pt}{\isacharcolon}{\kern0pt}nat{\isacharparenright}{\kern0pt}\ {\isacharminus}{\kern0pt}\ {\isadigit{0}}{\isacharparenright}{\kern0pt}{\isachardot}{\kern0pt}{\isachardot}{\kern0pt}{\isacharless}{\kern0pt}{\isadigit{1}}{\isacharbraceright}{\kern0pt}{\isachardot}{\kern0pt}\ x\ u\ {\isacharequal}{\kern0pt}\ t{\isacharparenright}{\kern0pt}\ {\isasymand}\ t\ {\isasymnotin}\ x\ {\isacharbackquote}{\kern0pt}\ {\isacharbraceleft}{\kern0pt}{\isachardot}{\kern0pt}{\isachardot}{\kern0pt}{\isacharless}{\kern0pt}{\isacharparenleft}{\kern0pt}{\isadigit{1}}\ {\isacharminus}{\kern0pt}\ {\isacharparenleft}{\kern0pt}{\isadigit{0}}{\isacharcolon}{\kern0pt}{\isacharcolon}{\kern0pt}nat{\isacharparenright}{\kern0pt}{\isacharparenright}{\kern0pt}{\isacharbraceright}{\kern0pt}{\isacharbraceright}{\kern0pt}{\isachardoublequoteclose}\ \isacommand{using}\isamarkupfalse%
\ classes{\isacharunderscore}{\kern0pt}def\ \isacommand{by}\isamarkupfalse%
\ simp\isanewline
\ \ \ \ \ \ \ \ \isacommand{also}\isamarkupfalse%
\ \isacommand{have}\isamarkupfalse%
\ {\isachardoublequoteopen}{\isachardot}{\kern0pt}{\isachardot}{\kern0pt}{\isachardot}{\kern0pt}\ {\isacharequal}{\kern0pt}\ {\isacharbraceleft}{\kern0pt}x\ {\isachardot}{\kern0pt}\ x\ {\isasymin}\ cube\ {\isadigit{1}}\ {\isacharparenleft}{\kern0pt}t{\isacharplus}{\kern0pt}{\isadigit{1}}{\isacharparenright}{\kern0pt}\ {\isasymand}\ t\ {\isasymnotin}\ x\ {\isacharbackquote}{\kern0pt}\ {\isacharbraceleft}{\kern0pt}{\isachardot}{\kern0pt}{\isachardot}{\kern0pt}{\isacharless}{\kern0pt}{\isacharparenleft}{\kern0pt}{\isadigit{1}}{\isacharcolon}{\kern0pt}{\isacharcolon}{\kern0pt}nat{\isacharparenright}{\kern0pt}{\isacharbraceright}{\kern0pt}{\isacharbraceright}{\kern0pt}{\isachardoublequoteclose}\ \isacommand{by}\isamarkupfalse%
\ simp\isanewline
\ \ \ \ \ \ \ \ \isacommand{also}\isamarkupfalse%
\ \isacommand{have}\isamarkupfalse%
\ {\isachardoublequoteopen}{\isachardot}{\kern0pt}{\isachardot}{\kern0pt}{\isachardot}{\kern0pt}\ {\isacharequal}{\kern0pt}\ {\isacharbraceleft}{\kern0pt}x\ {\isachardot}{\kern0pt}\ x\ {\isasymin}\ cube\ {\isadigit{1}}\ {\isacharparenleft}{\kern0pt}t{\isacharplus}{\kern0pt}{\isadigit{1}}{\isacharparenright}{\kern0pt}\ {\isasymand}\ {\isacharparenleft}{\kern0pt}x\ {\isadigit{0}}\ {\isasymnoteq}\ t{\isacharparenright}{\kern0pt}{\isacharbraceright}{\kern0pt}{\isachardoublequoteclose}\ \isacommand{by}\isamarkupfalse%
\ blast\ \isanewline
\ \ \ \ \ \ \ \ \isacommand{also}\isamarkupfalse%
\ \isacommand{have}\isamarkupfalse%
\ {\isachardoublequoteopen}\ {\isachardot}{\kern0pt}{\isachardot}{\kern0pt}{\isachardot}{\kern0pt}\ {\isacharequal}{\kern0pt}\ {\isacharbraceleft}{\kern0pt}x\ {\isachardot}{\kern0pt}\ x\ {\isasymin}\ cube\ {\isadigit{1}}\ {\isacharparenleft}{\kern0pt}t{\isacharplus}{\kern0pt}{\isadigit{1}}{\isacharparenright}{\kern0pt}\ {\isasymand}\ {\isacharparenleft}{\kern0pt}x\ {\isadigit{0}}\ {\isasymin}\ {\isacharbraceleft}{\kern0pt}{\isachardot}{\kern0pt}{\isachardot}{\kern0pt}{\isacharless}{\kern0pt}t{\isacharplus}{\kern0pt}{\isadigit{1}}{\isacharbraceright}{\kern0pt}\ {\isasymand}\ x\ {\isadigit{0}}\ {\isasymnoteq}\ t{\isacharparenright}{\kern0pt}{\isacharbraceright}{\kern0pt}{\isachardoublequoteclose}\ \isacommand{unfolding}\isamarkupfalse%
\ cube{\isacharunderscore}{\kern0pt}def\ \isacommand{by}\isamarkupfalse%
\ blast\isanewline
\ \ \ \ \ \ \ \ \isacommand{also}\isamarkupfalse%
\ \isacommand{have}\isamarkupfalse%
\ {\isachardoublequoteopen}\ {\isachardot}{\kern0pt}{\isachardot}{\kern0pt}{\isachardot}{\kern0pt}\ {\isacharequal}{\kern0pt}\ {\isacharbraceleft}{\kern0pt}x\ {\isachardot}{\kern0pt}\ x\ {\isasymin}\ cube\ {\isadigit{1}}\ {\isacharparenleft}{\kern0pt}t{\isacharplus}{\kern0pt}{\isadigit{1}}{\isacharparenright}{\kern0pt}\ {\isasymand}\ {\isacharparenleft}{\kern0pt}x\ {\isadigit{0}}\ {\isasymin}\ {\isacharbraceleft}{\kern0pt}{\isachardot}{\kern0pt}{\isachardot}{\kern0pt}{\isacharless}{\kern0pt}t{\isacharbraceright}{\kern0pt}{\isacharparenright}{\kern0pt}{\isacharbraceright}{\kern0pt}{\isachardoublequoteclose}\ \isacommand{using}\isamarkupfalse%
\ {\isacharasterisk}{\kern0pt}\ \isacommand{by}\isamarkupfalse%
\ simp\isanewline
\ \ \ \ \ \ \ \ \isacommand{finally}\isamarkupfalse%
\ \isacommand{have}\isamarkupfalse%
\ redef{\isacharcolon}{\kern0pt}\ {\isachardoublequoteopen}classes\ {\isadigit{1}}\ t\ {\isadigit{0}}\ {\isacharequal}{\kern0pt}\ {\isacharbraceleft}{\kern0pt}x\ {\isachardot}{\kern0pt}\ x\ {\isasymin}\ cube\ {\isadigit{1}}\ {\isacharparenleft}{\kern0pt}t{\isacharplus}{\kern0pt}{\isadigit{1}}{\isacharparenright}{\kern0pt}\ {\isasymand}\ {\isacharparenleft}{\kern0pt}x\ {\isadigit{0}}\ {\isasymin}\ {\isacharbraceleft}{\kern0pt}{\isachardot}{\kern0pt}{\isachardot}{\kern0pt}{\isacharless}{\kern0pt}t{\isacharbraceright}{\kern0pt}{\isacharparenright}{\kern0pt}{\isacharbraceright}{\kern0pt}{\isachardoublequoteclose}\ \isacommand{by}\isamarkupfalse%
\ simp\isanewline
\ \ \ \ \ \ \ \ \isacommand{have}\isamarkupfalse%
\ {\isachardoublequoteopen}{\isacharbraceleft}{\kern0pt}x\ {\isadigit{0}}\ {\isacharbar}{\kern0pt}\ x\ {\isachardot}{\kern0pt}\ x\ {\isasymin}\ classes\ {\isadigit{1}}\ t\ {\isadigit{0}}{\isacharbraceright}{\kern0pt}\ {\isasymsubseteq}\ {\isacharbraceleft}{\kern0pt}{\isachardot}{\kern0pt}{\isachardot}{\kern0pt}{\isacharless}{\kern0pt}t{\isacharbraceright}{\kern0pt}{\isachardoublequoteclose}\ \isacommand{using}\isamarkupfalse%
\ redef\ \isacommand{by}\isamarkupfalse%
\ auto\isanewline
\ \ \ \ \ \ \ \ \isacommand{moreover}\isamarkupfalse%
\ \isacommand{have}\isamarkupfalse%
\ {\isachardoublequoteopen}{\isacharbraceleft}{\kern0pt}{\isachardot}{\kern0pt}{\isachardot}{\kern0pt}{\isacharless}{\kern0pt}t{\isacharbraceright}{\kern0pt}\ {\isasymsubseteq}\ {\isacharbraceleft}{\kern0pt}x\ {\isadigit{0}}\ {\isacharbar}{\kern0pt}\ x\ {\isachardot}{\kern0pt}\ x\ {\isasymin}\ classes\ {\isadigit{1}}\ t\ {\isadigit{0}}{\isacharbraceright}{\kern0pt}{\isachardoublequoteclose}\ \isanewline
\ \ \ \ \ \ \ \ \isacommand{proof}\isamarkupfalse%
\isanewline
\ \ \ \ \ \ \ \ \ \ \isacommand{fix}\isamarkupfalse%
\ x\ \isacommand{assume}\isamarkupfalse%
\ x{\isacharcolon}{\kern0pt}\ {\isachardoublequoteopen}x\ {\isasymin}\ {\isacharbraceleft}{\kern0pt}{\isachardot}{\kern0pt}{\isachardot}{\kern0pt}{\isacharless}{\kern0pt}t{\isacharbraceright}{\kern0pt}{\isachardoublequoteclose}\isanewline
\ \ \ \ \ \ \ \ \ \ \isacommand{hence}\isamarkupfalse%
\ {\isachardoublequoteopen}{\isasymexists}a{\isasymin}cube\ {\isadigit{1}}\ t{\isachardot}{\kern0pt}\ a\ {\isadigit{0}}\ {\isacharequal}{\kern0pt}\ x{\isachardoublequoteclose}\isanewline
\ \ \ \ \ \ \ \ \ \ \ \ \isacommand{unfolding}\isamarkupfalse%
\ cube{\isacharunderscore}{\kern0pt}def\ \isacommand{by}\isamarkupfalse%
\ {\isacharparenleft}{\kern0pt}intro\ fun{\isacharunderscore}{\kern0pt}ex{\isacharparenright}{\kern0pt}\ auto\isanewline
\ \ \ \ \ \ \ \ \ \ \isacommand{then}\isamarkupfalse%
\ \isacommand{show}\isamarkupfalse%
\ {\isachardoublequoteopen}x\ {\isasymin}\ {\isacharbraceleft}{\kern0pt}x\ {\isadigit{0}}\ {\isacharbar}{\kern0pt}x{\isachardot}{\kern0pt}\ x\ {\isasymin}\ classes\ {\isadigit{1}}\ t\ {\isadigit{0}}{\isacharbraceright}{\kern0pt}{\isachardoublequoteclose}\isanewline
\ \ \ \ \ \ \ \ \ \ \ \ \isacommand{using}\isamarkupfalse%
\ x\ cube{\isacharunderscore}{\kern0pt}subset\ \isacommand{unfolding}\isamarkupfalse%
\ redef\ \isacommand{by}\isamarkupfalse%
\ auto\isanewline
\ \ \ \ \ \ \ \ \isacommand{qed}\isamarkupfalse%
\isanewline
\ \ \ \ \ \ \ \ \isacommand{ultimately}\isamarkupfalse%
\ \isacommand{have}\isamarkupfalse%
\ {\isacharasterisk}{\kern0pt}{\isacharasterisk}{\kern0pt}{\isacharcolon}{\kern0pt}\ {\isachardoublequoteopen}{\isacharbraceleft}{\kern0pt}x\ {\isadigit{0}}\ {\isacharbar}{\kern0pt}\ x\ {\isachardot}{\kern0pt}\ x\ {\isasymin}\ classes\ {\isadigit{1}}\ t\ {\isadigit{0}}{\isacharbraceright}{\kern0pt}\ {\isacharequal}{\kern0pt}\ {\isacharbraceleft}{\kern0pt}{\isachardot}{\kern0pt}{\isachardot}{\kern0pt}{\isacharless}{\kern0pt}t{\isacharbraceright}{\kern0pt}{\isachardoublequoteclose}\ \isacommand{by}\isamarkupfalse%
\ blast\isanewline
\isanewline
\ \ \ \ \ \ \ \ \isacommand{have}\isamarkupfalse%
\ {\isachardoublequoteopen}{\isasymforall}x\ {\isasymin}\ classes\ {\isadigit{1}}\ t\ {\isadigit{0}}{\isachardot}{\kern0pt}\ {\isasymchi}\ {\isacharparenleft}{\kern0pt}S{\isadigit{1}}\ x{\isacharparenright}{\kern0pt}\ {\isacharequal}{\kern0pt}\ linecol{\isachardoublequoteclose}\isanewline
\ \ \ \ \ \ \ \ \isacommand{proof}\isamarkupfalse%
\isanewline
\ \ \ \ \ \ \ \ \ \ \isacommand{fix}\isamarkupfalse%
\ x\isanewline
\ \ \ \ \ \ \ \ \ \ \isacommand{assume}\isamarkupfalse%
\ {\isachardoublequoteopen}x\ {\isasymin}\ classes\ {\isadigit{1}}\ t\ {\isadigit{0}}{\isachardoublequoteclose}\isanewline
\ \ \ \ \ \ \ \ \ \ \isacommand{then}\isamarkupfalse%
\ \isacommand{have}\isamarkupfalse%
\ {\isachardoublequoteopen}x\ {\isasymin}\ cube\ {\isadigit{1}}\ {\isacharparenleft}{\kern0pt}t{\isacharplus}{\kern0pt}{\isadigit{1}}{\isacharparenright}{\kern0pt}{\isachardoublequoteclose}\ \isacommand{unfolding}\isamarkupfalse%
\ classes{\isacharunderscore}{\kern0pt}def\ \isacommand{by}\isamarkupfalse%
\ simp\isanewline
\ \ \ \ \ \ \ \ \ \ \isacommand{then}\isamarkupfalse%
\ \isacommand{have}\isamarkupfalse%
\ {\isachardoublequoteopen}S{\isadigit{1}}\ x\ {\isacharequal}{\kern0pt}\ L{\isacharprime}{\kern0pt}\ {\isacharparenleft}{\kern0pt}x\ {\isadigit{0}}{\isacharparenright}{\kern0pt}{\isachardoublequoteclose}\ \isacommand{unfolding}\isamarkupfalse%
\ S{\isadigit{1}}{\isacharunderscore}{\kern0pt}def\ \isacommand{by}\isamarkupfalse%
\ simp\isanewline
\ \ \ \ \ \ \ \ \ \ \isacommand{moreover}\isamarkupfalse%
\ \isacommand{have}\isamarkupfalse%
\ {\isachardoublequoteopen}x\ {\isadigit{0}}\ {\isasymin}\ {\isacharbraceleft}{\kern0pt}{\isachardot}{\kern0pt}{\isachardot}{\kern0pt}{\isacharless}{\kern0pt}t{\isacharbraceright}{\kern0pt}{\isachardoublequoteclose}\ \isacommand{using}\isamarkupfalse%
\ {\isacharasterisk}{\kern0pt}{\isacharasterisk}{\kern0pt}\ \isacommand{using}\isamarkupfalse%
\ {\isacartoucheopen}x\ {\isasymin}\ classes\ {\isadigit{1}}\ t\ {\isadigit{0}}{\isacartoucheclose}\ \isacommand{by}\isamarkupfalse%
\ blast\isanewline
\ \ \ \ \ \ \ \ \ \ \isacommand{ultimately}\isamarkupfalse%
\ \isacommand{show}\isamarkupfalse%
\ {\isachardoublequoteopen}{\isasymchi}\ {\isacharparenleft}{\kern0pt}S{\isadigit{1}}\ x{\isacharparenright}{\kern0pt}\ {\isacharequal}{\kern0pt}\ linecol{\isachardoublequoteclose}\ \isacommand{using}\isamarkupfalse%
\ lc{\isacharunderscore}{\kern0pt}def\ \isanewline
\ \ \ \ \ \ \ \ \ \ \ \ \isacommand{using}\isamarkupfalse%
\ fun{\isacharunderscore}{\kern0pt}upd{\isacharunderscore}{\kern0pt}triv\ image{\isacharunderscore}{\kern0pt}eqI\ \isacommand{by}\isamarkupfalse%
\ blast\isanewline
\ \ \ \ \ \ \ \ \isacommand{qed}\isamarkupfalse%
\isanewline
\ \ \ \ \ \ \ \ \isacommand{then}\isamarkupfalse%
\ \isacommand{show}\isamarkupfalse%
\ {\isacharquery}{\kern0pt}thesis\ \isacommand{using}\isamarkupfalse%
\ lc{\isacharunderscore}{\kern0pt}def\ {\isacartoucheopen}i\ {\isacharequal}{\kern0pt}\ {\isadigit{0}}{\isacartoucheclose}\ \isacommand{by}\isamarkupfalse%
\ auto\isanewline
\ \ \ \ \ \ \isacommand{next}\isamarkupfalse%
\isanewline
\ \ \ \ \ \ \ \ \isacommand{case}\isamarkupfalse%
\ {\isadigit{2}}\ \isanewline
\ \ \ \ \ \ \ \ \isacommand{assume}\isamarkupfalse%
\ {\isachardoublequoteopen}i\ {\isacharequal}{\kern0pt}\ {\isadigit{1}}{\isachardoublequoteclose}\isanewline
\ \ \ \ \ \ \ \ \isacommand{have}\isamarkupfalse%
\ {\isachardoublequoteopen}classes\ {\isadigit{1}}\ t\ {\isadigit{1}}\ {\isacharequal}{\kern0pt}\ {\isacharbraceleft}{\kern0pt}x\ {\isachardot}{\kern0pt}\ x\ {\isasymin}\ {\isacharparenleft}{\kern0pt}cube\ {\isadigit{1}}\ {\isacharparenleft}{\kern0pt}t\ {\isacharplus}{\kern0pt}\ {\isadigit{1}}{\isacharparenright}{\kern0pt}{\isacharparenright}{\kern0pt}\ {\isasymand}\ {\isacharparenleft}{\kern0pt}{\isasymforall}u\ {\isasymin}\ {\isacharbraceleft}{\kern0pt}{\isadigit{0}}{\isacharcolon}{\kern0pt}{\isacharcolon}{\kern0pt}nat{\isachardot}{\kern0pt}{\isachardot}{\kern0pt}{\isacharless}{\kern0pt}{\isadigit{1}}{\isacharbraceright}{\kern0pt}{\isachardot}{\kern0pt}\ x\ u\ {\isacharequal}{\kern0pt}\ t{\isacharparenright}{\kern0pt}\ {\isasymand}\ t\ {\isasymnotin}\ x\ {\isacharbackquote}{\kern0pt}\ {\isacharbraceleft}{\kern0pt}{\isachardot}{\kern0pt}{\isachardot}{\kern0pt}{\isacharless}{\kern0pt}{\isadigit{0}}{\isacharbraceright}{\kern0pt}{\isacharbraceright}{\kern0pt}{\isachardoublequoteclose}\ \isacommand{unfolding}\isamarkupfalse%
\ classes{\isacharunderscore}{\kern0pt}def\ \isacommand{by}\isamarkupfalse%
\ simp\isanewline
\ \ \ \ \ \ \ \ \isacommand{also}\isamarkupfalse%
\ \isacommand{have}\isamarkupfalse%
\ {\isachardoublequoteopen}\ {\isachardot}{\kern0pt}{\isachardot}{\kern0pt}{\isachardot}{\kern0pt}\ {\isacharequal}{\kern0pt}\ {\isacharbraceleft}{\kern0pt}x\ {\isachardot}{\kern0pt}\ x\ {\isasymin}\ cube\ {\isadigit{1}}\ {\isacharparenleft}{\kern0pt}t{\isacharplus}{\kern0pt}{\isadigit{1}}{\isacharparenright}{\kern0pt}\ {\isasymand}\ {\isacharparenleft}{\kern0pt}{\isasymforall}u\ {\isasymin}\ {\isacharbraceleft}{\kern0pt}{\isadigit{0}}{\isacharbraceright}{\kern0pt}{\isachardot}{\kern0pt}\ x\ u\ {\isacharequal}{\kern0pt}\ t{\isacharparenright}{\kern0pt}{\isacharbraceright}{\kern0pt}{\isachardoublequoteclose}\ \isacommand{by}\isamarkupfalse%
\ simp\isanewline
\ \ \ \ \ \ \ \ \isacommand{finally}\isamarkupfalse%
\ \isacommand{have}\isamarkupfalse%
\ redef{\isacharcolon}{\kern0pt}\ {\isachardoublequoteopen}classes\ {\isadigit{1}}\ t\ {\isadigit{1}}\ {\isacharequal}{\kern0pt}\ {\isacharbraceleft}{\kern0pt}x\ {\isachardot}{\kern0pt}\ x\ {\isasymin}\ cube\ {\isadigit{1}}\ {\isacharparenleft}{\kern0pt}t{\isacharplus}{\kern0pt}{\isadigit{1}}{\isacharparenright}{\kern0pt}\ {\isasymand}\ {\isacharparenleft}{\kern0pt}x\ {\isadigit{0}}\ {\isacharequal}{\kern0pt}\ t{\isacharparenright}{\kern0pt}{\isacharbraceright}{\kern0pt}{\isachardoublequoteclose}\ \isacommand{by}\isamarkupfalse%
\ auto\isanewline
\ \ \ \ \ \ \ \ \isacommand{have}\isamarkupfalse%
\ {\isachardoublequoteopen}{\isasymforall}s\ {\isasymin}\ {\isacharbraceleft}{\kern0pt}{\isachardot}{\kern0pt}{\isachardot}{\kern0pt}{\isacharless}{\kern0pt}t{\isacharplus}{\kern0pt}{\isadigit{1}}{\isacharbraceright}{\kern0pt}{\isachardot}{\kern0pt}\ {\isasymexists}{\isacharbang}{\kern0pt}x\ {\isasymin}\ cube\ {\isadigit{1}}\ {\isacharparenleft}{\kern0pt}t{\isacharplus}{\kern0pt}{\isadigit{1}}{\isacharparenright}{\kern0pt}{\isachardot}{\kern0pt}\ {\isacharparenleft}{\kern0pt}{\isasymlambda}p{\isachardot}{\kern0pt}\ {\isasymlambda}y{\isasymin}{\isacharbraceleft}{\kern0pt}{\isachardot}{\kern0pt}{\isachardot}{\kern0pt}{\isacharless}{\kern0pt}{\isadigit{1}}{\isacharcolon}{\kern0pt}{\isacharcolon}{\kern0pt}nat{\isacharbraceright}{\kern0pt}{\isachardot}{\kern0pt}\ p{\isacharparenright}{\kern0pt}\ s\ {\isacharequal}{\kern0pt}\ x{\isachardoublequoteclose}\ \isacommand{using}\isamarkupfalse%
\ nat{\isacharunderscore}{\kern0pt}set{\isacharunderscore}{\kern0pt}eq{\isacharunderscore}{\kern0pt}one{\isacharunderscore}{\kern0pt}dim{\isacharunderscore}{\kern0pt}cube{\isacharbrackleft}{\kern0pt}of\ {\isachardoublequoteopen}t{\isacharplus}{\kern0pt}{\isadigit{1}}{\isachardoublequoteclose}{\isacharbrackright}{\kern0pt}\ \isacommand{unfolding}\isamarkupfalse%
\ bij{\isacharunderscore}{\kern0pt}betw{\isacharunderscore}{\kern0pt}def\ \isacommand{by}\isamarkupfalse%
\ blast\isanewline
\ \ \ \ \ \ \ \ \isacommand{then}\isamarkupfalse%
\ \isacommand{have}\isamarkupfalse%
\ {\isachardoublequoteopen}{\isasymexists}{\isacharbang}{\kern0pt}x\ {\isasymin}cube\ {\isadigit{1}}\ {\isacharparenleft}{\kern0pt}t{\isacharplus}{\kern0pt}{\isadigit{1}}{\isacharparenright}{\kern0pt}{\isachardot}{\kern0pt}\ {\isacharparenleft}{\kern0pt}{\isasymlambda}p{\isachardot}{\kern0pt}\ {\isasymlambda}y{\isasymin}{\isacharbraceleft}{\kern0pt}{\isachardot}{\kern0pt}{\isachardot}{\kern0pt}{\isacharless}{\kern0pt}{\isadigit{1}}{\isacharcolon}{\kern0pt}{\isacharcolon}{\kern0pt}nat{\isacharbraceright}{\kern0pt}{\isachardot}{\kern0pt}\ p{\isacharparenright}{\kern0pt}\ t\ {\isacharequal}{\kern0pt}\ x{\isachardoublequoteclose}\ \isacommand{by}\isamarkupfalse%
\ auto\isanewline
\ \ \ \ \ \ \ \ \isacommand{then}\isamarkupfalse%
\ \isacommand{obtain}\isamarkupfalse%
\ x\ \isakeyword{where}\ x{\isacharunderscore}{\kern0pt}prop{\isacharcolon}{\kern0pt}\ {\isachardoublequoteopen}x\ {\isasymin}\ cube\ {\isadigit{1}}\ {\isacharparenleft}{\kern0pt}t{\isacharplus}{\kern0pt}{\isadigit{1}}{\isacharparenright}{\kern0pt}{\isachardoublequoteclose}\ \isakeyword{and}\ {\isachardoublequoteopen}{\isacharparenleft}{\kern0pt}{\isasymlambda}p{\isachardot}{\kern0pt}\ {\isasymlambda}y{\isasymin}{\isacharbraceleft}{\kern0pt}{\isachardot}{\kern0pt}{\isachardot}{\kern0pt}{\isacharless}{\kern0pt}{\isadigit{1}}{\isacharcolon}{\kern0pt}{\isacharcolon}{\kern0pt}nat{\isacharbraceright}{\kern0pt}{\isachardot}{\kern0pt}\ p{\isacharparenright}{\kern0pt}\ t\ {\isacharequal}{\kern0pt}\ x{\isachardoublequoteclose}\ \isakeyword{and}\ {\isachardoublequoteopen}{\isasymforall}z\ {\isasymin}\ cube\ {\isadigit{1}}\ {\isacharparenleft}{\kern0pt}t{\isacharplus}{\kern0pt}{\isadigit{1}}{\isacharparenright}{\kern0pt}{\isachardot}{\kern0pt}\ {\isacharparenleft}{\kern0pt}{\isasymlambda}p{\isachardot}{\kern0pt}\ {\isasymlambda}y{\isasymin}{\isacharbraceleft}{\kern0pt}{\isachardot}{\kern0pt}{\isachardot}{\kern0pt}{\isacharless}{\kern0pt}{\isadigit{1}}{\isacharcolon}{\kern0pt}{\isacharcolon}{\kern0pt}nat{\isacharbraceright}{\kern0pt}{\isachardot}{\kern0pt}\ p{\isacharparenright}{\kern0pt}\ t\ {\isacharequal}{\kern0pt}\ z\ {\isasymlongrightarrow}\ z\ {\isacharequal}{\kern0pt}\ x{\isachardoublequoteclose}\ \isacommand{by}\isamarkupfalse%
\ blast\isanewline
\ \ \ \ \ \ \ \ \isacommand{then}\isamarkupfalse%
\ \isacommand{have}\isamarkupfalse%
\ {\isachardoublequoteopen}{\isacharparenleft}{\kern0pt}{\isasymlambda}p{\isachardot}{\kern0pt}\ {\isasymlambda}y{\isasymin}{\isacharbraceleft}{\kern0pt}{\isadigit{0}}{\isacharbraceright}{\kern0pt}{\isachardot}{\kern0pt}\ p{\isacharparenright}{\kern0pt}\ \ t\ \ {\isacharequal}{\kern0pt}\ x\ {\isasymand}\ {\isacharparenleft}{\kern0pt}{\isasymforall}z\ {\isasymin}\ cube\ {\isadigit{1}}\ {\isacharparenleft}{\kern0pt}t{\isacharplus}{\kern0pt}{\isadigit{1}}{\isacharparenright}{\kern0pt}{\isachardot}{\kern0pt}\ {\isacharparenleft}{\kern0pt}{\isasymlambda}p{\isachardot}{\kern0pt}\ {\isasymlambda}y{\isasymin}{\isacharbraceleft}{\kern0pt}{\isadigit{0}}{\isacharbraceright}{\kern0pt}{\isachardot}{\kern0pt}\ p{\isacharparenright}{\kern0pt}\ t\ {\isacharequal}{\kern0pt}\ z\ {\isasymlongrightarrow}\ z\ {\isacharequal}{\kern0pt}\ x{\isacharparenright}{\kern0pt}{\isachardoublequoteclose}\ \ \isacommand{by}\isamarkupfalse%
\ force\isanewline
\ \ \ \ \ \ \ \ \isacommand{then}\isamarkupfalse%
\ \isacommand{have}\isamarkupfalse%
\ {\isacharasterisk}{\kern0pt}{\isacharcolon}{\kern0pt}{\isachardoublequoteopen}{\isacharparenleft}{\kern0pt}{\isacharparenleft}{\kern0pt}{\isasymlambda}p{\isachardot}{\kern0pt}\ {\isasymlambda}y{\isasymin}{\isacharbraceleft}{\kern0pt}{\isadigit{0}}{\isacharbraceright}{\kern0pt}{\isachardot}{\kern0pt}\ p{\isacharparenright}{\kern0pt}\ t{\isacharparenright}{\kern0pt}\ {\isadigit{0}}\ \ {\isacharequal}{\kern0pt}\ x\ {\isadigit{0}}\ {\isasymand}\ {\isacharparenleft}{\kern0pt}{\isasymforall}z\ {\isasymin}\ cube\ {\isadigit{1}}\ {\isacharparenleft}{\kern0pt}t{\isacharplus}{\kern0pt}{\isadigit{1}}{\isacharparenright}{\kern0pt}{\isachardot}{\kern0pt}\ {\isacharparenleft}{\kern0pt}{\isasymlambda}p{\isachardot}{\kern0pt}\ {\isasymlambda}y{\isasymin}{\isacharbraceleft}{\kern0pt}{\isadigit{0}}{\isacharbraceright}{\kern0pt}{\isachardot}{\kern0pt}\ p{\isacharparenright}{\kern0pt}\ t\ \ {\isacharequal}{\kern0pt}\ z\ \ {\isasymlongrightarrow}\ z\ {\isacharequal}{\kern0pt}\ x{\isacharparenright}{\kern0pt}{\isachardoublequoteclose}\ \ \isanewline
\ \ \ \ \ \ \ \ \ \ \isacommand{using}\isamarkupfalse%
\ x{\isacharunderscore}{\kern0pt}prop\ \isacommand{by}\isamarkupfalse%
\ force\isanewline
\isanewline
\ \ \ \ \ \ \ \ \isacommand{then}\isamarkupfalse%
\ \isacommand{have}\isamarkupfalse%
\ {\isachardoublequoteopen}{\isasymexists}{\isacharbang}{\kern0pt}y\ {\isasymin}\ cube\ {\isadigit{1}}\ {\isacharparenleft}{\kern0pt}t\ {\isacharplus}{\kern0pt}\ {\isadigit{1}}{\isacharparenright}{\kern0pt}{\isachardot}{\kern0pt}\ y\ {\isadigit{0}}\ {\isacharequal}{\kern0pt}\ t{\isachardoublequoteclose}\ \isanewline
\ \ \ \ \ \ \ \ \isacommand{proof}\isamarkupfalse%
\ {\isacharparenleft}{\kern0pt}intro\ ex{\isadigit{1}}I{\isacharunderscore}{\kern0pt}alt{\isacharparenright}{\kern0pt}\isanewline
\ \ \ \ \ \ \ \ \ \ \isacommand{define}\isamarkupfalse%
\ y\ \isakeyword{where}\ {\isachardoublequoteopen}y\ {\isasymequiv}\ {\isacharparenleft}{\kern0pt}{\isasymlambda}x{\isacharcolon}{\kern0pt}{\isacharcolon}{\kern0pt}nat{\isachardot}{\kern0pt}\ {\isasymlambda}y{\isasymin}{\isacharbraceleft}{\kern0pt}{\isachardot}{\kern0pt}{\isachardot}{\kern0pt}{\isacharless}{\kern0pt}{\isadigit{1}}{\isacharcolon}{\kern0pt}{\isacharcolon}{\kern0pt}nat{\isacharbraceright}{\kern0pt}{\isachardot}{\kern0pt}\ x{\isacharparenright}{\kern0pt}{\isachardoublequoteclose}\ \isanewline
\ \ \ \ \ \ \ \ \ \ \isacommand{have}\isamarkupfalse%
\ {\isachardoublequoteopen}y\ t\ {\isasymin}\ {\isacharparenleft}{\kern0pt}cube\ {\isadigit{1}}\ {\isacharparenleft}{\kern0pt}t\ {\isacharplus}{\kern0pt}\ {\isadigit{1}}{\isacharparenright}{\kern0pt}{\isacharparenright}{\kern0pt}{\isachardoublequoteclose}\ \isacommand{unfolding}\isamarkupfalse%
\ cube{\isacharunderscore}{\kern0pt}def\ y{\isacharunderscore}{\kern0pt}def\ \isacommand{by}\isamarkupfalse%
\ simp\ \isanewline
\ \ \ \ \ \ \ \ \ \ \isacommand{moreover}\isamarkupfalse%
\ \isacommand{have}\isamarkupfalse%
\ {\isachardoublequoteopen}y\ t\ {\isadigit{0}}\ {\isacharequal}{\kern0pt}\ t{\isachardoublequoteclose}\ \isacommand{unfolding}\isamarkupfalse%
\ y{\isacharunderscore}{\kern0pt}def\ \isacommand{by}\isamarkupfalse%
\ auto\isanewline
\ \ \ \ \ \ \ \ \ \ \isacommand{moreover}\isamarkupfalse%
\ \isacommand{have}\isamarkupfalse%
\ {\isachardoublequoteopen}z\ {\isacharequal}{\kern0pt}\ y\ t{\isachardoublequoteclose}\ \isakeyword{if}\ {\isachardoublequoteopen}z\ {\isasymin}\ cube\ {\isadigit{1}}\ {\isacharparenleft}{\kern0pt}t\ {\isacharplus}{\kern0pt}\ {\isadigit{1}}{\isacharparenright}{\kern0pt}{\isachardoublequoteclose}\ \isakeyword{and}\ {\isachardoublequoteopen}z\ {\isadigit{0}}\ {\isacharequal}{\kern0pt}\ t{\isachardoublequoteclose}\ \isakeyword{for}\ z\isanewline
\ \ \ \ \ \ \ \ \ \ \isacommand{proof}\isamarkupfalse%
\ {\isacharparenleft}{\kern0pt}rule\ ccontr{\isacharparenright}{\kern0pt}\isanewline
\ \ \ \ \ \ \ \ \ \ \ \ \isacommand{assume}\isamarkupfalse%
\ {\isachardoublequoteopen}z\ {\isasymnoteq}\ y\ t{\isachardoublequoteclose}\ \isanewline
\ \ \ \ \ \ \ \ \ \ \ \ \isacommand{then}\isamarkupfalse%
\ \isacommand{obtain}\isamarkupfalse%
\ l\ \isakeyword{where}\ l{\isacharunderscore}{\kern0pt}prop{\isacharcolon}{\kern0pt}\ {\isachardoublequoteopen}z\ l\ {\isasymnoteq}\ y\ t\ l{\isachardoublequoteclose}\ \isacommand{by}\isamarkupfalse%
\ blast\isanewline
\ \ \ \ \ \ \ \ \ \ \ \ \isacommand{consider}\isamarkupfalse%
\ {\isachardoublequoteopen}l\ {\isasymin}\ {\isacharbraceleft}{\kern0pt}{\isachardot}{\kern0pt}{\isachardot}{\kern0pt}{\isacharless}{\kern0pt}{\isadigit{1}}{\isacharcolon}{\kern0pt}{\isacharcolon}{\kern0pt}nat{\isacharbraceright}{\kern0pt}{\isachardoublequoteclose}\ {\isacharbar}{\kern0pt}\ {\isachardoublequoteopen}l\ {\isasymnotin}\ {\isacharbraceleft}{\kern0pt}{\isachardot}{\kern0pt}{\isachardot}{\kern0pt}{\isacharless}{\kern0pt}{\isadigit{1}}{\isacharcolon}{\kern0pt}{\isacharcolon}{\kern0pt}nat{\isacharbraceright}{\kern0pt}{\isachardoublequoteclose}\ \isacommand{by}\isamarkupfalse%
\ blast\isanewline
\ \ \ \ \ \ \ \ \ \ \ \ \isacommand{then}\isamarkupfalse%
\ \isacommand{show}\isamarkupfalse%
\ False\isanewline
\ \ \ \ \ \ \ \ \ \ \ \ \isacommand{proof}\isamarkupfalse%
\ cases\isanewline
\ \ \ \ \ \ \ \ \ \ \ \ \ \ \isacommand{case}\isamarkupfalse%
\ {\isadigit{1}}\isanewline
\ \ \ \ \ \ \ \ \ \ \ \ \ \ \isacommand{then}\isamarkupfalse%
\ \isacommand{show}\isamarkupfalse%
\ {\isacharquery}{\kern0pt}thesis\ \isacommand{using}\isamarkupfalse%
\ l{\isacharunderscore}{\kern0pt}prop\ that{\isacharparenleft}{\kern0pt}{\isadigit{2}}{\isacharparenright}{\kern0pt}\ \isacommand{unfolding}\isamarkupfalse%
\ y{\isacharunderscore}{\kern0pt}def\ \isacommand{by}\isamarkupfalse%
\ auto\isanewline
\ \ \ \ \ \ \ \ \ \ \ \ \isacommand{next}\isamarkupfalse%
\isanewline
\ \ \ \ \ \ \ \ \ \ \ \ \ \ \isacommand{case}\isamarkupfalse%
\ {\isadigit{2}}\isanewline
\ \ \ \ \ \ \ \ \ \ \ \ \ \ \isacommand{then}\isamarkupfalse%
\ \isacommand{have}\isamarkupfalse%
\ {\isachardoublequoteopen}z\ l\ {\isacharequal}{\kern0pt}\ undefined{\isachardoublequoteclose}\ \isacommand{using}\isamarkupfalse%
\ that\ \isacommand{unfolding}\isamarkupfalse%
\ cube{\isacharunderscore}{\kern0pt}def\ \isacommand{by}\isamarkupfalse%
\ blast\isanewline
\ \ \ \ \ \ \ \ \ \ \ \ \ \ \isacommand{moreover}\isamarkupfalse%
\ \isacommand{have}\isamarkupfalse%
\ {\isachardoublequoteopen}y\ t\ l\ {\isacharequal}{\kern0pt}\ undefined{\isachardoublequoteclose}\ \isacommand{unfolding}\isamarkupfalse%
\ y{\isacharunderscore}{\kern0pt}def\ \isacommand{using}\isamarkupfalse%
\ {\isadigit{2}}\ \isacommand{by}\isamarkupfalse%
\ auto\isanewline
\ \ \ \ \ \ \ \ \ \ \ \ \ \ \isacommand{ultimately}\isamarkupfalse%
\ \isacommand{show}\isamarkupfalse%
\ {\isacharquery}{\kern0pt}thesis\ \isacommand{using}\isamarkupfalse%
\ l{\isacharunderscore}{\kern0pt}prop\ \isacommand{by}\isamarkupfalse%
\ presburger\isanewline
\ \ \ \ \ \ \ \ \ \ \ \ \isacommand{qed}\isamarkupfalse%
\isanewline
\ \ \ \ \ \ \ \ \ \ \isacommand{qed}\isamarkupfalse%
\isanewline
\ \ \ \ \ \ \ \ \ \ \isacommand{ultimately}\isamarkupfalse%
\ \isacommand{show}\isamarkupfalse%
\ {\isachardoublequoteopen}{\isasymexists}y{\isachardot}{\kern0pt}\ {\isacharparenleft}{\kern0pt}y\ {\isasymin}\ cube\ {\isadigit{1}}\ {\isacharparenleft}{\kern0pt}t\ {\isacharplus}{\kern0pt}\ {\isadigit{1}}{\isacharparenright}{\kern0pt}\ {\isasymand}\ y\ {\isadigit{0}}\ {\isacharequal}{\kern0pt}\ t{\isacharparenright}{\kern0pt}\ {\isasymand}\ {\isacharparenleft}{\kern0pt}{\isasymforall}ya{\isachardot}{\kern0pt}\ ya\ {\isasymin}\ cube\ {\isadigit{1}}\ {\isacharparenleft}{\kern0pt}t\ {\isacharplus}{\kern0pt}\ {\isadigit{1}}{\isacharparenright}{\kern0pt}\ {\isasymand}\ ya\ {\isadigit{0}}\ {\isacharequal}{\kern0pt}\ t\ {\isasymlongrightarrow}\ y\ {\isacharequal}{\kern0pt}\ ya{\isacharparenright}{\kern0pt}{\isachardoublequoteclose}\ \isacommand{by}\isamarkupfalse%
\ blast\isanewline
\ \ \ \ \ \ \ \ \isacommand{qed}\isamarkupfalse%
\isanewline
\ \ \ \ \ \ \ \ \isacommand{then}\isamarkupfalse%
\ \isacommand{have}\isamarkupfalse%
\ {\isachardoublequoteopen}{\isasymexists}{\isacharbang}{\kern0pt}x\ {\isasymin}\ classes\ {\isadigit{1}}\ t\ {\isadigit{1}}{\isachardot}{\kern0pt}\ True{\isachardoublequoteclose}\ \isacommand{using}\isamarkupfalse%
\ redef\ \isacommand{by}\isamarkupfalse%
\ simp\isanewline
\ \ \ \ \ \ \ \ \isacommand{then}\isamarkupfalse%
\ \isacommand{obtain}\isamarkupfalse%
\ x\ \isakeyword{where}\ x{\isacharunderscore}{\kern0pt}def{\isacharcolon}{\kern0pt}\ {\isachardoublequoteopen}x\ {\isasymin}\ classes\ {\isadigit{1}}\ t\ {\isadigit{1}}\ {\isasymand}\ {\isacharparenleft}{\kern0pt}{\isasymforall}y\ {\isasymin}\ classes\ {\isadigit{1}}\ t\ {\isadigit{1}}{\isachardot}{\kern0pt}\ x\ {\isacharequal}{\kern0pt}\ y{\isacharparenright}{\kern0pt}{\isachardoublequoteclose}\ \isacommand{by}\isamarkupfalse%
\ auto\isanewline
\isanewline
\ \ \ \ \ \ \ \ \isacommand{have}\isamarkupfalse%
\ {\isachardoublequoteopen}{\isasymexists}c\ {\isacharless}{\kern0pt}\ r{\isachardot}{\kern0pt}\ {\isasymforall}x\ {\isasymin}\ classes\ {\isadigit{1}}\ t\ {\isadigit{1}}{\isachardot}{\kern0pt}\ {\isasymchi}\ {\isacharparenleft}{\kern0pt}S{\isadigit{1}}\ x{\isacharparenright}{\kern0pt}\ {\isacharequal}{\kern0pt}\ c{\isachardoublequoteclose}\ \isanewline
\ \ \ \ \ \ \ \ \isacommand{proof}\isamarkupfalse%
{\isacharminus}{\kern0pt}\isanewline
\ \ \ \ \ \ \ \ \ \ \isacommand{have}\isamarkupfalse%
\ {\isachardoublequoteopen}{\isasymforall}y\ {\isasymin}\ classes\ {\isadigit{1}}\ t\ {\isadigit{1}}{\isachardot}{\kern0pt}\ y\ {\isacharequal}{\kern0pt}\ x{\isachardoublequoteclose}\ \isacommand{using}\isamarkupfalse%
\ x{\isacharunderscore}{\kern0pt}def\ \isacommand{by}\isamarkupfalse%
\ auto\isanewline
\ \ \ \ \ \ \ \ \ \ \isacommand{then}\isamarkupfalse%
\ \isacommand{have}\isamarkupfalse%
\ {\isachardoublequoteopen}{\isasymforall}y{\isasymin}classes\ {\isadigit{1}}\ t\ {\isadigit{1}}{\isachardot}{\kern0pt}\ {\isasymchi}\ {\isacharparenleft}{\kern0pt}S{\isadigit{1}}\ y{\isacharparenright}{\kern0pt}\ {\isacharequal}{\kern0pt}\ {\isasymchi}\ {\isacharparenleft}{\kern0pt}S{\isadigit{1}}\ x{\isacharparenright}{\kern0pt}{\isachardoublequoteclose}\ \isacommand{by}\isamarkupfalse%
\ auto\isanewline
\ \ \ \ \ \ \ \ \ \ \isacommand{moreover}\isamarkupfalse%
\ \isacommand{have}\isamarkupfalse%
\ {\isachardoublequoteopen}x\ {\isasymin}\ cube\ {\isadigit{1}}\ {\isacharparenleft}{\kern0pt}t{\isacharplus}{\kern0pt}{\isadigit{1}}{\isacharparenright}{\kern0pt}{\isachardoublequoteclose}\ \isacommand{using}\isamarkupfalse%
\ x{\isacharunderscore}{\kern0pt}def\ \ \isacommand{using}\isamarkupfalse%
\ redef\ \isacommand{by}\isamarkupfalse%
\ simp\isanewline
\ \ \ \ \ \ \ \ \ \ \isacommand{moreover}\isamarkupfalse%
\ \isacommand{have}\isamarkupfalse%
\ {\isachardoublequoteopen}S{\isadigit{1}}\ x\ {\isasymin}\ cube\ N{\isacharprime}{\kern0pt}\ {\isacharparenleft}{\kern0pt}t{\isacharplus}{\kern0pt}{\isadigit{1}}{\isacharparenright}{\kern0pt}{\isachardoublequoteclose}\ \isacommand{unfolding}\isamarkupfalse%
\ S{\isadigit{1}}{\isacharunderscore}{\kern0pt}def\ is{\isacharunderscore}{\kern0pt}line{\isacharunderscore}{\kern0pt}def\ \isacommand{using}\isamarkupfalse%
\ line{\isacharunderscore}{\kern0pt}prop\ line{\isacharunderscore}{\kern0pt}points{\isacharunderscore}{\kern0pt}in{\isacharunderscore}{\kern0pt}cube\ redef\ x{\isacharunderscore}{\kern0pt}def\ \isacommand{by}\isamarkupfalse%
\ fastforce\isanewline
\ \ \ \ \ \ \ \ \ \ \isacommand{moreover}\isamarkupfalse%
\ \isacommand{have}\isamarkupfalse%
\ {\isachardoublequoteopen}{\isasymchi}\ {\isacharparenleft}{\kern0pt}S{\isadigit{1}}\ x{\isacharparenright}{\kern0pt}\ {\isacharless}{\kern0pt}\ r{\isachardoublequoteclose}\ \ \isacommand{using}\isamarkupfalse%
\ asm\ calculation\ \isacommand{unfolding}\isamarkupfalse%
\ cube{\isacharunderscore}{\kern0pt}def\ \isacommand{by}\isamarkupfalse%
\ auto\isanewline
\ \ \ \ \ \ \ \ \ \ \isacommand{ultimately}\isamarkupfalse%
\ \isacommand{show}\isamarkupfalse%
\ {\isachardoublequoteopen}{\isasymexists}c\ {\isacharless}{\kern0pt}\ r{\isachardot}{\kern0pt}\ {\isasymforall}x\ {\isasymin}\ classes\ {\isadigit{1}}\ t\ {\isadigit{1}}{\isachardot}{\kern0pt}\ {\isasymchi}\ {\isacharparenleft}{\kern0pt}S{\isadigit{1}}\ x{\isacharparenright}{\kern0pt}\ {\isacharequal}{\kern0pt}\ c{\isachardoublequoteclose}\ \isacommand{by}\isamarkupfalse%
\ auto\isanewline
\ \ \ \ \ \ \ \ \isacommand{qed}\isamarkupfalse%
\isanewline
\ \ \ \ \ \ \ \ \isacommand{then}\isamarkupfalse%
\ \isacommand{show}\isamarkupfalse%
\ {\isacharquery}{\kern0pt}thesis\ \isacommand{using}\isamarkupfalse%
\ lc{\isacharunderscore}{\kern0pt}def\ {\isacartoucheopen}i\ {\isacharequal}{\kern0pt}\ {\isadigit{1}}{\isacartoucheclose}\ \isacommand{by}\isamarkupfalse%
\ auto\isanewline
\ \ \ \ \ \ \isacommand{qed}\isamarkupfalse%
\isanewline
\isanewline
\isanewline
\ \ \ \ \isacommand{qed}\isamarkupfalse%
\isanewline
\ \ \ \ \isacommand{show}\isamarkupfalse%
\ {\isachardoublequoteopen}{\isacharparenleft}{\kern0pt}{\isasymexists}S{\isachardot}{\kern0pt}\ is{\isacharunderscore}{\kern0pt}subspace\ S\ {\isadigit{1}}\ N{\isacharprime}{\kern0pt}\ {\isacharparenleft}{\kern0pt}t\ {\isacharplus}{\kern0pt}\ {\isadigit{1}}{\isacharparenright}{\kern0pt}\ {\isasymand}\ {\isacharparenleft}{\kern0pt}{\isasymforall}i\ {\isasymin}\ {\isacharbraceleft}{\kern0pt}{\isachardot}{\kern0pt}{\isachardot}{\kern0pt}{\isadigit{1}}{\isacharbraceright}{\kern0pt}{\isachardot}{\kern0pt}\ {\isasymexists}c\ {\isacharless}{\kern0pt}\ r{\isachardot}{\kern0pt}\ {\isacharparenleft}{\kern0pt}{\isasymforall}x\ {\isasymin}\ classes\ {\isadigit{1}}\ t\ i{\isachardot}{\kern0pt}\ {\isasymchi}\ {\isacharparenleft}{\kern0pt}S\ x{\isacharparenright}{\kern0pt}\ {\isacharequal}{\kern0pt}\ c{\isacharparenright}{\kern0pt}{\isacharparenright}{\kern0pt}{\isacharparenright}{\kern0pt}\ {\isachardoublequoteclose}\ \isacommand{using}\isamarkupfalse%
\ F{\isadigit{1}}\ F{\isadigit{2}}\ \isacommand{unfolding}\isamarkupfalse%
\ S{\isadigit{1}}{\isacharunderscore}{\kern0pt}def\ \isacommand{by}\isamarkupfalse%
\ blast\isanewline
\ \ \isacommand{qed}\isamarkupfalse%
\isanewline
\ \ \isacommand{then}\isamarkupfalse%
\ \isacommand{show}\isamarkupfalse%
\ {\isacharquery}{\kern0pt}thesis\ \isacommand{using}\isamarkupfalse%
\ N{\isacharunderscore}{\kern0pt}def\ \isacommand{unfolding}\isamarkupfalse%
\ layered{\isacharunderscore}{\kern0pt}subspace{\isacharunderscore}{\kern0pt}def\ lhj{\isacharunderscore}{\kern0pt}def\ \isacommand{by}\isamarkupfalse%
\ auto\isanewline
\isacommand{qed}\isamarkupfalse%
%
\endisatagproof
{\isafoldproof}%
%
\isadelimproof
%
\endisadelimproof
%
\begin{isamarkuptext}%
Claiming k-dimensional subspaces of (cube n t) are isomorphic to (cube k t)%
\end{isamarkuptext}\isamarkuptrue%
\isacommand{definition}\isamarkupfalse%
\ is{\isacharunderscore}{\kern0pt}subspace{\isacharunderscore}{\kern0pt}alt\isanewline
\ \ \isakeyword{where}\ {\isachardoublequoteopen}is{\isacharunderscore}{\kern0pt}subspace{\isacharunderscore}{\kern0pt}alt\ S\ k\ n\ t\ {\isasymequiv}\ {\isacharparenleft}{\kern0pt}{\isasymexists}{\isasymphi}{\isachardot}{\kern0pt}\ k\ {\isasymle}\ n\ {\isasymand}\ bij{\isacharunderscore}{\kern0pt}betw\ {\isasymphi}\ S\ {\isacharparenleft}{\kern0pt}cube\ k\ t{\isacharparenright}{\kern0pt}{\isacharparenright}{\kern0pt}{\isachardoublequoteclose}%
\begin{isamarkuptext}%
Some useful facts about 1-dimensional subspaces.%
\end{isamarkuptext}\isamarkuptrue%
\isacommand{lemma}\isamarkupfalse%
\ dim{\isadigit{1}}{\isacharunderscore}{\kern0pt}subspace{\isacharunderscore}{\kern0pt}elims{\isacharcolon}{\kern0pt}\ \isanewline
\ \ \isakeyword{assumes}\ {\isachardoublequoteopen}disjoint{\isacharunderscore}{\kern0pt}family{\isacharunderscore}{\kern0pt}on\ B\ {\isacharbraceleft}{\kern0pt}{\isachardot}{\kern0pt}{\isachardot}{\kern0pt}{\isadigit{1}}{\isacharcolon}{\kern0pt}{\isacharcolon}{\kern0pt}nat{\isacharbraceright}{\kern0pt}{\isachardoublequoteclose}\ \isakeyword{and}\ {\isachardoublequoteopen}{\isasymUnion}{\isacharparenleft}{\kern0pt}B\ {\isacharbackquote}{\kern0pt}\ {\isacharbraceleft}{\kern0pt}{\isachardot}{\kern0pt}{\isachardot}{\kern0pt}{\isadigit{1}}{\isacharcolon}{\kern0pt}{\isacharcolon}{\kern0pt}nat{\isacharbraceright}{\kern0pt}{\isacharparenright}{\kern0pt}\ {\isacharequal}{\kern0pt}\ {\isacharbraceleft}{\kern0pt}{\isachardot}{\kern0pt}{\isachardot}{\kern0pt}{\isacharless}{\kern0pt}n{\isacharbraceright}{\kern0pt}{\isachardoublequoteclose}\ \isakeyword{and}\ {\isachardoublequoteopen}{\isacharparenleft}{\kern0pt}{\isacharbraceleft}{\kern0pt}{\isacharbraceright}{\kern0pt}\ {\isasymnotin}\ B\ {\isacharbackquote}{\kern0pt}\ {\isacharbraceleft}{\kern0pt}{\isachardot}{\kern0pt}{\isachardot}{\kern0pt}{\isacharless}{\kern0pt}{\isadigit{1}}{\isacharcolon}{\kern0pt}{\isacharcolon}{\kern0pt}nat{\isacharbraceright}{\kern0pt}{\isacharparenright}{\kern0pt}{\isachardoublequoteclose}\ \isakeyword{and}\ \ {\isachardoublequoteopen}f\ {\isasymin}\ {\isacharparenleft}{\kern0pt}B\ {\isadigit{1}}{\isacharparenright}{\kern0pt}\ {\isasymrightarrow}\isactrlsub E\ {\isacharbraceleft}{\kern0pt}{\isachardot}{\kern0pt}{\isachardot}{\kern0pt}{\isacharless}{\kern0pt}t{\isacharbraceright}{\kern0pt}{\isachardoublequoteclose}\ \isakeyword{and}\ {\isachardoublequoteopen}S\ {\isasymin}\ {\isacharparenleft}{\kern0pt}cube\ {\isadigit{1}}\ t{\isacharparenright}{\kern0pt}\ {\isasymrightarrow}\isactrlsub E\ {\isacharparenleft}{\kern0pt}cube\ n\ t{\isacharparenright}{\kern0pt}{\isachardoublequoteclose}\ \isakeyword{and}\ {\isachardoublequoteopen}{\isacharparenleft}{\kern0pt}{\isasymforall}y\ {\isasymin}\ cube\ {\isadigit{1}}\ t{\isachardot}{\kern0pt}\ {\isacharparenleft}{\kern0pt}{\isasymforall}i\ {\isasymin}\ B\ {\isadigit{1}}{\isachardot}{\kern0pt}\ S\ y\ i\ {\isacharequal}{\kern0pt}\ f\ i{\isacharparenright}{\kern0pt}\ {\isasymand}\ {\isacharparenleft}{\kern0pt}{\isasymforall}j{\isacharless}{\kern0pt}{\isadigit{1}}{\isachardot}{\kern0pt}\ {\isasymforall}i\ {\isasymin}\ B\ j{\isachardot}{\kern0pt}\ {\isacharparenleft}{\kern0pt}S\ y{\isacharparenright}{\kern0pt}\ i\ {\isacharequal}{\kern0pt}\ y\ j{\isacharparenright}{\kern0pt}{\isacharparenright}{\kern0pt}{\isachardoublequoteclose}\isanewline
\ \ \isakeyword{shows}\ {\isachardoublequoteopen}B\ {\isadigit{0}}\ {\isasymunion}\ B\ {\isadigit{1}}\ {\isacharequal}{\kern0pt}\ {\isacharbraceleft}{\kern0pt}{\isachardot}{\kern0pt}{\isachardot}{\kern0pt}{\isacharless}{\kern0pt}n{\isacharbraceright}{\kern0pt}{\isachardoublequoteclose}\isanewline
\ \ \ \ \isakeyword{and}\ {\isachardoublequoteopen}B\ {\isadigit{0}}\ {\isasyminter}\ B\ {\isadigit{1}}\ {\isacharequal}{\kern0pt}\ {\isacharbraceleft}{\kern0pt}{\isacharbraceright}{\kern0pt}{\isachardoublequoteclose}\isanewline
\ \ \ \ \isakeyword{and}\ {\isachardoublequoteopen}{\isacharparenleft}{\kern0pt}{\isasymforall}y\ {\isasymin}\ cube\ {\isadigit{1}}\ t{\isachardot}{\kern0pt}\ {\isacharparenleft}{\kern0pt}{\isasymforall}i\ {\isasymin}\ B\ {\isadigit{1}}{\isachardot}{\kern0pt}\ S\ y\ i\ {\isacharequal}{\kern0pt}\ f\ i{\isacharparenright}{\kern0pt}\ {\isasymand}\ {\isacharparenleft}{\kern0pt}{\isasymforall}i\ {\isasymin}\ B\ {\isadigit{0}}{\isachardot}{\kern0pt}\ {\isacharparenleft}{\kern0pt}S\ y{\isacharparenright}{\kern0pt}\ i\ {\isacharequal}{\kern0pt}\ y\ {\isadigit{0}}{\isacharparenright}{\kern0pt}{\isacharparenright}{\kern0pt}{\isachardoublequoteclose}\isanewline
\ \ \ \ \isakeyword{and}\ {\isachardoublequoteopen}B\ {\isadigit{0}}\ {\isasymnoteq}\ {\isacharbraceleft}{\kern0pt}{\isacharbraceright}{\kern0pt}{\isachardoublequoteclose}\isanewline
%
\isadelimproof
%
\endisadelimproof
%
\isatagproof
\isacommand{proof}\isamarkupfalse%
\ {\isacharminus}{\kern0pt}\isanewline
\ \ \isacommand{have}\isamarkupfalse%
\ {\isachardoublequoteopen}{\isacharbraceleft}{\kern0pt}{\isachardot}{\kern0pt}{\isachardot}{\kern0pt}{\isadigit{1}}{\isacharbraceright}{\kern0pt}\ {\isacharequal}{\kern0pt}\ {\isacharbraceleft}{\kern0pt}{\isadigit{0}}{\isacharcolon}{\kern0pt}{\isacharcolon}{\kern0pt}nat{\isacharcomma}{\kern0pt}\ {\isadigit{1}}{\isacharbraceright}{\kern0pt}{\isachardoublequoteclose}\ \isacommand{by}\isamarkupfalse%
\ auto\isanewline
\ \ \isacommand{then}\isamarkupfalse%
\ \isacommand{show}\isamarkupfalse%
\ {\isachardoublequoteopen}B\ {\isadigit{0}}\ {\isasymunion}\ B\ {\isadigit{1}}\ {\isacharequal}{\kern0pt}\ {\isacharbraceleft}{\kern0pt}{\isachardot}{\kern0pt}{\isachardot}{\kern0pt}{\isacharless}{\kern0pt}n{\isacharbraceright}{\kern0pt}{\isachardoublequoteclose}\ \ \isacommand{using}\isamarkupfalse%
\ assms{\isacharparenleft}{\kern0pt}{\isadigit{2}}{\isacharparenright}{\kern0pt}\ \isacommand{by}\isamarkupfalse%
\ simp\isanewline
\isacommand{next}\isamarkupfalse%
\isanewline
\ \ \isacommand{show}\isamarkupfalse%
\ {\isachardoublequoteopen}B\ {\isadigit{0}}\ {\isasyminter}\ B\ {\isadigit{1}}\ {\isacharequal}{\kern0pt}\ {\isacharbraceleft}{\kern0pt}{\isacharbraceright}{\kern0pt}{\isachardoublequoteclose}\ \isacommand{using}\isamarkupfalse%
\ assms{\isacharparenleft}{\kern0pt}{\isadigit{1}}{\isacharparenright}{\kern0pt}\ \isacommand{unfolding}\isamarkupfalse%
\ disjoint{\isacharunderscore}{\kern0pt}family{\isacharunderscore}{\kern0pt}on{\isacharunderscore}{\kern0pt}def\ \isacommand{by}\isamarkupfalse%
\ simp\isanewline
\isacommand{next}\isamarkupfalse%
\isanewline
\ \ \isacommand{show}\isamarkupfalse%
\ {\isachardoublequoteopen}{\isacharparenleft}{\kern0pt}{\isasymforall}y\ {\isasymin}\ cube\ {\isadigit{1}}\ t{\isachardot}{\kern0pt}\ {\isacharparenleft}{\kern0pt}{\isasymforall}i\ {\isasymin}\ B\ {\isadigit{1}}{\isachardot}{\kern0pt}\ S\ y\ i\ {\isacharequal}{\kern0pt}\ f\ i{\isacharparenright}{\kern0pt}\ {\isasymand}\ {\isacharparenleft}{\kern0pt}{\isasymforall}i\ {\isasymin}\ B\ {\isadigit{0}}{\isachardot}{\kern0pt}\ {\isacharparenleft}{\kern0pt}S\ y{\isacharparenright}{\kern0pt}\ i\ {\isacharequal}{\kern0pt}\ y\ {\isadigit{0}}{\isacharparenright}{\kern0pt}{\isacharparenright}{\kern0pt}{\isachardoublequoteclose}\ \isacommand{using}\isamarkupfalse%
\ assms{\isacharparenleft}{\kern0pt}{\isadigit{6}}{\isacharparenright}{\kern0pt}\ \isacommand{by}\isamarkupfalse%
\ simp\isanewline
\isacommand{next}\isamarkupfalse%
\isanewline
\ \ \isacommand{show}\isamarkupfalse%
\ {\isachardoublequoteopen}B\ {\isadigit{0}}\ {\isasymnoteq}\ {\isacharbraceleft}{\kern0pt}{\isacharbraceright}{\kern0pt}{\isachardoublequoteclose}\ \isacommand{using}\isamarkupfalse%
\ assms{\isacharparenleft}{\kern0pt}{\isadigit{3}}{\isacharparenright}{\kern0pt}\ \isacommand{by}\isamarkupfalse%
\ auto\isanewline
\isacommand{qed}\isamarkupfalse%
%
\endisatagproof
{\isafoldproof}%
%
\isadelimproof
%
\endisadelimproof
%
\begin{isamarkuptext}%
Useful properties about cubes.%
\end{isamarkuptext}\isamarkuptrue%
\isacommand{lemma}\isamarkupfalse%
\ cube{\isacharunderscore}{\kern0pt}props{\isacharcolon}{\kern0pt}\isanewline
\ \ \isakeyword{shows}\ {\isachardoublequoteopen}{\isasymforall}s\ {\isasymin}\ {\isacharbraceleft}{\kern0pt}{\isachardot}{\kern0pt}{\isachardot}{\kern0pt}{\isacharless}{\kern0pt}t{\isacharbraceright}{\kern0pt}{\isachardot}{\kern0pt}\ {\isasymexists}p\ {\isasymin}\ cube\ {\isadigit{1}}\ t{\isachardot}{\kern0pt}\ p\ {\isadigit{0}}\ {\isacharequal}{\kern0pt}\ s{\isachardoublequoteclose}\isanewline
\ \ \ \ \isakeyword{and}\ {\isachardoublequoteopen}{\isasymforall}s\ {\isasymin}\ {\isacharbraceleft}{\kern0pt}{\isachardot}{\kern0pt}{\isachardot}{\kern0pt}{\isacharless}{\kern0pt}t{\isacharbraceright}{\kern0pt}{\isachardot}{\kern0pt}\ {\isacharparenleft}{\kern0pt}SOME\ p{\isachardot}{\kern0pt}\ p\ {\isasymin}\ cube\ {\isadigit{1}}\ t\ {\isasymand}\ p\ {\isadigit{0}}\ {\isacharequal}{\kern0pt}\ s{\isacharparenright}{\kern0pt}\ {\isadigit{0}}\ {\isacharequal}{\kern0pt}\ s{\isachardoublequoteclose}\isanewline
\ \ \ \ \isakeyword{and}\ {\isachardoublequoteopen}{\isasymforall}s\ {\isasymin}\ {\isacharbraceleft}{\kern0pt}{\isachardot}{\kern0pt}{\isachardot}{\kern0pt}{\isacharless}{\kern0pt}t{\isacharbraceright}{\kern0pt}{\isachardot}{\kern0pt}\ {\isacharparenleft}{\kern0pt}{\isasymlambda}s{\isasymin}{\isacharbraceleft}{\kern0pt}{\isachardot}{\kern0pt}{\isachardot}{\kern0pt}{\isacharless}{\kern0pt}t{\isacharbraceright}{\kern0pt}{\isachardot}{\kern0pt}\ S\ {\isacharparenleft}{\kern0pt}SOME\ p{\isachardot}{\kern0pt}\ p{\isasymin}cube\ {\isadigit{1}}\ t\ {\isasymand}\ p\ {\isadigit{0}}\ {\isacharequal}{\kern0pt}\ s{\isacharparenright}{\kern0pt}{\isacharparenright}{\kern0pt}\ s\ {\isacharequal}{\kern0pt}\ {\isacharparenleft}{\kern0pt}{\isasymlambda}s{\isasymin}{\isacharbraceleft}{\kern0pt}{\isachardot}{\kern0pt}{\isachardot}{\kern0pt}{\isacharless}{\kern0pt}t{\isacharbraceright}{\kern0pt}{\isachardot}{\kern0pt}\ S\ {\isacharparenleft}{\kern0pt}SOME\ p{\isachardot}{\kern0pt}\ p{\isasymin}cube\ {\isadigit{1}}\ t\ {\isasymand}\ p\ {\isadigit{0}}\ {\isacharequal}{\kern0pt}\ s{\isacharparenright}{\kern0pt}{\isacharparenright}{\kern0pt}\ {\isacharparenleft}{\kern0pt}{\isacharparenleft}{\kern0pt}SOME\ p{\isachardot}{\kern0pt}\ p\ {\isasymin}\ cube\ {\isadigit{1}}\ t\ {\isasymand}\ p\ {\isadigit{0}}\ {\isacharequal}{\kern0pt}\ s{\isacharparenright}{\kern0pt}\ {\isadigit{0}}{\isacharparenright}{\kern0pt}{\isachardoublequoteclose}\isanewline
\ \ \ \ \isakeyword{and}\ {\isachardoublequoteopen}{\isasymforall}s\ {\isasymin}\ {\isacharbraceleft}{\kern0pt}{\isachardot}{\kern0pt}{\isachardot}{\kern0pt}{\isacharless}{\kern0pt}t{\isacharbraceright}{\kern0pt}{\isachardot}{\kern0pt}\ {\isacharparenleft}{\kern0pt}SOME\ p{\isachardot}{\kern0pt}\ p\ {\isasymin}\ cube\ {\isadigit{1}}\ t\ {\isasymand}\ p\ {\isadigit{0}}\ {\isacharequal}{\kern0pt}\ s{\isacharparenright}{\kern0pt}\ {\isasymin}\ cube\ {\isadigit{1}}\ t{\isachardoublequoteclose}\isanewline
%
\isadelimproof
%
\endisadelimproof
%
\isatagproof
\isacommand{proof}\isamarkupfalse%
\ {\isacharminus}{\kern0pt}\isanewline
\ \ \isacommand{show}\isamarkupfalse%
\ {\isadigit{1}}{\isacharcolon}{\kern0pt}\ {\isachardoublequoteopen}{\isasymforall}s\ {\isasymin}\ {\isacharbraceleft}{\kern0pt}{\isachardot}{\kern0pt}{\isachardot}{\kern0pt}{\isacharless}{\kern0pt}t{\isacharbraceright}{\kern0pt}{\isachardot}{\kern0pt}\ {\isasymexists}p\ {\isasymin}\ cube\ {\isadigit{1}}\ t{\isachardot}{\kern0pt}\ p\ {\isadigit{0}}\ {\isacharequal}{\kern0pt}\ s{\isachardoublequoteclose}\ \isacommand{unfolding}\isamarkupfalse%
\ cube{\isacharunderscore}{\kern0pt}def\ \isacommand{by}\isamarkupfalse%
\ {\isacharparenleft}{\kern0pt}simp\ add{\isacharcolon}{\kern0pt}\ fun{\isacharunderscore}{\kern0pt}ex{\isacharparenright}{\kern0pt}\isanewline
\ \ \isacommand{show}\isamarkupfalse%
\ {\isadigit{2}}{\isacharcolon}{\kern0pt}\ {\isachardoublequoteopen}{\isasymforall}s\ {\isasymin}\ {\isacharbraceleft}{\kern0pt}{\isachardot}{\kern0pt}{\isachardot}{\kern0pt}{\isacharless}{\kern0pt}t{\isacharbraceright}{\kern0pt}{\isachardot}{\kern0pt}\ {\isacharparenleft}{\kern0pt}SOME\ p{\isachardot}{\kern0pt}\ p\ {\isasymin}\ cube\ {\isadigit{1}}\ t\ {\isasymand}\ p\ {\isadigit{0}}\ {\isacharequal}{\kern0pt}\ s{\isacharparenright}{\kern0pt}\ {\isadigit{0}}\ {\isacharequal}{\kern0pt}\ s{\isachardoublequoteclose}\isanewline
\ \ \isacommand{proof}\isamarkupfalse%
{\isacharparenleft}{\kern0pt}safe{\isacharparenright}{\kern0pt}\isanewline
\ \ \ \ \isacommand{fix}\isamarkupfalse%
\ s\isanewline
\ \ \ \ \isacommand{assume}\isamarkupfalse%
\ {\isachardoublequoteopen}s\ {\isacharless}{\kern0pt}\ t{\isachardoublequoteclose}\isanewline
\ \ \ \ \isacommand{then}\isamarkupfalse%
\ \isacommand{have}\isamarkupfalse%
\ {\isachardoublequoteopen}{\isasymexists}p\ {\isasymin}\ cube\ {\isadigit{1}}\ t{\isachardot}{\kern0pt}\ p\ {\isadigit{0}}\ {\isacharequal}{\kern0pt}\ s{\isachardoublequoteclose}\ \isanewline
\ \ \ \ \ \ \isacommand{using}\isamarkupfalse%
\ {\isacartoucheopen}{\isasymforall}s{\isasymin}{\isacharbraceleft}{\kern0pt}{\isachardot}{\kern0pt}{\isachardot}{\kern0pt}{\isacharless}{\kern0pt}t{\isacharbraceright}{\kern0pt}{\isachardot}{\kern0pt}\ {\isasymexists}p{\isasymin}cube\ {\isadigit{1}}\ t{\isachardot}{\kern0pt}\ p\ {\isadigit{0}}\ {\isacharequal}{\kern0pt}\ s{\isacartoucheclose}\ \isacommand{by}\isamarkupfalse%
\ blast\isanewline
\ \ \ \ \isacommand{then}\isamarkupfalse%
\ \isacommand{show}\isamarkupfalse%
\ {\isachardoublequoteopen}{\isacharparenleft}{\kern0pt}SOME\ p{\isachardot}{\kern0pt}\ p\ {\isasymin}\ cube\ {\isadigit{1}}\ t\ {\isasymand}\ \ p\ {\isadigit{0}}\ {\isacharequal}{\kern0pt}\ s{\isacharparenright}{\kern0pt}\ {\isadigit{0}}\ {\isacharequal}{\kern0pt}\ s{\isachardoublequoteclose}\ \isacommand{using}\isamarkupfalse%
\ someI{\isacharunderscore}{\kern0pt}ex{\isacharbrackleft}{\kern0pt}of\ {\isachardoublequoteopen}{\isasymlambda}x{\isachardot}{\kern0pt}\ x\ {\isasymin}\ cube\ {\isadigit{1}}\ t\ {\isasymand}\ x\ {\isadigit{0}}\ {\isacharequal}{\kern0pt}\ s{\isachardoublequoteclose}{\isacharbrackright}{\kern0pt}\ \isacommand{by}\isamarkupfalse%
\ auto\isanewline
\ \ \isacommand{qed}\isamarkupfalse%
\isanewline
\isanewline
\ \ \isacommand{show}\isamarkupfalse%
\ {\isadigit{3}}{\isacharcolon}{\kern0pt}\ {\isachardoublequoteopen}{\isasymforall}s\ {\isasymin}\ {\isacharbraceleft}{\kern0pt}{\isachardot}{\kern0pt}{\isachardot}{\kern0pt}{\isacharless}{\kern0pt}t{\isacharbraceright}{\kern0pt}{\isachardot}{\kern0pt}\ {\isacharparenleft}{\kern0pt}{\isasymlambda}s{\isasymin}{\isacharbraceleft}{\kern0pt}{\isachardot}{\kern0pt}{\isachardot}{\kern0pt}{\isacharless}{\kern0pt}t{\isacharbraceright}{\kern0pt}{\isachardot}{\kern0pt}\ S\ {\isacharparenleft}{\kern0pt}SOME\ p{\isachardot}{\kern0pt}\ p{\isasymin}cube\ {\isadigit{1}}\ t\ {\isasymand}\ p\ {\isadigit{0}}\ {\isacharequal}{\kern0pt}\ s{\isacharparenright}{\kern0pt}{\isacharparenright}{\kern0pt}\ s\ {\isacharequal}{\kern0pt}\ {\isacharparenleft}{\kern0pt}{\isasymlambda}s{\isasymin}{\isacharbraceleft}{\kern0pt}{\isachardot}{\kern0pt}{\isachardot}{\kern0pt}{\isacharless}{\kern0pt}t{\isacharbraceright}{\kern0pt}{\isachardot}{\kern0pt}\ S\ {\isacharparenleft}{\kern0pt}SOME\ p{\isachardot}{\kern0pt}\ p{\isasymin}cube\ {\isadigit{1}}\ t\ {\isasymand}\ p\ {\isadigit{0}}\ {\isacharequal}{\kern0pt}\ s{\isacharparenright}{\kern0pt}{\isacharparenright}{\kern0pt}\ {\isacharparenleft}{\kern0pt}{\isacharparenleft}{\kern0pt}SOME\ p{\isachardot}{\kern0pt}\ p\ {\isasymin}\ cube\ {\isadigit{1}}\ t\ {\isasymand}\ p\ {\isadigit{0}}\ {\isacharequal}{\kern0pt}\ s{\isacharparenright}{\kern0pt}\ {\isadigit{0}}{\isacharparenright}{\kern0pt}{\isachardoublequoteclose}\ \isacommand{using}\isamarkupfalse%
\ {\isadigit{2}}\ \isacommand{by}\isamarkupfalse%
\ simp\isanewline
\ \ \isacommand{have}\isamarkupfalse%
\ {\isadigit{4}}{\isacharcolon}{\kern0pt}\ {\isachardoublequoteopen}{\isacharparenleft}{\kern0pt}SOME\ p{\isachardot}{\kern0pt}\ p\ {\isasymin}\ cube\ {\isadigit{1}}\ t\ {\isasymand}\ p\ {\isadigit{0}}\ {\isacharequal}{\kern0pt}\ s{\isacharparenright}{\kern0pt}\ {\isasymin}\ cube\ {\isadigit{1}}\ t{\isachardoublequoteclose}\ \isakeyword{if}\ {\isachardoublequoteopen}s\ {\isasymin}\ {\isacharbraceleft}{\kern0pt}{\isachardot}{\kern0pt}{\isachardot}{\kern0pt}{\isacharless}{\kern0pt}t{\isacharbraceright}{\kern0pt}{\isachardoublequoteclose}\ \isakeyword{for}\ s\ \isacommand{using}\isamarkupfalse%
\ {\isadigit{1}}\ someI{\isacharunderscore}{\kern0pt}ex{\isacharbrackleft}{\kern0pt}of\ {\isachardoublequoteopen}{\isasymlambda}p{\isachardot}{\kern0pt}\ p\ {\isasymin}\ cube\ {\isadigit{1}}\ t\ {\isasymand}\ p\ {\isadigit{0}}\ {\isacharequal}{\kern0pt}\ s{\isachardoublequoteclose}{\isacharbrackright}{\kern0pt}\ that\ \isacommand{by}\isamarkupfalse%
\ blast\isanewline
\ \ \isacommand{then}\isamarkupfalse%
\ \isacommand{show}\isamarkupfalse%
\ {\isachardoublequoteopen}{\isasymforall}s\ {\isasymin}\ {\isacharbraceleft}{\kern0pt}{\isachardot}{\kern0pt}{\isachardot}{\kern0pt}{\isacharless}{\kern0pt}t{\isacharbraceright}{\kern0pt}{\isachardot}{\kern0pt}\ {\isacharparenleft}{\kern0pt}SOME\ p{\isachardot}{\kern0pt}\ p\ {\isasymin}\ cube\ {\isadigit{1}}\ t\ {\isasymand}\ p\ {\isadigit{0}}\ {\isacharequal}{\kern0pt}\ s{\isacharparenright}{\kern0pt}\ {\isasymin}\ cube\ {\isadigit{1}}\ t{\isachardoublequoteclose}\ \isacommand{by}\isamarkupfalse%
\ simp\isanewline
\isacommand{qed}\isamarkupfalse%
%
\endisatagproof
{\isafoldproof}%
%
\isadelimproof
\isanewline
%
\endisadelimproof
\isanewline
\isacommand{lemma}\isamarkupfalse%
\ dim{\isadigit{1}}{\isacharunderscore}{\kern0pt}subspace{\isacharunderscore}{\kern0pt}is{\isacharunderscore}{\kern0pt}line{\isacharcolon}{\kern0pt}\ \isanewline
\ \ \isakeyword{assumes}\ {\isachardoublequoteopen}t\ {\isachargreater}{\kern0pt}\ {\isadigit{0}}{\isachardoublequoteclose}\ \isanewline
\ \ \ \ \isakeyword{and}\ {\isachardoublequoteopen}is{\isacharunderscore}{\kern0pt}subspace\ S\ {\isadigit{1}}\ n\ t{\isachardoublequoteclose}\ \isanewline
\ \ \isakeyword{shows}\ \ \ {\isachardoublequoteopen}is{\isacharunderscore}{\kern0pt}line\ {\isacharparenleft}{\kern0pt}{\isasymlambda}s{\isasymin}{\isacharbraceleft}{\kern0pt}{\isachardot}{\kern0pt}{\isachardot}{\kern0pt}{\isacharless}{\kern0pt}t{\isacharbraceright}{\kern0pt}{\isachardot}{\kern0pt}\ S\ {\isacharparenleft}{\kern0pt}SOME\ p{\isachardot}{\kern0pt}\ p{\isasymin}cube\ {\isadigit{1}}\ t\ {\isasymand}\ p\ {\isadigit{0}}\ {\isacharequal}{\kern0pt}\ s{\isacharparenright}{\kern0pt}{\isacharparenright}{\kern0pt}\ n\ t{\isachardoublequoteclose}\isanewline
%
\isadelimproof
%
\endisadelimproof
%
\isatagproof
\isacommand{proof}\isamarkupfalse%
{\isacharminus}{\kern0pt}\isanewline
\ \ \isacommand{define}\isamarkupfalse%
\ L\ \isakeyword{where}\ {\isachardoublequoteopen}L\ {\isasymequiv}\ {\isacharparenleft}{\kern0pt}{\isasymlambda}s{\isasymin}{\isacharbraceleft}{\kern0pt}{\isachardot}{\kern0pt}{\isachardot}{\kern0pt}{\isacharless}{\kern0pt}t{\isacharbraceright}{\kern0pt}{\isachardot}{\kern0pt}\ S\ {\isacharparenleft}{\kern0pt}SOME\ p{\isachardot}{\kern0pt}\ p{\isasymin}cube\ {\isadigit{1}}\ t\ {\isasymand}\ p\ {\isadigit{0}}\ {\isacharequal}{\kern0pt}\ s{\isacharparenright}{\kern0pt}{\isacharparenright}{\kern0pt}{\isachardoublequoteclose}\isanewline
\ \ \isacommand{have}\isamarkupfalse%
\ {\isachardoublequoteopen}{\isacharbraceleft}{\kern0pt}{\isachardot}{\kern0pt}{\isachardot}{\kern0pt}{\isadigit{1}}{\isacharbraceright}{\kern0pt}\ {\isacharequal}{\kern0pt}\ {\isacharbraceleft}{\kern0pt}{\isadigit{0}}{\isacharcolon}{\kern0pt}{\isacharcolon}{\kern0pt}nat{\isacharcomma}{\kern0pt}\ {\isadigit{1}}{\isacharbraceright}{\kern0pt}{\isachardoublequoteclose}\ \isacommand{by}\isamarkupfalse%
\ auto\isanewline
\ \ \isacommand{obtain}\isamarkupfalse%
\ B\ f\ \isakeyword{where}\ Bf{\isacharunderscore}{\kern0pt}props{\isacharcolon}{\kern0pt}\ {\isachardoublequoteopen}disjoint{\isacharunderscore}{\kern0pt}family{\isacharunderscore}{\kern0pt}on\ B\ {\isacharbraceleft}{\kern0pt}{\isachardot}{\kern0pt}{\isachardot}{\kern0pt}{\isadigit{1}}{\isacharcolon}{\kern0pt}{\isacharcolon}{\kern0pt}nat{\isacharbraceright}{\kern0pt}\ {\isasymand}\ {\isasymUnion}{\isacharparenleft}{\kern0pt}B\ {\isacharbackquote}{\kern0pt}\ {\isacharbraceleft}{\kern0pt}{\isachardot}{\kern0pt}{\isachardot}{\kern0pt}{\isadigit{1}}{\isacharcolon}{\kern0pt}{\isacharcolon}{\kern0pt}nat{\isacharbraceright}{\kern0pt}{\isacharparenright}{\kern0pt}\ {\isacharequal}{\kern0pt}\ {\isacharbraceleft}{\kern0pt}{\isachardot}{\kern0pt}{\isachardot}{\kern0pt}{\isacharless}{\kern0pt}n{\isacharbraceright}{\kern0pt}\ {\isasymand}\ {\isacharparenleft}{\kern0pt}{\isacharbraceleft}{\kern0pt}{\isacharbraceright}{\kern0pt}\ {\isasymnotin}\ B\ {\isacharbackquote}{\kern0pt}\ {\isacharbraceleft}{\kern0pt}{\isachardot}{\kern0pt}{\isachardot}{\kern0pt}{\isacharless}{\kern0pt}{\isadigit{1}}{\isacharcolon}{\kern0pt}{\isacharcolon}{\kern0pt}nat{\isacharbraceright}{\kern0pt}{\isacharparenright}{\kern0pt}\ {\isasymand}\ f\ {\isasymin}\ {\isacharparenleft}{\kern0pt}B\ {\isadigit{1}}{\isacharparenright}{\kern0pt}\ {\isasymrightarrow}\isactrlsub E\ {\isacharbraceleft}{\kern0pt}{\isachardot}{\kern0pt}{\isachardot}{\kern0pt}{\isacharless}{\kern0pt}t{\isacharbraceright}{\kern0pt}\ {\isasymand}\ S\ {\isasymin}\ {\isacharparenleft}{\kern0pt}cube\ {\isadigit{1}}\ t{\isacharparenright}{\kern0pt}\ {\isasymrightarrow}\isactrlsub E\ {\isacharparenleft}{\kern0pt}cube\ n\ t{\isacharparenright}{\kern0pt}\ {\isasymand}\ {\isacharparenleft}{\kern0pt}{\isasymforall}y\ {\isasymin}\ cube\ {\isadigit{1}}\ t{\isachardot}{\kern0pt}\ {\isacharparenleft}{\kern0pt}{\isasymforall}i\ {\isasymin}\ B\ {\isadigit{1}}{\isachardot}{\kern0pt}\ S\ y\ i\ {\isacharequal}{\kern0pt}\ f\ i{\isacharparenright}{\kern0pt}\ {\isasymand}\ {\isacharparenleft}{\kern0pt}{\isasymforall}j{\isacharless}{\kern0pt}{\isadigit{1}}{\isachardot}{\kern0pt}\ {\isasymforall}i\ {\isasymin}\ B\ j{\isachardot}{\kern0pt}\ {\isacharparenleft}{\kern0pt}S\ y{\isacharparenright}{\kern0pt}\ i\ {\isacharequal}{\kern0pt}\ y\ j{\isacharparenright}{\kern0pt}{\isacharparenright}{\kern0pt}{\isachardoublequoteclose}\ \isacommand{using}\isamarkupfalse%
\ assms{\isacharparenleft}{\kern0pt}{\isadigit{2}}{\isacharparenright}{\kern0pt}\ \isacommand{unfolding}\isamarkupfalse%
\ is{\isacharunderscore}{\kern0pt}subspace{\isacharunderscore}{\kern0pt}def\ \isacommand{by}\isamarkupfalse%
\ auto\isanewline
\ \ \isacommand{then}\isamarkupfalse%
\ \isacommand{have}\isamarkupfalse%
\ {\isadigit{1}}{\isacharcolon}{\kern0pt}\ {\isachardoublequoteopen}B\ {\isadigit{0}}\ {\isasymunion}\ B\ {\isadigit{1}}\ {\isacharequal}{\kern0pt}\ {\isacharbraceleft}{\kern0pt}{\isachardot}{\kern0pt}{\isachardot}{\kern0pt}{\isacharless}{\kern0pt}n{\isacharbraceright}{\kern0pt}\ {\isasymand}\ B\ {\isadigit{0}}\ {\isasyminter}\ B\ {\isadigit{1}}\ {\isacharequal}{\kern0pt}\ {\isacharbraceleft}{\kern0pt}{\isacharbraceright}{\kern0pt}{\isachardoublequoteclose}\ \isacommand{using}\isamarkupfalse%
\ dim{\isadigit{1}}{\isacharunderscore}{\kern0pt}subspace{\isacharunderscore}{\kern0pt}elims{\isacharparenleft}{\kern0pt}{\isadigit{1}}{\isacharcomma}{\kern0pt}\ {\isadigit{2}}{\isacharparenright}{\kern0pt}{\isacharbrackleft}{\kern0pt}of\ B\ n\ f\ t\ S{\isacharbrackright}{\kern0pt}\ \isacommand{by}\isamarkupfalse%
\ simp\isanewline
\isanewline
\ \ \isacommand{have}\isamarkupfalse%
\ {\isachardoublequoteopen}L\ {\isasymin}\ {\isacharbraceleft}{\kern0pt}{\isachardot}{\kern0pt}{\isachardot}{\kern0pt}{\isacharless}{\kern0pt}t{\isacharbraceright}{\kern0pt}\ {\isasymrightarrow}\isactrlsub E\ cube\ n\ t{\isachardoublequoteclose}\isanewline
\ \ \isacommand{proof}\isamarkupfalse%
\isanewline
\ \ \ \ \isacommand{fix}\isamarkupfalse%
\ s\ \isacommand{assume}\isamarkupfalse%
\ a{\isacharcolon}{\kern0pt}\ {\isachardoublequoteopen}s\ {\isasymin}\ {\isacharbraceleft}{\kern0pt}{\isachardot}{\kern0pt}{\isachardot}{\kern0pt}{\isacharless}{\kern0pt}t{\isacharbraceright}{\kern0pt}{\isachardoublequoteclose}\isanewline
\ \ \ \ \isacommand{then}\isamarkupfalse%
\ \isacommand{have}\isamarkupfalse%
\ {\isachardoublequoteopen}L\ s\ {\isacharequal}{\kern0pt}\ S\ {\isacharparenleft}{\kern0pt}SOME\ p{\isachardot}{\kern0pt}\ p{\isasymin}cube\ {\isadigit{1}}\ t\ {\isasymand}\ p\ {\isadigit{0}}\ {\isacharequal}{\kern0pt}\ s{\isacharparenright}{\kern0pt}{\isachardoublequoteclose}\ \isacommand{unfolding}\isamarkupfalse%
\ L{\isacharunderscore}{\kern0pt}def\ \isacommand{by}\isamarkupfalse%
\ simp\isanewline
\ \ \ \ \isacommand{moreover}\isamarkupfalse%
\ \isacommand{have}\isamarkupfalse%
\ {\isachardoublequoteopen}{\isacharparenleft}{\kern0pt}SOME\ p{\isachardot}{\kern0pt}\ p{\isasymin}cube\ {\isadigit{1}}\ t\ {\isasymand}\ p\ {\isadigit{0}}\ {\isacharequal}{\kern0pt}\ s{\isacharparenright}{\kern0pt}\ {\isasymin}\ cube\ {\isadigit{1}}\ t{\isachardoublequoteclose}\ \isacommand{using}\isamarkupfalse%
\ cube{\isacharunderscore}{\kern0pt}props{\isacharparenleft}{\kern0pt}{\isadigit{1}}{\isacharparenright}{\kern0pt}\ a\ someI{\isacharunderscore}{\kern0pt}ex{\isacharbrackleft}{\kern0pt}of\ {\isachardoublequoteopen}{\isasymlambda}p{\isachardot}{\kern0pt}\ p\ {\isasymin}\ cube\ {\isadigit{1}}\ t\ {\isasymand}\ p\ {\isadigit{0}}\ {\isacharequal}{\kern0pt}\ s{\isachardoublequoteclose}{\isacharbrackright}{\kern0pt}\ \isacommand{by}\isamarkupfalse%
\ blast\isanewline
\ \ \ \ \isacommand{moreover}\isamarkupfalse%
\ \isacommand{have}\isamarkupfalse%
\ {\isachardoublequoteopen}S\ {\isacharparenleft}{\kern0pt}SOME\ p{\isachardot}{\kern0pt}\ p{\isasymin}cube\ {\isadigit{1}}\ t\ {\isasymand}\ p\ {\isadigit{0}}\ {\isacharequal}{\kern0pt}\ s{\isacharparenright}{\kern0pt}\ {\isasymin}\ cube\ n\ t{\isachardoublequoteclose}\isanewline
\ \ \ \ \ \ \isacommand{using}\isamarkupfalse%
\ assms{\isacharparenleft}{\kern0pt}{\isadigit{2}}{\isacharparenright}{\kern0pt}\ calculation{\isacharparenleft}{\kern0pt}{\isadigit{2}}{\isacharparenright}{\kern0pt}\ is{\isacharunderscore}{\kern0pt}subspace{\isacharunderscore}{\kern0pt}def\ \isacommand{by}\isamarkupfalse%
\ auto\isanewline
\ \ \ \ \isacommand{ultimately}\isamarkupfalse%
\ \isacommand{show}\isamarkupfalse%
\ {\isachardoublequoteopen}L\ s\ {\isasymin}\ cube\ n\ t{\isachardoublequoteclose}\ \isacommand{by}\isamarkupfalse%
\ simp\isanewline
\ \ \isacommand{next}\isamarkupfalse%
\isanewline
\ \ \ \ \isacommand{fix}\isamarkupfalse%
\ s\ \isacommand{assume}\isamarkupfalse%
\ a{\isacharcolon}{\kern0pt}\ {\isachardoublequoteopen}s\ {\isasymnotin}\ {\isacharbraceleft}{\kern0pt}{\isachardot}{\kern0pt}{\isachardot}{\kern0pt}{\isacharless}{\kern0pt}t{\isacharbraceright}{\kern0pt}{\isachardoublequoteclose}\isanewline
\ \ \ \ \isacommand{then}\isamarkupfalse%
\ \isacommand{show}\isamarkupfalse%
\ {\isachardoublequoteopen}L\ s\ {\isacharequal}{\kern0pt}\ undefined{\isachardoublequoteclose}\ \isacommand{unfolding}\isamarkupfalse%
\ L{\isacharunderscore}{\kern0pt}def\ \isacommand{by}\isamarkupfalse%
\ simp\isanewline
\ \ \isacommand{qed}\isamarkupfalse%
\isanewline
\ \ \isacommand{moreover}\isamarkupfalse%
\ \isacommand{have}\isamarkupfalse%
\ {\isachardoublequoteopen}{\isacharparenleft}{\kern0pt}{\isasymforall}x{\isacharless}{\kern0pt}t{\isachardot}{\kern0pt}\ {\isasymforall}y{\isacharless}{\kern0pt}t{\isachardot}{\kern0pt}\ L\ x\ j\ {\isacharequal}{\kern0pt}\ L\ y\ j{\isacharparenright}{\kern0pt}\ {\isasymor}\ {\isacharparenleft}{\kern0pt}{\isasymforall}s{\isacharless}{\kern0pt}t{\isachardot}{\kern0pt}\ L\ s\ j\ {\isacharequal}{\kern0pt}\ s{\isacharparenright}{\kern0pt}{\isachardoublequoteclose}\ \isakeyword{if}\ {\isachardoublequoteopen}j\ {\isacharless}{\kern0pt}\ n{\isachardoublequoteclose}\ \isakeyword{for}\ j\isanewline
\ \ \isacommand{proof}\isamarkupfalse%
{\isacharminus}{\kern0pt}\isanewline
\ \ \ \ \isacommand{consider}\isamarkupfalse%
\ {\isachardoublequoteopen}j\ {\isasymin}\ B\ {\isadigit{0}}{\isachardoublequoteclose}\ {\isacharbar}{\kern0pt}\ {\isachardoublequoteopen}j\ {\isasymin}\ B\ {\isadigit{1}}{\isachardoublequoteclose}\ \isacommand{using}\isamarkupfalse%
\ {\isacartoucheopen}j\ {\isacharless}{\kern0pt}\ n{\isacartoucheclose}\ {\isadigit{1}}\ \isacommand{by}\isamarkupfalse%
\ blast\ \isanewline
\ \ \ \ \isacommand{then}\isamarkupfalse%
\ \isacommand{show}\isamarkupfalse%
\ {\isachardoublequoteopen}{\isacharparenleft}{\kern0pt}{\isasymforall}x{\isacharless}{\kern0pt}t{\isachardot}{\kern0pt}\ {\isasymforall}y{\isacharless}{\kern0pt}t{\isachardot}{\kern0pt}\ L\ x\ j\ {\isacharequal}{\kern0pt}\ L\ y\ j{\isacharparenright}{\kern0pt}\ {\isasymor}\ {\isacharparenleft}{\kern0pt}{\isasymforall}s{\isacharless}{\kern0pt}t{\isachardot}{\kern0pt}\ L\ s\ j\ {\isacharequal}{\kern0pt}\ s{\isacharparenright}{\kern0pt}{\isachardoublequoteclose}\isanewline
\ \ \ \ \isacommand{proof}\isamarkupfalse%
\ {\isacharparenleft}{\kern0pt}cases{\isacharparenright}{\kern0pt}\isanewline
\ \ \ \ \ \ \isacommand{case}\isamarkupfalse%
\ {\isadigit{1}}\isanewline
\ \ \ \ \ \ \isacommand{have}\isamarkupfalse%
\ {\isachardoublequoteopen}L\ s\ j\ {\isacharequal}{\kern0pt}\ s{\isachardoublequoteclose}\ \isakeyword{if}\ {\isachardoublequoteopen}s\ {\isacharless}{\kern0pt}\ t{\isachardoublequoteclose}\ \isakeyword{for}\ s\isanewline
\ \ \ \ \ \ \isacommand{proof}\isamarkupfalse%
{\isacharminus}{\kern0pt}\isanewline
\ \ \ \ \ \ \ \ \isacommand{have}\isamarkupfalse%
\ {\isachardoublequoteopen}{\isasymforall}y\ {\isasymin}\ cube\ {\isadigit{1}}\ t{\isachardot}{\kern0pt}\ {\isacharparenleft}{\kern0pt}S\ y{\isacharparenright}{\kern0pt}\ j\ {\isacharequal}{\kern0pt}\ y\ {\isadigit{0}}{\isachardoublequoteclose}\ \isacommand{using}\isamarkupfalse%
\ Bf{\isacharunderscore}{\kern0pt}props\ {\isadigit{1}}\ \isacommand{by}\isamarkupfalse%
\ simp\isanewline
\ \ \ \ \ \ \ \ \isacommand{then}\isamarkupfalse%
\ \isacommand{show}\isamarkupfalse%
\ {\isachardoublequoteopen}L\ s\ j\ {\isacharequal}{\kern0pt}\ s{\isachardoublequoteclose}\ \isacommand{using}\isamarkupfalse%
\ that\ cube{\isacharunderscore}{\kern0pt}props{\isacharparenleft}{\kern0pt}{\isadigit{2}}{\isacharcomma}{\kern0pt}{\isadigit{4}}{\isacharparenright}{\kern0pt}\ \ \isacommand{unfolding}\isamarkupfalse%
\ L{\isacharunderscore}{\kern0pt}def\ \isacommand{by}\isamarkupfalse%
\ auto\isanewline
\ \ \ \ \ \ \isacommand{qed}\isamarkupfalse%
\isanewline
\ \ \ \ \ \ \isacommand{then}\isamarkupfalse%
\ \isacommand{show}\isamarkupfalse%
\ {\isacharquery}{\kern0pt}thesis\ \isacommand{by}\isamarkupfalse%
\ blast\isanewline
\ \ \ \ \isacommand{next}\isamarkupfalse%
\isanewline
\ \ \ \ \ \ \isacommand{case}\isamarkupfalse%
\ {\isadigit{2}}\isanewline
\ \ \ \ \ \ \isacommand{have}\isamarkupfalse%
\ {\isachardoublequoteopen}L\ x\ j\ {\isacharequal}{\kern0pt}\ L\ y\ j{\isachardoublequoteclose}\ \isakeyword{if}\ {\isachardoublequoteopen}x\ {\isacharless}{\kern0pt}\ t{\isachardoublequoteclose}\ \isakeyword{and}\ {\isachardoublequoteopen}y\ {\isacharless}{\kern0pt}\ t{\isachardoublequoteclose}\ \isakeyword{for}\ x\ y\isanewline
\ \ \ \ \ \ \isacommand{proof}\isamarkupfalse%
{\isacharminus}{\kern0pt}\isanewline
\ \ \ \ \ \ \ \ \isacommand{have}\isamarkupfalse%
\ {\isacharasterisk}{\kern0pt}{\isacharcolon}{\kern0pt}\ {\isachardoublequoteopen}S\ y\ j\ {\isacharequal}{\kern0pt}\ f\ j{\isachardoublequoteclose}\ \isakeyword{if}\ {\isachardoublequoteopen}y\ {\isasymin}\ cube\ {\isadigit{1}}\ t{\isachardoublequoteclose}\ \isakeyword{for}\ y\ \isacommand{using}\isamarkupfalse%
\ {\isadigit{2}}\ that\ Bf{\isacharunderscore}{\kern0pt}props\ \isacommand{by}\isamarkupfalse%
\ simp\isanewline
\ \ \ \ \ \ \ \ \isacommand{then}\isamarkupfalse%
\ \isacommand{have}\isamarkupfalse%
\ {\isachardoublequoteopen}L\ y\ j\ {\isacharequal}{\kern0pt}\ f\ j{\isachardoublequoteclose}\ \isacommand{using}\isamarkupfalse%
\ that{\isacharparenleft}{\kern0pt}{\isadigit{2}}{\isacharparenright}{\kern0pt}\ cube{\isacharunderscore}{\kern0pt}props{\isacharparenleft}{\kern0pt}{\isadigit{2}}{\isacharcomma}{\kern0pt}{\isadigit{4}}{\isacharparenright}{\kern0pt}\ lessThan{\isacharunderscore}{\kern0pt}iff\ restrict{\isacharunderscore}{\kern0pt}apply\ \isacommand{unfolding}\isamarkupfalse%
\ L{\isacharunderscore}{\kern0pt}def\ \isacommand{by}\isamarkupfalse%
\ fastforce\isanewline
\ \ \ \ \ \ \ \ \isacommand{moreover}\isamarkupfalse%
\ \isacommand{from}\isamarkupfalse%
\ {\isacharasterisk}{\kern0pt}\ \isacommand{have}\isamarkupfalse%
\ {\isachardoublequoteopen}L\ x\ j\ {\isacharequal}{\kern0pt}\ f\ j{\isachardoublequoteclose}\ \isacommand{using}\isamarkupfalse%
\ that{\isacharparenleft}{\kern0pt}{\isadigit{1}}{\isacharparenright}{\kern0pt}\ cube{\isacharunderscore}{\kern0pt}props{\isacharparenleft}{\kern0pt}{\isadigit{2}}{\isacharcomma}{\kern0pt}{\isadigit{4}}{\isacharparenright}{\kern0pt}\ lessThan{\isacharunderscore}{\kern0pt}iff\ restrict{\isacharunderscore}{\kern0pt}apply\ \isacommand{unfolding}\isamarkupfalse%
\ L{\isacharunderscore}{\kern0pt}def\ \isacommand{by}\isamarkupfalse%
\ fastforce\isanewline
\ \ \ \ \ \ \ \ \isacommand{ultimately}\isamarkupfalse%
\ \isacommand{show}\isamarkupfalse%
\ {\isachardoublequoteopen}L\ x\ j\ {\isacharequal}{\kern0pt}\ L\ y\ j{\isachardoublequoteclose}\ \isacommand{by}\isamarkupfalse%
\ simp\isanewline
\ \ \ \ \ \ \isacommand{qed}\isamarkupfalse%
\isanewline
\ \ \ \ \ \ \isacommand{then}\isamarkupfalse%
\ \isacommand{show}\isamarkupfalse%
\ {\isacharquery}{\kern0pt}thesis\ \isacommand{by}\isamarkupfalse%
\ blast\isanewline
\ \ \ \ \isacommand{qed}\isamarkupfalse%
\isanewline
\ \ \isacommand{qed}\isamarkupfalse%
\isanewline
\ \ \isacommand{moreover}\isamarkupfalse%
\ \isacommand{have}\isamarkupfalse%
\ {\isachardoublequoteopen}{\isacharparenleft}{\kern0pt}{\isasymexists}j{\isacharless}{\kern0pt}n{\isachardot}{\kern0pt}\ {\isasymforall}s{\isacharless}{\kern0pt}t{\isachardot}{\kern0pt}\ {\isacharparenleft}{\kern0pt}L\ s\ j\ {\isacharequal}{\kern0pt}\ s{\isacharparenright}{\kern0pt}{\isacharparenright}{\kern0pt}{\isachardoublequoteclose}\isanewline
\ \ \isacommand{proof}\isamarkupfalse%
\ {\isacharminus}{\kern0pt}\isanewline
\ \ \ \ \isacommand{obtain}\isamarkupfalse%
\ j\ \isakeyword{where}\ j{\isacharunderscore}{\kern0pt}prop{\isacharcolon}{\kern0pt}\ {\isachardoublequoteopen}j\ {\isasymin}\ B\ {\isadigit{0}}\ {\isasymand}\ j\ {\isacharless}{\kern0pt}\ n{\isachardoublequoteclose}\ \isacommand{using}\isamarkupfalse%
\ Bf{\isacharunderscore}{\kern0pt}props\ \isacommand{by}\isamarkupfalse%
\ blast\isanewline
\ \ \ \ \isacommand{then}\isamarkupfalse%
\ \isacommand{have}\isamarkupfalse%
\ {\isachardoublequoteopen}{\isacharparenleft}{\kern0pt}S\ y{\isacharparenright}{\kern0pt}\ j\ {\isacharequal}{\kern0pt}\ y\ {\isadigit{0}}{\isachardoublequoteclose}\ \isakeyword{if}\ {\isachardoublequoteopen}y\ {\isasymin}\ cube\ {\isadigit{1}}\ t{\isachardoublequoteclose}\ \isakeyword{for}\ y\ \isacommand{using}\isamarkupfalse%
\ that\ Bf{\isacharunderscore}{\kern0pt}props\ \isacommand{by}\isamarkupfalse%
\ auto\isanewline
\ \ \ \ \isacommand{then}\isamarkupfalse%
\ \isacommand{have}\isamarkupfalse%
\ {\isachardoublequoteopen}L\ s\ j\ {\isacharequal}{\kern0pt}\ s{\isachardoublequoteclose}\ \isakeyword{if}\ {\isachardoublequoteopen}s\ {\isacharless}{\kern0pt}\ t{\isachardoublequoteclose}\ \isakeyword{for}\ s\ \isacommand{using}\isamarkupfalse%
\ that\ cube{\isacharunderscore}{\kern0pt}props{\isacharparenleft}{\kern0pt}{\isadigit{2}}{\isacharcomma}{\kern0pt}{\isadigit{4}}{\isacharparenright}{\kern0pt}\ \isacommand{unfolding}\isamarkupfalse%
\ L{\isacharunderscore}{\kern0pt}def\ \isacommand{by}\isamarkupfalse%
\ auto\isanewline
\ \ \ \ \isacommand{then}\isamarkupfalse%
\ \isacommand{show}\isamarkupfalse%
\ {\isachardoublequoteopen}{\isasymexists}j{\isacharless}{\kern0pt}n{\isachardot}{\kern0pt}\ {\isasymforall}s{\isacharless}{\kern0pt}t{\isachardot}{\kern0pt}\ {\isacharparenleft}{\kern0pt}L\ s\ j\ {\isacharequal}{\kern0pt}\ s{\isacharparenright}{\kern0pt}{\isachardoublequoteclose}\ \isacommand{using}\isamarkupfalse%
\ j{\isacharunderscore}{\kern0pt}prop\ \isacommand{by}\isamarkupfalse%
\ blast\isanewline
\ \ \isacommand{qed}\isamarkupfalse%
\isanewline
\ \ \isacommand{ultimately}\isamarkupfalse%
\ \isacommand{show}\isamarkupfalse%
\ {\isachardoublequoteopen}is{\isacharunderscore}{\kern0pt}line\ {\isacharparenleft}{\kern0pt}{\isasymlambda}s{\isasymin}{\isacharbraceleft}{\kern0pt}{\isachardot}{\kern0pt}{\isachardot}{\kern0pt}{\isacharless}{\kern0pt}t{\isacharbraceright}{\kern0pt}{\isachardot}{\kern0pt}\ S\ {\isacharparenleft}{\kern0pt}SOME\ p{\isachardot}{\kern0pt}\ p{\isasymin}cube\ {\isadigit{1}}\ t\ {\isasymand}\ p\ {\isadigit{0}}\ {\isacharequal}{\kern0pt}\ s{\isacharparenright}{\kern0pt}{\isacharparenright}{\kern0pt}\ n\ t{\isachardoublequoteclose}\ \isacommand{unfolding}\isamarkupfalse%
\ L{\isacharunderscore}{\kern0pt}def\ is{\isacharunderscore}{\kern0pt}line{\isacharunderscore}{\kern0pt}def\ \isacommand{by}\isamarkupfalse%
\ auto\isanewline
\isacommand{qed}\isamarkupfalse%
%
\endisatagproof
{\isafoldproof}%
%
\isadelimproof
\isanewline
%
\endisadelimproof
\isanewline
\isacommand{lemma}\isamarkupfalse%
\ invinto{\isacharcolon}{\kern0pt}\ {\isachardoublequoteopen}bij{\isacharunderscore}{\kern0pt}betw\ f\ A\ B\ {\isasymLongrightarrow}\ {\isacharparenleft}{\kern0pt}{\isasymforall}x\ {\isasymin}\ B{\isachardot}{\kern0pt}\ {\isasymexists}{\isacharbang}{\kern0pt}y\ {\isasymin}\ A{\isachardot}{\kern0pt}\ {\isacharparenleft}{\kern0pt}the{\isacharunderscore}{\kern0pt}inv{\isacharunderscore}{\kern0pt}into\ A\ f{\isacharparenright}{\kern0pt}\ x\ {\isacharequal}{\kern0pt}\ y{\isacharparenright}{\kern0pt}{\isachardoublequoteclose}\ \isanewline
%
\isadelimproof
\ \ %
\endisadelimproof
%
\isatagproof
\isacommand{unfolding}\isamarkupfalse%
\ bij{\isacharunderscore}{\kern0pt}betw{\isacharunderscore}{\kern0pt}def\ inj{\isacharunderscore}{\kern0pt}on{\isacharunderscore}{\kern0pt}def\ the{\isacharunderscore}{\kern0pt}inv{\isacharunderscore}{\kern0pt}into{\isacharunderscore}{\kern0pt}def\ \isacommand{by}\isamarkupfalse%
\ blast%
\endisatagproof
{\isafoldproof}%
%
\isadelimproof
\isanewline
%
\endisadelimproof
\isanewline
\isacommand{lemma}\isamarkupfalse%
\ invintoprops{\isacharcolon}{\kern0pt}\isanewline
\ \ \isakeyword{assumes}\ {\isachardoublequoteopen}s\ {\isacharless}{\kern0pt}\ t{\isachardoublequoteclose}\isanewline
\ \ \isakeyword{shows}\ {\isachardoublequoteopen}the{\isacharunderscore}{\kern0pt}inv{\isacharunderscore}{\kern0pt}into\ {\isacharparenleft}{\kern0pt}cube\ {\isadigit{1}}\ t{\isacharparenright}{\kern0pt}\ {\isacharparenleft}{\kern0pt}{\isasymlambda}f{\isachardot}{\kern0pt}\ f\ {\isadigit{0}}{\isacharparenright}{\kern0pt}\ s\ {\isasymin}\ cube\ {\isadigit{1}}\ t{\isachardoublequoteclose}\ \isanewline
\ \ \ \ \isakeyword{and}\ {\isachardoublequoteopen}the{\isacharunderscore}{\kern0pt}inv{\isacharunderscore}{\kern0pt}into\ {\isacharparenleft}{\kern0pt}cube\ {\isadigit{1}}\ t{\isacharparenright}{\kern0pt}\ {\isacharparenleft}{\kern0pt}{\isasymlambda}f{\isachardot}{\kern0pt}\ f\ {\isadigit{0}}{\isacharparenright}{\kern0pt}\ s\ {\isadigit{0}}\ {\isacharequal}{\kern0pt}\ s{\isachardoublequoteclose}\isanewline
%
\isadelimproof
\ \ %
\endisadelimproof
%
\isatagproof
\isacommand{using}\isamarkupfalse%
\ assms\ invinto\ one{\isacharunderscore}{\kern0pt}dim{\isacharunderscore}{\kern0pt}cube{\isacharunderscore}{\kern0pt}eq{\isacharunderscore}{\kern0pt}nat{\isacharunderscore}{\kern0pt}set\ \isacommand{apply}\isamarkupfalse%
\ auto\isanewline
\ \ \isacommand{using}\isamarkupfalse%
\ f{\isacharunderscore}{\kern0pt}the{\isacharunderscore}{\kern0pt}inv{\isacharunderscore}{\kern0pt}into{\isacharunderscore}{\kern0pt}f{\isacharunderscore}{\kern0pt}bij{\isacharunderscore}{\kern0pt}betw\ \isacommand{by}\isamarkupfalse%
\ fastforce%
\endisatagproof
{\isafoldproof}%
%
\isadelimproof
\isanewline
%
\endisadelimproof
\isanewline
\isanewline
\isanewline
\isanewline
\isacommand{lemma}\isamarkupfalse%
\ some{\isacharunderscore}{\kern0pt}inv{\isacharunderscore}{\kern0pt}into{\isacharcolon}{\kern0pt}\ \isakeyword{assumes}\ {\isachardoublequoteopen}s\ {\isacharless}{\kern0pt}\ t{\isachardoublequoteclose}\ \isakeyword{shows}\ {\isachardoublequoteopen}{\isacharparenleft}{\kern0pt}SOME\ p{\isachardot}{\kern0pt}\ p{\isasymin}cube\ {\isadigit{1}}\ t\ {\isasymand}\ p\ {\isadigit{0}}\ {\isacharequal}{\kern0pt}\ s{\isacharparenright}{\kern0pt}\ {\isacharequal}{\kern0pt}\ {\isacharparenleft}{\kern0pt}the{\isacharunderscore}{\kern0pt}inv{\isacharunderscore}{\kern0pt}into\ {\isacharparenleft}{\kern0pt}cube\ {\isadigit{1}}\ t{\isacharparenright}{\kern0pt}\ {\isacharparenleft}{\kern0pt}{\isasymlambda}f{\isachardot}{\kern0pt}\ f\ {\isadigit{0}}{\isacharparenright}{\kern0pt}\ s{\isacharparenright}{\kern0pt}{\isachardoublequoteclose}\isanewline
%
\isadelimproof
\ \ %
\endisadelimproof
%
\isatagproof
\isacommand{using}\isamarkupfalse%
\ invintoprops{\isacharbrackleft}{\kern0pt}of\ s\ t{\isacharbrackright}{\kern0pt}\ one{\isacharunderscore}{\kern0pt}dim{\isacharunderscore}{\kern0pt}cube{\isacharunderscore}{\kern0pt}eq{\isacharunderscore}{\kern0pt}nat{\isacharunderscore}{\kern0pt}set{\isacharbrackleft}{\kern0pt}of\ t{\isacharbrackright}{\kern0pt}\ assms\ \isacommand{unfolding}\isamarkupfalse%
\ bij{\isacharunderscore}{\kern0pt}betw{\isacharunderscore}{\kern0pt}def\ inj{\isacharunderscore}{\kern0pt}on{\isacharunderscore}{\kern0pt}def\ \isacommand{by}\isamarkupfalse%
\ auto%
\endisatagproof
{\isafoldproof}%
%
\isadelimproof
\isanewline
%
\endisadelimproof
\isanewline
\isacommand{lemma}\isamarkupfalse%
\ some{\isacharunderscore}{\kern0pt}inv{\isacharunderscore}{\kern0pt}into{\isacharunderscore}{\kern0pt}{\isadigit{2}}{\isacharcolon}{\kern0pt}\ \isakeyword{assumes}\ {\isachardoublequoteopen}s\ {\isacharless}{\kern0pt}\ t{\isachardoublequoteclose}\ \isakeyword{shows}\ {\isachardoublequoteopen}{\isacharparenleft}{\kern0pt}SOME\ p{\isachardot}{\kern0pt}\ p{\isasymin}cube\ {\isadigit{1}}\ {\isacharparenleft}{\kern0pt}t{\isacharplus}{\kern0pt}{\isadigit{1}}{\isacharparenright}{\kern0pt}\ {\isasymand}\ p\ {\isadigit{0}}\ {\isacharequal}{\kern0pt}\ s{\isacharparenright}{\kern0pt}\ {\isacharequal}{\kern0pt}\ {\isacharparenleft}{\kern0pt}the{\isacharunderscore}{\kern0pt}inv{\isacharunderscore}{\kern0pt}into\ {\isacharparenleft}{\kern0pt}cube\ {\isadigit{1}}\ t{\isacharparenright}{\kern0pt}\ {\isacharparenleft}{\kern0pt}{\isasymlambda}f{\isachardot}{\kern0pt}\ f\ {\isadigit{0}}{\isacharparenright}{\kern0pt}\ s{\isacharparenright}{\kern0pt}{\isachardoublequoteclose}\isanewline
%
\isadelimproof
%
\endisadelimproof
%
\isatagproof
\isacommand{proof}\isamarkupfalse%
{\isacharminus}{\kern0pt}\isanewline
\ \ \isacommand{have}\isamarkupfalse%
\ {\isacharasterisk}{\kern0pt}{\isacharcolon}{\kern0pt}\ {\isachardoublequoteopen}{\isacharparenleft}{\kern0pt}SOME\ p{\isachardot}{\kern0pt}\ p{\isasymin}cube\ {\isadigit{1}}\ {\isacharparenleft}{\kern0pt}t{\isacharplus}{\kern0pt}{\isadigit{1}}{\isacharparenright}{\kern0pt}\ {\isasymand}\ p\ {\isadigit{0}}\ {\isacharequal}{\kern0pt}\ s{\isacharparenright}{\kern0pt}\ {\isasymin}\ cube\ {\isadigit{1}}\ {\isacharparenleft}{\kern0pt}t{\isacharplus}{\kern0pt}{\isadigit{1}}{\isacharparenright}{\kern0pt}{\isachardoublequoteclose}\ \isacommand{using}\isamarkupfalse%
\ cube{\isacharunderscore}{\kern0pt}props\ assms\ \isacommand{by}\isamarkupfalse%
\ simp\isanewline
\ \ \isacommand{then}\isamarkupfalse%
\ \isacommand{have}\isamarkupfalse%
\ {\isachardoublequoteopen}{\isacharparenleft}{\kern0pt}SOME\ p{\isachardot}{\kern0pt}\ p{\isasymin}cube\ {\isadigit{1}}\ {\isacharparenleft}{\kern0pt}t{\isacharplus}{\kern0pt}{\isadigit{1}}{\isacharparenright}{\kern0pt}\ {\isasymand}\ p\ {\isadigit{0}}\ {\isacharequal}{\kern0pt}\ s{\isacharparenright}{\kern0pt}\ {\isadigit{0}}\ {\isacharequal}{\kern0pt}\ s{\isachardoublequoteclose}\ \isacommand{using}\isamarkupfalse%
\ cube{\isacharunderscore}{\kern0pt}props\ assms\ \isacommand{by}\isamarkupfalse%
\ simp\isanewline
\ \ \isacommand{moreover}\isamarkupfalse%
\isanewline
\ \ \isacommand{{\isacharbraceleft}{\kern0pt}}\isamarkupfalse%
\isanewline
\ \ \ \ \isacommand{have}\isamarkupfalse%
\ {\isachardoublequoteopen}{\isacharparenleft}{\kern0pt}SOME\ p{\isachardot}{\kern0pt}\ p{\isasymin}cube\ {\isadigit{1}}\ {\isacharparenleft}{\kern0pt}t{\isacharplus}{\kern0pt}{\isadigit{1}}{\isacharparenright}{\kern0pt}\ {\isasymand}\ p\ {\isadigit{0}}\ {\isacharequal}{\kern0pt}\ s{\isacharparenright}{\kern0pt}\ {\isacharbackquote}{\kern0pt}\ {\isacharbraceleft}{\kern0pt}{\isachardot}{\kern0pt}{\isachardot}{\kern0pt}{\isacharless}{\kern0pt}{\isadigit{1}}{\isacharbraceright}{\kern0pt}\ {\isasymsubseteq}\ {\isacharbraceleft}{\kern0pt}{\isachardot}{\kern0pt}{\isachardot}{\kern0pt}{\isacharless}{\kern0pt}t{\isacharbraceright}{\kern0pt}{\isachardoublequoteclose}\ \isacommand{using}\isamarkupfalse%
\ calculation\ assms\ \isacommand{by}\isamarkupfalse%
\ force\isanewline
\ \ \ \ \isacommand{then}\isamarkupfalse%
\ \isacommand{have}\isamarkupfalse%
\ {\isachardoublequoteopen}{\isacharparenleft}{\kern0pt}SOME\ p{\isachardot}{\kern0pt}\ p{\isasymin}cube\ {\isadigit{1}}\ {\isacharparenleft}{\kern0pt}t{\isacharplus}{\kern0pt}{\isadigit{1}}{\isacharparenright}{\kern0pt}\ {\isasymand}\ p\ {\isadigit{0}}\ {\isacharequal}{\kern0pt}\ s{\isacharparenright}{\kern0pt}\ {\isasymin}\ cube\ {\isadigit{1}}\ t{\isachardoublequoteclose}\ \isacommand{using}\isamarkupfalse%
\ {\isacharasterisk}{\kern0pt}\ \isacommand{unfolding}\isamarkupfalse%
\ cube{\isacharunderscore}{\kern0pt}def\ \isacommand{by}\isamarkupfalse%
\ auto\ \ \isanewline
\ \ \isacommand{{\isacharbraceright}{\kern0pt}}\isamarkupfalse%
\isanewline
\ \ \isacommand{moreover}\isamarkupfalse%
\ \isacommand{have}\isamarkupfalse%
\ {\isachardoublequoteopen}inj{\isacharunderscore}{\kern0pt}on\ {\isacharparenleft}{\kern0pt}{\isasymlambda}f{\isachardot}{\kern0pt}\ f\ {\isadigit{0}}{\isacharparenright}{\kern0pt}\ {\isacharparenleft}{\kern0pt}cube\ {\isadigit{1}}\ t{\isacharparenright}{\kern0pt}{\isachardoublequoteclose}\ \isacommand{using}\isamarkupfalse%
\ one{\isacharunderscore}{\kern0pt}dim{\isacharunderscore}{\kern0pt}cube{\isacharunderscore}{\kern0pt}eq{\isacharunderscore}{\kern0pt}nat{\isacharunderscore}{\kern0pt}set{\isacharbrackleft}{\kern0pt}of\ t{\isacharbrackright}{\kern0pt}\ \isacommand{unfolding}\isamarkupfalse%
\ bij{\isacharunderscore}{\kern0pt}betw{\isacharunderscore}{\kern0pt}def\ inj{\isacharunderscore}{\kern0pt}on{\isacharunderscore}{\kern0pt}def\ \isacommand{by}\isamarkupfalse%
\ auto\ \isanewline
\ \ \isacommand{ultimately}\isamarkupfalse%
\ \isacommand{show}\isamarkupfalse%
\ {\isachardoublequoteopen}{\isacharparenleft}{\kern0pt}SOME\ p{\isachardot}{\kern0pt}\ p{\isasymin}cube\ {\isadigit{1}}\ {\isacharparenleft}{\kern0pt}t{\isacharplus}{\kern0pt}{\isadigit{1}}{\isacharparenright}{\kern0pt}\ {\isasymand}\ p\ {\isadigit{0}}\ {\isacharequal}{\kern0pt}\ s{\isacharparenright}{\kern0pt}\ {\isacharequal}{\kern0pt}\ {\isacharparenleft}{\kern0pt}the{\isacharunderscore}{\kern0pt}inv{\isacharunderscore}{\kern0pt}into\ {\isacharparenleft}{\kern0pt}cube\ {\isadigit{1}}\ t{\isacharparenright}{\kern0pt}\ {\isacharparenleft}{\kern0pt}{\isasymlambda}f{\isachardot}{\kern0pt}\ f\ {\isadigit{0}}{\isacharparenright}{\kern0pt}\ s{\isacharparenright}{\kern0pt}{\isachardoublequoteclose}\ \isacommand{using}\isamarkupfalse%
\ the{\isacharunderscore}{\kern0pt}inv{\isacharunderscore}{\kern0pt}into{\isacharunderscore}{\kern0pt}f{\isacharunderscore}{\kern0pt}eq\ {\isacharbrackleft}{\kern0pt}of\ {\isachardoublequoteopen}{\isasymlambda}f{\isachardot}{\kern0pt}\ f\ {\isadigit{0}}{\isachardoublequoteclose}\ {\isachardoublequoteopen}cube\ {\isadigit{1}}\ t{\isachardoublequoteclose}\ {\isachardoublequoteopen}{\isacharparenleft}{\kern0pt}SOME\ p{\isachardot}{\kern0pt}\ p{\isasymin}cube\ {\isadigit{1}}\ {\isacharparenleft}{\kern0pt}t{\isacharplus}{\kern0pt}{\isadigit{1}}{\isacharparenright}{\kern0pt}\ {\isasymand}\ p\ {\isadigit{0}}\ {\isacharequal}{\kern0pt}\ s{\isacharparenright}{\kern0pt}{\isachardoublequoteclose}\ s{\isacharbrackright}{\kern0pt}\ \isacommand{by}\isamarkupfalse%
\ auto\isanewline
\isacommand{qed}\isamarkupfalse%
%
\endisatagproof
{\isafoldproof}%
%
\isadelimproof
\isanewline
%
\endisadelimproof
\isanewline
\isacommand{lemma}\isamarkupfalse%
\ dim{\isadigit{1}}{\isacharunderscore}{\kern0pt}layered{\isacharunderscore}{\kern0pt}subspace{\isacharunderscore}{\kern0pt}as{\isacharunderscore}{\kern0pt}line{\isacharcolon}{\kern0pt}\isanewline
\ \ \isakeyword{assumes}\ {\isachardoublequoteopen}t\ {\isachargreater}{\kern0pt}\ {\isadigit{0}}{\isachardoublequoteclose}\isanewline
\ \ \ \ \isakeyword{and}\ {\isachardoublequoteopen}layered{\isacharunderscore}{\kern0pt}subspace\ S\ {\isadigit{1}}\ n\ t\ r\ {\isasymchi}{\isachardoublequoteclose}\isanewline
\ \ \isakeyword{shows}\ {\isachardoublequoteopen}{\isasymexists}c{\isadigit{1}}\ c{\isadigit{2}}{\isachardot}{\kern0pt}\ c{\isadigit{1}}{\isacharless}{\kern0pt}r\ {\isasymand}\ c{\isadigit{2}}{\isacharless}{\kern0pt}r\ {\isasymand}\ {\isacharparenleft}{\kern0pt}{\isasymforall}s{\isacharless}{\kern0pt}t{\isachardot}{\kern0pt}\ {\isasymchi}\ {\isacharparenleft}{\kern0pt}S\ {\isacharparenleft}{\kern0pt}SOME\ p{\isachardot}{\kern0pt}\ p{\isasymin}cube\ {\isadigit{1}}\ {\isacharparenleft}{\kern0pt}t{\isacharplus}{\kern0pt}{\isadigit{1}}{\isacharparenright}{\kern0pt}\ {\isasymand}\ p\ {\isadigit{0}}\ {\isacharequal}{\kern0pt}\ s{\isacharparenright}{\kern0pt}{\isacharparenright}{\kern0pt}\ {\isacharequal}{\kern0pt}\ c{\isadigit{1}}{\isacharparenright}{\kern0pt}\ {\isasymand}\ {\isasymchi}\ {\isacharparenleft}{\kern0pt}S\ {\isacharparenleft}{\kern0pt}SOME\ p{\isachardot}{\kern0pt}\ p{\isasymin}cube\ {\isadigit{1}}\ {\isacharparenleft}{\kern0pt}t{\isacharplus}{\kern0pt}{\isadigit{1}}{\isacharparenright}{\kern0pt}\ {\isasymand}\ p\ {\isadigit{0}}\ {\isacharequal}{\kern0pt}\ t{\isacharparenright}{\kern0pt}{\isacharparenright}{\kern0pt}\ {\isacharequal}{\kern0pt}\ c{\isadigit{2}}{\isachardoublequoteclose}\isanewline
%
\isadelimproof
%
\endisadelimproof
%
\isatagproof
\isacommand{proof}\isamarkupfalse%
\ {\isacharminus}{\kern0pt}\isanewline
\ \ \isacommand{have}\isamarkupfalse%
\ {\isachardoublequoteopen}x\ u\ {\isacharless}{\kern0pt}\ t{\isachardoublequoteclose}\ \isakeyword{if}\ {\isachardoublequoteopen}x\ {\isasymin}\ classes\ {\isadigit{1}}\ t\ {\isadigit{0}}{\isachardoublequoteclose}\ \isakeyword{and}\ {\isachardoublequoteopen}u\ {\isacharless}{\kern0pt}\ {\isadigit{1}}{\isachardoublequoteclose}\ \isakeyword{for}\ x\ u\ \isanewline
\ \ \isacommand{proof}\isamarkupfalse%
\ {\isacharminus}{\kern0pt}\isanewline
\ \ \ \ \isacommand{have}\isamarkupfalse%
\ {\isachardoublequoteopen}x\ {\isasymin}\ cube\ {\isadigit{1}}\ {\isacharparenleft}{\kern0pt}t{\isacharplus}{\kern0pt}{\isadigit{1}}{\isacharparenright}{\kern0pt}{\isachardoublequoteclose}\ \isacommand{using}\isamarkupfalse%
\ that\ \isacommand{unfolding}\isamarkupfalse%
\ classes{\isacharunderscore}{\kern0pt}def\ \isacommand{by}\isamarkupfalse%
\ blast\isanewline
\ \ \ \ \isacommand{then}\isamarkupfalse%
\ \isacommand{have}\isamarkupfalse%
\ {\isachardoublequoteopen}x\ u\ {\isasymin}\ {\isacharbraceleft}{\kern0pt}{\isachardot}{\kern0pt}{\isachardot}{\kern0pt}{\isacharless}{\kern0pt}t{\isacharplus}{\kern0pt}{\isadigit{1}}{\isacharbraceright}{\kern0pt}{\isachardoublequoteclose}\ \isacommand{using}\isamarkupfalse%
\ that\ \isacommand{unfolding}\isamarkupfalse%
\ cube{\isacharunderscore}{\kern0pt}def\ \isacommand{by}\isamarkupfalse%
\ blast\isanewline
\ \ \ \ \isacommand{then}\isamarkupfalse%
\ \isacommand{have}\isamarkupfalse%
\ {\isachardoublequoteopen}x\ u\ {\isasymin}\ {\isacharbraceleft}{\kern0pt}{\isachardot}{\kern0pt}{\isachardot}{\kern0pt}{\isacharless}{\kern0pt}t{\isacharbraceright}{\kern0pt}{\isachardoublequoteclose}\ \isacommand{using}\isamarkupfalse%
\ that\isanewline
\ \ \ \ \ \ \isacommand{using}\isamarkupfalse%
\ that\ less{\isacharunderscore}{\kern0pt}Suc{\isacharunderscore}{\kern0pt}eq\ \isacommand{unfolding}\isamarkupfalse%
\ classes{\isacharunderscore}{\kern0pt}def\ \isacommand{by}\isamarkupfalse%
\ auto\isanewline
\ \ \ \ \isacommand{then}\isamarkupfalse%
\ \isacommand{show}\isamarkupfalse%
\ {\isachardoublequoteopen}x\ u\ {\isacharless}{\kern0pt}\ t{\isachardoublequoteclose}\ \isacommand{by}\isamarkupfalse%
\ simp\isanewline
\ \ \isacommand{qed}\isamarkupfalse%
\isanewline
\ \ \isacommand{then}\isamarkupfalse%
\ \isacommand{have}\isamarkupfalse%
\ {\isachardoublequoteopen}classes\ {\isadigit{1}}\ t\ {\isadigit{0}}\ {\isasymsubseteq}\ cube\ {\isadigit{1}}\ t{\isachardoublequoteclose}\ \isacommand{unfolding}\isamarkupfalse%
\ cube{\isacharunderscore}{\kern0pt}def\ classes{\isacharunderscore}{\kern0pt}def\ \isacommand{by}\isamarkupfalse%
\ auto\isanewline
\ \ \isacommand{moreover}\isamarkupfalse%
\ \isacommand{have}\isamarkupfalse%
\ {\isachardoublequoteopen}cube\ {\isadigit{1}}\ t\ {\isasymsubseteq}\ classes\ {\isadigit{1}}\ t\ {\isadigit{0}}{\isachardoublequoteclose}\ \isacommand{using}\isamarkupfalse%
\ cube{\isacharunderscore}{\kern0pt}subset{\isacharbrackleft}{\kern0pt}of\ {\isadigit{1}}\ t{\isacharbrackright}{\kern0pt}\ \isacommand{unfolding}\isamarkupfalse%
\ cube{\isacharunderscore}{\kern0pt}def\ classes{\isacharunderscore}{\kern0pt}def\ \isacommand{by}\isamarkupfalse%
\ auto\isanewline
\ \ \isacommand{ultimately}\isamarkupfalse%
\ \isacommand{have}\isamarkupfalse%
\ X{\isacharcolon}{\kern0pt}\ {\isachardoublequoteopen}classes\ {\isadigit{1}}\ t\ {\isadigit{0}}\ {\isacharequal}{\kern0pt}\ cube\ {\isadigit{1}}\ t{\isachardoublequoteclose}\ \isacommand{by}\isamarkupfalse%
\ blast\isanewline
\isanewline
\ \ \isacommand{obtain}\isamarkupfalse%
\ c{\isadigit{1}}\ \isakeyword{where}\ c{\isadigit{1}}{\isacharunderscore}{\kern0pt}prop{\isacharcolon}{\kern0pt}\ {\isachardoublequoteopen}c{\isadigit{1}}\ {\isacharless}{\kern0pt}\ r\ {\isasymand}\ {\isacharparenleft}{\kern0pt}{\isasymforall}x{\isasymin}classes\ {\isadigit{1}}\ t\ {\isadigit{0}}{\isachardot}{\kern0pt}\ {\isasymchi}\ {\isacharparenleft}{\kern0pt}S\ x{\isacharparenright}{\kern0pt}\ {\isacharequal}{\kern0pt}\ c{\isadigit{1}}{\isacharparenright}{\kern0pt}{\isachardoublequoteclose}\ \isacommand{using}\isamarkupfalse%
\ assms{\isacharparenleft}{\kern0pt}{\isadigit{2}}{\isacharparenright}{\kern0pt}\ \isacommand{unfolding}\isamarkupfalse%
\ layered{\isacharunderscore}{\kern0pt}subspace{\isacharunderscore}{\kern0pt}def\ \isacommand{by}\isamarkupfalse%
\ blast\isanewline
\ \ \isacommand{then}\isamarkupfalse%
\ \isacommand{have}\isamarkupfalse%
\ {\isachardoublequoteopen}{\isacharparenleft}{\kern0pt}{\isasymchi}\ {\isacharparenleft}{\kern0pt}S\ x{\isacharparenright}{\kern0pt}\ {\isacharequal}{\kern0pt}\ c{\isadigit{1}}{\isacharparenright}{\kern0pt}{\isachardoublequoteclose}\ \isakeyword{if}\ {\isachardoublequoteopen}x\ {\isasymin}\ cube\ {\isadigit{1}}\ t{\isachardoublequoteclose}\ \isakeyword{for}\ x\ \isacommand{using}\isamarkupfalse%
\ X\ that\ \isacommand{by}\isamarkupfalse%
\ blast\isanewline
\ \ \isacommand{then}\isamarkupfalse%
\ \isacommand{have}\isamarkupfalse%
\ {\isachardoublequoteopen}{\isasymchi}\ {\isacharparenleft}{\kern0pt}S\ {\isacharparenleft}{\kern0pt}the{\isacharunderscore}{\kern0pt}inv{\isacharunderscore}{\kern0pt}into\ {\isacharparenleft}{\kern0pt}cube\ {\isadigit{1}}\ t{\isacharparenright}{\kern0pt}\ {\isacharparenleft}{\kern0pt}{\isasymlambda}f{\isachardot}{\kern0pt}\ f\ {\isadigit{0}}{\isacharparenright}{\kern0pt}\ s{\isacharparenright}{\kern0pt}{\isacharparenright}{\kern0pt}\ {\isacharequal}{\kern0pt}\ c{\isadigit{1}}{\isachardoublequoteclose}\ \isakeyword{if}\ {\isachardoublequoteopen}s\ {\isacharless}{\kern0pt}\ t{\isachardoublequoteclose}\ \isakeyword{for}\ s\ \isacommand{using}\isamarkupfalse%
\ one{\isacharunderscore}{\kern0pt}dim{\isacharunderscore}{\kern0pt}cube{\isacharunderscore}{\kern0pt}eq{\isacharunderscore}{\kern0pt}nat{\isacharunderscore}{\kern0pt}set{\isacharbrackleft}{\kern0pt}of\ t{\isacharbrackright}{\kern0pt}\ \isanewline
\ \ \ \ \isacommand{by}\isamarkupfalse%
\ {\isacharparenleft}{\kern0pt}meson\ that\ bij{\isacharunderscore}{\kern0pt}betwE\ bij{\isacharunderscore}{\kern0pt}betw{\isacharunderscore}{\kern0pt}the{\isacharunderscore}{\kern0pt}inv{\isacharunderscore}{\kern0pt}into\ lessThan{\isacharunderscore}{\kern0pt}iff{\isacharparenright}{\kern0pt}\isanewline
\ \ \isacommand{then}\isamarkupfalse%
\ \isacommand{have}\isamarkupfalse%
\ K{\isadigit{1}}{\isacharcolon}{\kern0pt}\ {\isachardoublequoteopen}{\isasymchi}\ {\isacharparenleft}{\kern0pt}S\ {\isacharparenleft}{\kern0pt}SOME\ p{\isachardot}{\kern0pt}\ p{\isasymin}cube\ {\isadigit{1}}\ {\isacharparenleft}{\kern0pt}t{\isacharplus}{\kern0pt}{\isadigit{1}}{\isacharparenright}{\kern0pt}\ {\isasymand}\ p\ {\isadigit{0}}\ {\isacharequal}{\kern0pt}\ s{\isacharparenright}{\kern0pt}{\isacharparenright}{\kern0pt}\ {\isacharequal}{\kern0pt}\ c{\isadigit{1}}{\isachardoublequoteclose}\ \isakeyword{if}\ {\isachardoublequoteopen}s\ {\isacharless}{\kern0pt}\ t{\isachardoublequoteclose}\ \isakeyword{for}\ s\ \isacommand{using}\isamarkupfalse%
\ that\ some{\isacharunderscore}{\kern0pt}inv{\isacharunderscore}{\kern0pt}into{\isacharunderscore}{\kern0pt}{\isadigit{2}}\ \isacommand{by}\isamarkupfalse%
\ simp\isanewline
\isanewline
\ \ \isacommand{have}\isamarkupfalse%
\ {\isacharasterisk}{\kern0pt}{\isacharcolon}{\kern0pt}\ {\isachardoublequoteopen}{\isasymexists}c{\isacharless}{\kern0pt}r{\isachardot}{\kern0pt}\ {\isasymforall}x\ {\isasymin}\ classes\ {\isadigit{1}}\ t\ {\isadigit{1}}{\isachardot}{\kern0pt}\ {\isasymchi}\ {\isacharparenleft}{\kern0pt}S\ x{\isacharparenright}{\kern0pt}\ {\isacharequal}{\kern0pt}\ c{\isachardoublequoteclose}\ \isacommand{using}\isamarkupfalse%
\ assms{\isacharparenleft}{\kern0pt}{\isadigit{2}}{\isacharparenright}{\kern0pt}\ \isacommand{unfolding}\isamarkupfalse%
\ layered{\isacharunderscore}{\kern0pt}subspace{\isacharunderscore}{\kern0pt}def\ \isacommand{by}\isamarkupfalse%
\ blast\isanewline
\isanewline
\ \ \isacommand{have}\isamarkupfalse%
\ {\isachardoublequoteopen}x\ {\isadigit{0}}\ {\isacharequal}{\kern0pt}\ t{\isachardoublequoteclose}\ \isakeyword{if}\ {\isachardoublequoteopen}x\ {\isasymin}\ classes\ {\isadigit{1}}\ t\ {\isadigit{1}}{\isachardoublequoteclose}\ \isakeyword{for}\ x\ \isacommand{using}\isamarkupfalse%
\ that\ \isacommand{unfolding}\isamarkupfalse%
\ classes{\isacharunderscore}{\kern0pt}def\ \isacommand{by}\isamarkupfalse%
\ simp\isanewline
\ \ \isacommand{moreover}\isamarkupfalse%
\ \isacommand{have}\isamarkupfalse%
\ {\isachardoublequoteopen}{\isasymexists}{\isacharbang}{\kern0pt}x\ {\isasymin}\ cube\ {\isadigit{1}}\ {\isacharparenleft}{\kern0pt}t{\isacharplus}{\kern0pt}{\isadigit{1}}{\isacharparenright}{\kern0pt}{\isachardot}{\kern0pt}\ x\ {\isadigit{0}}\ {\isacharequal}{\kern0pt}\ t{\isachardoublequoteclose}\ \isacommand{using}\isamarkupfalse%
\ one{\isacharunderscore}{\kern0pt}dim{\isacharunderscore}{\kern0pt}cube{\isacharunderscore}{\kern0pt}eq{\isacharunderscore}{\kern0pt}nat{\isacharunderscore}{\kern0pt}set{\isacharbrackleft}{\kern0pt}of\ {\isachardoublequoteopen}t{\isacharplus}{\kern0pt}{\isadigit{1}}{\isachardoublequoteclose}{\isacharbrackright}{\kern0pt}\ \isacommand{unfolding}\isamarkupfalse%
\ bij{\isacharunderscore}{\kern0pt}betw{\isacharunderscore}{\kern0pt}def\ inj{\isacharunderscore}{\kern0pt}on{\isacharunderscore}{\kern0pt}def\ \isanewline
\ \ \ \ \isacommand{using}\isamarkupfalse%
\ invintoprops{\isacharparenleft}{\kern0pt}{\isadigit{1}}{\isacharparenright}{\kern0pt}\ invintoprops{\isacharparenleft}{\kern0pt}{\isadigit{2}}{\isacharparenright}{\kern0pt}\ \isacommand{by}\isamarkupfalse%
\ force\ \isanewline
\ \ \isacommand{moreover}\isamarkupfalse%
\ \isacommand{have}\isamarkupfalse%
\ {\isacharasterisk}{\kern0pt}{\isacharasterisk}{\kern0pt}{\isacharcolon}{\kern0pt}\ {\isachardoublequoteopen}{\isasymexists}{\isacharbang}{\kern0pt}x{\isachardot}{\kern0pt}\ x\ \ {\isasymin}\ classes\ {\isadigit{1}}\ t\ {\isadigit{1}}{\isachardoublequoteclose}\ \isacommand{unfolding}\isamarkupfalse%
\ classes{\isacharunderscore}{\kern0pt}def\ \isacommand{using}\isamarkupfalse%
\ calculation{\isacharparenleft}{\kern0pt}{\isadigit{2}}{\isacharparenright}{\kern0pt}\ \isacommand{by}\isamarkupfalse%
\ simp\isanewline
\ \ \isacommand{ultimately}\isamarkupfalse%
\ \isacommand{have}\isamarkupfalse%
\ {\isachardoublequoteopen}the{\isacharunderscore}{\kern0pt}inv{\isacharunderscore}{\kern0pt}into\ {\isacharparenleft}{\kern0pt}cube\ {\isadigit{1}}\ {\isacharparenleft}{\kern0pt}t{\isacharplus}{\kern0pt}{\isadigit{1}}{\isacharparenright}{\kern0pt}{\isacharparenright}{\kern0pt}\ {\isacharparenleft}{\kern0pt}{\isasymlambda}f{\isachardot}{\kern0pt}\ f\ {\isadigit{0}}{\isacharparenright}{\kern0pt}\ t\ {\isasymin}\ classes\ {\isadigit{1}}\ t\ {\isadigit{1}}{\isachardoublequoteclose}\ \isacommand{using}\isamarkupfalse%
\ invintoprops{\isacharbrackleft}{\kern0pt}of\ t\ {\isachardoublequoteopen}t{\isacharplus}{\kern0pt}{\isadigit{1}}{\isachardoublequoteclose}{\isacharbrackright}{\kern0pt}\ \isacommand{unfolding}\isamarkupfalse%
\ classes{\isacharunderscore}{\kern0pt}def\ \isacommand{by}\isamarkupfalse%
\ simp\isanewline
\isanewline
\ \ \isacommand{then}\isamarkupfalse%
\ \isacommand{have}\isamarkupfalse%
\ {\isachardoublequoteopen}{\isasymexists}c{\isadigit{2}}{\isachardot}{\kern0pt}\ c{\isadigit{2}}\ {\isacharless}{\kern0pt}\ r\ {\isasymand}\ {\isasymchi}\ {\isacharparenleft}{\kern0pt}S\ {\isacharparenleft}{\kern0pt}the{\isacharunderscore}{\kern0pt}inv{\isacharunderscore}{\kern0pt}into\ {\isacharparenleft}{\kern0pt}cube\ {\isadigit{1}}\ {\isacharparenleft}{\kern0pt}t{\isacharplus}{\kern0pt}{\isadigit{1}}{\isacharparenright}{\kern0pt}{\isacharparenright}{\kern0pt}\ {\isacharparenleft}{\kern0pt}{\isasymlambda}f{\isachardot}{\kern0pt}\ f\ {\isadigit{0}}{\isacharparenright}{\kern0pt}\ t{\isacharparenright}{\kern0pt}{\isacharparenright}{\kern0pt}\ {\isacharequal}{\kern0pt}\ c{\isadigit{2}}{\isachardoublequoteclose}\ \isacommand{using}\isamarkupfalse%
\ {\isacharasterisk}{\kern0pt}\ {\isacharasterisk}{\kern0pt}{\isacharasterisk}{\kern0pt}\ \isacommand{by}\isamarkupfalse%
\ blast\isanewline
\ \ \isacommand{then}\isamarkupfalse%
\ \isacommand{have}\isamarkupfalse%
\ K{\isadigit{2}}{\isacharcolon}{\kern0pt}\ {\isachardoublequoteopen}{\isasymexists}c{\isadigit{2}}{\isachardot}{\kern0pt}\ c{\isadigit{2}}\ {\isacharless}{\kern0pt}\ r\ {\isasymand}\ {\isasymchi}\ {\isacharparenleft}{\kern0pt}S\ {\isacharparenleft}{\kern0pt}SOME\ p{\isachardot}{\kern0pt}\ p{\isasymin}cube\ {\isadigit{1}}\ {\isacharparenleft}{\kern0pt}t{\isacharplus}{\kern0pt}{\isadigit{1}}{\isacharparenright}{\kern0pt}\ {\isasymand}\ p\ {\isadigit{0}}\ {\isacharequal}{\kern0pt}\ t{\isacharparenright}{\kern0pt}{\isacharparenright}{\kern0pt}\ {\isacharequal}{\kern0pt}\ c{\isadigit{2}}{\isachardoublequoteclose}\ \isacommand{using}\isamarkupfalse%
\ some{\isacharunderscore}{\kern0pt}inv{\isacharunderscore}{\kern0pt}into\ \isacommand{by}\isamarkupfalse%
\ simp\isanewline
\isanewline
\ \ \isacommand{from}\isamarkupfalse%
\ K{\isadigit{1}}\ K{\isadigit{2}}\ \isacommand{show}\isamarkupfalse%
\ {\isacharquery}{\kern0pt}thesis\ \isanewline
\ \ \ \ \isacommand{using}\isamarkupfalse%
\ c{\isadigit{1}}{\isacharunderscore}{\kern0pt}prop\ \isacommand{by}\isamarkupfalse%
\ blast\isanewline
\isacommand{qed}\isamarkupfalse%
%
\endisatagproof
{\isafoldproof}%
%
\isadelimproof
\isanewline
%
\endisadelimproof
\isanewline
\isacommand{lemma}\isamarkupfalse%
\ dim{\isadigit{1}}{\isacharunderscore}{\kern0pt}layered{\isacharunderscore}{\kern0pt}subspace{\isacharunderscore}{\kern0pt}mono{\isacharunderscore}{\kern0pt}line{\isacharcolon}{\kern0pt}\ \isakeyword{assumes}\ {\isachardoublequoteopen}t\ {\isachargreater}{\kern0pt}\ {\isadigit{0}}{\isachardoublequoteclose}\ \isakeyword{and}\ {\isachardoublequoteopen}layered{\isacharunderscore}{\kern0pt}subspace\ S\ {\isadigit{1}}\ n\ t\ r\ {\isasymchi}{\isachardoublequoteclose}\isanewline
\ \ \isakeyword{shows}\ {\isachardoublequoteopen}{\isasymforall}s{\isacharless}{\kern0pt}t{\isachardot}{\kern0pt}\ {\isasymforall}l{\isacharless}{\kern0pt}t{\isachardot}{\kern0pt}\ \ {\isasymchi}\ {\isacharparenleft}{\kern0pt}S\ {\isacharparenleft}{\kern0pt}SOME\ p{\isachardot}{\kern0pt}\ p{\isasymin}cube\ {\isadigit{1}}\ {\isacharparenleft}{\kern0pt}t{\isacharplus}{\kern0pt}{\isadigit{1}}{\isacharparenright}{\kern0pt}\ {\isasymand}\ p\ {\isadigit{0}}\ {\isacharequal}{\kern0pt}\ s{\isacharparenright}{\kern0pt}{\isacharparenright}{\kern0pt}\ {\isacharequal}{\kern0pt}\ \ {\isasymchi}\ {\isacharparenleft}{\kern0pt}S\ {\isacharparenleft}{\kern0pt}SOME\ p{\isachardot}{\kern0pt}\ p{\isasymin}cube\ {\isadigit{1}}\ {\isacharparenleft}{\kern0pt}t{\isacharplus}{\kern0pt}{\isadigit{1}}{\isacharparenright}{\kern0pt}\ {\isasymand}\ p\ {\isadigit{0}}\ {\isacharequal}{\kern0pt}\ l{\isacharparenright}{\kern0pt}{\isacharparenright}{\kern0pt}\ {\isasymand}\ \ {\isasymchi}\ {\isacharparenleft}{\kern0pt}S\ {\isacharparenleft}{\kern0pt}SOME\ p{\isachardot}{\kern0pt}\ p{\isasymin}cube\ {\isadigit{1}}\ {\isacharparenleft}{\kern0pt}t{\isacharplus}{\kern0pt}{\isadigit{1}}{\isacharparenright}{\kern0pt}\ {\isasymand}\ p\ {\isadigit{0}}\ {\isacharequal}{\kern0pt}\ s{\isacharparenright}{\kern0pt}{\isacharparenright}{\kern0pt}\ {\isacharless}{\kern0pt}\ r{\isachardoublequoteclose}\isanewline
%
\isadelimproof
\ \ %
\endisadelimproof
%
\isatagproof
\isacommand{using}\isamarkupfalse%
\ dim{\isadigit{1}}{\isacharunderscore}{\kern0pt}layered{\isacharunderscore}{\kern0pt}subspace{\isacharunderscore}{\kern0pt}as{\isacharunderscore}{\kern0pt}line{\isacharbrackleft}{\kern0pt}of\ t\ S\ n\ r\ {\isasymchi}{\isacharbrackright}{\kern0pt}\ assms\ \isacommand{by}\isamarkupfalse%
\ auto%
\endisatagproof
{\isafoldproof}%
%
\isadelimproof
\ \ \isanewline
%
\endisadelimproof
\isanewline
\isacommand{definition}\isamarkupfalse%
\ join\ {\isacharcolon}{\kern0pt}{\isacharcolon}{\kern0pt}\ {\isachardoublequoteopen}{\isacharparenleft}{\kern0pt}nat\ {\isasymRightarrow}\ {\isacharprime}{\kern0pt}a{\isacharparenright}{\kern0pt}\ {\isasymRightarrow}\ {\isacharparenleft}{\kern0pt}nat\ {\isasymRightarrow}\ {\isacharprime}{\kern0pt}a{\isacharparenright}{\kern0pt}\ {\isasymRightarrow}\ nat\ {\isasymRightarrow}\ nat\ {\isasymRightarrow}\ {\isacharparenleft}{\kern0pt}nat\ {\isasymRightarrow}\ {\isacharprime}{\kern0pt}a{\isacharparenright}{\kern0pt}{\isachardoublequoteclose}\isanewline
\ \ \isakeyword{where}\isanewline
\ \ \ \ {\isachardoublequoteopen}join\ f\ g\ n\ m\ {\isasymequiv}\ {\isacharparenleft}{\kern0pt}{\isasymlambda}x{\isachardot}{\kern0pt}\ if\ x\ {\isasymin}\ {\isacharbraceleft}{\kern0pt}{\isachardot}{\kern0pt}{\isachardot}{\kern0pt}{\isacharless}{\kern0pt}n{\isacharbraceright}{\kern0pt}\ then\ f\ x\ else\ {\isacharparenleft}{\kern0pt}if\ x\ {\isasymin}\ {\isacharbraceleft}{\kern0pt}n{\isachardot}{\kern0pt}{\isachardot}{\kern0pt}{\isacharless}{\kern0pt}n{\isacharplus}{\kern0pt}m{\isacharbraceright}{\kern0pt}\ then\ g\ {\isacharparenleft}{\kern0pt}x\ {\isacharminus}{\kern0pt}\ n{\isacharparenright}{\kern0pt}\ else\ undefined{\isacharparenright}{\kern0pt}{\isacharparenright}{\kern0pt}{\isachardoublequoteclose}\isanewline
\isanewline
\isacommand{lemma}\isamarkupfalse%
\ join{\isacharunderscore}{\kern0pt}cubes{\isacharcolon}{\kern0pt}\ \isakeyword{assumes}\ {\isachardoublequoteopen}f\ {\isasymin}\ cube\ n\ {\isacharparenleft}{\kern0pt}t{\isacharplus}{\kern0pt}{\isadigit{1}}{\isacharparenright}{\kern0pt}{\isachardoublequoteclose}\ \isakeyword{and}\ {\isachardoublequoteopen}g\ {\isasymin}\ cube\ m\ {\isacharparenleft}{\kern0pt}t{\isacharplus}{\kern0pt}{\isadigit{1}}{\isacharparenright}{\kern0pt}{\isachardoublequoteclose}\ \isakeyword{shows}\ {\isachardoublequoteopen}join\ f\ g\ n\ m\ {\isasymin}\ cube\ {\isacharparenleft}{\kern0pt}n{\isacharplus}{\kern0pt}m{\isacharparenright}{\kern0pt}\ {\isacharparenleft}{\kern0pt}t{\isacharplus}{\kern0pt}{\isadigit{1}}{\isacharparenright}{\kern0pt}{\isachardoublequoteclose}\isanewline
%
\isadelimproof
%
\endisadelimproof
%
\isatagproof
\isacommand{proof}\isamarkupfalse%
\ {\isacharparenleft}{\kern0pt}unfold\ cube{\isacharunderscore}{\kern0pt}def{\isacharsemicolon}{\kern0pt}\ intro\ PiE{\isacharunderscore}{\kern0pt}I{\isacharparenright}{\kern0pt}\isanewline
\ \ \isacommand{fix}\isamarkupfalse%
\ i\isanewline
\ \ \isacommand{assume}\isamarkupfalse%
\ {\isachardoublequoteopen}i\ {\isasymin}\ {\isacharbraceleft}{\kern0pt}{\isachardot}{\kern0pt}{\isachardot}{\kern0pt}{\isacharless}{\kern0pt}n{\isacharplus}{\kern0pt}m{\isacharbraceright}{\kern0pt}{\isachardoublequoteclose}\isanewline
\ \ \isacommand{then}\isamarkupfalse%
\ \isacommand{consider}\isamarkupfalse%
\ {\isachardoublequoteopen}i\ {\isacharless}{\kern0pt}\ n{\isachardoublequoteclose}\ {\isacharbar}{\kern0pt}\ {\isachardoublequoteopen}i\ {\isasymge}\ n\ {\isasymand}\ i\ {\isacharless}{\kern0pt}\ n{\isacharplus}{\kern0pt}m{\isachardoublequoteclose}\ \isacommand{by}\isamarkupfalse%
\ fastforce\isanewline
\ \ \isacommand{then}\isamarkupfalse%
\ \isacommand{show}\isamarkupfalse%
\ {\isachardoublequoteopen}join\ f\ g\ n\ m\ i\ {\isasymin}\ {\isacharbraceleft}{\kern0pt}{\isachardot}{\kern0pt}{\isachardot}{\kern0pt}{\isacharless}{\kern0pt}t\ {\isacharplus}{\kern0pt}\ {\isadigit{1}}{\isacharbraceright}{\kern0pt}{\isachardoublequoteclose}\isanewline
\ \ \isacommand{proof}\isamarkupfalse%
\ {\isacharparenleft}{\kern0pt}cases{\isacharparenright}{\kern0pt}\isanewline
\ \ \ \ \isacommand{case}\isamarkupfalse%
\ {\isadigit{1}}\isanewline
\ \ \ \ \isacommand{then}\isamarkupfalse%
\ \isacommand{have}\isamarkupfalse%
\ {\isachardoublequoteopen}join\ f\ g\ n\ m\ i\ {\isacharequal}{\kern0pt}\ f\ i{\isachardoublequoteclose}\ \isacommand{unfolding}\isamarkupfalse%
\ join{\isacharunderscore}{\kern0pt}def\ \isacommand{by}\isamarkupfalse%
\ simp\isanewline
\ \ \ \ \isacommand{moreover}\isamarkupfalse%
\ \isacommand{have}\isamarkupfalse%
\ {\isachardoublequoteopen}f\ i\ {\isasymin}\ {\isacharbraceleft}{\kern0pt}{\isachardot}{\kern0pt}{\isachardot}{\kern0pt}{\isacharless}{\kern0pt}t{\isacharplus}{\kern0pt}{\isadigit{1}}{\isacharbraceright}{\kern0pt}{\isachardoublequoteclose}\ \isacommand{using}\isamarkupfalse%
\ assms{\isacharparenleft}{\kern0pt}{\isadigit{1}}{\isacharparenright}{\kern0pt}\ {\isadigit{1}}\ \isacommand{unfolding}\isamarkupfalse%
\ cube{\isacharunderscore}{\kern0pt}def\ \isacommand{by}\isamarkupfalse%
\ blast\isanewline
\ \ \ \ \isacommand{ultimately}\isamarkupfalse%
\ \isacommand{show}\isamarkupfalse%
\ {\isacharquery}{\kern0pt}thesis\ \isacommand{by}\isamarkupfalse%
\ simp\isanewline
\ \ \isacommand{next}\isamarkupfalse%
\isanewline
\ \ \ \ \isacommand{case}\isamarkupfalse%
\ {\isadigit{2}}\isanewline
\ \ \ \ \isacommand{then}\isamarkupfalse%
\ \isacommand{have}\isamarkupfalse%
\ {\isachardoublequoteopen}join\ f\ g\ n\ m\ i\ {\isacharequal}{\kern0pt}\ g\ {\isacharparenleft}{\kern0pt}i\ {\isacharminus}{\kern0pt}\ n{\isacharparenright}{\kern0pt}{\isachardoublequoteclose}\ \isacommand{unfolding}\isamarkupfalse%
\ join{\isacharunderscore}{\kern0pt}def\ \isacommand{by}\isamarkupfalse%
\ simp\isanewline
\ \ \ \ \isacommand{moreover}\isamarkupfalse%
\ \isacommand{have}\isamarkupfalse%
\ {\isachardoublequoteopen}i\ {\isacharminus}{\kern0pt}\ n\ {\isasymin}\ {\isacharbraceleft}{\kern0pt}{\isachardot}{\kern0pt}{\isachardot}{\kern0pt}{\isacharless}{\kern0pt}m{\isacharbraceright}{\kern0pt}{\isachardoublequoteclose}\ \isacommand{using}\isamarkupfalse%
\ {\isadigit{2}}\ \isacommand{by}\isamarkupfalse%
\ auto\isanewline
\ \ \ \ \isacommand{moreover}\isamarkupfalse%
\ \isacommand{have}\isamarkupfalse%
\ {\isachardoublequoteopen}g\ {\isacharparenleft}{\kern0pt}i\ {\isacharminus}{\kern0pt}\ n{\isacharparenright}{\kern0pt}\ {\isasymin}\ {\isacharbraceleft}{\kern0pt}{\isachardot}{\kern0pt}{\isachardot}{\kern0pt}{\isacharless}{\kern0pt}t{\isacharplus}{\kern0pt}{\isadigit{1}}{\isacharbraceright}{\kern0pt}{\isachardoublequoteclose}\ \isacommand{using}\isamarkupfalse%
\ calculation{\isacharparenleft}{\kern0pt}{\isadigit{2}}{\isacharparenright}{\kern0pt}\ assms{\isacharparenleft}{\kern0pt}{\isadigit{2}}{\isacharparenright}{\kern0pt}\ \isacommand{unfolding}\isamarkupfalse%
\ cube{\isacharunderscore}{\kern0pt}def\ \isacommand{by}\isamarkupfalse%
\ blast\isanewline
\ \ \ \ \isacommand{ultimately}\isamarkupfalse%
\ \isacommand{show}\isamarkupfalse%
\ {\isacharquery}{\kern0pt}thesis\ \isacommand{by}\isamarkupfalse%
\ simp\isanewline
\ \ \isacommand{qed}\isamarkupfalse%
\isanewline
\isacommand{next}\isamarkupfalse%
\isanewline
\ \ \isacommand{fix}\isamarkupfalse%
\ i\isanewline
\ \ \isacommand{assume}\isamarkupfalse%
\ {\isachardoublequoteopen}i\ {\isasymnotin}\ {\isacharbraceleft}{\kern0pt}{\isachardot}{\kern0pt}{\isachardot}{\kern0pt}{\isacharless}{\kern0pt}n{\isacharplus}{\kern0pt}m{\isacharbraceright}{\kern0pt}{\isachardoublequoteclose}\isanewline
\ \ \isacommand{then}\isamarkupfalse%
\ \isacommand{show}\isamarkupfalse%
\ {\isachardoublequoteopen}join\ f\ g\ n\ m\ i\ {\isacharequal}{\kern0pt}\ undefined{\isachardoublequoteclose}\ \isacommand{unfolding}\isamarkupfalse%
\ join{\isacharunderscore}{\kern0pt}def\ \isacommand{by}\isamarkupfalse%
\ simp\isanewline
\isacommand{qed}\isamarkupfalse%
%
\endisatagproof
{\isafoldproof}%
%
\isadelimproof
\isanewline
%
\endisadelimproof
\isanewline
\isacommand{lemma}\isamarkupfalse%
\ subspace{\isacharunderscore}{\kern0pt}elems{\isacharunderscore}{\kern0pt}embed{\isacharcolon}{\kern0pt}\ \isakeyword{assumes}\ {\isachardoublequoteopen}is{\isacharunderscore}{\kern0pt}subspace\ S\ k\ n\ t{\isachardoublequoteclose}\isanewline
\ \ \isakeyword{shows}\ {\isachardoublequoteopen}S\ {\isacharbackquote}{\kern0pt}\ {\isacharparenleft}{\kern0pt}cube\ k\ t{\isacharparenright}{\kern0pt}\ {\isasymsubseteq}\ cube\ n\ t{\isachardoublequoteclose}\isanewline
%
\isadelimproof
\ \ %
\endisadelimproof
%
\isatagproof
\isacommand{using}\isamarkupfalse%
\ assms\ \isacommand{unfolding}\isamarkupfalse%
\ cube{\isacharunderscore}{\kern0pt}def\ is{\isacharunderscore}{\kern0pt}subspace{\isacharunderscore}{\kern0pt}def\ \isacommand{by}\isamarkupfalse%
\ blast%
\endisatagproof
{\isafoldproof}%
%
\isadelimproof
%
\endisadelimproof
%
\begin{isamarkuptext}%
The induction step of theorem 4. Heart of the proof%
\end{isamarkuptext}\isamarkuptrue%
%
\begin{isamarkuptext}%
Proof sketch/idea:
  * we prove lhj r t (k+1) for all r at once. That means we assume hj r t for all r, and lhj r t k' for all r (and all dimensions k' less than k+1)
  * remember: hj -> statement about monochromatic lines, lhj -> statement about layered subspaces (k-dimensional)
  * core idea: to construct (k+1)-dimensional subspace, use (by induction) k-dimensional subspace and (by assumption) 1-dimensional subspace (line) in some natural way (ensuring the colorings satisfy the requisite conditions)

In detail:
  - let \isa{{\isasymchi}} be an r-coloring, for which we wish to show that there exists a layered (k+1)-dimensional subspace.
  - (SECTION 1) by our assumptions, we can obtain a layered k-dimensional subspace S (w.r.t. r-colorings) and a layered line L (w.r.t. to s-colorings, where s=f(r) is constructed from r to facilitate our main proof; details irrelevant)
  - let m be the dimension of the cube in which the layered k-dimensional subspace S exists
  - let n' be the dimension of the cube in which the layered line L exists
  - we claim that the layered (k+1)-dimensional subspace we are looking for exists in the (m+n')-dimensional cube
    - concretely, we construct these (m+n')-dimensional elements (i.e. tuples) by setting the first n' coordinates to points on the line, and the last m coordinates to points on the subspace.
    - (SECTION 2) this construction yields a subspace (i.e. satisfying the subspace properties). We call this T''. 
    -  We prove it is a subspace in SECTION 3. In SECTION 4, we show it is layered.%
\end{isamarkuptext}\isamarkuptrue%
\isacommand{lemma}\isamarkupfalse%
\ thm{\isadigit{4}}{\isacharunderscore}{\kern0pt}step{\isacharcolon}{\kern0pt}\ \isanewline
\ \ \isakeyword{fixes}\ \ \ r\ k\isanewline
\ \ \isakeyword{assumes}\ {\isachardoublequoteopen}t\ {\isachargreater}{\kern0pt}\ {\isadigit{0}}{\isachardoublequoteclose}\isanewline
\ \ \ \ \isakeyword{and}\ {\isachardoublequoteopen}k\ {\isasymge}\ {\isadigit{1}}{\isachardoublequoteclose}\isanewline
\ \ \ \ \isakeyword{and}\ {\isachardoublequoteopen}True{\isachardoublequoteclose}\ \isanewline
\ \ \ \ \isakeyword{and}\ {\isachardoublequoteopen}{\isacharparenleft}{\kern0pt}{\isasymAnd}r\ k{\isacharprime}{\kern0pt}{\isachardot}{\kern0pt}\ k{\isacharprime}{\kern0pt}\ {\isasymle}\ k\ {\isasymLongrightarrow}\ lhj\ r\ t\ k{\isacharprime}{\kern0pt}{\isacharparenright}{\kern0pt}{\isachardoublequoteclose}\ \isanewline
\ \ \ \ \isakeyword{and}\ {\isachardoublequoteopen}r\ {\isachargreater}{\kern0pt}\ {\isadigit{0}}{\isachardoublequoteclose}\isanewline
\ \ \isakeyword{shows}\ \ \ {\isachardoublequoteopen}lhj\ r\ t\ {\isacharparenleft}{\kern0pt}k{\isacharplus}{\kern0pt}{\isadigit{1}}{\isacharparenright}{\kern0pt}{\isachardoublequoteclose}\isanewline
%
\isadelimproof
%
\endisadelimproof
%
\isatagproof
\isacommand{proof}\isamarkupfalse%
{\isacharminus}{\kern0pt}\isanewline
\ \ \isacommand{obtain}\isamarkupfalse%
\ m\ \isakeyword{where}\ m{\isacharunderscore}{\kern0pt}props{\isacharcolon}{\kern0pt}\ {\isachardoublequoteopen}{\isacharparenleft}{\kern0pt}m\ {\isachargreater}{\kern0pt}\ {\isadigit{0}}\ {\isasymand}\ {\isacharparenleft}{\kern0pt}{\isasymforall}M{\isacharprime}{\kern0pt}\ {\isasymge}\ m{\isachardot}{\kern0pt}\ {\isasymforall}{\isasymchi}{\isachardot}{\kern0pt}\ {\isasymchi}\ {\isasymin}\ {\isacharparenleft}{\kern0pt}cube\ M{\isacharprime}{\kern0pt}\ {\isacharparenleft}{\kern0pt}t\ {\isacharplus}{\kern0pt}\ {\isadigit{1}}{\isacharparenright}{\kern0pt}{\isacharparenright}{\kern0pt}\ {\isasymrightarrow}\isactrlsub E\ {\isacharbraceleft}{\kern0pt}{\isachardot}{\kern0pt}{\isachardot}{\kern0pt}{\isacharless}{\kern0pt}r{\isacharcolon}{\kern0pt}{\isacharcolon}{\kern0pt}nat{\isacharbraceright}{\kern0pt}\ {\isasymlongrightarrow}\ {\isacharparenleft}{\kern0pt}{\isasymexists}S{\isachardot}{\kern0pt}\ layered{\isacharunderscore}{\kern0pt}subspace\ S\ k\ M{\isacharprime}{\kern0pt}\ t\ r\ {\isasymchi}{\isacharparenright}{\kern0pt}{\isacharparenright}{\kern0pt}{\isacharparenright}{\kern0pt}{\isachardoublequoteclose}\ \isacommand{using}\isamarkupfalse%
\ assms{\isacharparenleft}{\kern0pt}{\isadigit{4}}{\isacharparenright}{\kern0pt}{\isacharbrackleft}{\kern0pt}of\ {\isachardoublequoteopen}k{\isachardoublequoteclose}\ {\isachardoublequoteopen}r{\isachardoublequoteclose}{\isacharbrackright}{\kern0pt}\ \isacommand{unfolding}\isamarkupfalse%
\ lhj{\isacharunderscore}{\kern0pt}def\ \ \isacommand{by}\isamarkupfalse%
\ blast\isanewline
\ \ \isacommand{define}\isamarkupfalse%
\ s\ \isakeyword{where}\ {\isachardoublequoteopen}s\ {\isasymequiv}\ r{\isacharcircum}{\kern0pt}{\isacharparenleft}{\kern0pt}{\isacharparenleft}{\kern0pt}t\ {\isacharplus}{\kern0pt}\ {\isadigit{1}}{\isacharparenright}{\kern0pt}{\isacharcircum}{\kern0pt}m{\isacharparenright}{\kern0pt}{\isachardoublequoteclose}\isanewline
\ \ \isacommand{obtain}\isamarkupfalse%
\ n{\isacharprime}{\kern0pt}\ \isakeyword{where}\ n{\isacharprime}{\kern0pt}{\isacharunderscore}{\kern0pt}props{\isacharcolon}{\kern0pt}\ {\isachardoublequoteopen}{\isacharparenleft}{\kern0pt}n{\isacharprime}{\kern0pt}\ {\isachargreater}{\kern0pt}\ {\isadigit{0}}\ {\isasymand}\ {\isacharparenleft}{\kern0pt}{\isasymforall}N\ {\isasymge}\ n{\isacharprime}{\kern0pt}{\isachardot}{\kern0pt}\ {\isasymforall}{\isasymchi}{\isachardot}{\kern0pt}\ {\isasymchi}\ {\isasymin}\ {\isacharparenleft}{\kern0pt}cube\ N\ {\isacharparenleft}{\kern0pt}t\ {\isacharplus}{\kern0pt}\ {\isadigit{1}}{\isacharparenright}{\kern0pt}{\isacharparenright}{\kern0pt}\ {\isasymrightarrow}\isactrlsub E\ {\isacharbraceleft}{\kern0pt}{\isachardot}{\kern0pt}{\isachardot}{\kern0pt}{\isacharless}{\kern0pt}s{\isacharcolon}{\kern0pt}{\isacharcolon}{\kern0pt}nat{\isacharbraceright}{\kern0pt}\ {\isasymlongrightarrow}\ {\isacharparenleft}{\kern0pt}{\isasymexists}S{\isachardot}{\kern0pt}\ layered{\isacharunderscore}{\kern0pt}subspace\ S\ {\isadigit{1}}\ N\ t\ s\ {\isasymchi}{\isacharparenright}{\kern0pt}{\isacharparenright}{\kern0pt}{\isacharparenright}{\kern0pt}{\isachardoublequoteclose}\ \isacommand{using}\isamarkupfalse%
\ assms{\isacharparenleft}{\kern0pt}{\isadigit{2}}{\isacharparenright}{\kern0pt}\ assms{\isacharparenleft}{\kern0pt}{\isadigit{4}}{\isacharparenright}{\kern0pt}{\isacharbrackleft}{\kern0pt}of\ {\isachardoublequoteopen}{\isadigit{1}}{\isachardoublequoteclose}\ {\isachardoublequoteopen}s{\isachardoublequoteclose}{\isacharbrackright}{\kern0pt}\ \isacommand{unfolding}\isamarkupfalse%
\ lhj{\isacharunderscore}{\kern0pt}def\ \isacommand{by}\isamarkupfalse%
\ auto\ \isanewline
\isanewline
\ \ \isacommand{have}\isamarkupfalse%
\ {\isachardoublequoteopen}{\isacharparenleft}{\kern0pt}{\isasymexists}T{\isachardot}{\kern0pt}\ layered{\isacharunderscore}{\kern0pt}subspace\ T\ {\isacharparenleft}{\kern0pt}k\ {\isacharplus}{\kern0pt}\ {\isadigit{1}}{\isacharparenright}{\kern0pt}\ {\isacharparenleft}{\kern0pt}M{\isacharprime}{\kern0pt}{\isacharparenright}{\kern0pt}\ t\ r\ {\isasymchi}{\isacharparenright}{\kern0pt}{\isachardoublequoteclose}\ \isakeyword{if}\ {\isasymchi}{\isacharunderscore}{\kern0pt}prop{\isacharcolon}{\kern0pt}\ {\isachardoublequoteopen}{\isasymchi}\ {\isasymin}\ cube\ M{\isacharprime}{\kern0pt}\ {\isacharparenleft}{\kern0pt}t\ {\isacharplus}{\kern0pt}\ {\isadigit{1}}{\isacharparenright}{\kern0pt}\ {\isasymrightarrow}\isactrlsub E\ {\isacharbraceleft}{\kern0pt}{\isachardot}{\kern0pt}{\isachardot}{\kern0pt}{\isacharless}{\kern0pt}r{\isacharbraceright}{\kern0pt}{\isachardoublequoteclose}\ \isakeyword{and}\ M{\isacharprime}{\kern0pt}{\isacharunderscore}{\kern0pt}prop{\isacharcolon}{\kern0pt}\ {\isachardoublequoteopen}M{\isacharprime}{\kern0pt}\ {\isasymge}\ n{\isacharprime}{\kern0pt}\ {\isacharplus}{\kern0pt}\ m{\isachardoublequoteclose}\ \isakeyword{for}\ {\isasymchi}\ M{\isacharprime}{\kern0pt}\isanewline
\ \ \isacommand{proof}\isamarkupfalse%
\ {\isacharminus}{\kern0pt}\isanewline
\ \ \ \ \isacommand{define}\isamarkupfalse%
\ d\ \isakeyword{where}\ {\isachardoublequoteopen}d\ {\isasymequiv}\ M{\isacharprime}{\kern0pt}\ {\isacharminus}{\kern0pt}\ {\isacharparenleft}{\kern0pt}n{\isacharprime}{\kern0pt}\ {\isacharplus}{\kern0pt}\ m{\isacharparenright}{\kern0pt}{\isachardoublequoteclose}\isanewline
\ \ \ \ \isacommand{define}\isamarkupfalse%
\ n\ \isakeyword{where}\ {\isachardoublequoteopen}n\ {\isasymequiv}\ n{\isacharprime}{\kern0pt}\ {\isacharplus}{\kern0pt}\ d{\isachardoublequoteclose}\isanewline
\ \ \ \ \isacommand{have}\isamarkupfalse%
\ {\isachardoublequoteopen}n\ {\isasymge}\ n{\isacharprime}{\kern0pt}{\isachardoublequoteclose}\ \isacommand{unfolding}\isamarkupfalse%
\ n{\isacharunderscore}{\kern0pt}def\ d{\isacharunderscore}{\kern0pt}def\ \isacommand{by}\isamarkupfalse%
\ simp\isanewline
\ \ \ \ \isacommand{have}\isamarkupfalse%
\ {\isachardoublequoteopen}n\ {\isacharplus}{\kern0pt}\ m\ {\isacharequal}{\kern0pt}\ M{\isacharprime}{\kern0pt}{\isachardoublequoteclose}\ \isacommand{unfolding}\isamarkupfalse%
\ n{\isacharunderscore}{\kern0pt}def\ d{\isacharunderscore}{\kern0pt}def\ \isacommand{using}\isamarkupfalse%
\ M{\isacharprime}{\kern0pt}{\isacharunderscore}{\kern0pt}prop\ \isacommand{by}\isamarkupfalse%
\ simp\isanewline
\ \ \ \ \isacommand{have}\isamarkupfalse%
\ {\isachardoublequoteopen}{\isasymforall}{\isasymchi}{\isachardot}{\kern0pt}\ {\isasymchi}\ {\isasymin}\ {\isacharparenleft}{\kern0pt}cube\ n\ {\isacharparenleft}{\kern0pt}t\ {\isacharplus}{\kern0pt}\ {\isadigit{1}}{\isacharparenright}{\kern0pt}{\isacharparenright}{\kern0pt}\ {\isasymrightarrow}\isactrlsub E\ {\isacharbraceleft}{\kern0pt}{\isachardot}{\kern0pt}{\isachardot}{\kern0pt}{\isacharless}{\kern0pt}s{\isacharcolon}{\kern0pt}{\isacharcolon}{\kern0pt}nat{\isacharbraceright}{\kern0pt}\ {\isasymlongrightarrow}\ {\isacharparenleft}{\kern0pt}{\isasymexists}S{\isachardot}{\kern0pt}\ layered{\isacharunderscore}{\kern0pt}subspace\ S\ {\isadigit{1}}\ n\ t\ s\ {\isasymchi}{\isacharparenright}{\kern0pt}{\isachardoublequoteclose}\ \isacommand{using}\isamarkupfalse%
\ n{\isacharprime}{\kern0pt}{\isacharunderscore}{\kern0pt}props\ {\isacartoucheopen}n\ {\isasymge}\ n{\isacharprime}{\kern0pt}{\isacartoucheclose}\ \isacommand{by}\isamarkupfalse%
\ blast\isanewline
\ \ \ \ \isacommand{have}\isamarkupfalse%
\ line{\isacharunderscore}{\kern0pt}subspace{\isacharunderscore}{\kern0pt}s{\isacharcolon}{\kern0pt}\ {\isachardoublequoteopen}{\isasymforall}{\isasymchi}{\isachardot}{\kern0pt}\ {\isasymchi}\ {\isasymin}\ {\isacharparenleft}{\kern0pt}cube\ n\ {\isacharparenleft}{\kern0pt}t\ {\isacharplus}{\kern0pt}\ {\isadigit{1}}{\isacharparenright}{\kern0pt}{\isacharparenright}{\kern0pt}\ {\isasymrightarrow}\isactrlsub E\ {\isacharbraceleft}{\kern0pt}{\isachardot}{\kern0pt}{\isachardot}{\kern0pt}{\isacharless}{\kern0pt}s{\isacharcolon}{\kern0pt}{\isacharcolon}{\kern0pt}nat{\isacharbraceright}{\kern0pt}\ {\isasymlongrightarrow}\ {\isacharparenleft}{\kern0pt}{\isasymexists}S{\isachardot}{\kern0pt}\ layered{\isacharunderscore}{\kern0pt}subspace\ S\ {\isadigit{1}}\ n\ t\ s\ {\isasymchi}\ {\isasymand}\ is{\isacharunderscore}{\kern0pt}line\ {\isacharparenleft}{\kern0pt}{\isasymlambda}s{\isasymin}{\isacharbraceleft}{\kern0pt}{\isachardot}{\kern0pt}{\isachardot}{\kern0pt}{\isacharless}{\kern0pt}t{\isacharplus}{\kern0pt}{\isadigit{1}}{\isacharbraceright}{\kern0pt}{\isachardot}{\kern0pt}\ S\ {\isacharparenleft}{\kern0pt}SOME\ p{\isachardot}{\kern0pt}\ p{\isasymin}cube\ {\isadigit{1}}\ {\isacharparenleft}{\kern0pt}t{\isacharplus}{\kern0pt}{\isadigit{1}}{\isacharparenright}{\kern0pt}\ {\isasymand}\ p\ {\isadigit{0}}\ {\isacharequal}{\kern0pt}\ s{\isacharparenright}{\kern0pt}{\isacharparenright}{\kern0pt}\ n\ {\isacharparenleft}{\kern0pt}t{\isacharplus}{\kern0pt}{\isadigit{1}}{\isacharparenright}{\kern0pt}{\isacharparenright}{\kern0pt}{\isachardoublequoteclose}\isanewline
\ \ \ \ \isacommand{proof}\isamarkupfalse%
{\isacharparenleft}{\kern0pt}safe{\isacharparenright}{\kern0pt}\isanewline
\ \ \ \ \ \ \isacommand{fix}\isamarkupfalse%
\ {\isasymchi}\isanewline
\ \ \ \ \ \ \isacommand{assume}\isamarkupfalse%
\ a{\isacharcolon}{\kern0pt}\ {\isachardoublequoteopen}{\isasymchi}\ {\isasymin}\ cube\ n\ {\isacharparenleft}{\kern0pt}t\ {\isacharplus}{\kern0pt}\ {\isadigit{1}}{\isacharparenright}{\kern0pt}\ {\isasymrightarrow}\isactrlsub E\ {\isacharbraceleft}{\kern0pt}{\isachardot}{\kern0pt}{\isachardot}{\kern0pt}{\isacharless}{\kern0pt}s{\isacharbraceright}{\kern0pt}{\isachardoublequoteclose}\isanewline
\ \ \ \ \ \ \isacommand{then}\isamarkupfalse%
\ \isacommand{have}\isamarkupfalse%
\ {\isachardoublequoteopen}{\isacharparenleft}{\kern0pt}{\isasymexists}S{\isachardot}{\kern0pt}\ layered{\isacharunderscore}{\kern0pt}subspace\ S\ {\isadigit{1}}\ n\ t\ s\ {\isasymchi}{\isacharparenright}{\kern0pt}{\isachardoublequoteclose}\ \isanewline
\ \ \ \ \ \ \ \ \isacommand{using}\isamarkupfalse%
\ {\isacartoucheopen}{\isasymforall}{\isasymchi}{\isachardot}{\kern0pt}\ {\isasymchi}\ {\isasymin}\ cube\ n\ {\isacharparenleft}{\kern0pt}t\ {\isacharplus}{\kern0pt}\ {\isadigit{1}}{\isacharparenright}{\kern0pt}\ {\isasymrightarrow}\isactrlsub E\ {\isacharbraceleft}{\kern0pt}{\isachardot}{\kern0pt}{\isachardot}{\kern0pt}{\isacharless}{\kern0pt}s{\isacharbraceright}{\kern0pt}\ {\isasymlongrightarrow}\ {\isacharparenleft}{\kern0pt}{\isasymexists}S{\isachardot}{\kern0pt}\ layered{\isacharunderscore}{\kern0pt}subspace\ S\ {\isadigit{1}}\ n\ t\ s\ {\isasymchi}{\isacharparenright}{\kern0pt}{\isacartoucheclose}\ \isacommand{by}\isamarkupfalse%
\ presburger\isanewline
\ \ \ \ \ \ \isacommand{then}\isamarkupfalse%
\ \isacommand{obtain}\isamarkupfalse%
\ L\ \isakeyword{where}\ {\isachardoublequoteopen}layered{\isacharunderscore}{\kern0pt}subspace\ L\ {\isadigit{1}}\ n\ t\ s\ {\isasymchi}{\isachardoublequoteclose}\ \isacommand{by}\isamarkupfalse%
\ blast\isanewline
\ \ \ \ \ \ \isacommand{then}\isamarkupfalse%
\ \isacommand{have}\isamarkupfalse%
\ {\isachardoublequoteopen}is{\isacharunderscore}{\kern0pt}subspace\ L\ {\isadigit{1}}\ n\ {\isacharparenleft}{\kern0pt}t{\isacharplus}{\kern0pt}{\isadigit{1}}{\isacharparenright}{\kern0pt}{\isachardoublequoteclose}\ \isacommand{unfolding}\isamarkupfalse%
\ layered{\isacharunderscore}{\kern0pt}subspace{\isacharunderscore}{\kern0pt}def\ \isacommand{by}\isamarkupfalse%
\ simp\isanewline
\ \ \ \ \ \ \isacommand{then}\isamarkupfalse%
\ \isacommand{have}\isamarkupfalse%
\ {\isachardoublequoteopen}is{\isacharunderscore}{\kern0pt}line\ {\isacharparenleft}{\kern0pt}{\isasymlambda}s{\isasymin}{\isacharbraceleft}{\kern0pt}{\isachardot}{\kern0pt}{\isachardot}{\kern0pt}{\isacharless}{\kern0pt}t{\isacharplus}{\kern0pt}{\isadigit{1}}{\isacharbraceright}{\kern0pt}{\isachardot}{\kern0pt}\ L\ {\isacharparenleft}{\kern0pt}SOME\ p{\isachardot}{\kern0pt}\ p{\isasymin}cube\ {\isadigit{1}}\ {\isacharparenleft}{\kern0pt}t{\isacharplus}{\kern0pt}{\isadigit{1}}{\isacharparenright}{\kern0pt}\ {\isasymand}\ p\ {\isadigit{0}}\ {\isacharequal}{\kern0pt}\ s{\isacharparenright}{\kern0pt}{\isacharparenright}{\kern0pt}\ n\ {\isacharparenleft}{\kern0pt}t\ {\isacharplus}{\kern0pt}\ {\isadigit{1}}{\isacharparenright}{\kern0pt}{\isachardoublequoteclose}\ \isacommand{using}\isamarkupfalse%
\ dim{\isadigit{1}}{\isacharunderscore}{\kern0pt}subspace{\isacharunderscore}{\kern0pt}is{\isacharunderscore}{\kern0pt}line{\isacharbrackleft}{\kern0pt}of\ {\isachardoublequoteopen}t{\isacharplus}{\kern0pt}{\isadigit{1}}{\isachardoublequoteclose}\ {\isachardoublequoteopen}L{\isachardoublequoteclose}\ {\isachardoublequoteopen}n{\isachardoublequoteclose}{\isacharbrackright}{\kern0pt}\ assms{\isacharparenleft}{\kern0pt}{\isadigit{1}}{\isacharparenright}{\kern0pt}\ \isacommand{by}\isamarkupfalse%
\ simp\isanewline
\ \ \ \ \ \ \isacommand{then}\isamarkupfalse%
\ \isacommand{show}\isamarkupfalse%
\ {\isachardoublequoteopen}{\isasymexists}S{\isachardot}{\kern0pt}\ layered{\isacharunderscore}{\kern0pt}subspace\ S\ {\isadigit{1}}\ n\ t\ s\ {\isasymchi}\ {\isasymand}\ is{\isacharunderscore}{\kern0pt}line\ {\isacharparenleft}{\kern0pt}{\isasymlambda}s{\isasymin}{\isacharbraceleft}{\kern0pt}{\isachardot}{\kern0pt}{\isachardot}{\kern0pt}{\isacharless}{\kern0pt}t\ {\isacharplus}{\kern0pt}\ {\isadigit{1}}{\isacharbraceright}{\kern0pt}{\isachardot}{\kern0pt}\ S\ {\isacharparenleft}{\kern0pt}SOME\ p{\isachardot}{\kern0pt}\ p\ {\isasymin}\ cube\ {\isadigit{1}}\ {\isacharparenleft}{\kern0pt}t{\isacharplus}{\kern0pt}{\isadigit{1}}{\isacharparenright}{\kern0pt}\ {\isasymand}\ p\ {\isadigit{0}}\ {\isacharequal}{\kern0pt}\ s{\isacharparenright}{\kern0pt}{\isacharparenright}{\kern0pt}\ n\ {\isacharparenleft}{\kern0pt}t\ {\isacharplus}{\kern0pt}\ {\isadigit{1}}{\isacharparenright}{\kern0pt}{\isachardoublequoteclose}\ \isacommand{using}\isamarkupfalse%
\ {\isacartoucheopen}layered{\isacharunderscore}{\kern0pt}subspace\ L\ {\isadigit{1}}\ n\ t\ s\ {\isasymchi}{\isacartoucheclose}\ \isacommand{by}\isamarkupfalse%
\ auto\isanewline
\ \ \ \ \isacommand{qed}\isamarkupfalse%
\isanewline
\isanewline
\isanewline
\ \ \isanewline
\isanewline
\isanewline
\ \ \ \ \isacommand{define}\isamarkupfalse%
\ {\isasymchi}L\ \isakeyword{where}\ {\isachardoublequoteopen}{\isasymchi}L\ {\isasymequiv}\ {\isacharparenleft}{\kern0pt}{\isasymlambda}x\ {\isasymin}\ cube\ n\ {\isacharparenleft}{\kern0pt}t{\isacharplus}{\kern0pt}{\isadigit{1}}{\isacharparenright}{\kern0pt}{\isachardot}{\kern0pt}\ {\isacharparenleft}{\kern0pt}{\isasymlambda}y\ {\isasymin}\ cube\ m\ {\isacharparenleft}{\kern0pt}t\ {\isacharplus}{\kern0pt}\ {\isadigit{1}}{\isacharparenright}{\kern0pt}{\isachardot}{\kern0pt}\ {\isasymchi}\ {\isacharparenleft}{\kern0pt}join\ x\ y\ n\ m{\isacharparenright}{\kern0pt}{\isacharparenright}{\kern0pt}{\isacharparenright}{\kern0pt}{\isachardoublequoteclose}\isanewline
\ \ \ \ \isacommand{have}\isamarkupfalse%
\ A{\isacharcolon}{\kern0pt}\ {\isachardoublequoteopen}{\isasymforall}x\ {\isasymin}\ cube\ n\ {\isacharparenleft}{\kern0pt}t{\isacharplus}{\kern0pt}{\isadigit{1}}{\isacharparenright}{\kern0pt}{\isachardot}{\kern0pt}\ {\isasymforall}y\ {\isasymin}\ cube\ m\ {\isacharparenleft}{\kern0pt}t{\isacharplus}{\kern0pt}{\isadigit{1}}{\isacharparenright}{\kern0pt}{\isachardot}{\kern0pt}\ {\isasymchi}\ {\isacharparenleft}{\kern0pt}join\ x\ y\ n\ m{\isacharparenright}{\kern0pt}\ {\isasymin}\ {\isacharbraceleft}{\kern0pt}{\isachardot}{\kern0pt}{\isachardot}{\kern0pt}{\isacharless}{\kern0pt}r{\isacharbraceright}{\kern0pt}{\isachardoublequoteclose}\isanewline
\ \ \ \ \isacommand{proof}\isamarkupfalse%
{\isacharparenleft}{\kern0pt}safe{\isacharparenright}{\kern0pt}\isanewline
\ \ \ \ \ \ \isacommand{fix}\isamarkupfalse%
\ x\ y\isanewline
\ \ \ \ \ \ \isacommand{assume}\isamarkupfalse%
\ {\isachardoublequoteopen}x\ {\isasymin}\ cube\ n\ {\isacharparenleft}{\kern0pt}t{\isacharplus}{\kern0pt}{\isadigit{1}}{\isacharparenright}{\kern0pt}{\isachardoublequoteclose}\ {\isachardoublequoteopen}y\ {\isasymin}\ cube\ m\ {\isacharparenleft}{\kern0pt}t{\isacharplus}{\kern0pt}{\isadigit{1}}{\isacharparenright}{\kern0pt}{\isachardoublequoteclose}\isanewline
\ \ \ \ \ \ \isacommand{then}\isamarkupfalse%
\ \isacommand{have}\isamarkupfalse%
\ {\isachardoublequoteopen}join\ x\ y\ n\ m\ {\isasymin}\ cube\ {\isacharparenleft}{\kern0pt}n{\isacharplus}{\kern0pt}m{\isacharparenright}{\kern0pt}\ {\isacharparenleft}{\kern0pt}t{\isacharplus}{\kern0pt}{\isadigit{1}}{\isacharparenright}{\kern0pt}{\isachardoublequoteclose}\ \isacommand{using}\isamarkupfalse%
\ join{\isacharunderscore}{\kern0pt}cubes{\isacharbrackleft}{\kern0pt}of\ x\ n\ t\ y\ m{\isacharbrackright}{\kern0pt}\ \isacommand{by}\isamarkupfalse%
\ simp\isanewline
\ \ \ \ \ \ \isacommand{then}\isamarkupfalse%
\ \isacommand{show}\isamarkupfalse%
\ {\isachardoublequoteopen}{\isasymchi}\ {\isacharparenleft}{\kern0pt}join\ x\ y\ n\ m{\isacharparenright}{\kern0pt}\ {\isacharless}{\kern0pt}\ r{\isachardoublequoteclose}\ \isacommand{using}\isamarkupfalse%
\ {\isasymchi}{\isacharunderscore}{\kern0pt}prop\ {\isacartoucheopen}n\ {\isacharplus}{\kern0pt}\ m\ {\isacharequal}{\kern0pt}\ M{\isacharprime}{\kern0pt}{\isacartoucheclose}\ \isacommand{by}\isamarkupfalse%
\ blast\ \isanewline
\ \ \ \ \isacommand{qed}\isamarkupfalse%
\isanewline
\ \ \ \ \isacommand{have}\isamarkupfalse%
\ {\isasymchi}L{\isacharunderscore}{\kern0pt}prop{\isacharcolon}{\kern0pt}\ {\isachardoublequoteopen}{\isasymchi}L\ {\isasymin}\ cube\ n\ {\isacharparenleft}{\kern0pt}t{\isacharplus}{\kern0pt}{\isadigit{1}}{\isacharparenright}{\kern0pt}\ {\isasymrightarrow}\isactrlsub E\ cube\ m\ {\isacharparenleft}{\kern0pt}t{\isacharplus}{\kern0pt}{\isadigit{1}}{\isacharparenright}{\kern0pt}\ {\isasymrightarrow}\isactrlsub E\ {\isacharbraceleft}{\kern0pt}{\isachardot}{\kern0pt}{\isachardot}{\kern0pt}{\isacharless}{\kern0pt}r{\isacharbraceright}{\kern0pt}{\isachardoublequoteclose}\ \isacommand{using}\isamarkupfalse%
\ A\ \isacommand{by}\isamarkupfalse%
\ {\isacharparenleft}{\kern0pt}auto\ simp{\isacharcolon}{\kern0pt}\ {\isasymchi}L{\isacharunderscore}{\kern0pt}def{\isacharparenright}{\kern0pt}\isanewline
\isanewline
\ \ \ \ \isacommand{have}\isamarkupfalse%
\ {\isachardoublequoteopen}card\ {\isacharparenleft}{\kern0pt}cube\ m\ {\isacharparenleft}{\kern0pt}t{\isacharplus}{\kern0pt}{\isadigit{1}}{\isacharparenright}{\kern0pt}\ {\isasymrightarrow}\isactrlsub E\ {\isacharbraceleft}{\kern0pt}{\isachardot}{\kern0pt}{\isachardot}{\kern0pt}{\isacharless}{\kern0pt}r{\isacharbraceright}{\kern0pt}{\isacharparenright}{\kern0pt}\ {\isacharequal}{\kern0pt}\ {\isacharparenleft}{\kern0pt}card\ {\isacharbraceleft}{\kern0pt}{\isachardot}{\kern0pt}{\isachardot}{\kern0pt}{\isacharless}{\kern0pt}r{\isacharbraceright}{\kern0pt}{\isacharparenright}{\kern0pt}\ {\isacharcircum}{\kern0pt}\ {\isacharparenleft}{\kern0pt}card\ {\isacharparenleft}{\kern0pt}cube\ m\ {\isacharparenleft}{\kern0pt}t{\isacharplus}{\kern0pt}{\isadigit{1}}{\isacharparenright}{\kern0pt}{\isacharparenright}{\kern0pt}{\isacharparenright}{\kern0pt}{\isachardoublequoteclose}\ \ \isacommand{apply}\isamarkupfalse%
\ {\isacharparenleft}{\kern0pt}subst\ card{\isacharunderscore}{\kern0pt}PiE{\isacharparenright}{\kern0pt}\ \isacommand{unfolding}\isamarkupfalse%
\ cube{\isacharunderscore}{\kern0pt}def\ \isacommand{apply}\isamarkupfalse%
\ {\isacharparenleft}{\kern0pt}meson\ finite{\isacharunderscore}{\kern0pt}PiE\ finite{\isacharunderscore}{\kern0pt}lessThan{\isacharparenright}{\kern0pt}\ \ \isanewline
\ \ \ \ \ \ \isacommand{using}\isamarkupfalse%
\ prod{\isacharunderscore}{\kern0pt}constant\ \isacommand{by}\isamarkupfalse%
\ blast\isanewline
\ \ \ \ \isacommand{also}\isamarkupfalse%
\ \isacommand{have}\isamarkupfalse%
\ {\isachardoublequoteopen}{\isachardot}{\kern0pt}{\isachardot}{\kern0pt}{\isachardot}{\kern0pt}\ {\isacharequal}{\kern0pt}\ r\ {\isacharcircum}{\kern0pt}\ {\isacharparenleft}{\kern0pt}card\ {\isacharparenleft}{\kern0pt}cube\ m\ {\isacharparenleft}{\kern0pt}t{\isacharplus}{\kern0pt}{\isadigit{1}}{\isacharparenright}{\kern0pt}{\isacharparenright}{\kern0pt}{\isacharparenright}{\kern0pt}{\isachardoublequoteclose}\ \isacommand{by}\isamarkupfalse%
\ simp\isanewline
\ \ \ \ \isacommand{also}\isamarkupfalse%
\ \isacommand{have}\isamarkupfalse%
\ {\isachardoublequoteopen}{\isachardot}{\kern0pt}{\isachardot}{\kern0pt}{\isachardot}{\kern0pt}\ {\isacharequal}{\kern0pt}\ r\ {\isacharcircum}{\kern0pt}\ {\isacharparenleft}{\kern0pt}{\isacharparenleft}{\kern0pt}t{\isacharplus}{\kern0pt}{\isadigit{1}}{\isacharparenright}{\kern0pt}{\isacharcircum}{\kern0pt}m{\isacharparenright}{\kern0pt}{\isachardoublequoteclose}\ \isacommand{using}\isamarkupfalse%
\ cube{\isacharunderscore}{\kern0pt}card\ \isacommand{unfolding}\isamarkupfalse%
\ cube{\isacharunderscore}{\kern0pt}def\ \isacommand{by}\isamarkupfalse%
\ simp\isanewline
\ \ \ \ \isacommand{finally}\isamarkupfalse%
\ \isacommand{have}\isamarkupfalse%
\ {\isachardoublequoteopen}card\ {\isacharparenleft}{\kern0pt}cube\ m\ {\isacharparenleft}{\kern0pt}t{\isacharplus}{\kern0pt}{\isadigit{1}}{\isacharparenright}{\kern0pt}\ {\isasymrightarrow}\isactrlsub E\ {\isacharbraceleft}{\kern0pt}{\isachardot}{\kern0pt}{\isachardot}{\kern0pt}{\isacharless}{\kern0pt}r{\isacharbraceright}{\kern0pt}{\isacharparenright}{\kern0pt}\ {\isacharequal}{\kern0pt}\ r\ {\isacharcircum}{\kern0pt}\ {\isacharparenleft}{\kern0pt}{\isacharparenleft}{\kern0pt}t{\isacharplus}{\kern0pt}{\isadigit{1}}{\isacharparenright}{\kern0pt}{\isacharcircum}{\kern0pt}m{\isacharparenright}{\kern0pt}{\isachardoublequoteclose}\ \isacommand{{\isachardot}{\kern0pt}}\isamarkupfalse%
\isanewline
\ \ \ \ \isacommand{then}\isamarkupfalse%
\ \isacommand{have}\isamarkupfalse%
\ s{\isacharunderscore}{\kern0pt}colored{\isacharcolon}{\kern0pt}\ {\isachardoublequoteopen}card\ {\isacharparenleft}{\kern0pt}cube\ m\ {\isacharparenleft}{\kern0pt}t{\isacharplus}{\kern0pt}{\isadigit{1}}{\isacharparenright}{\kern0pt}\ {\isasymrightarrow}\isactrlsub E\ {\isacharbraceleft}{\kern0pt}{\isachardot}{\kern0pt}{\isachardot}{\kern0pt}{\isacharless}{\kern0pt}r{\isacharbraceright}{\kern0pt}{\isacharparenright}{\kern0pt}\ {\isacharequal}{\kern0pt}\ s{\isachardoublequoteclose}\ \isacommand{unfolding}\isamarkupfalse%
\ s{\isacharunderscore}{\kern0pt}def\ \isacommand{by}\isamarkupfalse%
\ simp\isanewline
\ \ \ \ \isacommand{have}\isamarkupfalse%
\ {\isachardoublequoteopen}s\ {\isachargreater}{\kern0pt}\ {\isadigit{0}}{\isachardoublequoteclose}\ \isacommand{using}\isamarkupfalse%
\ assms{\isacharparenleft}{\kern0pt}{\isadigit{5}}{\isacharparenright}{\kern0pt}\ \isacommand{unfolding}\isamarkupfalse%
\ s{\isacharunderscore}{\kern0pt}def\ \isacommand{by}\isamarkupfalse%
\ simp\isanewline
\ \ \ \ \isacommand{then}\isamarkupfalse%
\ \isacommand{obtain}\isamarkupfalse%
\ {\isasymphi}\ \isakeyword{where}\ {\isasymphi}{\isacharunderscore}{\kern0pt}prop{\isacharcolon}{\kern0pt}\ {\isachardoublequoteopen}bij{\isacharunderscore}{\kern0pt}betw\ {\isasymphi}\ {\isacharparenleft}{\kern0pt}cube\ m\ {\isacharparenleft}{\kern0pt}t{\isacharplus}{\kern0pt}{\isadigit{1}}{\isacharparenright}{\kern0pt}\ {\isasymrightarrow}\isactrlsub E\ {\isacharbraceleft}{\kern0pt}{\isachardot}{\kern0pt}{\isachardot}{\kern0pt}{\isacharless}{\kern0pt}r{\isacharbraceright}{\kern0pt}{\isacharparenright}{\kern0pt}\ {\isacharbraceleft}{\kern0pt}{\isachardot}{\kern0pt}{\isachardot}{\kern0pt}{\isacharless}{\kern0pt}s{\isacharbraceright}{\kern0pt}{\isachardoublequoteclose}\ \isacommand{using}\isamarkupfalse%
\ ex{\isacharunderscore}{\kern0pt}bij{\isacharunderscore}{\kern0pt}betw{\isacharunderscore}{\kern0pt}nat{\isacharunderscore}{\kern0pt}finite{\isacharunderscore}{\kern0pt}{\isadigit{2}}{\isacharbrackleft}{\kern0pt}of\ {\isachardoublequoteopen}cube\ m\ {\isacharparenleft}{\kern0pt}t{\isacharplus}{\kern0pt}{\isadigit{1}}{\isacharparenright}{\kern0pt}\ {\isasymrightarrow}\isactrlsub E\ {\isacharbraceleft}{\kern0pt}{\isachardot}{\kern0pt}{\isachardot}{\kern0pt}{\isacharless}{\kern0pt}r{\isacharbraceright}{\kern0pt}{\isachardoublequoteclose}\ {\isachardoublequoteopen}s{\isachardoublequoteclose}{\isacharbrackright}{\kern0pt}\ s{\isacharunderscore}{\kern0pt}colored\ \isacommand{by}\isamarkupfalse%
\ blast\isanewline
\ \ \ \ \isacommand{define}\isamarkupfalse%
\ {\isasymchi}L{\isacharunderscore}{\kern0pt}s\ \isakeyword{where}\ {\isachardoublequoteopen}{\isasymchi}L{\isacharunderscore}{\kern0pt}s\ {\isasymequiv}\ {\isacharparenleft}{\kern0pt}{\isasymlambda}x{\isasymin}cube\ n\ {\isacharparenleft}{\kern0pt}t{\isacharplus}{\kern0pt}{\isadigit{1}}{\isacharparenright}{\kern0pt}{\isachardot}{\kern0pt}\ {\isasymphi}\ {\isacharparenleft}{\kern0pt}{\isasymchi}L\ x{\isacharparenright}{\kern0pt}{\isacharparenright}{\kern0pt}{\isachardoublequoteclose}\isanewline
\ \ \ \ \isacommand{have}\isamarkupfalse%
\ {\isachardoublequoteopen}{\isasymchi}L{\isacharunderscore}{\kern0pt}s\ {\isasymin}\ cube\ n\ {\isacharparenleft}{\kern0pt}t{\isacharplus}{\kern0pt}{\isadigit{1}}{\isacharparenright}{\kern0pt}\ {\isasymrightarrow}\isactrlsub E\ {\isacharbraceleft}{\kern0pt}{\isachardot}{\kern0pt}{\isachardot}{\kern0pt}{\isacharless}{\kern0pt}s{\isacharbraceright}{\kern0pt}{\isachardoublequoteclose}\isanewline
\ \ \ \ \isacommand{proof}\isamarkupfalse%
\isanewline
\ \ \ \ \ \ \isacommand{fix}\isamarkupfalse%
\ x\ \isacommand{assume}\isamarkupfalse%
\ a{\isacharcolon}{\kern0pt}\ {\isachardoublequoteopen}x\ {\isasymin}\ cube\ n\ {\isacharparenleft}{\kern0pt}t{\isacharplus}{\kern0pt}{\isadigit{1}}{\isacharparenright}{\kern0pt}{\isachardoublequoteclose}\isanewline
\ \ \ \ \ \ \isacommand{then}\isamarkupfalse%
\ \isacommand{have}\isamarkupfalse%
\ {\isachardoublequoteopen}{\isasymchi}L{\isacharunderscore}{\kern0pt}s\ x\ {\isacharequal}{\kern0pt}\ {\isasymphi}\ {\isacharparenleft}{\kern0pt}{\isasymchi}L\ x{\isacharparenright}{\kern0pt}{\isachardoublequoteclose}\ \isacommand{unfolding}\isamarkupfalse%
\ {\isasymchi}L{\isacharunderscore}{\kern0pt}s{\isacharunderscore}{\kern0pt}def\ \isacommand{by}\isamarkupfalse%
\ simp\isanewline
\ \ \ \ \ \ \isacommand{moreover}\isamarkupfalse%
\ \isacommand{have}\isamarkupfalse%
\ {\isachardoublequoteopen}{\isasymchi}L\ x\ {\isasymin}\ {\isacharparenleft}{\kern0pt}cube\ m\ {\isacharparenleft}{\kern0pt}t{\isacharplus}{\kern0pt}{\isadigit{1}}{\isacharparenright}{\kern0pt}\ {\isasymrightarrow}\isactrlsub E\ {\isacharbraceleft}{\kern0pt}{\isachardot}{\kern0pt}{\isachardot}{\kern0pt}{\isacharless}{\kern0pt}r{\isacharbraceright}{\kern0pt}{\isacharparenright}{\kern0pt}{\isachardoublequoteclose}\ \isacommand{using}\isamarkupfalse%
\ a\ {\isasymchi}L{\isacharunderscore}{\kern0pt}def\ {\isasymchi}L{\isacharunderscore}{\kern0pt}prop\ \isacommand{unfolding}\isamarkupfalse%
\ {\isasymchi}L{\isacharunderscore}{\kern0pt}def\ \isacommand{by}\isamarkupfalse%
\ blast\isanewline
\ \ \ \ \ \ \isacommand{moreover}\isamarkupfalse%
\ \isacommand{have}\isamarkupfalse%
\ {\isachardoublequoteopen}{\isasymphi}\ {\isacharparenleft}{\kern0pt}{\isasymchi}L\ x{\isacharparenright}{\kern0pt}\ {\isasymin}\ {\isacharbraceleft}{\kern0pt}{\isachardot}{\kern0pt}{\isachardot}{\kern0pt}{\isacharless}{\kern0pt}s{\isacharbraceright}{\kern0pt}{\isachardoublequoteclose}\ \isacommand{using}\isamarkupfalse%
\ {\isasymphi}{\isacharunderscore}{\kern0pt}prop\ calculation{\isacharparenleft}{\kern0pt}{\isadigit{2}}{\isacharparenright}{\kern0pt}\ \isacommand{unfolding}\isamarkupfalse%
\ bij{\isacharunderscore}{\kern0pt}betw{\isacharunderscore}{\kern0pt}def\ \isacommand{by}\isamarkupfalse%
\ blast\isanewline
\ \ \ \ \ \ \isacommand{ultimately}\isamarkupfalse%
\ \isacommand{show}\isamarkupfalse%
\ {\isachardoublequoteopen}{\isasymchi}L{\isacharunderscore}{\kern0pt}s\ x\ {\isasymin}\ {\isacharbraceleft}{\kern0pt}{\isachardot}{\kern0pt}{\isachardot}{\kern0pt}{\isacharless}{\kern0pt}s{\isacharbraceright}{\kern0pt}{\isachardoublequoteclose}\ \isacommand{by}\isamarkupfalse%
\ auto\isanewline
\ \ \ \ \isacommand{qed}\isamarkupfalse%
\ {\isacharparenleft}{\kern0pt}auto\ simp{\isacharcolon}{\kern0pt}\ {\isasymchi}L{\isacharunderscore}{\kern0pt}s{\isacharunderscore}{\kern0pt}def{\isacharparenright}{\kern0pt}\isanewline
\ \ \ \ \ \ \isanewline
\ \ \ \ \isacommand{then}\isamarkupfalse%
\ \isacommand{obtain}\isamarkupfalse%
\ L\ \isakeyword{where}\ L{\isacharunderscore}{\kern0pt}prop{\isacharcolon}{\kern0pt}\ {\isachardoublequoteopen}layered{\isacharunderscore}{\kern0pt}subspace\ L\ {\isadigit{1}}\ n\ t\ s\ {\isasymchi}L{\isacharunderscore}{\kern0pt}s{\isachardoublequoteclose}\ \isacommand{using}\isamarkupfalse%
\ line{\isacharunderscore}{\kern0pt}subspace{\isacharunderscore}{\kern0pt}s\ \isacommand{by}\isamarkupfalse%
\ blast\isanewline
\ \ \ \ \isacommand{define}\isamarkupfalse%
\ L{\isacharunderscore}{\kern0pt}line\ \isakeyword{where}\ {\isachardoublequoteopen}L{\isacharunderscore}{\kern0pt}line\ {\isasymequiv}\ {\isacharparenleft}{\kern0pt}{\isasymlambda}s{\isasymin}{\isacharbraceleft}{\kern0pt}{\isachardot}{\kern0pt}{\isachardot}{\kern0pt}{\isacharless}{\kern0pt}t{\isacharplus}{\kern0pt}{\isadigit{1}}{\isacharbraceright}{\kern0pt}{\isachardot}{\kern0pt}\ L\ {\isacharparenleft}{\kern0pt}SOME\ p{\isachardot}{\kern0pt}\ p{\isasymin}cube\ {\isadigit{1}}\ {\isacharparenleft}{\kern0pt}t{\isacharplus}{\kern0pt}{\isadigit{1}}{\isacharparenright}{\kern0pt}\ {\isasymand}\ p\ {\isadigit{0}}\ {\isacharequal}{\kern0pt}\ s{\isacharparenright}{\kern0pt}{\isacharparenright}{\kern0pt}{\isachardoublequoteclose}\isanewline
\ \ \ \ \isacommand{have}\isamarkupfalse%
\ L{\isacharunderscore}{\kern0pt}line{\isacharunderscore}{\kern0pt}base{\isacharunderscore}{\kern0pt}prop{\isacharcolon}{\kern0pt}\ {\isachardoublequoteopen}{\isasymforall}s\ {\isasymin}\ {\isacharbraceleft}{\kern0pt}{\isachardot}{\kern0pt}{\isachardot}{\kern0pt}{\isacharless}{\kern0pt}t{\isacharplus}{\kern0pt}{\isadigit{1}}{\isacharbraceright}{\kern0pt}{\isachardot}{\kern0pt}\ L{\isacharunderscore}{\kern0pt}line\ s\ {\isasymin}\ cube\ n\ {\isacharparenleft}{\kern0pt}t{\isacharplus}{\kern0pt}{\isadigit{1}}{\isacharparenright}{\kern0pt}{\isachardoublequoteclose}\ \isacommand{using}\isamarkupfalse%
\ assms{\isacharparenleft}{\kern0pt}{\isadigit{1}}{\isacharparenright}{\kern0pt}\ dim{\isadigit{1}}{\isacharunderscore}{\kern0pt}subspace{\isacharunderscore}{\kern0pt}is{\isacharunderscore}{\kern0pt}line{\isacharbrackleft}{\kern0pt}of\ {\isachardoublequoteopen}t{\isacharplus}{\kern0pt}{\isadigit{1}}{\isachardoublequoteclose}\ {\isachardoublequoteopen}L{\isachardoublequoteclose}\ {\isachardoublequoteopen}n{\isachardoublequoteclose}{\isacharbrackright}{\kern0pt}\ L{\isacharunderscore}{\kern0pt}prop\ line{\isacharunderscore}{\kern0pt}points{\isacharunderscore}{\kern0pt}in{\isacharunderscore}{\kern0pt}cube{\isacharbrackleft}{\kern0pt}of\ L{\isacharunderscore}{\kern0pt}line\ n\ {\isachardoublequoteopen}t{\isacharplus}{\kern0pt}{\isadigit{1}}{\isachardoublequoteclose}{\isacharbrackright}{\kern0pt}\ \isacommand{unfolding}\isamarkupfalse%
\ layered{\isacharunderscore}{\kern0pt}subspace{\isacharunderscore}{\kern0pt}def\ L{\isacharunderscore}{\kern0pt}line{\isacharunderscore}{\kern0pt}def\ \isacommand{by}\isamarkupfalse%
\ auto\isanewline
\isanewline
\isanewline
\ \ \ \ \isacommand{define}\isamarkupfalse%
\ {\isasymchi}S\ \isakeyword{where}\ {\isachardoublequoteopen}{\isasymchi}S\ {\isasymequiv}\ {\isacharparenleft}{\kern0pt}{\isasymlambda}y{\isasymin}cube\ m\ {\isacharparenleft}{\kern0pt}t{\isacharplus}{\kern0pt}{\isadigit{1}}{\isacharparenright}{\kern0pt}{\isachardot}{\kern0pt}\ {\isasymchi}\ {\isacharparenleft}{\kern0pt}join\ {\isacharparenleft}{\kern0pt}L{\isacharunderscore}{\kern0pt}line\ {\isadigit{0}}{\isacharparenright}{\kern0pt}\ y\ n\ m{\isacharparenright}{\kern0pt}{\isacharparenright}{\kern0pt}{\isachardoublequoteclose}\isanewline
\ \ \ \ \isacommand{have}\isamarkupfalse%
\ {\isachardoublequoteopen}{\isasymchi}S\ {\isasymin}\ {\isacharparenleft}{\kern0pt}cube\ m\ {\isacharparenleft}{\kern0pt}t\ {\isacharplus}{\kern0pt}\ {\isadigit{1}}{\isacharparenright}{\kern0pt}{\isacharparenright}{\kern0pt}\ {\isasymrightarrow}\isactrlsub E\ {\isacharbraceleft}{\kern0pt}{\isachardot}{\kern0pt}{\isachardot}{\kern0pt}{\isacharless}{\kern0pt}r{\isacharcolon}{\kern0pt}{\isacharcolon}{\kern0pt}nat{\isacharbraceright}{\kern0pt}{\isachardoublequoteclose}\isanewline
\ \ \ \ \isacommand{proof}\isamarkupfalse%
\isanewline
\ \ \ \ \ \isacommand{fix}\isamarkupfalse%
\ x\ \isacommand{assume}\isamarkupfalse%
\ a{\isacharcolon}{\kern0pt}\ {\isachardoublequoteopen}x\ {\isasymin}\ cube\ m\ {\isacharparenleft}{\kern0pt}t{\isacharplus}{\kern0pt}{\isadigit{1}}{\isacharparenright}{\kern0pt}{\isachardoublequoteclose}\isanewline
\ \ \ \ \ \isacommand{then}\isamarkupfalse%
\ \isacommand{have}\isamarkupfalse%
\ {\isachardoublequoteopen}{\isasymchi}S\ x\ {\isacharequal}{\kern0pt}\ {\isasymchi}\ {\isacharparenleft}{\kern0pt}join\ {\isacharparenleft}{\kern0pt}L{\isacharunderscore}{\kern0pt}line\ {\isadigit{0}}{\isacharparenright}{\kern0pt}\ x\ n\ m{\isacharparenright}{\kern0pt}{\isachardoublequoteclose}\ \isacommand{unfolding}\isamarkupfalse%
\ {\isasymchi}S{\isacharunderscore}{\kern0pt}def\ \isacommand{by}\isamarkupfalse%
\ simp\isanewline
\ \ \ \ \ \isacommand{moreover}\isamarkupfalse%
\ \isacommand{have}\isamarkupfalse%
\ {\isachardoublequoteopen}L{\isacharunderscore}{\kern0pt}line\ {\isadigit{0}}\ {\isacharequal}{\kern0pt}\ L\ {\isacharparenleft}{\kern0pt}SOME\ p{\isachardot}{\kern0pt}\ p{\isasymin}cube\ {\isadigit{1}}\ {\isacharparenleft}{\kern0pt}t{\isacharplus}{\kern0pt}{\isadigit{1}}{\isacharparenright}{\kern0pt}\ {\isasymand}\ p\ {\isadigit{0}}\ {\isacharequal}{\kern0pt}\ {\isadigit{0}}{\isacharparenright}{\kern0pt}{\isachardoublequoteclose}\ \isacommand{using}\isamarkupfalse%
\ L{\isacharunderscore}{\kern0pt}prop\ assms{\isacharparenleft}{\kern0pt}{\isadigit{1}}{\isacharparenright}{\kern0pt}\ \isacommand{unfolding}\isamarkupfalse%
\ L{\isacharunderscore}{\kern0pt}line{\isacharunderscore}{\kern0pt}def\ \isacommand{by}\isamarkupfalse%
\ simp\isanewline
\ \ \ \ \ \isacommand{moreover}\isamarkupfalse%
\ \isacommand{have}\isamarkupfalse%
\ {\isachardoublequoteopen}{\isacharparenleft}{\kern0pt}SOME\ p{\isachardot}{\kern0pt}\ p{\isasymin}cube\ {\isadigit{1}}\ {\isacharparenleft}{\kern0pt}t{\isacharplus}{\kern0pt}{\isadigit{1}}{\isacharparenright}{\kern0pt}\ {\isasymand}\ p\ {\isadigit{0}}\ {\isacharequal}{\kern0pt}\ {\isadigit{0}}{\isacharparenright}{\kern0pt}\ {\isasymin}\ cube\ {\isadigit{1}}\ {\isacharparenleft}{\kern0pt}t{\isacharplus}{\kern0pt}{\isadigit{1}}{\isacharparenright}{\kern0pt}{\isachardoublequoteclose}\ \isacommand{using}\isamarkupfalse%
\ cube{\isacharunderscore}{\kern0pt}props{\isacharparenleft}{\kern0pt}{\isadigit{4}}{\isacharparenright}{\kern0pt}{\isacharbrackleft}{\kern0pt}of\ {\isachardoublequoteopen}t{\isacharplus}{\kern0pt}{\isadigit{1}}{\isachardoublequoteclose}{\isacharbrackright}{\kern0pt}\ \isacommand{using}\isamarkupfalse%
\ assms{\isacharparenleft}{\kern0pt}{\isadigit{1}}{\isacharparenright}{\kern0pt}\ \isacommand{by}\isamarkupfalse%
\ auto\ \isanewline
\ \ \ \ \ \isacommand{moreover}\isamarkupfalse%
\ \isacommand{have}\isamarkupfalse%
\ {\isachardoublequoteopen}L\ {\isasymin}\ cube\ {\isadigit{1}}\ {\isacharparenleft}{\kern0pt}t{\isacharplus}{\kern0pt}{\isadigit{1}}{\isacharparenright}{\kern0pt}\ {\isasymrightarrow}\isactrlsub E\ cube\ n\ {\isacharparenleft}{\kern0pt}t{\isacharplus}{\kern0pt}{\isadigit{1}}{\isacharparenright}{\kern0pt}{\isachardoublequoteclose}\ \isacommand{using}\isamarkupfalse%
\ L{\isacharunderscore}{\kern0pt}prop\ \isacommand{unfolding}\isamarkupfalse%
\ layered{\isacharunderscore}{\kern0pt}subspace{\isacharunderscore}{\kern0pt}def\ is{\isacharunderscore}{\kern0pt}subspace{\isacharunderscore}{\kern0pt}def\ \isacommand{by}\isamarkupfalse%
\ blast\isanewline
\ \ \ \ \ \isacommand{moreover}\isamarkupfalse%
\ \isacommand{have}\isamarkupfalse%
\ {\isachardoublequoteopen}L\ {\isacharparenleft}{\kern0pt}SOME\ p{\isachardot}{\kern0pt}\ p{\isasymin}cube\ {\isadigit{1}}\ {\isacharparenleft}{\kern0pt}t{\isacharplus}{\kern0pt}{\isadigit{1}}{\isacharparenright}{\kern0pt}\ {\isasymand}\ p\ {\isadigit{0}}\ {\isacharequal}{\kern0pt}\ {\isadigit{0}}{\isacharparenright}{\kern0pt}\ {\isasymin}\ cube\ n\ {\isacharparenleft}{\kern0pt}t{\isacharplus}{\kern0pt}{\isadigit{1}}{\isacharparenright}{\kern0pt}{\isachardoublequoteclose}\ \isacommand{using}\isamarkupfalse%
\ calculation\ {\isacharparenleft}{\kern0pt}{\isadigit{3}}{\isacharcomma}{\kern0pt}{\isadigit{4}}{\isacharparenright}{\kern0pt}\ \isacommand{unfolding}\isamarkupfalse%
\ cube{\isacharunderscore}{\kern0pt}def\ \isacommand{by}\isamarkupfalse%
\ auto\isanewline
\ \ \ \ \ \isacommand{moreover}\isamarkupfalse%
\ \isacommand{have}\isamarkupfalse%
\ {\isachardoublequoteopen}join\ {\isacharparenleft}{\kern0pt}L{\isacharunderscore}{\kern0pt}line\ {\isadigit{0}}{\isacharparenright}{\kern0pt}\ x\ n\ m\ {\isasymin}\ cube\ {\isacharparenleft}{\kern0pt}n\ {\isacharplus}{\kern0pt}\ m{\isacharparenright}{\kern0pt}\ {\isacharparenleft}{\kern0pt}t{\isacharplus}{\kern0pt}{\isadigit{1}}{\isacharparenright}{\kern0pt}{\isachardoublequoteclose}\ \isacommand{using}\isamarkupfalse%
\ join{\isacharunderscore}{\kern0pt}cubes\ a\ calculation{\isacharparenleft}{\kern0pt}{\isadigit{2}}{\isacharcomma}{\kern0pt}\ {\isadigit{5}}{\isacharparenright}{\kern0pt}\ \isacommand{by}\isamarkupfalse%
\ auto\isanewline
\ \ \ \ \ \isacommand{ultimately}\isamarkupfalse%
\ \isacommand{show}\isamarkupfalse%
\ {\isachardoublequoteopen}{\isasymchi}S\ x\ {\isasymin}\ {\isacharbraceleft}{\kern0pt}{\isachardot}{\kern0pt}{\isachardot}{\kern0pt}{\isacharless}{\kern0pt}r{\isacharbraceright}{\kern0pt}{\isachardoublequoteclose}\ \isacommand{using}\isamarkupfalse%
\ A\ a\ \isacommand{by}\isamarkupfalse%
\ fastforce\isanewline
\ \ \ \ \isacommand{qed}\isamarkupfalse%
\ {\isacharparenleft}{\kern0pt}auto\ simp{\isacharcolon}{\kern0pt}\ {\isasymchi}S{\isacharunderscore}{\kern0pt}def{\isacharparenright}{\kern0pt}\isanewline
\ \ \ \ \ \ \isanewline
\ \ \ \ \isacommand{then}\isamarkupfalse%
\ \isacommand{obtain}\isamarkupfalse%
\ S\ \isakeyword{where}\ S{\isacharunderscore}{\kern0pt}prop{\isacharcolon}{\kern0pt}\ {\isachardoublequoteopen}layered{\isacharunderscore}{\kern0pt}subspace\ S\ k\ m\ t\ r\ {\isasymchi}S{\isachardoublequoteclose}\ \isacommand{using}\isamarkupfalse%
\ assms{\isacharparenleft}{\kern0pt}{\isadigit{4}}{\isacharparenright}{\kern0pt}\ m{\isacharunderscore}{\kern0pt}props\ \isacommand{by}\isamarkupfalse%
\ blast%
\begin{isamarkuptext}%
04.07.2022 Having obtained our subspaces S and L, we define our new subspace very straightforwardly. Namely T = L \times S. Of course, since our way of representing tuples is through function sets C(n, t), we need an appropriate operator that mirrors \times for function sets. We call this join (and define it for elements of a FuncSet)%
\end{isamarkuptext}\isamarkuptrue%
\ \ \ \ \isacommand{define}\isamarkupfalse%
\ imT\ \isakeyword{where}\ {\isachardoublequoteopen}imT\ {\isasymequiv}\ {\isacharbraceleft}{\kern0pt}join\ {\isacharparenleft}{\kern0pt}L{\isacharunderscore}{\kern0pt}line\ i{\isacharparenright}{\kern0pt}\ s\ n\ m\ {\isacharbar}{\kern0pt}\ i\ s\ {\isachardot}{\kern0pt}\ i\ {\isasymin}\ {\isacharbraceleft}{\kern0pt}{\isachardot}{\kern0pt}{\isachardot}{\kern0pt}{\isacharless}{\kern0pt}t{\isacharplus}{\kern0pt}{\isadigit{1}}{\isacharbraceright}{\kern0pt}\ {\isasymand}\ s\ {\isasymin}\ S\ {\isacharbackquote}{\kern0pt}\ {\isacharparenleft}{\kern0pt}cube\ k\ {\isacharparenleft}{\kern0pt}t{\isacharplus}{\kern0pt}{\isadigit{1}}{\isacharparenright}{\kern0pt}{\isacharparenright}{\kern0pt}{\isacharbraceright}{\kern0pt}{\isachardoublequoteclose}\ \isanewline
\ \ \ \ \isacommand{define}\isamarkupfalse%
\ T{\isacharprime}{\kern0pt}\ \isakeyword{where}\ {\isachardoublequoteopen}T{\isacharprime}{\kern0pt}\ {\isasymequiv}\ {\isacharparenleft}{\kern0pt}{\isasymlambda}x\ {\isasymin}\ cube\ {\isadigit{1}}\ {\isacharparenleft}{\kern0pt}t{\isacharplus}{\kern0pt}{\isadigit{1}}{\isacharparenright}{\kern0pt}{\isachardot}{\kern0pt}\ {\isasymlambda}y\ {\isasymin}\ cube\ k\ {\isacharparenleft}{\kern0pt}t{\isacharplus}{\kern0pt}{\isadigit{1}}{\isacharparenright}{\kern0pt}{\isachardot}{\kern0pt}\ join\ {\isacharparenleft}{\kern0pt}L{\isacharunderscore}{\kern0pt}line\ {\isacharparenleft}{\kern0pt}x\ {\isadigit{0}}{\isacharparenright}{\kern0pt}{\isacharparenright}{\kern0pt}\ {\isacharparenleft}{\kern0pt}S\ y{\isacharparenright}{\kern0pt}\ n\ m{\isacharparenright}{\kern0pt}{\isachardoublequoteclose}\isanewline
\ \ \ \ \isacommand{have}\isamarkupfalse%
\ T{\isacharprime}{\kern0pt}{\isacharunderscore}{\kern0pt}prop{\isacharcolon}{\kern0pt}\ {\isachardoublequoteopen}T{\isacharprime}{\kern0pt}\ {\isasymin}\ cube\ {\isadigit{1}}\ {\isacharparenleft}{\kern0pt}t{\isacharplus}{\kern0pt}{\isadigit{1}}{\isacharparenright}{\kern0pt}\ {\isasymrightarrow}\isactrlsub E\ cube\ k\ {\isacharparenleft}{\kern0pt}t{\isacharplus}{\kern0pt}{\isadigit{1}}{\isacharparenright}{\kern0pt}\ {\isasymrightarrow}\isactrlsub E\ cube\ {\isacharparenleft}{\kern0pt}n\ {\isacharplus}{\kern0pt}\ m{\isacharparenright}{\kern0pt}\ {\isacharparenleft}{\kern0pt}t{\isacharplus}{\kern0pt}{\isadigit{1}}{\isacharparenright}{\kern0pt}{\isachardoublequoteclose}\isanewline
\ \ \ \ \isacommand{proof}\isamarkupfalse%
\isanewline
\ \ \ \ \ \ \isacommand{fix}\isamarkupfalse%
\ x\ \isacommand{assume}\isamarkupfalse%
\ a{\isacharcolon}{\kern0pt}\ {\isachardoublequoteopen}x\ {\isasymin}\ cube\ {\isadigit{1}}\ {\isacharparenleft}{\kern0pt}t{\isacharplus}{\kern0pt}{\isadigit{1}}{\isacharparenright}{\kern0pt}{\isachardoublequoteclose}\isanewline
\ \ \ \ \ \ \isacommand{show}\isamarkupfalse%
\ {\isachardoublequoteopen}T{\isacharprime}{\kern0pt}\ x\ {\isasymin}\ cube\ k\ {\isacharparenleft}{\kern0pt}t\ {\isacharplus}{\kern0pt}\ {\isadigit{1}}{\isacharparenright}{\kern0pt}\ {\isasymrightarrow}\isactrlsub E\ cube\ {\isacharparenleft}{\kern0pt}n\ {\isacharplus}{\kern0pt}\ m{\isacharparenright}{\kern0pt}\ {\isacharparenleft}{\kern0pt}t\ {\isacharplus}{\kern0pt}\ {\isadigit{1}}{\isacharparenright}{\kern0pt}{\isachardoublequoteclose}\isanewline
\ \ \ \ \ \ \isacommand{proof}\isamarkupfalse%
\isanewline
\ \ \ \ \ \ \ \ \isacommand{fix}\isamarkupfalse%
\ y\ \isacommand{assume}\isamarkupfalse%
\ b{\isacharcolon}{\kern0pt}\ {\isachardoublequoteopen}y\ {\isasymin}\ cube\ k\ {\isacharparenleft}{\kern0pt}t{\isacharplus}{\kern0pt}{\isadigit{1}}{\isacharparenright}{\kern0pt}{\isachardoublequoteclose}\isanewline
\ \ \ \ \ \ \ \ \isacommand{then}\isamarkupfalse%
\ \isacommand{have}\isamarkupfalse%
\ {\isachardoublequoteopen}T{\isacharprime}{\kern0pt}\ x\ y\ {\isacharequal}{\kern0pt}\ join\ {\isacharparenleft}{\kern0pt}L{\isacharunderscore}{\kern0pt}line\ {\isacharparenleft}{\kern0pt}x\ {\isadigit{0}}{\isacharparenright}{\kern0pt}{\isacharparenright}{\kern0pt}\ {\isacharparenleft}{\kern0pt}S\ y{\isacharparenright}{\kern0pt}\ n\ m{\isachardoublequoteclose}\ \isacommand{using}\isamarkupfalse%
\ a\ \isacommand{unfolding}\isamarkupfalse%
\ T{\isacharprime}{\kern0pt}{\isacharunderscore}{\kern0pt}def\ \isacommand{by}\isamarkupfalse%
\ simp\isanewline
\ \ \ \ \ \ \ \ \isacommand{moreover}\isamarkupfalse%
\ \isacommand{have}\isamarkupfalse%
\ {\isachardoublequoteopen}L{\isacharunderscore}{\kern0pt}line\ {\isacharparenleft}{\kern0pt}x\ {\isadigit{0}}{\isacharparenright}{\kern0pt}\ {\isasymin}\ cube\ n\ {\isacharparenleft}{\kern0pt}t{\isacharplus}{\kern0pt}{\isadigit{1}}{\isacharparenright}{\kern0pt}{\isachardoublequoteclose}\ \isacommand{using}\isamarkupfalse%
\ a\ L{\isacharunderscore}{\kern0pt}line{\isacharunderscore}{\kern0pt}base{\isacharunderscore}{\kern0pt}prop\ \isacommand{unfolding}\isamarkupfalse%
\ cube{\isacharunderscore}{\kern0pt}def\ \isacommand{by}\isamarkupfalse%
\ blast\isanewline
\ \ \ \ \ \ \ \ \isacommand{moreover}\isamarkupfalse%
\ \isacommand{have}\isamarkupfalse%
\ {\isachardoublequoteopen}S\ y\ {\isasymin}\ cube\ m\ {\isacharparenleft}{\kern0pt}t{\isacharplus}{\kern0pt}{\isadigit{1}}{\isacharparenright}{\kern0pt}{\isachardoublequoteclose}\ \isacommand{using}\isamarkupfalse%
\ subspace{\isacharunderscore}{\kern0pt}elems{\isacharunderscore}{\kern0pt}embed{\isacharbrackleft}{\kern0pt}of\ {\isachardoublequoteopen}S{\isachardoublequoteclose}\ {\isachardoublequoteopen}k{\isachardoublequoteclose}\ {\isachardoublequoteopen}m{\isachardoublequoteclose}\ {\isachardoublequoteopen}t{\isacharplus}{\kern0pt}{\isadigit{1}}{\isachardoublequoteclose}{\isacharbrackright}{\kern0pt}\ S{\isacharunderscore}{\kern0pt}prop\ b\ \ \isacommand{unfolding}\isamarkupfalse%
\ layered{\isacharunderscore}{\kern0pt}subspace{\isacharunderscore}{\kern0pt}def\ \isacommand{by}\isamarkupfalse%
\ blast\isanewline
\ \ \ \ \ \ \ \ \isacommand{ultimately}\isamarkupfalse%
\ \isacommand{show}\isamarkupfalse%
\ {\isachardoublequoteopen}T{\isacharprime}{\kern0pt}\ x\ y\ {\isasymin}\ cube\ {\isacharparenleft}{\kern0pt}n\ {\isacharplus}{\kern0pt}\ m{\isacharparenright}{\kern0pt}\ {\isacharparenleft}{\kern0pt}t\ {\isacharplus}{\kern0pt}\ {\isadigit{1}}{\isacharparenright}{\kern0pt}{\isachardoublequoteclose}\ \isacommand{using}\isamarkupfalse%
\ join{\isacharunderscore}{\kern0pt}cubes\ \isacommand{by}\isamarkupfalse%
\ presburger\isanewline
\ \ \ \ \ \ \isacommand{next}\isamarkupfalse%
\isanewline
\ \ \ \ \ \ \isacommand{qed}\isamarkupfalse%
\ {\isacharparenleft}{\kern0pt}unfold\ T{\isacharprime}{\kern0pt}{\isacharunderscore}{\kern0pt}def{\isacharsemicolon}{\kern0pt}\ use\ a\ \isakeyword{in}\ simp{\isacharparenright}{\kern0pt}\isanewline
\ \ \ \ \isacommand{qed}\isamarkupfalse%
\ {\isacharparenleft}{\kern0pt}auto\ simp{\isacharcolon}{\kern0pt}\ T{\isacharprime}{\kern0pt}{\isacharunderscore}{\kern0pt}def{\isacharparenright}{\kern0pt}\isanewline
\isanewline
\ \ \ \ \isacommand{define}\isamarkupfalse%
\ T\ \isakeyword{where}\ {\isachardoublequoteopen}T\ {\isasymequiv}\ {\isacharparenleft}{\kern0pt}{\isasymlambda}x\ {\isasymin}\ cube\ {\isacharparenleft}{\kern0pt}k\ {\isacharplus}{\kern0pt}\ {\isadigit{1}}{\isacharparenright}{\kern0pt}\ {\isacharparenleft}{\kern0pt}t{\isacharplus}{\kern0pt}{\isadigit{1}}{\isacharparenright}{\kern0pt}{\isachardot}{\kern0pt}\ T{\isacharprime}{\kern0pt}\ {\isacharparenleft}{\kern0pt}{\isasymlambda}y\ {\isasymin}\ {\isacharbraceleft}{\kern0pt}{\isachardot}{\kern0pt}{\isachardot}{\kern0pt}{\isacharless}{\kern0pt}{\isadigit{1}}{\isacharbraceright}{\kern0pt}{\isachardot}{\kern0pt}\ x\ y{\isacharparenright}{\kern0pt}\ {\isacharparenleft}{\kern0pt}{\isasymlambda}y\ {\isasymin}\ {\isacharbraceleft}{\kern0pt}{\isachardot}{\kern0pt}{\isachardot}{\kern0pt}{\isacharless}{\kern0pt}k{\isacharbraceright}{\kern0pt}{\isachardot}{\kern0pt}\ x\ {\isacharparenleft}{\kern0pt}y\ {\isacharplus}{\kern0pt}\ {\isadigit{1}}{\isacharparenright}{\kern0pt}{\isacharparenright}{\kern0pt}{\isacharparenright}{\kern0pt}{\isachardoublequoteclose}\isanewline
\ \ \ \ \isacommand{have}\isamarkupfalse%
\ T{\isacharunderscore}{\kern0pt}prop{\isacharcolon}{\kern0pt}\ {\isachardoublequoteopen}T\ {\isasymin}\ cube\ {\isacharparenleft}{\kern0pt}k{\isacharplus}{\kern0pt}{\isadigit{1}}{\isacharparenright}{\kern0pt}\ {\isacharparenleft}{\kern0pt}t{\isacharplus}{\kern0pt}{\isadigit{1}}{\isacharparenright}{\kern0pt}\ {\isasymrightarrow}\isactrlsub E\ cube\ {\isacharparenleft}{\kern0pt}n{\isacharplus}{\kern0pt}m{\isacharparenright}{\kern0pt}\ {\isacharparenleft}{\kern0pt}t{\isacharplus}{\kern0pt}{\isadigit{1}}{\isacharparenright}{\kern0pt}{\isachardoublequoteclose}\isanewline
\ \ \ \ \isacommand{proof}\isamarkupfalse%
\isanewline
\ \ \ \ \ \ \isacommand{fix}\isamarkupfalse%
\ x\ \isacommand{assume}\isamarkupfalse%
\ a{\isacharcolon}{\kern0pt}\ {\isachardoublequoteopen}x\ {\isasymin}\ cube\ {\isacharparenleft}{\kern0pt}k{\isacharplus}{\kern0pt}{\isadigit{1}}{\isacharparenright}{\kern0pt}\ {\isacharparenleft}{\kern0pt}t{\isacharplus}{\kern0pt}{\isadigit{1}}{\isacharparenright}{\kern0pt}{\isachardoublequoteclose}\isanewline
\ \ \ \ \ \ \isacommand{then}\isamarkupfalse%
\ \isacommand{have}\isamarkupfalse%
\ {\isachardoublequoteopen}T\ x\ {\isacharequal}{\kern0pt}\ T{\isacharprime}{\kern0pt}\ {\isacharparenleft}{\kern0pt}{\isasymlambda}y\ {\isasymin}\ {\isacharbraceleft}{\kern0pt}{\isachardot}{\kern0pt}{\isachardot}{\kern0pt}{\isacharless}{\kern0pt}{\isadigit{1}}{\isacharbraceright}{\kern0pt}{\isachardot}{\kern0pt}\ x\ y{\isacharparenright}{\kern0pt}\ {\isacharparenleft}{\kern0pt}{\isasymlambda}y\ {\isasymin}\ {\isacharbraceleft}{\kern0pt}{\isachardot}{\kern0pt}{\isachardot}{\kern0pt}{\isacharless}{\kern0pt}k{\isacharbraceright}{\kern0pt}{\isachardot}{\kern0pt}\ x\ {\isacharparenleft}{\kern0pt}y\ {\isacharplus}{\kern0pt}\ {\isadigit{1}}{\isacharparenright}{\kern0pt}{\isacharparenright}{\kern0pt}{\isachardoublequoteclose}\ \isacommand{unfolding}\isamarkupfalse%
\ T{\isacharunderscore}{\kern0pt}def\ \isacommand{by}\isamarkupfalse%
\ auto\isanewline
\ \ \ \ \ \ \isacommand{moreover}\isamarkupfalse%
\ \isacommand{have}\isamarkupfalse%
\ {\isachardoublequoteopen}{\isacharparenleft}{\kern0pt}{\isasymlambda}y\ {\isasymin}\ {\isacharbraceleft}{\kern0pt}{\isachardot}{\kern0pt}{\isachardot}{\kern0pt}{\isacharless}{\kern0pt}{\isadigit{1}}{\isacharbraceright}{\kern0pt}{\isachardot}{\kern0pt}\ x\ y{\isacharparenright}{\kern0pt}\ {\isasymin}\ cube\ {\isadigit{1}}\ {\isacharparenleft}{\kern0pt}t{\isacharplus}{\kern0pt}{\isadigit{1}}{\isacharparenright}{\kern0pt}{\isachardoublequoteclose}\ \isacommand{using}\isamarkupfalse%
\ a\ \isacommand{unfolding}\isamarkupfalse%
\ cube{\isacharunderscore}{\kern0pt}def\ \isacommand{by}\isamarkupfalse%
\ auto\isanewline
\ \ \ \ \ \ \isacommand{moreover}\isamarkupfalse%
\ \isacommand{have}\isamarkupfalse%
\ {\isachardoublequoteopen}{\isacharparenleft}{\kern0pt}{\isasymlambda}y\ {\isasymin}\ {\isacharbraceleft}{\kern0pt}{\isachardot}{\kern0pt}{\isachardot}{\kern0pt}{\isacharless}{\kern0pt}k{\isacharbraceright}{\kern0pt}{\isachardot}{\kern0pt}\ x\ {\isacharparenleft}{\kern0pt}y\ {\isacharplus}{\kern0pt}\ {\isadigit{1}}{\isacharparenright}{\kern0pt}{\isacharparenright}{\kern0pt}\ {\isasymin}\ cube\ k\ {\isacharparenleft}{\kern0pt}t{\isacharplus}{\kern0pt}{\isadigit{1}}{\isacharparenright}{\kern0pt}{\isachardoublequoteclose}\ \isacommand{using}\isamarkupfalse%
\ a\ \isacommand{unfolding}\isamarkupfalse%
\ cube{\isacharunderscore}{\kern0pt}def\ \isacommand{by}\isamarkupfalse%
\ auto\isanewline
\ \ \ \ \ \ \isacommand{moreover}\isamarkupfalse%
\ \isacommand{have}\isamarkupfalse%
\ {\isachardoublequoteopen}T{\isacharprime}{\kern0pt}\ {\isacharparenleft}{\kern0pt}{\isasymlambda}y\ {\isasymin}\ {\isacharbraceleft}{\kern0pt}{\isachardot}{\kern0pt}{\isachardot}{\kern0pt}{\isacharless}{\kern0pt}{\isadigit{1}}{\isacharbraceright}{\kern0pt}{\isachardot}{\kern0pt}\ x\ y{\isacharparenright}{\kern0pt}\ {\isacharparenleft}{\kern0pt}{\isasymlambda}y\ {\isasymin}\ {\isacharbraceleft}{\kern0pt}{\isachardot}{\kern0pt}{\isachardot}{\kern0pt}{\isacharless}{\kern0pt}k{\isacharbraceright}{\kern0pt}{\isachardot}{\kern0pt}\ x\ {\isacharparenleft}{\kern0pt}y\ {\isacharplus}{\kern0pt}\ {\isadigit{1}}{\isacharparenright}{\kern0pt}{\isacharparenright}{\kern0pt}\ {\isasymin}\ cube\ {\isacharparenleft}{\kern0pt}n\ {\isacharplus}{\kern0pt}\ m{\isacharparenright}{\kern0pt}\ {\isacharparenleft}{\kern0pt}t{\isacharplus}{\kern0pt}{\isadigit{1}}{\isacharparenright}{\kern0pt}{\isachardoublequoteclose}\ \isacommand{using}\isamarkupfalse%
\ T{\isacharprime}{\kern0pt}{\isacharunderscore}{\kern0pt}prop\ calculation\ \isacommand{unfolding}\isamarkupfalse%
\ T{\isacharprime}{\kern0pt}{\isacharunderscore}{\kern0pt}def\ \isacommand{by}\isamarkupfalse%
\ blast\isanewline
\ \ \ \ \ \ \isacommand{ultimately}\isamarkupfalse%
\ \isacommand{show}\isamarkupfalse%
\ {\isachardoublequoteopen}T\ x\ {\isasymin}\ cube\ {\isacharparenleft}{\kern0pt}n\ {\isacharplus}{\kern0pt}\ m{\isacharparenright}{\kern0pt}\ {\isacharparenleft}{\kern0pt}t{\isacharplus}{\kern0pt}{\isadigit{1}}{\isacharparenright}{\kern0pt}{\isachardoublequoteclose}\ \isacommand{by}\isamarkupfalse%
\ argo\isanewline
\ \ \ \ \isacommand{qed}\isamarkupfalse%
\ {\isacharparenleft}{\kern0pt}auto\ simp{\isacharcolon}{\kern0pt}\ T{\isacharunderscore}{\kern0pt}def{\isacharparenright}{\kern0pt}\isanewline
\isanewline
\ \ \ \ \isacommand{have}\isamarkupfalse%
\ im{\isacharunderscore}{\kern0pt}T{\isacharunderscore}{\kern0pt}eq{\isacharunderscore}{\kern0pt}imT{\isacharcolon}{\kern0pt}\ {\isachardoublequoteopen}T\ {\isacharbackquote}{\kern0pt}\ cube\ {\isacharparenleft}{\kern0pt}k{\isacharplus}{\kern0pt}{\isadigit{1}}{\isacharparenright}{\kern0pt}\ {\isacharparenleft}{\kern0pt}t{\isacharplus}{\kern0pt}{\isadigit{1}}{\isacharparenright}{\kern0pt}\ {\isacharequal}{\kern0pt}\ imT{\isachardoublequoteclose}\isanewline
\ \ \ \ \isacommand{proof}\isamarkupfalse%
\isanewline
\ \ \ \ \ \ \isacommand{show}\isamarkupfalse%
\ {\isachardoublequoteopen}T\ {\isacharbackquote}{\kern0pt}\ cube\ {\isacharparenleft}{\kern0pt}k\ {\isacharplus}{\kern0pt}\ {\isadigit{1}}{\isacharparenright}{\kern0pt}\ {\isacharparenleft}{\kern0pt}t\ {\isacharplus}{\kern0pt}\ {\isadigit{1}}{\isacharparenright}{\kern0pt}\ {\isasymsubseteq}\ imT{\isachardoublequoteclose}\isanewline
\ \ \ \ \ \ \isacommand{proof}\isamarkupfalse%
\isanewline
\ \ \ \ \ \ \ \ \isacommand{fix}\isamarkupfalse%
\ x\ \isacommand{assume}\isamarkupfalse%
\ {\isachardoublequoteopen}x\ {\isasymin}\ T\ {\isacharbackquote}{\kern0pt}\ cube\ {\isacharparenleft}{\kern0pt}k{\isacharplus}{\kern0pt}{\isadigit{1}}{\isacharparenright}{\kern0pt}\ {\isacharparenleft}{\kern0pt}t{\isacharplus}{\kern0pt}{\isadigit{1}}{\isacharparenright}{\kern0pt}{\isachardoublequoteclose}\isanewline
\ \ \ \ \ \ \ \ \isacommand{then}\isamarkupfalse%
\ \isacommand{obtain}\isamarkupfalse%
\ y\ \isakeyword{where}\ y{\isacharunderscore}{\kern0pt}prop{\isacharcolon}{\kern0pt}\ {\isachardoublequoteopen}y\ {\isasymin}\ cube\ {\isacharparenleft}{\kern0pt}k{\isacharplus}{\kern0pt}{\isadigit{1}}{\isacharparenright}{\kern0pt}\ {\isacharparenleft}{\kern0pt}t{\isacharplus}{\kern0pt}{\isadigit{1}}{\isacharparenright}{\kern0pt}\ {\isasymand}\ x\ {\isacharequal}{\kern0pt}\ T\ y{\isachardoublequoteclose}\ \isacommand{by}\isamarkupfalse%
\ blast\isanewline
\ \ \ \ \ \ \ \ \isacommand{then}\isamarkupfalse%
\ \isacommand{have}\isamarkupfalse%
\ {\isachardoublequoteopen}T\ y\ {\isacharequal}{\kern0pt}\ T{\isacharprime}{\kern0pt}\ {\isacharparenleft}{\kern0pt}{\isasymlambda}i\ {\isasymin}\ {\isacharbraceleft}{\kern0pt}{\isachardot}{\kern0pt}{\isachardot}{\kern0pt}{\isacharless}{\kern0pt}{\isadigit{1}}{\isacharbraceright}{\kern0pt}{\isachardot}{\kern0pt}\ y\ i{\isacharparenright}{\kern0pt}\ {\isacharparenleft}{\kern0pt}{\isasymlambda}i\ {\isasymin}\ {\isacharbraceleft}{\kern0pt}{\isachardot}{\kern0pt}{\isachardot}{\kern0pt}{\isacharless}{\kern0pt}k{\isacharbraceright}{\kern0pt}{\isachardot}{\kern0pt}\ y\ {\isacharparenleft}{\kern0pt}i\ {\isacharplus}{\kern0pt}\ {\isadigit{1}}{\isacharparenright}{\kern0pt}{\isacharparenright}{\kern0pt}{\isachardoublequoteclose}\ \isacommand{unfolding}\isamarkupfalse%
\ T{\isacharunderscore}{\kern0pt}def\ \isacommand{by}\isamarkupfalse%
\ simp\isanewline
\ \ \ \ \ \ \ \ \isacommand{moreover}\isamarkupfalse%
\ \isacommand{have}\isamarkupfalse%
\ {\isachardoublequoteopen}{\isacharparenleft}{\kern0pt}{\isasymlambda}i\ {\isasymin}\ {\isacharbraceleft}{\kern0pt}{\isachardot}{\kern0pt}{\isachardot}{\kern0pt}{\isacharless}{\kern0pt}{\isadigit{1}}{\isacharbraceright}{\kern0pt}{\isachardot}{\kern0pt}\ y\ i{\isacharparenright}{\kern0pt}\ {\isasymin}\ cube\ {\isadigit{1}}\ {\isacharparenleft}{\kern0pt}t{\isacharplus}{\kern0pt}{\isadigit{1}}{\isacharparenright}{\kern0pt}{\isachardoublequoteclose}\ \isacommand{using}\isamarkupfalse%
\ y{\isacharunderscore}{\kern0pt}prop\ \isacommand{unfolding}\isamarkupfalse%
\ cube{\isacharunderscore}{\kern0pt}def\ \isacommand{by}\isamarkupfalse%
\ auto\isanewline
\ \ \ \ \ \ \ \ \isacommand{moreover}\isamarkupfalse%
\ \isacommand{have}\isamarkupfalse%
\ {\isachardoublequoteopen}{\isacharparenleft}{\kern0pt}{\isasymlambda}i\ {\isasymin}\ {\isacharbraceleft}{\kern0pt}{\isachardot}{\kern0pt}{\isachardot}{\kern0pt}{\isacharless}{\kern0pt}k{\isacharbraceright}{\kern0pt}{\isachardot}{\kern0pt}\ y\ {\isacharparenleft}{\kern0pt}i\ {\isacharplus}{\kern0pt}\ {\isadigit{1}}{\isacharparenright}{\kern0pt}{\isacharparenright}{\kern0pt}\ {\isasymin}\ cube\ k\ {\isacharparenleft}{\kern0pt}t{\isacharplus}{\kern0pt}{\isadigit{1}}{\isacharparenright}{\kern0pt}{\isachardoublequoteclose}\ \isacommand{using}\isamarkupfalse%
\ y{\isacharunderscore}{\kern0pt}prop\ \isacommand{unfolding}\isamarkupfalse%
\ cube{\isacharunderscore}{\kern0pt}def\ \isacommand{by}\isamarkupfalse%
\ auto\isanewline
\ \ \ \ \ \ \ \ \isacommand{moreover}\isamarkupfalse%
\ \isacommand{have}\isamarkupfalse%
\ {\isachardoublequoteopen}\ T{\isacharprime}{\kern0pt}\ {\isacharparenleft}{\kern0pt}{\isasymlambda}i\ {\isasymin}\ {\isacharbraceleft}{\kern0pt}{\isachardot}{\kern0pt}{\isachardot}{\kern0pt}{\isacharless}{\kern0pt}{\isadigit{1}}{\isacharbraceright}{\kern0pt}{\isachardot}{\kern0pt}\ y\ i{\isacharparenright}{\kern0pt}\ {\isacharparenleft}{\kern0pt}{\isasymlambda}i\ {\isasymin}\ {\isacharbraceleft}{\kern0pt}{\isachardot}{\kern0pt}{\isachardot}{\kern0pt}{\isacharless}{\kern0pt}k{\isacharbraceright}{\kern0pt}{\isachardot}{\kern0pt}\ y\ {\isacharparenleft}{\kern0pt}i\ {\isacharplus}{\kern0pt}\ {\isadigit{1}}{\isacharparenright}{\kern0pt}{\isacharparenright}{\kern0pt}\ {\isacharequal}{\kern0pt}\ join\ {\isacharparenleft}{\kern0pt}L{\isacharunderscore}{\kern0pt}line\ {\isacharparenleft}{\kern0pt}{\isacharparenleft}{\kern0pt}{\isasymlambda}i\ {\isasymin}\ {\isacharbraceleft}{\kern0pt}{\isachardot}{\kern0pt}{\isachardot}{\kern0pt}{\isacharless}{\kern0pt}{\isadigit{1}}{\isacharbraceright}{\kern0pt}{\isachardot}{\kern0pt}\ y\ i{\isacharparenright}{\kern0pt}\ {\isadigit{0}}{\isacharparenright}{\kern0pt}{\isacharparenright}{\kern0pt}\ {\isacharparenleft}{\kern0pt}S\ {\isacharparenleft}{\kern0pt}{\isasymlambda}i\ {\isasymin}\ {\isacharbraceleft}{\kern0pt}{\isachardot}{\kern0pt}{\isachardot}{\kern0pt}{\isacharless}{\kern0pt}k{\isacharbraceright}{\kern0pt}{\isachardot}{\kern0pt}\ y\ {\isacharparenleft}{\kern0pt}i\ {\isacharplus}{\kern0pt}\ {\isadigit{1}}{\isacharparenright}{\kern0pt}{\isacharparenright}{\kern0pt}{\isacharparenright}{\kern0pt}\ n\ m{\isachardoublequoteclose}\ \isacommand{using}\isamarkupfalse%
\ calculation\ \isacommand{unfolding}\isamarkupfalse%
\ T{\isacharprime}{\kern0pt}{\isacharunderscore}{\kern0pt}def\ \isacommand{by}\isamarkupfalse%
\ auto\isanewline
\ \ \ \ \ \ \ \ \isacommand{ultimately}\isamarkupfalse%
\ \isacommand{have}\isamarkupfalse%
\ {\isacharasterisk}{\kern0pt}{\isacharcolon}{\kern0pt}\ {\isachardoublequoteopen}T\ y\ {\isacharequal}{\kern0pt}\ join\ {\isacharparenleft}{\kern0pt}L{\isacharunderscore}{\kern0pt}line\ {\isacharparenleft}{\kern0pt}{\isacharparenleft}{\kern0pt}{\isasymlambda}i\ {\isasymin}\ {\isacharbraceleft}{\kern0pt}{\isachardot}{\kern0pt}{\isachardot}{\kern0pt}{\isacharless}{\kern0pt}{\isadigit{1}}{\isacharbraceright}{\kern0pt}{\isachardot}{\kern0pt}\ y\ i{\isacharparenright}{\kern0pt}\ {\isadigit{0}}{\isacharparenright}{\kern0pt}{\isacharparenright}{\kern0pt}\ {\isacharparenleft}{\kern0pt}S\ {\isacharparenleft}{\kern0pt}{\isasymlambda}i\ {\isasymin}\ {\isacharbraceleft}{\kern0pt}{\isachardot}{\kern0pt}{\isachardot}{\kern0pt}{\isacharless}{\kern0pt}k{\isacharbraceright}{\kern0pt}{\isachardot}{\kern0pt}\ y\ {\isacharparenleft}{\kern0pt}i\ {\isacharplus}{\kern0pt}\ {\isadigit{1}}{\isacharparenright}{\kern0pt}{\isacharparenright}{\kern0pt}{\isacharparenright}{\kern0pt}\ n\ m{\isachardoublequoteclose}\ \isacommand{by}\isamarkupfalse%
\ simp\isanewline
\isanewline
\ \ \ \ \ \ \ \ \isacommand{have}\isamarkupfalse%
\ {\isachardoublequoteopen}{\isacharparenleft}{\kern0pt}{\isasymlambda}i\ {\isasymin}\ {\isacharbraceleft}{\kern0pt}{\isachardot}{\kern0pt}{\isachardot}{\kern0pt}{\isacharless}{\kern0pt}{\isadigit{1}}{\isacharbraceright}{\kern0pt}{\isachardot}{\kern0pt}\ y\ i{\isacharparenright}{\kern0pt}\ {\isadigit{0}}\ {\isasymin}\ {\isacharbraceleft}{\kern0pt}{\isachardot}{\kern0pt}{\isachardot}{\kern0pt}{\isacharless}{\kern0pt}t{\isacharplus}{\kern0pt}{\isadigit{1}}{\isacharbraceright}{\kern0pt}{\isachardoublequoteclose}\ \isacommand{using}\isamarkupfalse%
\ y{\isacharunderscore}{\kern0pt}prop\ \isacommand{unfolding}\isamarkupfalse%
\ cube{\isacharunderscore}{\kern0pt}def\ \isacommand{by}\isamarkupfalse%
\ auto\isanewline
\ \ \ \ \ \ \ \ \isacommand{moreover}\isamarkupfalse%
\ \isacommand{have}\isamarkupfalse%
\ {\isachardoublequoteopen}S\ {\isacharparenleft}{\kern0pt}{\isasymlambda}i\ {\isasymin}\ {\isacharbraceleft}{\kern0pt}{\isachardot}{\kern0pt}{\isachardot}{\kern0pt}{\isacharless}{\kern0pt}k{\isacharbraceright}{\kern0pt}{\isachardot}{\kern0pt}\ y\ {\isacharparenleft}{\kern0pt}i\ {\isacharplus}{\kern0pt}\ {\isadigit{1}}{\isacharparenright}{\kern0pt}{\isacharparenright}{\kern0pt}\ {\isasymin}\ S\ {\isacharbackquote}{\kern0pt}\ {\isacharparenleft}{\kern0pt}cube\ k\ {\isacharparenleft}{\kern0pt}t{\isacharplus}{\kern0pt}{\isadigit{1}}{\isacharparenright}{\kern0pt}{\isacharparenright}{\kern0pt}{\isachardoublequoteclose}\ \isanewline
\ \ \ \ \ \ \ \ \ \ \isacommand{using}\isamarkupfalse%
\ {\isacartoucheopen}{\isacharparenleft}{\kern0pt}{\isasymlambda}i{\isasymin}{\isacharbraceleft}{\kern0pt}{\isachardot}{\kern0pt}{\isachardot}{\kern0pt}{\isacharless}{\kern0pt}k{\isacharbraceright}{\kern0pt}{\isachardot}{\kern0pt}\ y\ {\isacharparenleft}{\kern0pt}i\ {\isacharplus}{\kern0pt}\ {\isadigit{1}}{\isacharparenright}{\kern0pt}{\isacharparenright}{\kern0pt}\ {\isasymin}\ cube\ k\ {\isacharparenleft}{\kern0pt}t\ {\isacharplus}{\kern0pt}\ {\isadigit{1}}{\isacharparenright}{\kern0pt}{\isacartoucheclose}\ \isacommand{by}\isamarkupfalse%
\ blast\isanewline
\ \ \ \ \ \ \ \ \isacommand{ultimately}\isamarkupfalse%
\ \isacommand{have}\isamarkupfalse%
\ {\isachardoublequoteopen}T\ y\ {\isasymin}\ imT{\isachardoublequoteclose}\ \isacommand{using}\isamarkupfalse%
\ {\isacharasterisk}{\kern0pt}\ \isacommand{unfolding}\isamarkupfalse%
\ imT{\isacharunderscore}{\kern0pt}def\ \isacommand{by}\isamarkupfalse%
\ blast\isanewline
\ \ \ \ \ \ \ \ \isacommand{then}\isamarkupfalse%
\ \isacommand{show}\isamarkupfalse%
\ {\isachardoublequoteopen}x\ {\isasymin}\ imT{\isachardoublequoteclose}\ \isacommand{using}\isamarkupfalse%
\ y{\isacharunderscore}{\kern0pt}prop\ \isacommand{by}\isamarkupfalse%
\ simp\isanewline
\ \ \ \ \ \ \isacommand{qed}\isamarkupfalse%
\isanewline
\isanewline
\ \ \ \ \ \ \isacommand{show}\isamarkupfalse%
\ {\isachardoublequoteopen}imT\ {\isasymsubseteq}\ T\ {\isacharbackquote}{\kern0pt}\ cube\ {\isacharparenleft}{\kern0pt}k\ {\isacharplus}{\kern0pt}\ {\isadigit{1}}{\isacharparenright}{\kern0pt}\ {\isacharparenleft}{\kern0pt}t\ {\isacharplus}{\kern0pt}\ {\isadigit{1}}{\isacharparenright}{\kern0pt}{\isachardoublequoteclose}\ \isanewline
\ \ \ \ \ \ \isacommand{proof}\isamarkupfalse%
\isanewline
\ \ \ \ \ \ \ \ \isacommand{fix}\isamarkupfalse%
\ x\ \isacommand{assume}\isamarkupfalse%
\ {\isachardoublequoteopen}x\ {\isasymin}\ imT{\isachardoublequoteclose}\isanewline
\ \ \ \ \ \ \ \ \isacommand{then}\isamarkupfalse%
\ \isacommand{obtain}\isamarkupfalse%
\ i\ sx\ sxinv\ \isakeyword{where}\ isx{\isacharunderscore}{\kern0pt}prop{\isacharcolon}{\kern0pt}\ {\isachardoublequoteopen}x\ {\isacharequal}{\kern0pt}\ join\ {\isacharparenleft}{\kern0pt}L{\isacharunderscore}{\kern0pt}line\ i{\isacharparenright}{\kern0pt}\ sx\ n\ m\ {\isasymand}\ i\ {\isasymin}\ {\isacharbraceleft}{\kern0pt}{\isachardot}{\kern0pt}{\isachardot}{\kern0pt}{\isacharless}{\kern0pt}t{\isacharplus}{\kern0pt}{\isadigit{1}}{\isacharbraceright}{\kern0pt}\ {\isasymand}\ sx\ {\isasymin}\ S\ {\isacharbackquote}{\kern0pt}\ {\isacharparenleft}{\kern0pt}cube\ k\ {\isacharparenleft}{\kern0pt}t{\isacharplus}{\kern0pt}{\isadigit{1}}{\isacharparenright}{\kern0pt}{\isacharparenright}{\kern0pt}\ {\isasymand}\ sxinv\ {\isasymin}\ cube\ k\ {\isacharparenleft}{\kern0pt}t{\isacharplus}{\kern0pt}{\isadigit{1}}{\isacharparenright}{\kern0pt}\ {\isasymand}\ S\ sxinv\ {\isacharequal}{\kern0pt}\ sx{\isachardoublequoteclose}\ \isacommand{unfolding}\isamarkupfalse%
\ imT{\isacharunderscore}{\kern0pt}def\ \isacommand{by}\isamarkupfalse%
\ blast\isanewline
\ \ \ \ \ \ \ \ \isacommand{let}\isamarkupfalse%
\ {\isacharquery}{\kern0pt}f{\isadigit{1}}\ {\isacharequal}{\kern0pt}\ {\isachardoublequoteopen}{\isacharparenleft}{\kern0pt}{\isasymlambda}j\ {\isasymin}\ {\isacharbraceleft}{\kern0pt}{\isachardot}{\kern0pt}{\isachardot}{\kern0pt}{\isacharless}{\kern0pt}{\isadigit{1}}{\isacharcolon}{\kern0pt}{\isacharcolon}{\kern0pt}nat{\isacharbraceright}{\kern0pt}{\isachardot}{\kern0pt}\ i{\isacharparenright}{\kern0pt}{\isachardoublequoteclose}\isanewline
\ \ \ \ \ \ \ \ \isacommand{let}\isamarkupfalse%
\ {\isacharquery}{\kern0pt}f{\isadigit{2}}\ {\isacharequal}{\kern0pt}\ {\isachardoublequoteopen}sxinv{\isachardoublequoteclose}\isanewline
\ \ \ \ \ \ \ \ \isacommand{have}\isamarkupfalse%
\ {\isachardoublequoteopen}{\isacharquery}{\kern0pt}f{\isadigit{1}}\ {\isasymin}\ cube\ {\isadigit{1}}\ {\isacharparenleft}{\kern0pt}t{\isacharplus}{\kern0pt}{\isadigit{1}}{\isacharparenright}{\kern0pt}{\isachardoublequoteclose}\ \isacommand{using}\isamarkupfalse%
\ isx{\isacharunderscore}{\kern0pt}prop\ \isacommand{unfolding}\isamarkupfalse%
\ cube{\isacharunderscore}{\kern0pt}def\ \isacommand{by}\isamarkupfalse%
\ simp\isanewline
\ \ \ \ \ \ \ \ \isacommand{moreover}\isamarkupfalse%
\ \isacommand{have}\isamarkupfalse%
\ {\isachardoublequoteopen}{\isacharquery}{\kern0pt}f{\isadigit{2}}\ {\isasymin}\ cube\ k\ {\isacharparenleft}{\kern0pt}t{\isacharplus}{\kern0pt}{\isadigit{1}}{\isacharparenright}{\kern0pt}{\isachardoublequoteclose}\ \isacommand{using}\isamarkupfalse%
\ isx{\isacharunderscore}{\kern0pt}prop\ \isacommand{by}\isamarkupfalse%
\ blast\isanewline
\ \ \ \ \ \ \ \ \isacommand{moreover}\isamarkupfalse%
\ \isacommand{have}\isamarkupfalse%
\ {\isachardoublequoteopen}x\ {\isacharequal}{\kern0pt}\ join\ {\isacharparenleft}{\kern0pt}L{\isacharunderscore}{\kern0pt}line\ {\isacharparenleft}{\kern0pt}{\isacharquery}{\kern0pt}f{\isadigit{1}}\ {\isadigit{0}}{\isacharparenright}{\kern0pt}{\isacharparenright}{\kern0pt}\ {\isacharparenleft}{\kern0pt}S\ {\isacharquery}{\kern0pt}f{\isadigit{2}}{\isacharparenright}{\kern0pt}\ n\ m{\isachardoublequoteclose}\ \isacommand{by}\isamarkupfalse%
\ {\isacharparenleft}{\kern0pt}simp\ add{\isacharcolon}{\kern0pt}\ isx{\isacharunderscore}{\kern0pt}prop{\isacharparenright}{\kern0pt}\isanewline
\ \ \ \ \ \ \ \ \isacommand{ultimately}\isamarkupfalse%
\ \isacommand{have}\isamarkupfalse%
\ {\isacharasterisk}{\kern0pt}{\isacharcolon}{\kern0pt}\ {\isachardoublequoteopen}x\ {\isacharequal}{\kern0pt}\ T{\isacharprime}{\kern0pt}\ {\isacharquery}{\kern0pt}f{\isadigit{1}}\ {\isacharquery}{\kern0pt}f{\isadigit{2}}{\isachardoublequoteclose}\ \isacommand{unfolding}\isamarkupfalse%
\ T{\isacharprime}{\kern0pt}{\isacharunderscore}{\kern0pt}def\ \isacommand{by}\isamarkupfalse%
\ simp\ \isanewline
\isanewline
\ \ \ \ \ \ \ \ \isacommand{define}\isamarkupfalse%
\ f\ \isakeyword{where}\ {\isachardoublequoteopen}f\ {\isasymequiv}\ {\isacharparenleft}{\kern0pt}{\isasymlambda}j\ {\isasymin}\ {\isacharbraceleft}{\kern0pt}{\isadigit{1}}{\isachardot}{\kern0pt}{\isachardot}{\kern0pt}{\isacharless}{\kern0pt}k{\isacharplus}{\kern0pt}{\isadigit{1}}{\isacharbraceright}{\kern0pt}{\isachardot}{\kern0pt}\ {\isacharquery}{\kern0pt}f{\isadigit{2}}\ {\isacharparenleft}{\kern0pt}j\ {\isacharminus}{\kern0pt}\ {\isadigit{1}}{\isacharparenright}{\kern0pt}{\isacharparenright}{\kern0pt}{\isacharparenleft}{\kern0pt}{\isadigit{0}}{\isacharcolon}{\kern0pt}{\isacharequal}{\kern0pt}i{\isacharparenright}{\kern0pt}{\isachardoublequoteclose}\isanewline
\ \ \ \ \ \ \ \ \isacommand{have}\isamarkupfalse%
\ {\isachardoublequoteopen}f\ {\isasymin}\ cube\ {\isacharparenleft}{\kern0pt}k{\isacharplus}{\kern0pt}{\isadigit{1}}{\isacharparenright}{\kern0pt}\ {\isacharparenleft}{\kern0pt}t{\isacharplus}{\kern0pt}{\isadigit{1}}{\isacharparenright}{\kern0pt}{\isachardoublequoteclose}\isanewline
\ \ \ \ \ \ \ \ \isacommand{proof}\isamarkupfalse%
\ {\isacharparenleft}{\kern0pt}unfold\ cube{\isacharunderscore}{\kern0pt}def{\isacharsemicolon}{\kern0pt}\ intro\ PiE{\isacharunderscore}{\kern0pt}I{\isacharparenright}{\kern0pt}\isanewline
\ \ \ \ \ \ \ \ \ \ \isacommand{fix}\isamarkupfalse%
\ j\ \isacommand{assume}\isamarkupfalse%
\ {\isachardoublequoteopen}j\ {\isasymin}\ {\isacharbraceleft}{\kern0pt}{\isachardot}{\kern0pt}{\isachardot}{\kern0pt}{\isacharless}{\kern0pt}k{\isacharplus}{\kern0pt}{\isadigit{1}}{\isacharbraceright}{\kern0pt}{\isachardoublequoteclose}\isanewline
\ \ \ \ \ \ \ \ \ \ \isacommand{then}\isamarkupfalse%
\ \isacommand{consider}\isamarkupfalse%
\ {\isachardoublequoteopen}j\ {\isacharequal}{\kern0pt}\ {\isadigit{0}}{\isachardoublequoteclose}\ {\isacharbar}{\kern0pt}\ {\isachardoublequoteopen}j\ {\isasymin}\ {\isacharbraceleft}{\kern0pt}{\isadigit{1}}{\isachardot}{\kern0pt}{\isachardot}{\kern0pt}{\isacharless}{\kern0pt}k{\isacharplus}{\kern0pt}{\isadigit{1}}{\isacharbraceright}{\kern0pt}{\isachardoublequoteclose}\ \isacommand{by}\isamarkupfalse%
\ fastforce\isanewline
\ \ \ \ \ \ \ \ \ \ \isacommand{then}\isamarkupfalse%
\ \isacommand{show}\isamarkupfalse%
\ {\isachardoublequoteopen}f\ j\ {\isasymin}\ {\isacharbraceleft}{\kern0pt}{\isachardot}{\kern0pt}{\isachardot}{\kern0pt}{\isacharless}{\kern0pt}t{\isacharplus}{\kern0pt}{\isadigit{1}}{\isacharbraceright}{\kern0pt}{\isachardoublequoteclose}\isanewline
\ \ \ \ \ \ \ \ \ \ \isacommand{proof}\isamarkupfalse%
\ {\isacharparenleft}{\kern0pt}cases{\isacharparenright}{\kern0pt}\isanewline
\ \ \ \ \ \ \ \ \ \ \ \ \isacommand{case}\isamarkupfalse%
\ {\isadigit{1}}\isanewline
\ \ \ \ \ \ \ \ \ \ \ \ \isacommand{then}\isamarkupfalse%
\ \isacommand{have}\isamarkupfalse%
\ {\isachardoublequoteopen}f\ j\ {\isacharequal}{\kern0pt}\ i{\isachardoublequoteclose}\ \isacommand{unfolding}\isamarkupfalse%
\ f{\isacharunderscore}{\kern0pt}def\ \isacommand{by}\isamarkupfalse%
\ simp\isanewline
\ \ \ \ \ \ \ \ \ \ \ \ \isacommand{then}\isamarkupfalse%
\ \isacommand{show}\isamarkupfalse%
\ {\isacharquery}{\kern0pt}thesis\ \isacommand{using}\isamarkupfalse%
\ isx{\isacharunderscore}{\kern0pt}prop\ \isacommand{by}\isamarkupfalse%
\ simp\isanewline
\ \ \ \ \ \ \ \ \ \ \isacommand{next}\isamarkupfalse%
\isanewline
\ \ \ \ \ \ \ \ \ \ \ \ \isacommand{case}\isamarkupfalse%
\ {\isadigit{2}}\isanewline
\ \ \ \ \ \ \ \ \ \ \ \ \isacommand{then}\isamarkupfalse%
\ \isacommand{have}\isamarkupfalse%
\ {\isachardoublequoteopen}j\ {\isacharminus}{\kern0pt}\ {\isadigit{1}}\ {\isasymin}\ {\isacharbraceleft}{\kern0pt}{\isachardot}{\kern0pt}{\isachardot}{\kern0pt}{\isacharless}{\kern0pt}k{\isacharbraceright}{\kern0pt}{\isachardoublequoteclose}\ \isacommand{by}\isamarkupfalse%
\ auto\isanewline
\ \ \ \ \ \ \ \ \ \ \ \ \isacommand{moreover}\isamarkupfalse%
\ \isacommand{have}\isamarkupfalse%
\ {\isachardoublequoteopen}f\ j\ {\isacharequal}{\kern0pt}\ {\isacharquery}{\kern0pt}f{\isadigit{2}}\ {\isacharparenleft}{\kern0pt}j\ {\isacharminus}{\kern0pt}\ {\isadigit{1}}{\isacharparenright}{\kern0pt}{\isachardoublequoteclose}\ \isacommand{using}\isamarkupfalse%
\ {\isadigit{2}}\ \isacommand{unfolding}\isamarkupfalse%
\ f{\isacharunderscore}{\kern0pt}def\ \isacommand{by}\isamarkupfalse%
\ simp\isanewline
\ \ \ \ \ \ \ \ \ \ \ \ \isacommand{moreover}\isamarkupfalse%
\ \isacommand{have}\isamarkupfalse%
\ {\isachardoublequoteopen}{\isacharquery}{\kern0pt}f{\isadigit{2}}\ {\isacharparenleft}{\kern0pt}j\ {\isacharminus}{\kern0pt}\ {\isadigit{1}}{\isacharparenright}{\kern0pt}\ {\isasymin}\ {\isacharbraceleft}{\kern0pt}{\isachardot}{\kern0pt}{\isachardot}{\kern0pt}{\isacharless}{\kern0pt}t{\isacharplus}{\kern0pt}{\isadigit{1}}{\isacharbraceright}{\kern0pt}{\isachardoublequoteclose}\ \isacommand{using}\isamarkupfalse%
\ calculation{\isacharparenleft}{\kern0pt}{\isadigit{1}}{\isacharparenright}{\kern0pt}\ isx{\isacharunderscore}{\kern0pt}prop\ \isacommand{unfolding}\isamarkupfalse%
\ cube{\isacharunderscore}{\kern0pt}def\ \isacommand{by}\isamarkupfalse%
\ blast\isanewline
\ \ \ \ \ \ \ \ \ \ \ \ \isacommand{ultimately}\isamarkupfalse%
\ \isacommand{show}\isamarkupfalse%
\ {\isacharquery}{\kern0pt}thesis\ \isacommand{by}\isamarkupfalse%
\ simp\isanewline
\ \ \ \ \ \ \ \ \ \ \isacommand{qed}\isamarkupfalse%
\isanewline
\ \ \ \ \ \ \ \ \isacommand{qed}\isamarkupfalse%
\ {\isacharparenleft}{\kern0pt}auto\ simp{\isacharcolon}{\kern0pt}\ f{\isacharunderscore}{\kern0pt}def{\isacharparenright}{\kern0pt}\isanewline
\ \ \ \ \ \ \ \ \isacommand{have}\isamarkupfalse%
\ {\isachardoublequoteopen}{\isacharquery}{\kern0pt}f{\isadigit{1}}\ {\isacharequal}{\kern0pt}\ {\isacharparenleft}{\kern0pt}{\isasymlambda}j\ {\isasymin}\ {\isacharbraceleft}{\kern0pt}{\isachardot}{\kern0pt}{\isachardot}{\kern0pt}{\isacharless}{\kern0pt}{\isadigit{1}}{\isacharbraceright}{\kern0pt}{\isachardot}{\kern0pt}\ f\ j{\isacharparenright}{\kern0pt}{\isachardoublequoteclose}\ \isacommand{unfolding}\isamarkupfalse%
\ f{\isacharunderscore}{\kern0pt}def\ \isacommand{using}\isamarkupfalse%
\ isx{\isacharunderscore}{\kern0pt}prop\ \isacommand{by}\isamarkupfalse%
\ auto\isanewline
\ \ \ \ \ \ \ \ \isacommand{moreover}\isamarkupfalse%
\ \isacommand{have}\isamarkupfalse%
\ {\isachardoublequoteopen}{\isacharquery}{\kern0pt}f{\isadigit{2}}\ {\isacharequal}{\kern0pt}\ {\isacharparenleft}{\kern0pt}{\isasymlambda}j{\isasymin}{\isacharbraceleft}{\kern0pt}{\isachardot}{\kern0pt}{\isachardot}{\kern0pt}{\isacharless}{\kern0pt}k{\isacharbraceright}{\kern0pt}{\isachardot}{\kern0pt}\ f\ {\isacharparenleft}{\kern0pt}j{\isacharplus}{\kern0pt}{\isadigit{1}}{\isacharparenright}{\kern0pt}{\isacharparenright}{\kern0pt}{\isachardoublequoteclose}\ \isacommand{using}\isamarkupfalse%
\ calculation\ isx{\isacharunderscore}{\kern0pt}prop\ \isacommand{unfolding}\isamarkupfalse%
\ cube{\isacharunderscore}{\kern0pt}def\ f{\isacharunderscore}{\kern0pt}def\ \isacommand{by}\isamarkupfalse%
\ fastforce\isanewline
\ \ \ \ \ \ \ \ \isacommand{ultimately}\isamarkupfalse%
\ \isacommand{have}\isamarkupfalse%
\ {\isachardoublequoteopen}T{\isacharprime}{\kern0pt}\ {\isacharquery}{\kern0pt}f{\isadigit{1}}\ {\isacharquery}{\kern0pt}f{\isadigit{2}}\ {\isacharequal}{\kern0pt}\ T\ f{\isachardoublequoteclose}\ \isacommand{using}\isamarkupfalse%
\ {\isacartoucheopen}f\ {\isasymin}\ cube\ {\isacharparenleft}{\kern0pt}k{\isacharplus}{\kern0pt}{\isadigit{1}}{\isacharparenright}{\kern0pt}\ {\isacharparenleft}{\kern0pt}t{\isacharplus}{\kern0pt}{\isadigit{1}}{\isacharparenright}{\kern0pt}{\isacartoucheclose}\ \isacommand{unfolding}\isamarkupfalse%
\ T{\isacharunderscore}{\kern0pt}def\ \isacommand{by}\isamarkupfalse%
\ simp\isanewline
\ \ \ \ \ \ \ \ \isacommand{then}\isamarkupfalse%
\ \isacommand{show}\isamarkupfalse%
\ {\isachardoublequoteopen}x\ {\isasymin}\ T\ {\isacharbackquote}{\kern0pt}\ cube\ {\isacharparenleft}{\kern0pt}k\ {\isacharplus}{\kern0pt}\ {\isadigit{1}}{\isacharparenright}{\kern0pt}\ {\isacharparenleft}{\kern0pt}t\ {\isacharplus}{\kern0pt}\ {\isadigit{1}}{\isacharparenright}{\kern0pt}{\isachardoublequoteclose}\ \isacommand{using}\isamarkupfalse%
\ {\isacharasterisk}{\kern0pt}\ \isanewline
\ \ \ \ \ \ \ \ \ \ \isacommand{using}\isamarkupfalse%
\ {\isacartoucheopen}f\ {\isasymin}\ cube\ {\isacharparenleft}{\kern0pt}k\ {\isacharplus}{\kern0pt}\ {\isadigit{1}}{\isacharparenright}{\kern0pt}\ {\isacharparenleft}{\kern0pt}t\ {\isacharplus}{\kern0pt}\ {\isadigit{1}}{\isacharparenright}{\kern0pt}{\isacartoucheclose}\ \isacommand{by}\isamarkupfalse%
\ blast\isanewline
\ \ \ \ \ \ \isacommand{qed}\isamarkupfalse%
\isanewline
\isanewline
\isanewline
\ \ \ \ \isacommand{qed}\isamarkupfalse%
\isanewline
\ \ \ \ \isacommand{have}\isamarkupfalse%
\ {\isachardoublequoteopen}imT\ {\isasymsubseteq}\ cube\ {\isacharparenleft}{\kern0pt}n\ {\isacharplus}{\kern0pt}\ m{\isacharparenright}{\kern0pt}\ {\isacharparenleft}{\kern0pt}t{\isacharplus}{\kern0pt}{\isadigit{1}}{\isacharparenright}{\kern0pt}{\isachardoublequoteclose}\isanewline
\ \ \ \ \isacommand{proof}\isamarkupfalse%
\isanewline
\ \ \ \ \ \ \isacommand{fix}\isamarkupfalse%
\ x\ \isacommand{assume}\isamarkupfalse%
\ a{\isacharcolon}{\kern0pt}\ {\isachardoublequoteopen}x{\isasymin}imT{\isachardoublequoteclose}\isanewline
\ \ \ \ \ \ \isacommand{then}\isamarkupfalse%
\ \isacommand{obtain}\isamarkupfalse%
\ i\ sx\ \isakeyword{where}\ isx{\isacharunderscore}{\kern0pt}props{\isacharcolon}{\kern0pt}\ {\isachardoublequoteopen}x\ {\isacharequal}{\kern0pt}\ join\ {\isacharparenleft}{\kern0pt}L{\isacharunderscore}{\kern0pt}line\ i{\isacharparenright}{\kern0pt}\ sx\ n\ m\ {\isasymand}\ i\ {\isasymin}\ {\isacharbraceleft}{\kern0pt}{\isachardot}{\kern0pt}{\isachardot}{\kern0pt}{\isacharless}{\kern0pt}t{\isacharplus}{\kern0pt}{\isadigit{1}}{\isacharbraceright}{\kern0pt}\ {\isasymand}\ sx\ {\isasymin}\ S\ {\isacharbackquote}{\kern0pt}\ {\isacharparenleft}{\kern0pt}cube\ k\ {\isacharparenleft}{\kern0pt}t{\isacharplus}{\kern0pt}{\isadigit{1}}{\isacharparenright}{\kern0pt}{\isacharparenright}{\kern0pt}{\isachardoublequoteclose}\ \isacommand{unfolding}\isamarkupfalse%
\ imT{\isacharunderscore}{\kern0pt}def\ \isacommand{by}\isamarkupfalse%
\ blast\isanewline
\ \ \ \ \ \ \isacommand{then}\isamarkupfalse%
\ \isacommand{have}\isamarkupfalse%
\ {\isachardoublequoteopen}L{\isacharunderscore}{\kern0pt}line\ i\ {\isasymin}\ cube\ n\ {\isacharparenleft}{\kern0pt}t{\isacharplus}{\kern0pt}{\isadigit{1}}{\isacharparenright}{\kern0pt}{\isachardoublequoteclose}\ \isacommand{using}\isamarkupfalse%
\ L{\isacharunderscore}{\kern0pt}line{\isacharunderscore}{\kern0pt}base{\isacharunderscore}{\kern0pt}prop\ \isacommand{by}\isamarkupfalse%
\ blast\isanewline
\ \ \ \ \ \ \isacommand{moreover}\isamarkupfalse%
\ \isacommand{have}\isamarkupfalse%
\ {\isachardoublequoteopen}sx\ {\isasymin}\ cube\ m\ {\isacharparenleft}{\kern0pt}t{\isacharplus}{\kern0pt}{\isadigit{1}}{\isacharparenright}{\kern0pt}{\isachardoublequoteclose}\ \isacommand{using}\isamarkupfalse%
\ subspace{\isacharunderscore}{\kern0pt}elems{\isacharunderscore}{\kern0pt}embed{\isacharbrackleft}{\kern0pt}of\ {\isachardoublequoteopen}S{\isachardoublequoteclose}\ {\isachardoublequoteopen}k{\isachardoublequoteclose}\ {\isachardoublequoteopen}m{\isachardoublequoteclose}\ {\isachardoublequoteopen}t{\isacharplus}{\kern0pt}{\isadigit{1}}{\isachardoublequoteclose}{\isacharbrackright}{\kern0pt}\ S{\isacharunderscore}{\kern0pt}prop\ isx{\isacharunderscore}{\kern0pt}props\ \isacommand{unfolding}\isamarkupfalse%
\ layered{\isacharunderscore}{\kern0pt}subspace{\isacharunderscore}{\kern0pt}def\ \isacommand{by}\isamarkupfalse%
\ blast\isanewline
\ \ \ \ \ \ \isacommand{ultimately}\isamarkupfalse%
\ \isacommand{show}\isamarkupfalse%
\ {\isachardoublequoteopen}x\ {\isasymin}\ cube\ {\isacharparenleft}{\kern0pt}n\ {\isacharplus}{\kern0pt}\ m{\isacharparenright}{\kern0pt}\ {\isacharparenleft}{\kern0pt}t{\isacharplus}{\kern0pt}{\isadigit{1}}{\isacharparenright}{\kern0pt}{\isachardoublequoteclose}\ \isacommand{using}\isamarkupfalse%
\ join{\isacharunderscore}{\kern0pt}cubes{\isacharbrackleft}{\kern0pt}of\ {\isachardoublequoteopen}L{\isacharunderscore}{\kern0pt}line\ i{\isachardoublequoteclose}\ {\isachardoublequoteopen}n{\isachardoublequoteclose}\ {\isachardoublequoteopen}t{\isachardoublequoteclose}\ sx\ m{\isacharbrackright}{\kern0pt}\ isx{\isacharunderscore}{\kern0pt}props\ \isacommand{by}\isamarkupfalse%
\ simp\ \isanewline
\ \ \ \ \isacommand{qed}\isamarkupfalse%
\isanewline
\ \ \ \ \ \ \isanewline
\isanewline
\isanewline
\isanewline
\isanewline
\ \ \ \ \isacommand{obtain}\isamarkupfalse%
\ BS\ fS\ \isakeyword{where}\ BfS{\isacharunderscore}{\kern0pt}props{\isacharcolon}{\kern0pt}\ {\isachardoublequoteopen}disjoint{\isacharunderscore}{\kern0pt}family{\isacharunderscore}{\kern0pt}on\ BS\ {\isacharbraceleft}{\kern0pt}{\isachardot}{\kern0pt}{\isachardot}{\kern0pt}k{\isacharbraceright}{\kern0pt}{\isachardoublequoteclose}\ {\isachardoublequoteopen}{\isasymUnion}{\isacharparenleft}{\kern0pt}BS\ {\isacharbackquote}{\kern0pt}\ {\isacharbraceleft}{\kern0pt}{\isachardot}{\kern0pt}{\isachardot}{\kern0pt}k{\isacharbraceright}{\kern0pt}{\isacharparenright}{\kern0pt}\ {\isacharequal}{\kern0pt}\ {\isacharbraceleft}{\kern0pt}{\isachardot}{\kern0pt}{\isachardot}{\kern0pt}{\isacharless}{\kern0pt}m{\isacharbraceright}{\kern0pt}{\isachardoublequoteclose}\ {\isachardoublequoteopen}{\isacharparenleft}{\kern0pt}{\isacharbraceleft}{\kern0pt}{\isacharbraceright}{\kern0pt}\ {\isasymnotin}\ BS\ {\isacharbackquote}{\kern0pt}\ {\isacharbraceleft}{\kern0pt}{\isachardot}{\kern0pt}{\isachardot}{\kern0pt}{\isacharless}{\kern0pt}k{\isacharbraceright}{\kern0pt}{\isacharparenright}{\kern0pt}{\isachardoublequoteclose}\ {\isachardoublequoteopen}\ fS\ {\isasymin}\ {\isacharparenleft}{\kern0pt}BS\ k{\isacharparenright}{\kern0pt}\ {\isasymrightarrow}\isactrlsub E\ {\isacharbraceleft}{\kern0pt}{\isachardot}{\kern0pt}{\isachardot}{\kern0pt}{\isacharless}{\kern0pt}t{\isacharplus}{\kern0pt}{\isadigit{1}}{\isacharbraceright}{\kern0pt}{\isachardoublequoteclose}\ {\isachardoublequoteopen}S\ {\isasymin}\ {\isacharparenleft}{\kern0pt}cube\ k\ {\isacharparenleft}{\kern0pt}t{\isacharplus}{\kern0pt}{\isadigit{1}}{\isacharparenright}{\kern0pt}{\isacharparenright}{\kern0pt}\ {\isasymrightarrow}\isactrlsub E\ {\isacharparenleft}{\kern0pt}cube\ m\ {\isacharparenleft}{\kern0pt}t{\isacharplus}{\kern0pt}{\isadigit{1}}{\isacharparenright}{\kern0pt}{\isacharparenright}{\kern0pt}\ {\isachardoublequoteclose}\ {\isachardoublequoteopen}{\isacharparenleft}{\kern0pt}{\isasymforall}y\ {\isasymin}\ cube\ k\ {\isacharparenleft}{\kern0pt}t{\isacharplus}{\kern0pt}{\isadigit{1}}{\isacharparenright}{\kern0pt}{\isachardot}{\kern0pt}\ {\isacharparenleft}{\kern0pt}{\isasymforall}i\ {\isasymin}\ BS\ k{\isachardot}{\kern0pt}\ S\ y\ i\ {\isacharequal}{\kern0pt}\ fS\ i{\isacharparenright}{\kern0pt}\ {\isasymand}\ {\isacharparenleft}{\kern0pt}{\isasymforall}j{\isacharless}{\kern0pt}k{\isachardot}{\kern0pt}\ {\isasymforall}i\ {\isasymin}\ BS\ j{\isachardot}{\kern0pt}\ {\isacharparenleft}{\kern0pt}S\ y{\isacharparenright}{\kern0pt}\ i\ {\isacharequal}{\kern0pt}\ y\ j{\isacharparenright}{\kern0pt}{\isacharparenright}{\kern0pt}{\isachardoublequoteclose}\ \isacommand{using}\isamarkupfalse%
\ S{\isacharunderscore}{\kern0pt}prop\ \isacommand{unfolding}\isamarkupfalse%
\ layered{\isacharunderscore}{\kern0pt}subspace{\isacharunderscore}{\kern0pt}def\ is{\isacharunderscore}{\kern0pt}subspace{\isacharunderscore}{\kern0pt}def\ \isacommand{by}\isamarkupfalse%
\ auto\isanewline
\isanewline
\ \ \ \ \isacommand{obtain}\isamarkupfalse%
\ BL\ fL\ \isakeyword{where}\ BfL{\isacharunderscore}{\kern0pt}props{\isacharcolon}{\kern0pt}\ {\isachardoublequoteopen}disjoint{\isacharunderscore}{\kern0pt}family{\isacharunderscore}{\kern0pt}on\ BL\ {\isacharbraceleft}{\kern0pt}{\isachardot}{\kern0pt}{\isachardot}{\kern0pt}{\isadigit{1}}{\isacharbraceright}{\kern0pt}{\isachardoublequoteclose}\ {\isachardoublequoteopen}{\isasymUnion}{\isacharparenleft}{\kern0pt}BL\ {\isacharbackquote}{\kern0pt}\ {\isacharbraceleft}{\kern0pt}{\isachardot}{\kern0pt}{\isachardot}{\kern0pt}{\isadigit{1}}{\isacharbraceright}{\kern0pt}{\isacharparenright}{\kern0pt}\ {\isacharequal}{\kern0pt}\ {\isacharbraceleft}{\kern0pt}{\isachardot}{\kern0pt}{\isachardot}{\kern0pt}{\isacharless}{\kern0pt}n{\isacharbraceright}{\kern0pt}{\isachardoublequoteclose}\ \ {\isachardoublequoteopen}{\isacharparenleft}{\kern0pt}{\isacharbraceleft}{\kern0pt}{\isacharbraceright}{\kern0pt}\ {\isasymnotin}\ BL\ {\isacharbackquote}{\kern0pt}\ {\isacharbraceleft}{\kern0pt}{\isachardot}{\kern0pt}{\isachardot}{\kern0pt}{\isacharless}{\kern0pt}{\isadigit{1}}{\isacharbraceright}{\kern0pt}{\isacharparenright}{\kern0pt}{\isachardoublequoteclose}\ {\isachardoublequoteopen}fL\ {\isasymin}\ {\isacharparenleft}{\kern0pt}BL\ {\isadigit{1}}{\isacharparenright}{\kern0pt}\ {\isasymrightarrow}\isactrlsub E\ {\isacharbraceleft}{\kern0pt}{\isachardot}{\kern0pt}{\isachardot}{\kern0pt}{\isacharless}{\kern0pt}t{\isacharplus}{\kern0pt}{\isadigit{1}}{\isacharbraceright}{\kern0pt}{\isachardoublequoteclose}\ {\isachardoublequoteopen}L\ {\isasymin}\ {\isacharparenleft}{\kern0pt}cube\ {\isadigit{1}}\ {\isacharparenleft}{\kern0pt}t{\isacharplus}{\kern0pt}{\isadigit{1}}{\isacharparenright}{\kern0pt}{\isacharparenright}{\kern0pt}\ {\isasymrightarrow}\isactrlsub E\ {\isacharparenleft}{\kern0pt}cube\ n\ {\isacharparenleft}{\kern0pt}t{\isacharplus}{\kern0pt}{\isadigit{1}}{\isacharparenright}{\kern0pt}{\isacharparenright}{\kern0pt}{\isachardoublequoteclose}\ {\isachardoublequoteopen}{\isacharparenleft}{\kern0pt}{\isasymforall}y\ {\isasymin}\ cube\ {\isadigit{1}}\ {\isacharparenleft}{\kern0pt}t{\isacharplus}{\kern0pt}{\isadigit{1}}{\isacharparenright}{\kern0pt}{\isachardot}{\kern0pt}\ {\isacharparenleft}{\kern0pt}{\isasymforall}i\ {\isasymin}\ BL\ {\isadigit{1}}{\isachardot}{\kern0pt}\ L\ y\ i\ {\isacharequal}{\kern0pt}\ fL\ i{\isacharparenright}{\kern0pt}\ {\isasymand}\ {\isacharparenleft}{\kern0pt}{\isasymforall}j{\isacharless}{\kern0pt}{\isadigit{1}}{\isachardot}{\kern0pt}\ {\isasymforall}i\ {\isasymin}\ BL\ j{\isachardot}{\kern0pt}\ {\isacharparenleft}{\kern0pt}L\ y{\isacharparenright}{\kern0pt}\ i\ {\isacharequal}{\kern0pt}\ y\ j{\isacharparenright}{\kern0pt}{\isacharparenright}{\kern0pt}{\isachardoublequoteclose}\ \isacommand{using}\isamarkupfalse%
\ L{\isacharunderscore}{\kern0pt}prop\ \isacommand{unfolding}\isamarkupfalse%
\ layered{\isacharunderscore}{\kern0pt}subspace{\isacharunderscore}{\kern0pt}def\ is{\isacharunderscore}{\kern0pt}subspace{\isacharunderscore}{\kern0pt}def\ \isacommand{by}\isamarkupfalse%
\ auto\isanewline
\isanewline
\ \ \ \ \isacommand{define}\isamarkupfalse%
\ Bstat\ \isakeyword{where}\ {\isachardoublequoteopen}Bstat\ {\isasymequiv}\ shiftset\ n\ {\isacharparenleft}{\kern0pt}BS\ k{\isacharparenright}{\kern0pt}\ {\isasymunion}\ BL\ {\isadigit{1}}{\isachardoublequoteclose}\isanewline
\ \ \ \ \isacommand{define}\isamarkupfalse%
\ Bvar\ \isakeyword{where}\ {\isachardoublequoteopen}Bvar\ {\isasymequiv}\ {\isacharparenleft}{\kern0pt}{\isasymlambda}i{\isacharcolon}{\kern0pt}{\isacharcolon}{\kern0pt}nat{\isachardot}{\kern0pt}\ {\isacharparenleft}{\kern0pt}if\ i\ {\isacharequal}{\kern0pt}\ {\isadigit{0}}\ then\ BL\ {\isadigit{0}}\ else\ shiftset\ n\ {\isacharparenleft}{\kern0pt}BS\ {\isacharparenleft}{\kern0pt}i\ {\isacharminus}{\kern0pt}\ {\isadigit{1}}{\isacharparenright}{\kern0pt}{\isacharparenright}{\kern0pt}{\isacharparenright}{\kern0pt}{\isacharparenright}{\kern0pt}{\isachardoublequoteclose}\isanewline
\ \ \ \ \isacommand{define}\isamarkupfalse%
\ BT\ \isakeyword{where}\ {\isachardoublequoteopen}BT\ {\isasymequiv}\ {\isacharparenleft}{\kern0pt}{\isasymlambda}i\ {\isasymin}\ {\isacharbraceleft}{\kern0pt}{\isachardot}{\kern0pt}{\isachardot}{\kern0pt}{\isacharless}{\kern0pt}k{\isacharplus}{\kern0pt}{\isadigit{1}}{\isacharbraceright}{\kern0pt}{\isachardot}{\kern0pt}\ Bvar\ i{\isacharparenright}{\kern0pt}{\isacharparenleft}{\kern0pt}{\isacharparenleft}{\kern0pt}k{\isacharplus}{\kern0pt}{\isadigit{1}}{\isacharparenright}{\kern0pt}{\isacharcolon}{\kern0pt}{\isacharequal}{\kern0pt}Bstat{\isacharparenright}{\kern0pt}{\isachardoublequoteclose}\isanewline
\ \ \ \ \isacommand{define}\isamarkupfalse%
\ fT\ \isakeyword{where}\ {\isachardoublequoteopen}fT\ {\isasymequiv}\ {\isacharparenleft}{\kern0pt}{\isasymlambda}x{\isachardot}{\kern0pt}\ {\isacharparenleft}{\kern0pt}if\ x\ {\isasymin}\ BL\ {\isadigit{1}}\ then\ fL\ x\ else\ {\isacharparenleft}{\kern0pt}if\ x\ {\isasymin}\ shiftset\ n\ {\isacharparenleft}{\kern0pt}BS\ k{\isacharparenright}{\kern0pt}\ then\ fS\ {\isacharparenleft}{\kern0pt}x\ {\isacharminus}{\kern0pt}\ n{\isacharparenright}{\kern0pt}\ else\ undefined{\isacharparenright}{\kern0pt}{\isacharparenright}{\kern0pt}{\isacharparenright}{\kern0pt}{\isachardoublequoteclose}\isanewline
\isanewline
\isanewline
\isanewline
\ \ \ \ \isacommand{have}\isamarkupfalse%
\ fax{\isadigit{1}}{\isacharcolon}{\kern0pt}\ {\isachardoublequoteopen}shiftset\ n\ {\isacharparenleft}{\kern0pt}BS\ k{\isacharparenright}{\kern0pt}\ {\isasyminter}\ BL\ {\isadigit{1}}\ {\isacharequal}{\kern0pt}\ {\isacharbraceleft}{\kern0pt}{\isacharbraceright}{\kern0pt}{\isachardoublequoteclose}\ \ \isacommand{using}\isamarkupfalse%
\ BfL{\isacharunderscore}{\kern0pt}props\ BfS{\isacharunderscore}{\kern0pt}props\ \isacommand{unfolding}\isamarkupfalse%
\ shiftset{\isacharunderscore}{\kern0pt}def\ \isacommand{by}\isamarkupfalse%
\ auto\isanewline
\ \ \ \ \isacommand{have}\isamarkupfalse%
\ fax{\isadigit{2}}{\isacharcolon}{\kern0pt}\ {\isachardoublequoteopen}BL\ {\isadigit{0}}\ {\isasyminter}\ {\isacharparenleft}{\kern0pt}{\isasymUnion}i{\isasymin}{\isacharbraceleft}{\kern0pt}{\isachardot}{\kern0pt}{\isachardot}{\kern0pt}{\isacharless}{\kern0pt}k{\isacharbraceright}{\kern0pt}{\isachardot}{\kern0pt}\ shiftset\ n\ {\isacharparenleft}{\kern0pt}BS\ i{\isacharparenright}{\kern0pt}{\isacharparenright}{\kern0pt}\ {\isacharequal}{\kern0pt}\ {\isacharbraceleft}{\kern0pt}{\isacharbraceright}{\kern0pt}{\isachardoublequoteclose}\ \isacommand{using}\isamarkupfalse%
\ BfL{\isacharunderscore}{\kern0pt}props\ BfS{\isacharunderscore}{\kern0pt}props\ \isacommand{unfolding}\isamarkupfalse%
\ shiftset{\isacharunderscore}{\kern0pt}def\ \isacommand{by}\isamarkupfalse%
\ auto\isanewline
\ \ \ \ \isacommand{have}\isamarkupfalse%
\ fax{\isadigit{3}}{\isacharcolon}{\kern0pt}\ {\isachardoublequoteopen}{\isasymforall}i\ {\isasymin}\ {\isacharbraceleft}{\kern0pt}{\isachardot}{\kern0pt}{\isachardot}{\kern0pt}{\isacharless}{\kern0pt}k{\isacharbraceright}{\kern0pt}{\isachardot}{\kern0pt}\ BL\ {\isadigit{0}}\ {\isasyminter}\ shiftset\ n\ {\isacharparenleft}{\kern0pt}BS\ i{\isacharparenright}{\kern0pt}\ {\isacharequal}{\kern0pt}\ {\isacharbraceleft}{\kern0pt}{\isacharbraceright}{\kern0pt}{\isachardoublequoteclose}\ \isacommand{using}\isamarkupfalse%
\ BfL{\isacharunderscore}{\kern0pt}props\ BfS{\isacharunderscore}{\kern0pt}props\ \isacommand{unfolding}\isamarkupfalse%
\ shiftset{\isacharunderscore}{\kern0pt}def\ \isacommand{by}\isamarkupfalse%
\ auto\isanewline
\ \ \ \ \isacommand{have}\isamarkupfalse%
\ fax{\isadigit{4}}{\isacharcolon}{\kern0pt}\ {\isachardoublequoteopen}{\isasymforall}i\ {\isasymin}\ {\isacharbraceleft}{\kern0pt}{\isachardot}{\kern0pt}{\isachardot}{\kern0pt}{\isacharless}{\kern0pt}k{\isacharplus}{\kern0pt}{\isadigit{1}}{\isacharbraceright}{\kern0pt}{\isachardot}{\kern0pt}\ {\isasymforall}j\ {\isasymin}\ {\isacharbraceleft}{\kern0pt}{\isachardot}{\kern0pt}{\isachardot}{\kern0pt}{\isacharless}{\kern0pt}k{\isacharplus}{\kern0pt}{\isadigit{1}}{\isacharbraceright}{\kern0pt}{\isachardot}{\kern0pt}\ i\ {\isasymnoteq}\ j\ {\isasymlongrightarrow}\ shiftset\ n\ {\isacharparenleft}{\kern0pt}BS\ i{\isacharparenright}{\kern0pt}\ {\isasyminter}\ shiftset\ n\ {\isacharparenleft}{\kern0pt}BS\ j{\isacharparenright}{\kern0pt}\ {\isacharequal}{\kern0pt}\ {\isacharbraceleft}{\kern0pt}{\isacharbraceright}{\kern0pt}{\isachardoublequoteclose}\ \isacommand{using}\isamarkupfalse%
\ shiftset{\isacharunderscore}{\kern0pt}disjoint{\isacharunderscore}{\kern0pt}family{\isacharbrackleft}{\kern0pt}of\ BS\ k{\isacharbrackright}{\kern0pt}\ BfS{\isacharunderscore}{\kern0pt}props\ \isacommand{unfolding}\isamarkupfalse%
\ disjoint{\isacharunderscore}{\kern0pt}family{\isacharunderscore}{\kern0pt}on{\isacharunderscore}{\kern0pt}def\ \isacommand{by}\isamarkupfalse%
\ simp\ \isanewline
\ \ \ \ \isacommand{have}\isamarkupfalse%
\ fax{\isadigit{5}}{\isacharcolon}{\kern0pt}\ {\isachardoublequoteopen}{\isasymforall}i\ {\isasymin}\ {\isacharbraceleft}{\kern0pt}{\isachardot}{\kern0pt}{\isachardot}{\kern0pt}{\isacharless}{\kern0pt}k{\isacharplus}{\kern0pt}{\isadigit{1}}{\isacharbraceright}{\kern0pt}{\isachardot}{\kern0pt}\ Bvar\ i\ {\isasyminter}\ Bstat\ {\isacharequal}{\kern0pt}\ {\isacharbraceleft}{\kern0pt}{\isacharbraceright}{\kern0pt}{\isachardoublequoteclose}\isanewline
\ \ \ \isacommand{proof}\isamarkupfalse%
\isanewline
\ \ \ \ \ \isacommand{fix}\isamarkupfalse%
\ i\ \isacommand{assume}\isamarkupfalse%
\ a{\isacharcolon}{\kern0pt}\ {\isachardoublequoteopen}i\ {\isasymin}\ {\isacharbraceleft}{\kern0pt}{\isachardot}{\kern0pt}{\isachardot}{\kern0pt}{\isacharless}{\kern0pt}k{\isacharplus}{\kern0pt}{\isadigit{1}}{\isacharbraceright}{\kern0pt}{\isachardoublequoteclose}\isanewline
\ \ \ \ \ \isacommand{show}\isamarkupfalse%
\ {\isachardoublequoteopen}Bvar\ i\ {\isasyminter}\ Bstat\ {\isacharequal}{\kern0pt}\ {\isacharbraceleft}{\kern0pt}{\isacharbraceright}{\kern0pt}{\isachardoublequoteclose}\isanewline
\ \ \ \ \ \isacommand{proof}\isamarkupfalse%
\ {\isacharparenleft}{\kern0pt}cases\ i{\isacharparenright}{\kern0pt}\isanewline
\ \ \ \ \ \ \ \isacommand{case}\isamarkupfalse%
\ {\isadigit{0}}\isanewline
\ \ \ \ \ \ \ \isacommand{then}\isamarkupfalse%
\ \isacommand{have}\isamarkupfalse%
\ {\isachardoublequoteopen}Bvar\ i\ {\isacharequal}{\kern0pt}\ BL\ {\isadigit{0}}{\isachardoublequoteclose}\ \isacommand{unfolding}\isamarkupfalse%
\ Bvar{\isacharunderscore}{\kern0pt}def\ \isacommand{by}\isamarkupfalse%
\ simp\isanewline
\ \ \ \ \ \ \ \isacommand{moreover}\isamarkupfalse%
\ \isacommand{have}\isamarkupfalse%
\ {\isachardoublequoteopen}BL\ {\isadigit{0}}\ {\isasyminter}\ BL\ {\isadigit{1}}\ {\isacharequal}{\kern0pt}\ {\isacharbraceleft}{\kern0pt}{\isacharbraceright}{\kern0pt}{\isachardoublequoteclose}\ \isacommand{using}\isamarkupfalse%
\ BfL{\isacharunderscore}{\kern0pt}props\ \isacommand{unfolding}\isamarkupfalse%
\ disjoint{\isacharunderscore}{\kern0pt}family{\isacharunderscore}{\kern0pt}on{\isacharunderscore}{\kern0pt}def\ \isacommand{by}\isamarkupfalse%
\ simp\isanewline
\ \ \ \ \ \ \ \isacommand{moreover}\isamarkupfalse%
\ \isacommand{have}\isamarkupfalse%
\ {\isachardoublequoteopen}shiftset\ n\ {\isacharparenleft}{\kern0pt}BS\ k{\isacharparenright}{\kern0pt}\ {\isasyminter}\ BL\ {\isadigit{0}}\ {\isacharequal}{\kern0pt}\ {\isacharbraceleft}{\kern0pt}{\isacharbraceright}{\kern0pt}{\isachardoublequoteclose}\ \isacommand{using}\isamarkupfalse%
\ BfL{\isacharunderscore}{\kern0pt}props\ BfS{\isacharunderscore}{\kern0pt}props\ \isacommand{unfolding}\isamarkupfalse%
\ shiftset{\isacharunderscore}{\kern0pt}def\ \isacommand{by}\isamarkupfalse%
\ auto\isanewline
\ \ \ \ \ \ \ \isacommand{ultimately}\isamarkupfalse%
\ \isacommand{show}\isamarkupfalse%
\ {\isacharquery}{\kern0pt}thesis\ \isacommand{unfolding}\isamarkupfalse%
\ Bstat{\isacharunderscore}{\kern0pt}def\ \isacommand{by}\isamarkupfalse%
\ blast\isanewline
\ \ \ \ \ \isacommand{next}\isamarkupfalse%
\isanewline
\ \ \ \ \ \ \ \isacommand{case}\isamarkupfalse%
\ {\isacharparenleft}{\kern0pt}Suc\ nat{\isacharparenright}{\kern0pt}\isanewline
\ \ \ \ \ \ \ \isacommand{then}\isamarkupfalse%
\ \isacommand{have}\isamarkupfalse%
\ {\isachardoublequoteopen}Bvar\ i\ {\isacharequal}{\kern0pt}\ shiftset\ n\ {\isacharparenleft}{\kern0pt}BS\ nat{\isacharparenright}{\kern0pt}{\isachardoublequoteclose}\ \isacommand{unfolding}\isamarkupfalse%
\ Bvar{\isacharunderscore}{\kern0pt}def\ \isacommand{by}\isamarkupfalse%
\ simp\isanewline
\ \ \ \ \ \ \ \isacommand{moreover}\isamarkupfalse%
\ \isacommand{have}\isamarkupfalse%
\ {\isachardoublequoteopen}shiftset\ n\ {\isacharparenleft}{\kern0pt}BS\ nat{\isacharparenright}{\kern0pt}\ {\isasyminter}\ BL\ {\isadigit{1}}\ {\isacharequal}{\kern0pt}\ {\isacharbraceleft}{\kern0pt}{\isacharbraceright}{\kern0pt}{\isachardoublequoteclose}\ \isacommand{using}\isamarkupfalse%
\ BfS{\isacharunderscore}{\kern0pt}props\ BfL{\isacharunderscore}{\kern0pt}props\ a\ Suc\ \isacommand{unfolding}\isamarkupfalse%
\ shiftset{\isacharunderscore}{\kern0pt}def\ \isacommand{by}\isamarkupfalse%
\ auto\isanewline
\ \ \ \ \ \ \ \isacommand{moreover}\isamarkupfalse%
\ \isacommand{have}\isamarkupfalse%
\ {\isachardoublequoteopen}shiftset\ n\ {\isacharparenleft}{\kern0pt}BS\ nat{\isacharparenright}{\kern0pt}\ {\isasyminter}\ shiftset\ n\ {\isacharparenleft}{\kern0pt}BS\ k{\isacharparenright}{\kern0pt}\ {\isacharequal}{\kern0pt}\ {\isacharbraceleft}{\kern0pt}{\isacharbraceright}{\kern0pt}{\isachardoublequoteclose}\ \isacommand{using}\isamarkupfalse%
\ a\ Suc\ fax{\isadigit{4}}\ \isacommand{by}\isamarkupfalse%
\ simp\isanewline
\ \ \ \ \ \ \ \isacommand{ultimately}\isamarkupfalse%
\ \isacommand{show}\isamarkupfalse%
\ {\isacharquery}{\kern0pt}thesis\ \isacommand{unfolding}\isamarkupfalse%
\ Bstat{\isacharunderscore}{\kern0pt}def\ \isacommand{by}\isamarkupfalse%
\ blast\isanewline
\ \ \ \ \ \isacommand{qed}\isamarkupfalse%
\isanewline
\ \ \ \isacommand{qed}\isamarkupfalse%
\isanewline
\isanewline
\ \ \ \isacommand{have}\isamarkupfalse%
\ shiftsetnotempty{\isacharcolon}{\kern0pt}\ {\isachardoublequoteopen}{\isasymforall}n\ i{\isachardot}{\kern0pt}\ BS\ i\ {\isasymnoteq}\ {\isacharbraceleft}{\kern0pt}{\isacharbraceright}{\kern0pt}\ {\isasymlongrightarrow}\ shiftset\ n\ {\isacharparenleft}{\kern0pt}BS\ i{\isacharparenright}{\kern0pt}\ {\isasymnoteq}\ {\isacharbraceleft}{\kern0pt}{\isacharbraceright}{\kern0pt}{\isachardoublequoteclose}\ \isacommand{unfolding}\isamarkupfalse%
\ shiftset{\isacharunderscore}{\kern0pt}def\ \isacommand{by}\isamarkupfalse%
\ blast\isanewline
\isanewline
\ \ \isanewline
\ \ \ \ \isacommand{have}\isamarkupfalse%
\ {\isachardoublequoteopen}Bvar\ {\isacharbackquote}{\kern0pt}\ {\isacharbraceleft}{\kern0pt}{\isachardot}{\kern0pt}{\isachardot}{\kern0pt}{\isacharless}{\kern0pt}k{\isacharplus}{\kern0pt}{\isadigit{1}}{\isacharbraceright}{\kern0pt}\ {\isacharequal}{\kern0pt}\ BL\ {\isacharbackquote}{\kern0pt}\ {\isacharbraceleft}{\kern0pt}{\isachardot}{\kern0pt}{\isachardot}{\kern0pt}{\isacharless}{\kern0pt}{\isadigit{1}}{\isacharbraceright}{\kern0pt}\ {\isasymunion}\ Bvar\ {\isacharbackquote}{\kern0pt}\ {\isacharbraceleft}{\kern0pt}{\isadigit{1}}{\isachardot}{\kern0pt}{\isachardot}{\kern0pt}{\isacharless}{\kern0pt}k{\isacharplus}{\kern0pt}{\isadigit{1}}{\isacharbraceright}{\kern0pt}{\isachardoublequoteclose}\ \isacommand{unfolding}\isamarkupfalse%
\ Bvar{\isacharunderscore}{\kern0pt}def\ \isacommand{by}\isamarkupfalse%
\ force\isanewline
\ \ \ \ \isacommand{also}\isamarkupfalse%
\ \isacommand{have}\isamarkupfalse%
\ {\isachardoublequoteopen}\ {\isachardot}{\kern0pt}{\isachardot}{\kern0pt}{\isachardot}{\kern0pt}\ {\isacharequal}{\kern0pt}\ BL\ {\isacharbackquote}{\kern0pt}\ {\isacharbraceleft}{\kern0pt}{\isachardot}{\kern0pt}{\isachardot}{\kern0pt}{\isacharless}{\kern0pt}{\isadigit{1}}{\isacharbraceright}{\kern0pt}\ {\isasymunion}\ {\isacharbraceleft}{\kern0pt}shiftset\ n\ {\isacharparenleft}{\kern0pt}BS\ i{\isacharparenright}{\kern0pt}\ {\isacharbar}{\kern0pt}\ i\ {\isachardot}{\kern0pt}\ i\ {\isasymin}\ {\isacharbraceleft}{\kern0pt}{\isachardot}{\kern0pt}{\isachardot}{\kern0pt}{\isacharless}{\kern0pt}k{\isacharbraceright}{\kern0pt}{\isacharbraceright}{\kern0pt}\ {\isachardoublequoteclose}\ \isacommand{unfolding}\isamarkupfalse%
\ Bvar{\isacharunderscore}{\kern0pt}def\ \isacommand{by}\isamarkupfalse%
\ fastforce\ \ \isanewline
\ \ \ \ \isacommand{moreover}\isamarkupfalse%
\ \isacommand{have}\isamarkupfalse%
\ {\isachardoublequoteopen}{\isacharbraceleft}{\kern0pt}{\isacharbraceright}{\kern0pt}\ {\isasymnotin}\ BL\ {\isacharbackquote}{\kern0pt}\ {\isacharbraceleft}{\kern0pt}{\isachardot}{\kern0pt}{\isachardot}{\kern0pt}{\isacharless}{\kern0pt}{\isadigit{1}}{\isacharbraceright}{\kern0pt}{\isachardoublequoteclose}\ \isacommand{using}\isamarkupfalse%
\ BfL{\isacharunderscore}{\kern0pt}props\ \isacommand{by}\isamarkupfalse%
\ auto\isanewline
\ \ \ \ \isacommand{moreover}\isamarkupfalse%
\ \isacommand{have}\isamarkupfalse%
\ {\isachardoublequoteopen}{\isacharbraceleft}{\kern0pt}{\isacharbraceright}{\kern0pt}\ {\isasymnotin}\ {\isacharbraceleft}{\kern0pt}shiftset\ n\ {\isacharparenleft}{\kern0pt}BS\ i{\isacharparenright}{\kern0pt}\ {\isacharbar}{\kern0pt}\ i\ {\isachardot}{\kern0pt}\ i\ {\isasymin}\ {\isacharbraceleft}{\kern0pt}{\isachardot}{\kern0pt}{\isachardot}{\kern0pt}{\isacharless}{\kern0pt}k{\isacharbraceright}{\kern0pt}{\isacharbraceright}{\kern0pt}{\isachardoublequoteclose}\ \isacommand{using}\isamarkupfalse%
\ BfS{\isacharunderscore}{\kern0pt}props{\isacharparenleft}{\kern0pt}{\isadigit{2}}{\isacharcomma}{\kern0pt}\ {\isadigit{3}}{\isacharparenright}{\kern0pt}\ shiftsetnotempty\ \isacommand{by}\isamarkupfalse%
\ fastforce\isanewline
\ \ \ \ \isacommand{ultimately}\isamarkupfalse%
\ \isacommand{have}\isamarkupfalse%
\ {\isachardoublequoteopen}{\isacharbraceleft}{\kern0pt}{\isacharbraceright}{\kern0pt}\ {\isasymnotin}\ Bvar\ {\isacharbackquote}{\kern0pt}\ {\isacharbraceleft}{\kern0pt}{\isachardot}{\kern0pt}{\isachardot}{\kern0pt}{\isacharless}{\kern0pt}k{\isacharplus}{\kern0pt}{\isadigit{1}}{\isacharbraceright}{\kern0pt}{\isachardoublequoteclose}\ \isacommand{by}\isamarkupfalse%
\ simp\isanewline
\ \ \ \ \isacommand{then}\isamarkupfalse%
\ \isacommand{have}\isamarkupfalse%
\ F{\isadigit{1}}{\isacharcolon}{\kern0pt}\ {\isachardoublequoteopen}{\isacharbraceleft}{\kern0pt}{\isacharbraceright}{\kern0pt}\ {\isasymnotin}\ BT\ {\isacharbackquote}{\kern0pt}\ {\isacharbraceleft}{\kern0pt}{\isachardot}{\kern0pt}{\isachardot}{\kern0pt}{\isacharless}{\kern0pt}k{\isacharplus}{\kern0pt}{\isadigit{1}}{\isacharbraceright}{\kern0pt}{\isachardoublequoteclose}\ \isacommand{unfolding}\isamarkupfalse%
\ BT{\isacharunderscore}{\kern0pt}def\ \isacommand{by}\isamarkupfalse%
\ simp\isanewline
\isanewline
\ \ \ \ \isacommand{have}\isamarkupfalse%
\ F{\isadigit{2}}{\isacharunderscore}{\kern0pt}aux{\isacharcolon}{\kern0pt}\ {\isachardoublequoteopen}disjoint{\isacharunderscore}{\kern0pt}family{\isacharunderscore}{\kern0pt}on\ Bvar\ {\isacharbraceleft}{\kern0pt}{\isachardot}{\kern0pt}{\isachardot}{\kern0pt}{\isacharless}{\kern0pt}k{\isacharplus}{\kern0pt}{\isadigit{1}}{\isacharbraceright}{\kern0pt}{\isachardoublequoteclose}\isanewline
\ \ \ \ \isacommand{proof}\isamarkupfalse%
\ {\isacharparenleft}{\kern0pt}unfold\ disjoint{\isacharunderscore}{\kern0pt}family{\isacharunderscore}{\kern0pt}on{\isacharunderscore}{\kern0pt}def{\isacharsemicolon}{\kern0pt}\ safe{\isacharparenright}{\kern0pt}\isanewline
\ \ \ \ \ \ \isacommand{fix}\isamarkupfalse%
\ m\ n\ x\ \isacommand{assume}\isamarkupfalse%
\ a{\isacharcolon}{\kern0pt}\ {\isachardoublequoteopen}m\ {\isacharless}{\kern0pt}\ k\ {\isacharplus}{\kern0pt}\ {\isadigit{1}}{\isachardoublequoteclose}\ {\isachardoublequoteopen}n\ {\isacharless}{\kern0pt}\ k\ {\isacharplus}{\kern0pt}\ {\isadigit{1}}{\isachardoublequoteclose}\ {\isachardoublequoteopen}m\ {\isasymnoteq}\ n{\isachardoublequoteclose}\ {\isachardoublequoteopen}x\ {\isasymin}\ Bvar\ m{\isachardoublequoteclose}\ {\isachardoublequoteopen}x\ {\isasymin}\ Bvar\ n{\isachardoublequoteclose}\isanewline
\ \ \ \ \ \ \isacommand{show}\isamarkupfalse%
\ {\isachardoublequoteopen}x\ {\isasymin}\ {\isacharbraceleft}{\kern0pt}{\isacharbraceright}{\kern0pt}{\isachardoublequoteclose}\isanewline
\ \ \ \ \ \ \isacommand{proof}\isamarkupfalse%
\ {\isacharparenleft}{\kern0pt}cases\ {\isachardoublequoteopen}n{\isachardoublequoteclose}{\isacharparenright}{\kern0pt}\isanewline
\ \ \ \ \ \ \ \ \isacommand{case}\isamarkupfalse%
\ {\isadigit{0}}\isanewline
\ \ \ \ \ \ \ \ \isacommand{then}\isamarkupfalse%
\ \isacommand{show}\isamarkupfalse%
\ {\isacharquery}{\kern0pt}thesis\ \isacommand{using}\isamarkupfalse%
\ a\ fax{\isadigit{3}}\ \isacommand{unfolding}\isamarkupfalse%
\ Bvar{\isacharunderscore}{\kern0pt}def\ \isacommand{by}\isamarkupfalse%
\ auto\isanewline
\ \ \ \ \ \ \isacommand{next}\isamarkupfalse%
\isanewline
\ \ \ \ \ \ \ \ \isacommand{case}\isamarkupfalse%
\ {\isacharparenleft}{\kern0pt}Suc\ nnat{\isacharparenright}{\kern0pt}\isanewline
\ \ \ \ \ \ \ \ \isacommand{then}\isamarkupfalse%
\ \isacommand{have}\isamarkupfalse%
\ {\isacharasterisk}{\kern0pt}{\isacharcolon}{\kern0pt}\ {\isachardoublequoteopen}n\ {\isacharequal}{\kern0pt}\ Suc\ nnat{\isachardoublequoteclose}\ \isacommand{by}\isamarkupfalse%
\ simp\isanewline
\ \ \ \ \ \ \ \ \isacommand{then}\isamarkupfalse%
\ \isacommand{show}\isamarkupfalse%
\ {\isacharquery}{\kern0pt}thesis\ \isanewline
\ \ \ \ \ \ \ \ \isacommand{proof}\isamarkupfalse%
\ {\isacharparenleft}{\kern0pt}cases\ m{\isacharparenright}{\kern0pt}\isanewline
\ \ \ \ \ \ \ \ \ \ \isacommand{case}\isamarkupfalse%
\ {\isadigit{0}}\isanewline
\ \ \ \ \ \ \ \ \ \ \isacommand{then}\isamarkupfalse%
\ \isacommand{show}\isamarkupfalse%
\ {\isacharquery}{\kern0pt}thesis\ \isacommand{using}\isamarkupfalse%
\ a\ fax{\isadigit{3}}\ \isacommand{unfolding}\isamarkupfalse%
\ Bvar{\isacharunderscore}{\kern0pt}def\ \isacommand{by}\isamarkupfalse%
\ auto\isanewline
\ \ \ \ \ \ \ \ \isacommand{next}\isamarkupfalse%
\isanewline
\ \ \ \ \ \ \ \ \ \ \isacommand{case}\isamarkupfalse%
\ {\isacharparenleft}{\kern0pt}Suc\ mnat{\isacharparenright}{\kern0pt}\isanewline
\ \ \ \ \ \ \ \ \ \ \isacommand{then}\isamarkupfalse%
\ \isacommand{show}\isamarkupfalse%
\ {\isacharquery}{\kern0pt}thesis\ \isacommand{using}\isamarkupfalse%
\ a\ fax{\isadigit{4}}\ \ {\isacharasterisk}{\kern0pt}\ \isacommand{unfolding}\isamarkupfalse%
\ Bvar{\isacharunderscore}{\kern0pt}def\ \isacommand{by}\isamarkupfalse%
\ fastforce\isanewline
\ \ \ \ \ \ \ \ \isacommand{qed}\isamarkupfalse%
\isanewline
\ \ \ \ \ \ \isacommand{qed}\isamarkupfalse%
\isanewline
\ \ \ \isacommand{qed}\isamarkupfalse%
\isanewline
\isanewline
\ \ \ \isacommand{have}\isamarkupfalse%
\ F{\isadigit{2}}{\isacharcolon}{\kern0pt}\ {\isachardoublequoteopen}disjoint{\isacharunderscore}{\kern0pt}family{\isacharunderscore}{\kern0pt}on\ BT\ {\isacharbraceleft}{\kern0pt}{\isachardot}{\kern0pt}{\isachardot}{\kern0pt}k{\isacharplus}{\kern0pt}{\isadigit{1}}{\isacharbraceright}{\kern0pt}{\isachardoublequoteclose}\isanewline
\ \ \ \isacommand{proof}\isamarkupfalse%
\isanewline
\ \ \ \ \ \isacommand{fix}\isamarkupfalse%
\ m\ n\ \isacommand{assume}\isamarkupfalse%
\ a{\isacharcolon}{\kern0pt}\ {\isachardoublequoteopen}m{\isasymin}{\isacharbraceleft}{\kern0pt}{\isachardot}{\kern0pt}{\isachardot}{\kern0pt}k{\isacharplus}{\kern0pt}{\isadigit{1}}{\isacharbraceright}{\kern0pt}{\isachardoublequoteclose}\ {\isachardoublequoteopen}n{\isasymin}{\isacharbraceleft}{\kern0pt}{\isachardot}{\kern0pt}{\isachardot}{\kern0pt}k{\isacharplus}{\kern0pt}{\isadigit{1}}{\isacharbraceright}{\kern0pt}{\isachardoublequoteclose}\ {\isachardoublequoteopen}m\ {\isasymnoteq}\ n{\isachardoublequoteclose}\isanewline
\ \ \ \ \ \isacommand{have}\isamarkupfalse%
\ {\isachardoublequoteopen}{\isasymforall}x{\isachardot}{\kern0pt}\ x\ {\isasymin}\ BT\ m\ {\isasyminter}\ BT\ n\ {\isasymlongrightarrow}\ x\ {\isasymin}\ {\isacharbraceleft}{\kern0pt}{\isacharbraceright}{\kern0pt}{\isachardoublequoteclose}\ \isanewline
\ \ \ \ \ \isacommand{proof}\isamarkupfalse%
\ {\isacharparenleft}{\kern0pt}intro\ allI\ impI{\isacharparenright}{\kern0pt}\isanewline
\ \ \ \ \ \ \ \isacommand{fix}\isamarkupfalse%
\ x\ \isacommand{assume}\isamarkupfalse%
\ b{\isacharcolon}{\kern0pt}\ {\isachardoublequoteopen}x\ {\isasymin}\ BT\ m\ {\isasyminter}\ BT\ n{\isachardoublequoteclose}\isanewline
\ \ \ \ \ \ \ \isacommand{have}\isamarkupfalse%
\ {\isachardoublequoteopen}m\ {\isacharless}{\kern0pt}\ k\ {\isacharplus}{\kern0pt}\ {\isadigit{1}}\ {\isasymand}\ n\ {\isacharless}{\kern0pt}\ k\ {\isacharplus}{\kern0pt}\ {\isadigit{1}}\ {\isasymor}\ m\ {\isacharequal}{\kern0pt}\ k\ {\isacharplus}{\kern0pt}\ {\isadigit{1}}\ {\isasymand}\ n\ {\isacharequal}{\kern0pt}\ k\ {\isacharplus}{\kern0pt}\ {\isadigit{1}}\ {\isasymor}\ m\ {\isacharless}{\kern0pt}\ k\ {\isacharplus}{\kern0pt}\ {\isadigit{1}}\ {\isasymand}\ n\ {\isacharequal}{\kern0pt}\ k\ {\isacharplus}{\kern0pt}\ {\isadigit{1}}\ {\isasymor}\ m\ {\isacharequal}{\kern0pt}\ k\ {\isacharplus}{\kern0pt}\ {\isadigit{1}}\ {\isasymand}\ n\ {\isacharless}{\kern0pt}\ k\ {\isacharplus}{\kern0pt}\ {\isadigit{1}}{\isachardoublequoteclose}\ \isacommand{using}\isamarkupfalse%
\ a\ le{\isacharunderscore}{\kern0pt}eq{\isacharunderscore}{\kern0pt}less{\isacharunderscore}{\kern0pt}or{\isacharunderscore}{\kern0pt}eq\ \isacommand{by}\isamarkupfalse%
\ auto\isanewline
\ \ \ \ \ \ \ \isacommand{then}\isamarkupfalse%
\ \isacommand{show}\isamarkupfalse%
\ {\isachardoublequoteopen}x\ {\isasymin}\ {\isacharbraceleft}{\kern0pt}{\isacharbraceright}{\kern0pt}{\isachardoublequoteclose}\isanewline
\ \ \ \ \ \ \ \isacommand{proof}\isamarkupfalse%
\ {\isacharparenleft}{\kern0pt}elim\ disjE{\isacharparenright}{\kern0pt}\isanewline
\ \ \ \ \ \ \ \ \ \isacommand{assume}\isamarkupfalse%
\ c{\isacharcolon}{\kern0pt}\ {\isachardoublequoteopen}m\ {\isacharless}{\kern0pt}\ k\ {\isacharplus}{\kern0pt}\ {\isadigit{1}}\ {\isasymand}\ n\ {\isacharless}{\kern0pt}\ k\ {\isacharplus}{\kern0pt}\ {\isadigit{1}}{\isachardoublequoteclose}\isanewline
\ \ \ \ \ \ \ \ \ \isacommand{then}\isamarkupfalse%
\ \isacommand{have}\isamarkupfalse%
\ {\isachardoublequoteopen}BT\ m\ {\isacharequal}{\kern0pt}\ Bvar\ m\ {\isasymand}\ BT\ n\ {\isacharequal}{\kern0pt}\ Bvar\ n{\isachardoublequoteclose}\ \isacommand{unfolding}\isamarkupfalse%
\ BT{\isacharunderscore}{\kern0pt}def\ \isacommand{by}\isamarkupfalse%
\ simp\isanewline
\ \ \ \ \ \ \ \ \ \isacommand{then}\isamarkupfalse%
\ \isacommand{show}\isamarkupfalse%
\ {\isachardoublequoteopen}x\ {\isasymin}\ {\isacharbraceleft}{\kern0pt}{\isacharbraceright}{\kern0pt}{\isachardoublequoteclose}\ \isacommand{using}\isamarkupfalse%
\ a\ b\ c\ fax{\isadigit{4}}\ F{\isadigit{2}}{\isacharunderscore}{\kern0pt}aux\ \isacommand{unfolding}\isamarkupfalse%
\ Bvar{\isacharunderscore}{\kern0pt}def\ disjoint{\isacharunderscore}{\kern0pt}family{\isacharunderscore}{\kern0pt}on{\isacharunderscore}{\kern0pt}def\ \isacommand{by}\isamarkupfalse%
\ auto\isanewline
\ \ \ \ \ \ \ \isacommand{qed}\isamarkupfalse%
\ {\isacharparenleft}{\kern0pt}use\ a\ b\ fax{\isadigit{5}}\ \isakeyword{in}\ {\isacartoucheopen}auto\ simp{\isacharcolon}{\kern0pt}\ BT{\isacharunderscore}{\kern0pt}def{\isacartoucheclose}{\isacharparenright}{\kern0pt}\isanewline
\ \ \ \ \ \isacommand{qed}\isamarkupfalse%
\isanewline
\ \ \ \ \ \isacommand{then}\isamarkupfalse%
\ \isacommand{show}\isamarkupfalse%
\ {\isachardoublequoteopen}BT\ m\ {\isasyminter}\ BT\ n\ {\isacharequal}{\kern0pt}\ {\isacharbraceleft}{\kern0pt}{\isacharbraceright}{\kern0pt}{\isachardoublequoteclose}\ \isacommand{by}\isamarkupfalse%
\ auto\isanewline
\ \ \ \isacommand{qed}\isamarkupfalse%
\isanewline
\isanewline
\isanewline
\ \ \ \isacommand{have}\isamarkupfalse%
\ F{\isadigit{3}}{\isacharcolon}{\kern0pt}\ {\isachardoublequoteopen}{\isasymUnion}{\isacharparenleft}{\kern0pt}BT\ {\isacharbackquote}{\kern0pt}\ {\isacharbraceleft}{\kern0pt}{\isachardot}{\kern0pt}{\isachardot}{\kern0pt}k{\isacharplus}{\kern0pt}{\isadigit{1}}{\isacharbraceright}{\kern0pt}{\isacharparenright}{\kern0pt}\ {\isacharequal}{\kern0pt}\ {\isacharbraceleft}{\kern0pt}{\isachardot}{\kern0pt}{\isachardot}{\kern0pt}{\isacharless}{\kern0pt}n\ {\isacharplus}{\kern0pt}\ m{\isacharbraceright}{\kern0pt}{\isachardoublequoteclose}\isanewline
\ \ \ \isacommand{proof}\isamarkupfalse%
\ \isanewline
\ \ \ \ \ \isacommand{show}\isamarkupfalse%
\ {\isachardoublequoteopen}{\isasymUnion}\ {\isacharparenleft}{\kern0pt}BT\ {\isacharbackquote}{\kern0pt}\ {\isacharbraceleft}{\kern0pt}{\isachardot}{\kern0pt}{\isachardot}{\kern0pt}k\ {\isacharplus}{\kern0pt}\ {\isadigit{1}}{\isacharbraceright}{\kern0pt}{\isacharparenright}{\kern0pt}\ {\isasymsubseteq}\ {\isacharbraceleft}{\kern0pt}{\isachardot}{\kern0pt}{\isachardot}{\kern0pt}{\isacharless}{\kern0pt}n\ {\isacharplus}{\kern0pt}\ m{\isacharbraceright}{\kern0pt}{\isachardoublequoteclose}\isanewline
\ \ \ \ \ \isacommand{proof}\isamarkupfalse%
\isanewline
\ \ \ \ \ \ \ \isacommand{fix}\isamarkupfalse%
\ x\ \isacommand{assume}\isamarkupfalse%
\ {\isachardoublequoteopen}x\ {\isasymin}\ {\isasymUnion}\ {\isacharparenleft}{\kern0pt}BT\ {\isacharbackquote}{\kern0pt}\ {\isacharbraceleft}{\kern0pt}{\isachardot}{\kern0pt}{\isachardot}{\kern0pt}k\ {\isacharplus}{\kern0pt}\ {\isadigit{1}}{\isacharbraceright}{\kern0pt}{\isacharparenright}{\kern0pt}{\isachardoublequoteclose}\isanewline
\ \ \ \ \ \ \ \isacommand{then}\isamarkupfalse%
\ \isacommand{obtain}\isamarkupfalse%
\ i\ \isakeyword{where}\ i{\isacharunderscore}{\kern0pt}prop{\isacharcolon}{\kern0pt}\ {\isachardoublequoteopen}i\ {\isasymin}\ {\isacharbraceleft}{\kern0pt}{\isachardot}{\kern0pt}{\isachardot}{\kern0pt}k{\isacharplus}{\kern0pt}{\isadigit{1}}{\isacharbraceright}{\kern0pt}\ {\isasymand}\ x\ {\isasymin}\ BT\ i{\isachardoublequoteclose}\ \isacommand{by}\isamarkupfalse%
\ blast\isanewline
\ \ \ \ \ \ \ \isacommand{then}\isamarkupfalse%
\ \isacommand{consider}\isamarkupfalse%
\ {\isachardoublequoteopen}i\ {\isacharequal}{\kern0pt}\ k\ {\isacharplus}{\kern0pt}{\isadigit{1}}{\isachardoublequoteclose}\ {\isacharbar}{\kern0pt}\ {\isachardoublequoteopen}i\ {\isasymin}\ {\isacharbraceleft}{\kern0pt}{\isachardot}{\kern0pt}{\isachardot}{\kern0pt}{\isacharless}{\kern0pt}k{\isacharplus}{\kern0pt}{\isadigit{1}}{\isacharbraceright}{\kern0pt}{\isachardoublequoteclose}\ \isacommand{by}\isamarkupfalse%
\ fastforce\isanewline
\ \ \ \ \ \ \ \isacommand{then}\isamarkupfalse%
\ \isacommand{show}\isamarkupfalse%
\ {\isachardoublequoteopen}x\ {\isasymin}\ {\isacharbraceleft}{\kern0pt}{\isachardot}{\kern0pt}{\isachardot}{\kern0pt}{\isacharless}{\kern0pt}n\ {\isacharplus}{\kern0pt}\ m{\isacharbraceright}{\kern0pt}{\isachardoublequoteclose}\isanewline
\ \ \ \ \ \ \ \isacommand{proof}\isamarkupfalse%
\ {\isacharparenleft}{\kern0pt}cases{\isacharparenright}{\kern0pt}\isanewline
\ \ \ \ \ \ \ \ \ \isacommand{case}\isamarkupfalse%
\ {\isadigit{1}}\isanewline
\ \ \ \ \ \ \ \ \ \isacommand{then}\isamarkupfalse%
\ \isacommand{have}\isamarkupfalse%
\ {\isachardoublequoteopen}x\ {\isasymin}\ Bstat{\isachardoublequoteclose}\ \isacommand{using}\isamarkupfalse%
\ i{\isacharunderscore}{\kern0pt}prop\ \isacommand{unfolding}\isamarkupfalse%
\ BT{\isacharunderscore}{\kern0pt}def\ \isacommand{by}\isamarkupfalse%
\ simp\isanewline
\ \ \ \ \ \ \ \ \ \isacommand{then}\isamarkupfalse%
\ \isacommand{have}\isamarkupfalse%
\ {\isachardoublequoteopen}x\ {\isasymin}\ BL\ {\isadigit{1}}\ {\isasymor}\ x\ {\isasymin}\ shiftset\ n\ {\isacharparenleft}{\kern0pt}BS\ k{\isacharparenright}{\kern0pt}{\isachardoublequoteclose}\ \isacommand{unfolding}\isamarkupfalse%
\ Bstat{\isacharunderscore}{\kern0pt}def\ \isacommand{by}\isamarkupfalse%
\ blast\isanewline
\ \ \ \ \ \ \ \ \ \isacommand{then}\isamarkupfalse%
\ \isacommand{have}\isamarkupfalse%
\ {\isachardoublequoteopen}x\ {\isasymin}\ {\isacharbraceleft}{\kern0pt}{\isachardot}{\kern0pt}{\isachardot}{\kern0pt}{\isacharless}{\kern0pt}n{\isacharbraceright}{\kern0pt}\ {\isasymor}\ x\ {\isasymin}\ {\isacharbraceleft}{\kern0pt}n{\isachardot}{\kern0pt}{\isachardot}{\kern0pt}{\isacharless}{\kern0pt}n{\isacharplus}{\kern0pt}m{\isacharbraceright}{\kern0pt}{\isachardoublequoteclose}\ \isacommand{using}\isamarkupfalse%
\ BfL{\isacharunderscore}{\kern0pt}props\ BfS{\isacharunderscore}{\kern0pt}props{\isacharparenleft}{\kern0pt}{\isadigit{2}}{\isacharparenright}{\kern0pt}\ shiftset{\isacharunderscore}{\kern0pt}image{\isacharbrackleft}{\kern0pt}of\ BS\ k\ m\ n{\isacharbrackright}{\kern0pt}\ \isacommand{by}\isamarkupfalse%
\ blast\isanewline
\ \ \ \ \ \ \ \ \ \isacommand{then}\isamarkupfalse%
\ \isacommand{show}\isamarkupfalse%
\ {\isacharquery}{\kern0pt}thesis\ \isacommand{by}\isamarkupfalse%
\ auto\isanewline
\ \ \ \ \ \ \ \isacommand{next}\isamarkupfalse%
\isanewline
\ \ \ \ \ \ \ \ \ \isacommand{case}\isamarkupfalse%
\ {\isadigit{2}}\isanewline
\ \ \ \ \ \ \ \ \ \isacommand{then}\isamarkupfalse%
\ \isacommand{have}\isamarkupfalse%
\ {\isachardoublequoteopen}x\ {\isasymin}\ Bvar\ i{\isachardoublequoteclose}\ \isacommand{using}\isamarkupfalse%
\ i{\isacharunderscore}{\kern0pt}prop\ \isacommand{unfolding}\isamarkupfalse%
\ BT{\isacharunderscore}{\kern0pt}def\ \isacommand{by}\isamarkupfalse%
\ simp\isanewline
\ \ \ \ \ \ \ \ \ \isacommand{then}\isamarkupfalse%
\ \isacommand{have}\isamarkupfalse%
\ {\isachardoublequoteopen}x\ {\isasymin}\ BL\ {\isadigit{0}}\ {\isasymor}\ x\ {\isasymin}\ shiftset\ n\ {\isacharparenleft}{\kern0pt}BS\ {\isacharparenleft}{\kern0pt}i\ {\isacharminus}{\kern0pt}\ {\isadigit{1}}{\isacharparenright}{\kern0pt}{\isacharparenright}{\kern0pt}{\isachardoublequoteclose}\ \isacommand{unfolding}\isamarkupfalse%
\ Bvar{\isacharunderscore}{\kern0pt}def\ \isacommand{by}\isamarkupfalse%
\ presburger\isanewline
\ \ \ \ \ \ \ \ \ \isacommand{then}\isamarkupfalse%
\ \isacommand{show}\isamarkupfalse%
\ {\isacharquery}{\kern0pt}thesis\isanewline
\ \ \ \ \ \ \ \ \ \isacommand{proof}\isamarkupfalse%
\ {\isacharparenleft}{\kern0pt}elim\ disjE{\isacharparenright}{\kern0pt}\isanewline
\ \ \ \ \ \ \ \ \ \ \ \isacommand{assume}\isamarkupfalse%
\ {\isachardoublequoteopen}x\ {\isasymin}\ BL\ {\isadigit{0}}{\isachardoublequoteclose}\isanewline
\ \ \ \ \ \ \ \ \ \ \ \isacommand{then}\isamarkupfalse%
\ \isacommand{have}\isamarkupfalse%
\ {\isachardoublequoteopen}x\ {\isasymin}\ {\isacharbraceleft}{\kern0pt}{\isachardot}{\kern0pt}{\isachardot}{\kern0pt}{\isacharless}{\kern0pt}n{\isacharbraceright}{\kern0pt}{\isachardoublequoteclose}\ \isacommand{using}\isamarkupfalse%
\ BfL{\isacharunderscore}{\kern0pt}props\ \isacommand{by}\isamarkupfalse%
\ auto\isanewline
\ \ \ \ \ \ \ \ \ \ \ \isacommand{then}\isamarkupfalse%
\ \isacommand{show}\isamarkupfalse%
\ {\isachardoublequoteopen}x\ {\isasymin}\ {\isacharbraceleft}{\kern0pt}{\isachardot}{\kern0pt}{\isachardot}{\kern0pt}{\isacharless}{\kern0pt}n\ {\isacharplus}{\kern0pt}\ m{\isacharbraceright}{\kern0pt}{\isachardoublequoteclose}\ \isacommand{by}\isamarkupfalse%
\ simp\isanewline
\ \ \ \ \ \ \ \ \ \isacommand{next}\isamarkupfalse%
\isanewline
\ \ \ \ \ \ \ \ \ \ \ \isacommand{assume}\isamarkupfalse%
\ a{\isacharcolon}{\kern0pt}\ {\isachardoublequoteopen}x\ {\isasymin}\ shiftset\ n\ {\isacharparenleft}{\kern0pt}BS\ {\isacharparenleft}{\kern0pt}i\ {\isacharminus}{\kern0pt}\ {\isadigit{1}}{\isacharparenright}{\kern0pt}{\isacharparenright}{\kern0pt}{\isachardoublequoteclose}\isanewline
\ \ \ \ \ \ \ \ \ \ \ \isacommand{then}\isamarkupfalse%
\ \isacommand{have}\isamarkupfalse%
\ {\isachardoublequoteopen}i\ {\isacharminus}{\kern0pt}\ {\isadigit{1}}\ {\isasymle}\ k{\isachardoublequoteclose}\ \isanewline
\ \ \ \ \ \ \ \ \ \ \ \ \ \isacommand{by}\isamarkupfalse%
\ {\isacharparenleft}{\kern0pt}meson\ atMost{\isacharunderscore}{\kern0pt}iff\ i{\isacharunderscore}{\kern0pt}prop\ le{\isacharunderscore}{\kern0pt}diff{\isacharunderscore}{\kern0pt}conv{\isacharparenright}{\kern0pt}\ \isanewline
\ \ \ \ \ \ \ \ \ \ \ \isacommand{then}\isamarkupfalse%
\ \isacommand{have}\isamarkupfalse%
\ {\isachardoublequoteopen}shiftset\ n\ {\isacharparenleft}{\kern0pt}BS\ {\isacharparenleft}{\kern0pt}i\ {\isacharminus}{\kern0pt}\ {\isadigit{1}}{\isacharparenright}{\kern0pt}{\isacharparenright}{\kern0pt}\ {\isasymsubseteq}\ {\isacharbraceleft}{\kern0pt}n{\isachardot}{\kern0pt}{\isachardot}{\kern0pt}{\isacharless}{\kern0pt}n{\isacharplus}{\kern0pt}m{\isacharbraceright}{\kern0pt}{\isachardoublequoteclose}\ \isacommand{using}\isamarkupfalse%
\ shiftset{\isacharunderscore}{\kern0pt}image{\isacharbrackleft}{\kern0pt}of\ BS\ k\ m\ n{\isacharbrackright}{\kern0pt}\ BfS{\isacharunderscore}{\kern0pt}props\ \isacommand{by}\isamarkupfalse%
\ auto\isanewline
\ \ \ \ \ \ \ \ \ \ \ \isacommand{then}\isamarkupfalse%
\ \isacommand{show}\isamarkupfalse%
\ {\isachardoublequoteopen}x\ {\isasymin}\ {\isacharbraceleft}{\kern0pt}{\isachardot}{\kern0pt}{\isachardot}{\kern0pt}{\isacharless}{\kern0pt}n{\isacharplus}{\kern0pt}m{\isacharbraceright}{\kern0pt}{\isachardoublequoteclose}\ \isacommand{using}\isamarkupfalse%
\ a\ \isacommand{by}\isamarkupfalse%
\ auto\isanewline
\ \ \ \ \ \ \ \ \ \isacommand{qed}\isamarkupfalse%
\isanewline
\ \ \ \ \ \ \ \isacommand{qed}\isamarkupfalse%
\isanewline
\ \ \ \ \ \isacommand{qed}\isamarkupfalse%
\isanewline
\isanewline
\ \ \ \ \ \isacommand{show}\isamarkupfalse%
\ {\isachardoublequoteopen}{\isacharbraceleft}{\kern0pt}{\isachardot}{\kern0pt}{\isachardot}{\kern0pt}{\isacharless}{\kern0pt}n\ {\isacharplus}{\kern0pt}\ m{\isacharbraceright}{\kern0pt}\ {\isasymsubseteq}\ {\isasymUnion}\ {\isacharparenleft}{\kern0pt}BT\ {\isacharbackquote}{\kern0pt}\ {\isacharbraceleft}{\kern0pt}{\isachardot}{\kern0pt}{\isachardot}{\kern0pt}k\ {\isacharplus}{\kern0pt}\ {\isadigit{1}}{\isacharbraceright}{\kern0pt}{\isacharparenright}{\kern0pt}{\isachardoublequoteclose}\isanewline
\ \ \ \ \ \isacommand{proof}\isamarkupfalse%
\ \isanewline
\ \ \ \ \ \ \ \isacommand{fix}\isamarkupfalse%
\ x\ \isacommand{assume}\isamarkupfalse%
\ {\isachardoublequoteopen}x\ {\isasymin}\ {\isacharbraceleft}{\kern0pt}{\isachardot}{\kern0pt}{\isachardot}{\kern0pt}{\isacharless}{\kern0pt}n\ {\isacharplus}{\kern0pt}\ m{\isacharbraceright}{\kern0pt}{\isachardoublequoteclose}\isanewline
\ \ \ \ \ \ \ \isacommand{then}\isamarkupfalse%
\ \isacommand{consider}\isamarkupfalse%
\ {\isachardoublequoteopen}x\ {\isasymin}\ {\isacharbraceleft}{\kern0pt}{\isachardot}{\kern0pt}{\isachardot}{\kern0pt}{\isacharless}{\kern0pt}n{\isacharbraceright}{\kern0pt}{\isachardoublequoteclose}\ {\isacharbar}{\kern0pt}\ {\isachardoublequoteopen}x\ {\isasymin}\ {\isacharbraceleft}{\kern0pt}n{\isachardot}{\kern0pt}{\isachardot}{\kern0pt}{\isacharless}{\kern0pt}n{\isacharplus}{\kern0pt}m{\isacharbraceright}{\kern0pt}{\isachardoublequoteclose}\ \isacommand{by}\isamarkupfalse%
\ fastforce\isanewline
\ \ \ \ \ \ \ \isacommand{then}\isamarkupfalse%
\ \isacommand{show}\isamarkupfalse%
\ {\isachardoublequoteopen}x\ {\isasymin}\ {\isasymUnion}\ {\isacharparenleft}{\kern0pt}BT\ {\isacharbackquote}{\kern0pt}\ {\isacharbraceleft}{\kern0pt}{\isachardot}{\kern0pt}{\isachardot}{\kern0pt}k\ {\isacharplus}{\kern0pt}\ {\isadigit{1}}{\isacharbraceright}{\kern0pt}{\isacharparenright}{\kern0pt}{\isachardoublequoteclose}\isanewline
\ \ \ \ \ \ \ \isacommand{proof}\isamarkupfalse%
\ {\isacharparenleft}{\kern0pt}cases{\isacharparenright}{\kern0pt}\isanewline
\ \ \ \ \ \ \ \ \ \isacommand{case}\isamarkupfalse%
\ {\isadigit{1}}\isanewline
\ \ \ \ \ \ \ \ \ \isacommand{have}\isamarkupfalse%
\ {\isacharasterisk}{\kern0pt}{\isacharcolon}{\kern0pt}\ {\isachardoublequoteopen}{\isacharbraceleft}{\kern0pt}{\isachardot}{\kern0pt}{\isachardot}{\kern0pt}{\isadigit{1}}{\isacharcolon}{\kern0pt}{\isacharcolon}{\kern0pt}nat{\isacharbraceright}{\kern0pt}\ {\isacharequal}{\kern0pt}\ {\isacharbraceleft}{\kern0pt}{\isadigit{0}}{\isacharcomma}{\kern0pt}\ {\isadigit{1}}{\isacharcolon}{\kern0pt}{\isacharcolon}{\kern0pt}nat{\isacharbraceright}{\kern0pt}{\isachardoublequoteclose}\ \isacommand{by}\isamarkupfalse%
\ auto\isanewline
\ \ \ \ \ \ \ \ \ \isacommand{from}\isamarkupfalse%
\ {\isadigit{1}}\ \isacommand{have}\isamarkupfalse%
\ {\isachardoublequoteopen}x\ {\isasymin}\ {\isasymUnion}\ {\isacharparenleft}{\kern0pt}BL\ {\isacharbackquote}{\kern0pt}\ {\isacharbraceleft}{\kern0pt}{\isachardot}{\kern0pt}{\isachardot}{\kern0pt}{\isadigit{1}}{\isacharcolon}{\kern0pt}{\isacharcolon}{\kern0pt}nat{\isacharbraceright}{\kern0pt}{\isacharparenright}{\kern0pt}{\isachardoublequoteclose}\ \isacommand{using}\isamarkupfalse%
\ BfL{\isacharunderscore}{\kern0pt}props\ \isacommand{by}\isamarkupfalse%
\ simp\isanewline
\ \ \ \ \ \ \ \ \ \isacommand{then}\isamarkupfalse%
\ \isacommand{have}\isamarkupfalse%
\ {\isachardoublequoteopen}x\ {\isasymin}\ BL\ {\isadigit{0}}\ {\isasymor}\ x\ {\isasymin}\ BL\ {\isadigit{1}}{\isachardoublequoteclose}\ \isacommand{using}\isamarkupfalse%
\ {\isacharasterisk}{\kern0pt}\ \isacommand{by}\isamarkupfalse%
\ simp\isanewline
\ \ \ \ \ \ \ \ \ \isacommand{then}\isamarkupfalse%
\ \isacommand{show}\isamarkupfalse%
\ {\isacharquery}{\kern0pt}thesis\ \isanewline
\ \ \ \ \ \ \ \ \ \isacommand{proof}\isamarkupfalse%
\ {\isacharparenleft}{\kern0pt}elim\ disjE{\isacharparenright}{\kern0pt}\isanewline
\ \ \ \ \ \ \ \ \ \ \ \isacommand{assume}\isamarkupfalse%
\ {\isachardoublequoteopen}x\ {\isasymin}\ BL\ {\isadigit{0}}{\isachardoublequoteclose}\isanewline
\ \ \ \ \ \ \ \ \ \ \ \isacommand{then}\isamarkupfalse%
\ \isacommand{have}\isamarkupfalse%
\ {\isachardoublequoteopen}x\ {\isasymin}\ Bvar\ {\isadigit{0}}{\isachardoublequoteclose}\ \isacommand{unfolding}\isamarkupfalse%
\ Bvar{\isacharunderscore}{\kern0pt}def\ \isacommand{by}\isamarkupfalse%
\ simp\isanewline
\ \ \ \ \ \ \ \ \ \ \ \isacommand{then}\isamarkupfalse%
\ \isacommand{have}\isamarkupfalse%
\ {\isachardoublequoteopen}x\ {\isasymin}\ BT\ {\isadigit{0}}{\isachardoublequoteclose}\ \isacommand{unfolding}\isamarkupfalse%
\ BT{\isacharunderscore}{\kern0pt}def\ \isacommand{by}\isamarkupfalse%
\ simp\isanewline
\ \ \ \ \ \ \ \ \ \ \ \isacommand{then}\isamarkupfalse%
\ \isacommand{show}\isamarkupfalse%
\ {\isachardoublequoteopen}x\ {\isasymin}\ {\isasymUnion}\ {\isacharparenleft}{\kern0pt}BT\ {\isacharbackquote}{\kern0pt}\ {\isacharbraceleft}{\kern0pt}{\isachardot}{\kern0pt}{\isachardot}{\kern0pt}k\ {\isacharplus}{\kern0pt}\ {\isadigit{1}}{\isacharbraceright}{\kern0pt}{\isacharparenright}{\kern0pt}{\isachardoublequoteclose}\ \isacommand{by}\isamarkupfalse%
\ auto\isanewline
\ \ \ \ \ \ \ \ \ \isacommand{next}\isamarkupfalse%
\isanewline
\ \ \ \ \ \ \ \ \ \ \ \isacommand{assume}\isamarkupfalse%
\ {\isachardoublequoteopen}x\ {\isasymin}\ BL\ {\isadigit{1}}{\isachardoublequoteclose}\isanewline
\ \ \ \ \ \ \ \ \ \ \ \isacommand{then}\isamarkupfalse%
\ \isacommand{have}\isamarkupfalse%
\ {\isachardoublequoteopen}x\ {\isasymin}\ Bstat{\isachardoublequoteclose}\ \isacommand{unfolding}\isamarkupfalse%
\ Bstat{\isacharunderscore}{\kern0pt}def\ \isacommand{by}\isamarkupfalse%
\ simp\isanewline
\ \ \ \ \ \ \ \ \ \ \ \isacommand{then}\isamarkupfalse%
\ \isacommand{have}\isamarkupfalse%
\ {\isachardoublequoteopen}x\ {\isasymin}\ BT\ {\isacharparenleft}{\kern0pt}k{\isacharplus}{\kern0pt}{\isadigit{1}}{\isacharparenright}{\kern0pt}{\isachardoublequoteclose}\ \isacommand{unfolding}\isamarkupfalse%
\ BT{\isacharunderscore}{\kern0pt}def\ \isacommand{by}\isamarkupfalse%
\ simp\isanewline
\ \ \ \ \ \ \ \ \ \ \ \isacommand{then}\isamarkupfalse%
\ \isacommand{show}\isamarkupfalse%
\ {\isachardoublequoteopen}x\ {\isasymin}\ {\isasymUnion}\ {\isacharparenleft}{\kern0pt}BT\ {\isacharbackquote}{\kern0pt}\ {\isacharbraceleft}{\kern0pt}{\isachardot}{\kern0pt}{\isachardot}{\kern0pt}k\ {\isacharplus}{\kern0pt}\ {\isadigit{1}}{\isacharbraceright}{\kern0pt}{\isacharparenright}{\kern0pt}{\isachardoublequoteclose}\ \isacommand{by}\isamarkupfalse%
\ auto\isanewline
\ \ \ \ \ \ \ \ \ \isacommand{qed}\isamarkupfalse%
\isanewline
\ \ \ \ \ \ \ \isacommand{next}\isamarkupfalse%
\isanewline
\ \ \ \ \ \ \ \ \ \isacommand{case}\isamarkupfalse%
\ {\isadigit{2}}\isanewline
\ \ \ \ \ \ \ \ \ \isacommand{then}\isamarkupfalse%
\ \isacommand{have}\isamarkupfalse%
\ {\isachardoublequoteopen}x\ {\isasymin}\ {\isacharparenleft}{\kern0pt}{\isasymUnion}i{\isasymle}k{\isachardot}{\kern0pt}\ shiftset\ n\ {\isacharparenleft}{\kern0pt}BS\ i{\isacharparenright}{\kern0pt}{\isacharparenright}{\kern0pt}{\isachardoublequoteclose}\ \isacommand{using}\isamarkupfalse%
\ shiftset{\isacharunderscore}{\kern0pt}image{\isacharbrackleft}{\kern0pt}of\ BS\ k\ m\ n{\isacharbrackright}{\kern0pt}\ BfS{\isacharunderscore}{\kern0pt}props\ \isacommand{by}\isamarkupfalse%
\ simp\isanewline
\ \ \ \ \ \ \ \ \ \isacommand{then}\isamarkupfalse%
\ \isacommand{obtain}\isamarkupfalse%
\ i\ \isakeyword{where}\ i{\isacharunderscore}{\kern0pt}prop{\isacharcolon}{\kern0pt}\ {\isachardoublequoteopen}i\ {\isasymle}\ k\ {\isasymand}\ x\ {\isasymin}\ shiftset\ n\ {\isacharparenleft}{\kern0pt}BS\ i{\isacharparenright}{\kern0pt}{\isachardoublequoteclose}\ \isacommand{by}\isamarkupfalse%
\ blast\isanewline
\ \ \ \ \ \ \ \ \ \isacommand{then}\isamarkupfalse%
\ \isacommand{consider}\isamarkupfalse%
\ {\isachardoublequoteopen}i\ {\isacharequal}{\kern0pt}\ k{\isachardoublequoteclose}\ {\isacharbar}{\kern0pt}\ {\isachardoublequoteopen}i\ {\isacharless}{\kern0pt}\ k{\isachardoublequoteclose}\ \isacommand{by}\isamarkupfalse%
\ fastforce\isanewline
\ \ \ \ \ \ \ \ \ \isacommand{then}\isamarkupfalse%
\ \isacommand{show}\isamarkupfalse%
\ {\isacharquery}{\kern0pt}thesis\isanewline
\ \ \ \ \ \ \ \ \ \isacommand{proof}\isamarkupfalse%
\ {\isacharparenleft}{\kern0pt}cases{\isacharparenright}{\kern0pt}\isanewline
\ \ \ \ \ \ \ \ \ \ \ \isacommand{case}\isamarkupfalse%
\ {\isadigit{1}}\isanewline
\ \ \ \ \ \ \ \ \ \ \ \isacommand{then}\isamarkupfalse%
\ \isacommand{have}\isamarkupfalse%
\ {\isachardoublequoteopen}x\ {\isasymin}\ Bstat{\isachardoublequoteclose}\ \isacommand{unfolding}\isamarkupfalse%
\ Bstat{\isacharunderscore}{\kern0pt}def\ \isacommand{using}\isamarkupfalse%
\ i{\isacharunderscore}{\kern0pt}prop\ \isacommand{by}\isamarkupfalse%
\ auto\isanewline
\ \ \ \ \ \ \ \ \ \ \ \isacommand{then}\isamarkupfalse%
\ \isacommand{have}\isamarkupfalse%
\ {\isachardoublequoteopen}x\ {\isasymin}\ BT\ {\isacharparenleft}{\kern0pt}k{\isacharplus}{\kern0pt}{\isadigit{1}}{\isacharparenright}{\kern0pt}{\isachardoublequoteclose}\ \isacommand{unfolding}\isamarkupfalse%
\ BT{\isacharunderscore}{\kern0pt}def\ \isacommand{by}\isamarkupfalse%
\ simp\isanewline
\ \ \ \ \ \ \ \ \ \ \ \isacommand{then}\isamarkupfalse%
\ \isacommand{show}\isamarkupfalse%
\ {\isacharquery}{\kern0pt}thesis\ \isacommand{by}\isamarkupfalse%
\ auto\isanewline
\ \ \ \ \ \ \ \ \ \isacommand{next}\isamarkupfalse%
\isanewline
\ \ \ \ \ \ \ \ \ \ \ \isacommand{case}\isamarkupfalse%
\ {\isadigit{2}}\isanewline
\ \ \ \ \ \ \ \ \ \ \ \isacommand{then}\isamarkupfalse%
\ \isacommand{have}\isamarkupfalse%
\ {\isachardoublequoteopen}x\ {\isasymin}\ Bvar\ {\isacharparenleft}{\kern0pt}i\ {\isacharplus}{\kern0pt}\ {\isadigit{1}}{\isacharparenright}{\kern0pt}{\isachardoublequoteclose}\ \isacommand{unfolding}\isamarkupfalse%
\ Bvar{\isacharunderscore}{\kern0pt}def\ \isacommand{using}\isamarkupfalse%
\ i{\isacharunderscore}{\kern0pt}prop\ \isacommand{by}\isamarkupfalse%
\ simp\isanewline
\ \ \ \ \ \ \ \ \ \ \ \isacommand{then}\isamarkupfalse%
\ \isacommand{have}\isamarkupfalse%
\ {\isachardoublequoteopen}x\ {\isasymin}\ BT\ {\isacharparenleft}{\kern0pt}i\ {\isacharplus}{\kern0pt}\ {\isadigit{1}}{\isacharparenright}{\kern0pt}{\isachardoublequoteclose}\ \isacommand{unfolding}\isamarkupfalse%
\ BT{\isacharunderscore}{\kern0pt}def\ \isacommand{using}\isamarkupfalse%
\ {\isadigit{2}}\ \isacommand{by}\isamarkupfalse%
\ force\isanewline
\ \ \ \ \ \ \ \ \ \ \ \isacommand{then}\isamarkupfalse%
\ \isacommand{show}\isamarkupfalse%
\ {\isacharquery}{\kern0pt}thesis\ \isacommand{using}\isamarkupfalse%
\ {\isadigit{2}}\ \isacommand{by}\isamarkupfalse%
\ auto\isanewline
\ \ \ \ \ \ \ \ \ \isacommand{qed}\isamarkupfalse%
\isanewline
\ \ \ \ \ \ \ \isacommand{qed}\isamarkupfalse%
\isanewline
\ \ \ \ \ \isacommand{qed}\isamarkupfalse%
\isanewline
\ \ \ \isacommand{qed}\isamarkupfalse%
\isanewline
\isanewline
\isanewline
\ \ \ \isacommand{have}\isamarkupfalse%
\ F{\isadigit{4}}{\isacharcolon}{\kern0pt}\ {\isachardoublequoteopen}fT\ {\isasymin}\ {\isacharparenleft}{\kern0pt}BT\ {\isacharparenleft}{\kern0pt}k{\isacharplus}{\kern0pt}{\isadigit{1}}{\isacharparenright}{\kern0pt}{\isacharparenright}{\kern0pt}\ {\isasymrightarrow}\isactrlsub E\ {\isacharbraceleft}{\kern0pt}{\isachardot}{\kern0pt}{\isachardot}{\kern0pt}{\isacharless}{\kern0pt}t{\isacharplus}{\kern0pt}{\isadigit{1}}{\isacharbraceright}{\kern0pt}{\isachardoublequoteclose}\isanewline
\ \ \ \isacommand{proof}\isamarkupfalse%
\isanewline
\ \ \ \ \ \isacommand{fix}\isamarkupfalse%
\ x\ \isacommand{assume}\isamarkupfalse%
\ {\isachardoublequoteopen}x\ {\isasymin}\ BT\ {\isacharparenleft}{\kern0pt}k{\isacharplus}{\kern0pt}{\isadigit{1}}{\isacharparenright}{\kern0pt}{\isachardoublequoteclose}\isanewline
\ \ \ \ \ \isacommand{then}\isamarkupfalse%
\ \isacommand{have}\isamarkupfalse%
\ {\isachardoublequoteopen}x\ {\isasymin}\ Bstat{\isachardoublequoteclose}\ \isacommand{unfolding}\isamarkupfalse%
\ BT{\isacharunderscore}{\kern0pt}def\ \isacommand{by}\isamarkupfalse%
\ simp\isanewline
\ \ \ \ \ \isacommand{then}\isamarkupfalse%
\ \isacommand{have}\isamarkupfalse%
\ {\isachardoublequoteopen}x\ {\isasymin}\ BL\ {\isadigit{1}}\ {\isasymor}\ x\ {\isasymin}\ shiftset\ n\ {\isacharparenleft}{\kern0pt}BS\ k{\isacharparenright}{\kern0pt}{\isachardoublequoteclose}\ \isacommand{unfolding}\isamarkupfalse%
\ Bstat{\isacharunderscore}{\kern0pt}def\ \isacommand{by}\isamarkupfalse%
\ auto\isanewline
\ \ \ \ \ \isacommand{then}\isamarkupfalse%
\ \isacommand{show}\isamarkupfalse%
\ {\isachardoublequoteopen}fT\ x\ {\isasymin}\ {\isacharbraceleft}{\kern0pt}{\isachardot}{\kern0pt}{\isachardot}{\kern0pt}{\isacharless}{\kern0pt}t\ {\isacharplus}{\kern0pt}\ {\isadigit{1}}{\isacharbraceright}{\kern0pt}{\isachardoublequoteclose}\isanewline
\ \ \ \ \ \isacommand{proof}\isamarkupfalse%
\ {\isacharparenleft}{\kern0pt}elim\ disjE{\isacharparenright}{\kern0pt}\isanewline
\ \ \ \ \ \ \ \isacommand{assume}\isamarkupfalse%
\ {\isachardoublequoteopen}x\ {\isasymin}\ BL\ {\isadigit{1}}{\isachardoublequoteclose}\isanewline
\ \ \ \ \ \ \ \isacommand{then}\isamarkupfalse%
\ \isacommand{have}\isamarkupfalse%
\ {\isachardoublequoteopen}fT\ x\ {\isacharequal}{\kern0pt}\ fL\ x{\isachardoublequoteclose}\ \isacommand{unfolding}\isamarkupfalse%
\ fT{\isacharunderscore}{\kern0pt}def\ \isacommand{by}\isamarkupfalse%
\ simp\isanewline
\ \ \ \ \ \ \ \isacommand{then}\isamarkupfalse%
\ \isacommand{show}\isamarkupfalse%
\ {\isachardoublequoteopen}fT\ x\ {\isasymin}\ {\isacharbraceleft}{\kern0pt}{\isachardot}{\kern0pt}{\isachardot}{\kern0pt}{\isacharless}{\kern0pt}t{\isacharplus}{\kern0pt}{\isadigit{1}}{\isacharbraceright}{\kern0pt}{\isachardoublequoteclose}\ \isacommand{using}\isamarkupfalse%
\ BfL{\isacharunderscore}{\kern0pt}props\ {\isacartoucheopen}x\ {\isasymin}\ BL\ {\isadigit{1}}{\isacartoucheclose}\ \isacommand{by}\isamarkupfalse%
\ auto\isanewline
\ \ \ \ \ \isacommand{next}\isamarkupfalse%
\isanewline
\ \ \ \ \ \ \ \isacommand{assume}\isamarkupfalse%
\ a{\isacharcolon}{\kern0pt}\ {\isachardoublequoteopen}x\ {\isasymin}\ shiftset\ n\ {\isacharparenleft}{\kern0pt}BS\ k{\isacharparenright}{\kern0pt}{\isachardoublequoteclose}\isanewline
\ \ \ \ \ \ \ \isacommand{then}\isamarkupfalse%
\ \isacommand{have}\isamarkupfalse%
\ {\isachardoublequoteopen}fT\ x\ {\isacharequal}{\kern0pt}\ fS\ {\isacharparenleft}{\kern0pt}x\ {\isacharminus}{\kern0pt}\ n{\isacharparenright}{\kern0pt}{\isachardoublequoteclose}\ \isacommand{using}\isamarkupfalse%
\ fax{\isadigit{1}}\ \isacommand{unfolding}\isamarkupfalse%
\ fT{\isacharunderscore}{\kern0pt}def\ \isacommand{by}\isamarkupfalse%
\ auto\isanewline
\ \ \ \ \ \ \ \isacommand{moreover}\isamarkupfalse%
\ \isacommand{have}\isamarkupfalse%
\ {\isachardoublequoteopen}x\ {\isacharminus}{\kern0pt}\ n\ {\isasymin}\ BS\ k{\isachardoublequoteclose}\ \isacommand{using}\isamarkupfalse%
\ a\ \isacommand{unfolding}\isamarkupfalse%
\ shiftset{\isacharunderscore}{\kern0pt}def\ \isacommand{by}\isamarkupfalse%
\ auto\isanewline
\ \ \ \ \ \ \ \isacommand{ultimately}\isamarkupfalse%
\ \isacommand{show}\isamarkupfalse%
\ {\isachardoublequoteopen}fT\ x\ {\isasymin}\ {\isacharbraceleft}{\kern0pt}{\isachardot}{\kern0pt}{\isachardot}{\kern0pt}{\isacharless}{\kern0pt}t{\isacharplus}{\kern0pt}{\isadigit{1}}{\isacharbraceright}{\kern0pt}{\isachardoublequoteclose}\ \isacommand{using}\isamarkupfalse%
\ BfS{\isacharunderscore}{\kern0pt}props\ \isacommand{by}\isamarkupfalse%
\ auto\isanewline
\ \ \ \ \ \isacommand{qed}\isamarkupfalse%
\isanewline
\ \ \ \isacommand{qed}\isamarkupfalse%
{\isacharparenleft}{\kern0pt}auto\ simp{\isacharcolon}{\kern0pt}\ BT{\isacharunderscore}{\kern0pt}def\ Bstat{\isacharunderscore}{\kern0pt}def\ fT{\isacharunderscore}{\kern0pt}def{\isacharparenright}{\kern0pt}\isanewline
\isanewline
\isanewline
\ \ \ \isacommand{have}\isamarkupfalse%
\ F{\isadigit{5}}{\isacharcolon}{\kern0pt}\ {\isachardoublequoteopen}{\isacharparenleft}{\kern0pt}{\isacharparenleft}{\kern0pt}{\isasymforall}i\ {\isasymin}\ BT\ {\isacharparenleft}{\kern0pt}k\ {\isacharplus}{\kern0pt}\ {\isadigit{1}}{\isacharparenright}{\kern0pt}{\isachardot}{\kern0pt}\ T\ y\ i\ {\isacharequal}{\kern0pt}\ fT\ i{\isacharparenright}{\kern0pt}\ {\isasymand}\ {\isacharparenleft}{\kern0pt}{\isasymforall}j{\isacharless}{\kern0pt}k{\isacharplus}{\kern0pt}{\isadigit{1}}{\isachardot}{\kern0pt}\ {\isasymforall}i\ {\isasymin}\ BT\ j{\isachardot}{\kern0pt}\ {\isacharparenleft}{\kern0pt}T\ y{\isacharparenright}{\kern0pt}\ i\ {\isacharequal}{\kern0pt}\ y\ j{\isacharparenright}{\kern0pt}{\isacharparenright}{\kern0pt}{\isachardoublequoteclose}\ \isakeyword{if}\ {\isachardoublequoteopen}y\ {\isasymin}\ cube\ {\isacharparenleft}{\kern0pt}k\ {\isacharplus}{\kern0pt}\ {\isadigit{1}}{\isacharparenright}{\kern0pt}\ {\isacharparenleft}{\kern0pt}t\ {\isacharplus}{\kern0pt}\ {\isadigit{1}}{\isacharparenright}{\kern0pt}{\isachardoublequoteclose}\ \isakeyword{for}\ y\isanewline
\ \ \ \isacommand{proof}\isamarkupfalse%
{\isacharparenleft}{\kern0pt}intro\ conjI\ allI\ impI\ ballI{\isacharparenright}{\kern0pt}\isanewline
\ \ \ \ \ \isacommand{fix}\isamarkupfalse%
\ i\ \isacommand{assume}\isamarkupfalse%
\ {\isachardoublequoteopen}i\ {\isasymin}\ BT\ {\isacharparenleft}{\kern0pt}k\ {\isacharplus}{\kern0pt}\ {\isadigit{1}}{\isacharparenright}{\kern0pt}{\isachardoublequoteclose}\isanewline
\ \ \ \ \ \isacommand{then}\isamarkupfalse%
\ \isacommand{have}\isamarkupfalse%
\ {\isachardoublequoteopen}i\ {\isasymin}\ Bstat{\isachardoublequoteclose}\ \isacommand{unfolding}\isamarkupfalse%
\ BT{\isacharunderscore}{\kern0pt}def\ \isacommand{by}\isamarkupfalse%
\ simp\isanewline
\ \ \ \ \ \isacommand{then}\isamarkupfalse%
\ \isacommand{consider}\isamarkupfalse%
\ {\isachardoublequoteopen}i\ {\isasymin}\ shiftset\ n\ {\isacharparenleft}{\kern0pt}BS\ k{\isacharparenright}{\kern0pt}{\isachardoublequoteclose}\ {\isacharbar}{\kern0pt}\ \ {\isachardoublequoteopen}i\ {\isasymin}\ BL\ {\isadigit{1}}{\isachardoublequoteclose}\ \isacommand{unfolding}\isamarkupfalse%
\ Bstat{\isacharunderscore}{\kern0pt}def\ \isacommand{by}\isamarkupfalse%
\ blast\isanewline
\ \ \ \ \ \isacommand{then}\isamarkupfalse%
\ \isacommand{show}\isamarkupfalse%
\ {\isachardoublequoteopen}T\ y\ i\ {\isacharequal}{\kern0pt}\ fT\ i{\isachardoublequoteclose}\isanewline
\ \ \ \ \ \isacommand{proof}\isamarkupfalse%
\ {\isacharparenleft}{\kern0pt}cases{\isacharparenright}{\kern0pt}\isanewline
\ \ \ \ \ \ \ \isacommand{case}\isamarkupfalse%
\ {\isadigit{1}}\isanewline
\ \ \ \ \ \ \ \isacommand{then}\isamarkupfalse%
\ \isacommand{have}\isamarkupfalse%
\ {\isachardoublequoteopen}{\isasymexists}s{\isacharless}{\kern0pt}m{\isachardot}{\kern0pt}\ i\ {\isacharequal}{\kern0pt}\ n\ {\isacharplus}{\kern0pt}\ s{\isachardoublequoteclose}\ \isacommand{unfolding}\isamarkupfalse%
\ shiftset{\isacharunderscore}{\kern0pt}def\ \isacommand{using}\isamarkupfalse%
\ BfS{\isacharunderscore}{\kern0pt}props{\isacharparenleft}{\kern0pt}{\isadigit{2}}{\isacharparenright}{\kern0pt}\ \isacommand{by}\isamarkupfalse%
\ auto\isanewline
\ \ \ \ \ \ \ \isacommand{then}\isamarkupfalse%
\ \isacommand{obtain}\isamarkupfalse%
\ s\ \isakeyword{where}\ s{\isacharunderscore}{\kern0pt}prop{\isacharcolon}{\kern0pt}\ {\isachardoublequoteopen}s\ {\isacharless}{\kern0pt}\ m\ {\isasymand}\ i\ {\isacharequal}{\kern0pt}\ n\ {\isacharplus}{\kern0pt}\ s{\isachardoublequoteclose}\ \isacommand{by}\isamarkupfalse%
\ blast\isanewline
\ \ \ \ \ \ \ \isacommand{then}\isamarkupfalse%
\ \isacommand{have}\isamarkupfalse%
\ {\isacharasterisk}{\kern0pt}{\isacharcolon}{\kern0pt}\ {\isachardoublequoteopen}\ i\ {\isasymin}\ {\isacharbraceleft}{\kern0pt}n{\isachardot}{\kern0pt}{\isachardot}{\kern0pt}{\isacharless}{\kern0pt}n{\isacharplus}{\kern0pt}m{\isacharbraceright}{\kern0pt}{\isachardoublequoteclose}\ \isacommand{by}\isamarkupfalse%
\ simp\isanewline
\ \ \ \ \ \ \ \isacommand{have}\isamarkupfalse%
\ {\isachardoublequoteopen}i\ {\isasymnotin}\ BL\ {\isadigit{1}}{\isachardoublequoteclose}\ \isacommand{using}\isamarkupfalse%
\ {\isadigit{1}}\ fax{\isadigit{1}}\ \isacommand{by}\isamarkupfalse%
\ auto\isanewline
\ \ \ \ \ \ \ \isacommand{then}\isamarkupfalse%
\ \isacommand{have}\isamarkupfalse%
\ {\isachardoublequoteopen}fT\ i\ {\isacharequal}{\kern0pt}\ fS\ {\isacharparenleft}{\kern0pt}i\ {\isacharminus}{\kern0pt}\ n{\isacharparenright}{\kern0pt}{\isachardoublequoteclose}\ \isacommand{using}\isamarkupfalse%
\ {\isadigit{1}}\ \isacommand{unfolding}\isamarkupfalse%
\ fT{\isacharunderscore}{\kern0pt}def\ \isacommand{by}\isamarkupfalse%
\ simp\isanewline
\ \ \ \ \ \ \ \isacommand{then}\isamarkupfalse%
\ \isacommand{have}\isamarkupfalse%
\ {\isacharasterisk}{\kern0pt}{\isacharasterisk}{\kern0pt}{\isacharcolon}{\kern0pt}\ {\isachardoublequoteopen}fT\ i\ {\isacharequal}{\kern0pt}\ fS\ s{\isachardoublequoteclose}\ \isacommand{using}\isamarkupfalse%
\ s{\isacharunderscore}{\kern0pt}prop\ \isacommand{by}\isamarkupfalse%
\ simp\isanewline
\isanewline
\ \ \ \ \ \ \ \isacommand{have}\isamarkupfalse%
\ XX{\isacharcolon}{\kern0pt}\ {\isachardoublequoteopen}{\isacharparenleft}{\kern0pt}{\isasymlambda}z\ {\isasymin}\ {\isacharbraceleft}{\kern0pt}{\isachardot}{\kern0pt}{\isachardot}{\kern0pt}{\isacharless}{\kern0pt}k{\isacharbraceright}{\kern0pt}{\isachardot}{\kern0pt}\ y\ {\isacharparenleft}{\kern0pt}z\ {\isacharplus}{\kern0pt}\ {\isadigit{1}}{\isacharparenright}{\kern0pt}{\isacharparenright}{\kern0pt}\ {\isasymin}\ cube\ k\ {\isacharparenleft}{\kern0pt}t{\isacharplus}{\kern0pt}{\isadigit{1}}{\isacharparenright}{\kern0pt}{\isachardoublequoteclose}\ \isacommand{using}\isamarkupfalse%
\ split{\isacharunderscore}{\kern0pt}cube\ that\ \isacommand{by}\isamarkupfalse%
\ simp\isanewline
\ \ \ \ \ \ \ \isacommand{have}\isamarkupfalse%
\ XY{\isacharcolon}{\kern0pt}\ {\isachardoublequoteopen}s\ {\isasymin}\ BS\ k{\isachardoublequoteclose}\ \isacommand{using}\isamarkupfalse%
\ \ s{\isacharunderscore}{\kern0pt}prop\ \ {\isadigit{1}}\ \isacommand{unfolding}\isamarkupfalse%
\ shiftset{\isacharunderscore}{\kern0pt}def\ \isacommand{by}\isamarkupfalse%
\ auto\isanewline
\isanewline
\ \ \ \ \ \ \ \isacommand{from}\isamarkupfalse%
\ that\ \isacommand{have}\isamarkupfalse%
\ {\isachardoublequoteopen}T\ y\ i\ {\isacharequal}{\kern0pt}\ {\isacharparenleft}{\kern0pt}T{\isacharprime}{\kern0pt}\ {\isacharparenleft}{\kern0pt}{\isasymlambda}z\ {\isasymin}\ {\isacharbraceleft}{\kern0pt}{\isachardot}{\kern0pt}{\isachardot}{\kern0pt}{\isacharless}{\kern0pt}{\isadigit{1}}{\isacharbraceright}{\kern0pt}{\isachardot}{\kern0pt}\ y\ z{\isacharparenright}{\kern0pt}\ {\isacharparenleft}{\kern0pt}{\isasymlambda}z\ {\isasymin}\ {\isacharbraceleft}{\kern0pt}{\isachardot}{\kern0pt}{\isachardot}{\kern0pt}{\isacharless}{\kern0pt}k{\isacharbraceright}{\kern0pt}{\isachardot}{\kern0pt}\ y\ {\isacharparenleft}{\kern0pt}z\ {\isacharplus}{\kern0pt}\ {\isadigit{1}}{\isacharparenright}{\kern0pt}{\isacharparenright}{\kern0pt}{\isacharparenright}{\kern0pt}\ i{\isachardoublequoteclose}\ \isacommand{unfolding}\isamarkupfalse%
\ T{\isacharunderscore}{\kern0pt}def\ \isacommand{by}\isamarkupfalse%
\ auto\isanewline
\ \ \ \ \ \ \ \isacommand{also}\isamarkupfalse%
\ \isacommand{have}\isamarkupfalse%
\ {\isachardoublequoteopen}{\isachardot}{\kern0pt}{\isachardot}{\kern0pt}{\isachardot}{\kern0pt}\ {\isacharequal}{\kern0pt}\ {\isacharparenleft}{\kern0pt}join\ {\isacharparenleft}{\kern0pt}L{\isacharunderscore}{\kern0pt}line\ {\isacharparenleft}{\kern0pt}{\isacharparenleft}{\kern0pt}{\isasymlambda}z\ {\isasymin}\ {\isacharbraceleft}{\kern0pt}{\isachardot}{\kern0pt}{\isachardot}{\kern0pt}{\isacharless}{\kern0pt}{\isadigit{1}}{\isacharbraceright}{\kern0pt}{\isachardot}{\kern0pt}\ y\ z{\isacharparenright}{\kern0pt}\ {\isadigit{0}}{\isacharparenright}{\kern0pt}{\isacharparenright}{\kern0pt}\ {\isacharparenleft}{\kern0pt}S\ {\isacharparenleft}{\kern0pt}{\isasymlambda}z\ {\isasymin}\ {\isacharbraceleft}{\kern0pt}{\isachardot}{\kern0pt}{\isachardot}{\kern0pt}{\isacharless}{\kern0pt}k{\isacharbraceright}{\kern0pt}{\isachardot}{\kern0pt}\ y\ {\isacharparenleft}{\kern0pt}z\ {\isacharplus}{\kern0pt}\ {\isadigit{1}}{\isacharparenright}{\kern0pt}{\isacharparenright}{\kern0pt}{\isacharparenright}{\kern0pt}\ n\ m{\isacharparenright}{\kern0pt}\ i{\isachardoublequoteclose}\ \isacommand{using}\isamarkupfalse%
\ split{\isacharunderscore}{\kern0pt}cube\ that\ \isacommand{unfolding}\isamarkupfalse%
\ T{\isacharprime}{\kern0pt}{\isacharunderscore}{\kern0pt}def\ \isacommand{by}\isamarkupfalse%
\ simp\isanewline
\ \ \ \ \ \ \ \isacommand{also}\isamarkupfalse%
\ \isacommand{have}\isamarkupfalse%
\ {\isachardoublequoteopen}{\isachardot}{\kern0pt}{\isachardot}{\kern0pt}{\isachardot}{\kern0pt}\ {\isacharequal}{\kern0pt}\ {\isacharparenleft}{\kern0pt}join\ {\isacharparenleft}{\kern0pt}L{\isacharunderscore}{\kern0pt}line\ {\isacharparenleft}{\kern0pt}y\ {\isadigit{0}}{\isacharparenright}{\kern0pt}{\isacharparenright}{\kern0pt}\ {\isacharparenleft}{\kern0pt}S\ {\isacharparenleft}{\kern0pt}{\isasymlambda}z\ {\isasymin}\ {\isacharbraceleft}{\kern0pt}{\isachardot}{\kern0pt}{\isachardot}{\kern0pt}{\isacharless}{\kern0pt}k{\isacharbraceright}{\kern0pt}{\isachardot}{\kern0pt}\ y\ {\isacharparenleft}{\kern0pt}z\ {\isacharplus}{\kern0pt}\ {\isadigit{1}}{\isacharparenright}{\kern0pt}{\isacharparenright}{\kern0pt}{\isacharparenright}{\kern0pt}\ n\ m{\isacharparenright}{\kern0pt}\ i{\isachardoublequoteclose}\ \isacommand{by}\isamarkupfalse%
\ simp\isanewline
\ \ \ \ \ \ \ \isacommand{also}\isamarkupfalse%
\ \isacommand{have}\isamarkupfalse%
\ {\isachardoublequoteopen}{\isachardot}{\kern0pt}{\isachardot}{\kern0pt}{\isachardot}{\kern0pt}\ {\isacharequal}{\kern0pt}\ {\isacharparenleft}{\kern0pt}S\ {\isacharparenleft}{\kern0pt}{\isasymlambda}z\ {\isasymin}\ {\isacharbraceleft}{\kern0pt}{\isachardot}{\kern0pt}{\isachardot}{\kern0pt}{\isacharless}{\kern0pt}k{\isacharbraceright}{\kern0pt}{\isachardot}{\kern0pt}\ y\ {\isacharparenleft}{\kern0pt}z\ {\isacharplus}{\kern0pt}\ {\isadigit{1}}{\isacharparenright}{\kern0pt}{\isacharparenright}{\kern0pt}{\isacharparenright}{\kern0pt}\ s{\isachardoublequoteclose}\ \isacommand{using}\isamarkupfalse%
\ {\isacharasterisk}{\kern0pt}\ s{\isacharunderscore}{\kern0pt}prop\ \isacommand{unfolding}\isamarkupfalse%
\ join{\isacharunderscore}{\kern0pt}def\ \isacommand{by}\isamarkupfalse%
\ simp\isanewline
\ \ \ \ \ \ \ \isacommand{also}\isamarkupfalse%
\ \isacommand{have}\isamarkupfalse%
\ {\isachardoublequoteopen}{\isachardot}{\kern0pt}{\isachardot}{\kern0pt}{\isachardot}{\kern0pt}\ {\isacharequal}{\kern0pt}\ fS\ s{\isachardoublequoteclose}\ \isacommand{using}\isamarkupfalse%
\ XX\ XY\ BfS{\isacharunderscore}{\kern0pt}props{\isacharparenleft}{\kern0pt}{\isadigit{6}}{\isacharparenright}{\kern0pt}\ \isacommand{by}\isamarkupfalse%
\ blast\isanewline
\ \ \ \ \ \ \ \isacommand{finally}\isamarkupfalse%
\ \isacommand{show}\isamarkupfalse%
\ {\isacharquery}{\kern0pt}thesis\ \isacommand{using}\isamarkupfalse%
\ {\isacharasterisk}{\kern0pt}{\isacharasterisk}{\kern0pt}\ \isacommand{by}\isamarkupfalse%
\ simp\isanewline
\ \ \ \ \ \isacommand{next}\isamarkupfalse%
\isanewline
\ \ \ \ \ \ \ \isacommand{case}\isamarkupfalse%
\ {\isadigit{2}}\isanewline
\ \ \ \ \ \ \ \isacommand{have}\isamarkupfalse%
\ XZ{\isacharcolon}{\kern0pt}\ {\isachardoublequoteopen}y\ {\isadigit{0}}\ {\isasymin}\ {\isacharbraceleft}{\kern0pt}{\isachardot}{\kern0pt}{\isachardot}{\kern0pt}{\isacharless}{\kern0pt}t{\isacharplus}{\kern0pt}{\isadigit{1}}{\isacharbraceright}{\kern0pt}{\isachardoublequoteclose}\ \isacommand{using}\isamarkupfalse%
\ that\ \isacommand{unfolding}\isamarkupfalse%
\ cube{\isacharunderscore}{\kern0pt}def\ \isacommand{by}\isamarkupfalse%
\ auto\isanewline
\ \ \ \ \ \ \ \isacommand{have}\isamarkupfalse%
\ XY{\isacharcolon}{\kern0pt}\ {\isachardoublequoteopen}i\ {\isasymin}\ {\isacharbraceleft}{\kern0pt}{\isachardot}{\kern0pt}{\isachardot}{\kern0pt}{\isacharless}{\kern0pt}n{\isacharbraceright}{\kern0pt}{\isachardoublequoteclose}\ \isacommand{using}\isamarkupfalse%
\ {\isadigit{2}}\ BfL{\isacharunderscore}{\kern0pt}props{\isacharparenleft}{\kern0pt}{\isadigit{2}}{\isacharparenright}{\kern0pt}\ \isacommand{by}\isamarkupfalse%
\ blast\isanewline
\ \ \ \ \ \ \ \isacommand{have}\isamarkupfalse%
\ XX{\isacharcolon}{\kern0pt}\ {\isachardoublequoteopen}{\isacharparenleft}{\kern0pt}{\isasymlambda}z\ {\isasymin}\ {\isacharbraceleft}{\kern0pt}{\isachardot}{\kern0pt}{\isachardot}{\kern0pt}{\isacharless}{\kern0pt}{\isadigit{1}}{\isacharbraceright}{\kern0pt}{\isachardot}{\kern0pt}\ y\ z{\isacharparenright}{\kern0pt}\ \ {\isasymin}\ cube\ {\isadigit{1}}\ {\isacharparenleft}{\kern0pt}t{\isacharplus}{\kern0pt}{\isadigit{1}}{\isacharparenright}{\kern0pt}{\isachardoublequoteclose}\ \isacommand{using}\isamarkupfalse%
\ that\ split{\isacharunderscore}{\kern0pt}cube\ \isacommand{by}\isamarkupfalse%
\ simp\isanewline
\isanewline
\ \ \ \ \ \ \ \ \isacommand{have}\isamarkupfalse%
\ some{\isacharunderscore}{\kern0pt}eq{\isacharunderscore}{\kern0pt}restrict{\isacharcolon}{\kern0pt}\ {\isachardoublequoteopen}{\isacharparenleft}{\kern0pt}SOME\ p{\isachardot}{\kern0pt}\ p{\isasymin}cube\ {\isadigit{1}}\ {\isacharparenleft}{\kern0pt}t{\isacharplus}{\kern0pt}{\isadigit{1}}{\isacharparenright}{\kern0pt}\ {\isasymand}\ p\ {\isadigit{0}}\ {\isacharequal}{\kern0pt}\ {\isacharparenleft}{\kern0pt}{\isacharparenleft}{\kern0pt}{\isasymlambda}z\ {\isasymin}\ {\isacharbraceleft}{\kern0pt}{\isachardot}{\kern0pt}{\isachardot}{\kern0pt}{\isacharless}{\kern0pt}{\isadigit{1}}{\isacharbraceright}{\kern0pt}{\isachardot}{\kern0pt}\ y\ z{\isacharparenright}{\kern0pt}\ {\isadigit{0}}{\isacharparenright}{\kern0pt}{\isacharparenright}{\kern0pt}\ {\isacharequal}{\kern0pt}\ {\isacharparenleft}{\kern0pt}{\isasymlambda}z\ {\isasymin}\ {\isacharbraceleft}{\kern0pt}{\isachardot}{\kern0pt}{\isachardot}{\kern0pt}{\isacharless}{\kern0pt}{\isadigit{1}}{\isacharbraceright}{\kern0pt}{\isachardot}{\kern0pt}\ y\ z{\isacharparenright}{\kern0pt}{\isachardoublequoteclose}\isanewline
\ \ \ \ \ \ \ \ \isacommand{proof}\isamarkupfalse%
\ \isanewline
\ \ \ \ \ \ \ \ \ \ \isacommand{show}\isamarkupfalse%
\ {\isachardoublequoteopen}restrict\ y\ {\isacharbraceleft}{\kern0pt}{\isachardot}{\kern0pt}{\isachardot}{\kern0pt}{\isacharless}{\kern0pt}{\isadigit{1}}{\isacharbraceright}{\kern0pt}\ {\isasymin}\ cube\ {\isadigit{1}}\ {\isacharparenleft}{\kern0pt}t\ {\isacharplus}{\kern0pt}\ {\isadigit{1}}{\isacharparenright}{\kern0pt}\ {\isasymand}\ restrict\ y\ {\isacharbraceleft}{\kern0pt}{\isachardot}{\kern0pt}{\isachardot}{\kern0pt}{\isacharless}{\kern0pt}{\isadigit{1}}{\isacharbraceright}{\kern0pt}\ {\isadigit{0}}\ {\isacharequal}{\kern0pt}\ restrict\ y\ {\isacharbraceleft}{\kern0pt}{\isachardot}{\kern0pt}{\isachardot}{\kern0pt}{\isacharless}{\kern0pt}{\isadigit{1}}{\isacharbraceright}{\kern0pt}\ {\isadigit{0}}{\isachardoublequoteclose}\ \isacommand{using}\isamarkupfalse%
\ XX\ \isacommand{by}\isamarkupfalse%
\ simp\isanewline
\ \ \ \ \ \ \ \ \isacommand{next}\isamarkupfalse%
\isanewline
\ \ \ \ \ \ \ \ \ \ \isacommand{fix}\isamarkupfalse%
\ p\isanewline
\ \ \ \ \ \ \ \ \ \ \isacommand{assume}\isamarkupfalse%
\ {\isachardoublequoteopen}p\ {\isasymin}\ cube\ {\isadigit{1}}\ {\isacharparenleft}{\kern0pt}t{\isacharplus}{\kern0pt}{\isadigit{1}}{\isacharparenright}{\kern0pt}\ {\isasymand}\ p\ {\isadigit{0}}\ {\isacharequal}{\kern0pt}\ restrict\ y\ {\isacharbraceleft}{\kern0pt}{\isachardot}{\kern0pt}{\isachardot}{\kern0pt}{\isacharless}{\kern0pt}{\isadigit{1}}{\isacharbraceright}{\kern0pt}\ {\isadigit{0}}{\isachardoublequoteclose}\isanewline
\ \ \ \ \ \ \ \ \ \ \isacommand{moreover}\isamarkupfalse%
\ \isacommand{have}\isamarkupfalse%
\ {\isachardoublequoteopen}p\ u\ {\isacharequal}{\kern0pt}\ restrict\ y\ {\isacharbraceleft}{\kern0pt}{\isachardot}{\kern0pt}{\isachardot}{\kern0pt}{\isacharless}{\kern0pt}{\isadigit{1}}{\isacharbraceright}{\kern0pt}\ u{\isachardoublequoteclose}\ \isakeyword{if}\ {\isachardoublequoteopen}u\ {\isasymnotin}\ {\isacharbraceleft}{\kern0pt}{\isachardot}{\kern0pt}{\isachardot}{\kern0pt}{\isacharless}{\kern0pt}{\isadigit{1}}{\isacharbraceright}{\kern0pt}{\isachardoublequoteclose}\ \isakeyword{for}\ u\ \isacommand{using}\isamarkupfalse%
\ that\ calculation\ XX\ \isacommand{unfolding}\isamarkupfalse%
\ cube{\isacharunderscore}{\kern0pt}def\ \isacommand{using}\isamarkupfalse%
\ PiE{\isacharunderscore}{\kern0pt}arb{\isacharbrackleft}{\kern0pt}of\ {\isachardoublequoteopen}restrict\ y\ {\isacharbraceleft}{\kern0pt}{\isachardot}{\kern0pt}{\isachardot}{\kern0pt}{\isacharless}{\kern0pt}{\isadigit{1}}{\isacharbraceright}{\kern0pt}{\isachardoublequoteclose}\ {\isachardoublequoteopen}{\isacharbraceleft}{\kern0pt}{\isachardot}{\kern0pt}{\isachardot}{\kern0pt}{\isacharless}{\kern0pt}{\isadigit{1}}{\isacharbraceright}{\kern0pt}{\isachardoublequoteclose}\ {\isachardoublequoteopen}{\isasymlambda}x{\isachardot}{\kern0pt}\ {\isacharbraceleft}{\kern0pt}{\isachardot}{\kern0pt}{\isachardot}{\kern0pt}{\isacharless}{\kern0pt}t\ {\isacharplus}{\kern0pt}\ {\isadigit{1}}{\isacharbraceright}{\kern0pt}{\isachardoublequoteclose}\ u{\isacharbrackright}{\kern0pt}\ \ PiE{\isacharunderscore}{\kern0pt}arb{\isacharbrackleft}{\kern0pt}of\ p\ {\isachardoublequoteopen}{\isacharbraceleft}{\kern0pt}{\isachardot}{\kern0pt}{\isachardot}{\kern0pt}{\isacharless}{\kern0pt}{\isadigit{1}}{\isacharbraceright}{\kern0pt}{\isachardoublequoteclose}\ {\isachardoublequoteopen}{\isasymlambda}x{\isachardot}{\kern0pt}\ {\isacharbraceleft}{\kern0pt}{\isachardot}{\kern0pt}{\isachardot}{\kern0pt}{\isacharless}{\kern0pt}t\ {\isacharplus}{\kern0pt}\ {\isadigit{1}}{\isacharbraceright}{\kern0pt}{\isachardoublequoteclose}\ u{\isacharbrackright}{\kern0pt}\ \isacommand{by}\isamarkupfalse%
\ simp\isanewline
\ \ \ \ \ \ \ \ \ \ \isacommand{ultimately}\isamarkupfalse%
\ \isacommand{show}\isamarkupfalse%
\ {\isachardoublequoteopen}p\ {\isacharequal}{\kern0pt}\ restrict\ y\ {\isacharbraceleft}{\kern0pt}{\isachardot}{\kern0pt}{\isachardot}{\kern0pt}{\isacharless}{\kern0pt}{\isadigit{1}}{\isacharbraceright}{\kern0pt}{\isachardoublequoteclose}\ \isacommand{by}\isamarkupfalse%
\ auto\ \isanewline
\ \ \ \ \ \ \ \ \isacommand{qed}\isamarkupfalse%
\isanewline
\isanewline
\ \ \ \ \ \ \ \isacommand{from}\isamarkupfalse%
\ that\ \isacommand{have}\isamarkupfalse%
\ {\isachardoublequoteopen}T\ y\ i\ {\isacharequal}{\kern0pt}\ {\isacharparenleft}{\kern0pt}T{\isacharprime}{\kern0pt}\ {\isacharparenleft}{\kern0pt}{\isasymlambda}z\ {\isasymin}\ {\isacharbraceleft}{\kern0pt}{\isachardot}{\kern0pt}{\isachardot}{\kern0pt}{\isacharless}{\kern0pt}{\isadigit{1}}{\isacharbraceright}{\kern0pt}{\isachardot}{\kern0pt}\ y\ z{\isacharparenright}{\kern0pt}\ {\isacharparenleft}{\kern0pt}{\isasymlambda}z\ {\isasymin}\ {\isacharbraceleft}{\kern0pt}{\isachardot}{\kern0pt}{\isachardot}{\kern0pt}{\isacharless}{\kern0pt}k{\isacharbraceright}{\kern0pt}{\isachardot}{\kern0pt}\ y\ {\isacharparenleft}{\kern0pt}z\ {\isacharplus}{\kern0pt}\ {\isadigit{1}}{\isacharparenright}{\kern0pt}{\isacharparenright}{\kern0pt}{\isacharparenright}{\kern0pt}\ i{\isachardoublequoteclose}\ \isacommand{unfolding}\isamarkupfalse%
\ T{\isacharunderscore}{\kern0pt}def\ \isacommand{by}\isamarkupfalse%
\ auto\isanewline
\ \ \ \ \ \ \ \isacommand{also}\isamarkupfalse%
\ \isacommand{have}\isamarkupfalse%
\ {\isachardoublequoteopen}{\isachardot}{\kern0pt}{\isachardot}{\kern0pt}{\isachardot}{\kern0pt}\ {\isacharequal}{\kern0pt}\ {\isacharparenleft}{\kern0pt}join\ {\isacharparenleft}{\kern0pt}L{\isacharunderscore}{\kern0pt}line\ {\isacharparenleft}{\kern0pt}{\isacharparenleft}{\kern0pt}{\isasymlambda}z\ {\isasymin}\ {\isacharbraceleft}{\kern0pt}{\isachardot}{\kern0pt}{\isachardot}{\kern0pt}{\isacharless}{\kern0pt}{\isadigit{1}}{\isacharbraceright}{\kern0pt}{\isachardot}{\kern0pt}\ y\ z{\isacharparenright}{\kern0pt}\ {\isadigit{0}}{\isacharparenright}{\kern0pt}{\isacharparenright}{\kern0pt}\ {\isacharparenleft}{\kern0pt}S\ {\isacharparenleft}{\kern0pt}{\isasymlambda}z\ {\isasymin}\ {\isacharbraceleft}{\kern0pt}{\isachardot}{\kern0pt}{\isachardot}{\kern0pt}{\isacharless}{\kern0pt}k{\isacharbraceright}{\kern0pt}{\isachardot}{\kern0pt}\ y\ {\isacharparenleft}{\kern0pt}z\ {\isacharplus}{\kern0pt}\ {\isadigit{1}}{\isacharparenright}{\kern0pt}{\isacharparenright}{\kern0pt}{\isacharparenright}{\kern0pt}\ n\ m{\isacharparenright}{\kern0pt}\ i{\isachardoublequoteclose}\ \isacommand{using}\isamarkupfalse%
\ split{\isacharunderscore}{\kern0pt}cube\ that\ \isacommand{unfolding}\isamarkupfalse%
\ T{\isacharprime}{\kern0pt}{\isacharunderscore}{\kern0pt}def\ \isacommand{by}\isamarkupfalse%
\ simp\isanewline
\ \ \ \ \ \ \ \isacommand{also}\isamarkupfalse%
\ \isacommand{have}\isamarkupfalse%
\ {\isachardoublequoteopen}{\isachardot}{\kern0pt}{\isachardot}{\kern0pt}{\isachardot}{\kern0pt}\ {\isacharequal}{\kern0pt}\ {\isacharparenleft}{\kern0pt}L{\isacharunderscore}{\kern0pt}line\ {\isacharparenleft}{\kern0pt}{\isacharparenleft}{\kern0pt}{\isasymlambda}z\ {\isasymin}\ {\isacharbraceleft}{\kern0pt}{\isachardot}{\kern0pt}{\isachardot}{\kern0pt}{\isacharless}{\kern0pt}{\isadigit{1}}{\isacharbraceright}{\kern0pt}{\isachardot}{\kern0pt}\ y\ z{\isacharparenright}{\kern0pt}\ {\isadigit{0}}{\isacharparenright}{\kern0pt}{\isacharparenright}{\kern0pt}\ i{\isachardoublequoteclose}\ \isacommand{using}\isamarkupfalse%
\ XY\ \isacommand{unfolding}\isamarkupfalse%
\ join{\isacharunderscore}{\kern0pt}def\ \isacommand{by}\isamarkupfalse%
\ simp\isanewline
\ \ \ \ \ \ \ \isacommand{also}\isamarkupfalse%
\ \isacommand{have}\isamarkupfalse%
\ {\isachardoublequoteopen}{\isachardot}{\kern0pt}{\isachardot}{\kern0pt}{\isachardot}{\kern0pt}\ {\isacharequal}{\kern0pt}\ L\ {\isacharparenleft}{\kern0pt}SOME\ p{\isachardot}{\kern0pt}\ p{\isasymin}cube\ {\isadigit{1}}\ {\isacharparenleft}{\kern0pt}t{\isacharplus}{\kern0pt}{\isadigit{1}}{\isacharparenright}{\kern0pt}\ {\isasymand}\ p\ {\isadigit{0}}\ {\isacharequal}{\kern0pt}\ {\isacharparenleft}{\kern0pt}{\isacharparenleft}{\kern0pt}{\isasymlambda}z\ {\isasymin}\ {\isacharbraceleft}{\kern0pt}{\isachardot}{\kern0pt}{\isachardot}{\kern0pt}{\isacharless}{\kern0pt}{\isadigit{1}}{\isacharbraceright}{\kern0pt}{\isachardot}{\kern0pt}\ y\ z{\isacharparenright}{\kern0pt}\ {\isadigit{0}}{\isacharparenright}{\kern0pt}{\isacharparenright}{\kern0pt}\ i{\isachardoublequoteclose}\ \isacommand{using}\isamarkupfalse%
\ XZ\ \isacommand{unfolding}\isamarkupfalse%
\ L{\isacharunderscore}{\kern0pt}line{\isacharunderscore}{\kern0pt}def\ \isacommand{by}\isamarkupfalse%
\ auto\isanewline
\ \ \ \ \ \ \ \isacommand{also}\isamarkupfalse%
\ \isacommand{have}\isamarkupfalse%
\ {\isachardoublequoteopen}{\isachardot}{\kern0pt}{\isachardot}{\kern0pt}{\isachardot}{\kern0pt}\ {\isacharequal}{\kern0pt}\ L\ {\isacharparenleft}{\kern0pt}{\isasymlambda}z\ {\isasymin}\ {\isacharbraceleft}{\kern0pt}{\isachardot}{\kern0pt}{\isachardot}{\kern0pt}{\isacharless}{\kern0pt}{\isadigit{1}}{\isacharbraceright}{\kern0pt}{\isachardot}{\kern0pt}\ y\ z{\isacharparenright}{\kern0pt}\ i{\isachardoublequoteclose}\ \isacommand{using}\isamarkupfalse%
\ some{\isacharunderscore}{\kern0pt}eq{\isacharunderscore}{\kern0pt}restrict\ \isacommand{by}\isamarkupfalse%
\ simp\isanewline
\ \ \ \ \ \ \ \isacommand{also}\isamarkupfalse%
\ \isacommand{have}\isamarkupfalse%
\ {\isachardoublequoteopen}{\isachardot}{\kern0pt}{\isachardot}{\kern0pt}{\isachardot}{\kern0pt}\ {\isacharequal}{\kern0pt}\ fL\ i{\isachardoublequoteclose}\ \isacommand{using}\isamarkupfalse%
\ BfL{\isacharunderscore}{\kern0pt}props{\isacharparenleft}{\kern0pt}{\isadigit{6}}{\isacharparenright}{\kern0pt}\ XX\ {\isadigit{2}}\ \isacommand{by}\isamarkupfalse%
\ blast\isanewline
\ \ \ \ \ \ \ \isacommand{also}\isamarkupfalse%
\ \isacommand{have}\isamarkupfalse%
\ {\isachardoublequoteopen}{\isachardot}{\kern0pt}{\isachardot}{\kern0pt}{\isachardot}{\kern0pt}\ {\isacharequal}{\kern0pt}\ fT\ i{\isachardoublequoteclose}\ \isacommand{using}\isamarkupfalse%
\ {\isadigit{2}}\ \isacommand{unfolding}\isamarkupfalse%
\ fT{\isacharunderscore}{\kern0pt}def\ \isacommand{by}\isamarkupfalse%
\ simp\isanewline
\ \ \ \ \ \ \ \isacommand{finally}\isamarkupfalse%
\ \isacommand{show}\isamarkupfalse%
\ {\isacharquery}{\kern0pt}thesis\ \isacommand{{\isachardot}{\kern0pt}}\isamarkupfalse%
\isanewline
\ \ \ \ \ \isacommand{qed}\isamarkupfalse%
\isanewline
\ \ \ \isacommand{next}\isamarkupfalse%
\isanewline
\ \ \ \ \ \isacommand{fix}\isamarkupfalse%
\ j\ i\ \isacommand{assume}\isamarkupfalse%
\ {\isachardoublequoteopen}j\ {\isacharless}{\kern0pt}\ k\ {\isacharplus}{\kern0pt}\ {\isadigit{1}}{\isachardoublequoteclose}\ {\isachardoublequoteopen}i\ {\isasymin}\ BT\ j{\isachardoublequoteclose}\isanewline
\ \ \ \ \ \isacommand{then}\isamarkupfalse%
\ \isacommand{have}\isamarkupfalse%
\ i{\isacharunderscore}{\kern0pt}prop{\isacharcolon}{\kern0pt}\ {\isachardoublequoteopen}i\ {\isasymin}\ Bvar\ j{\isachardoublequoteclose}\ \isacommand{unfolding}\isamarkupfalse%
\ BT{\isacharunderscore}{\kern0pt}def\ \isacommand{by}\isamarkupfalse%
\ auto\isanewline
\ \ \ \ \ \ \isacommand{consider}\isamarkupfalse%
\ {\isachardoublequoteopen}j\ {\isacharequal}{\kern0pt}\ {\isadigit{0}}{\isachardoublequoteclose}\ {\isacharbar}{\kern0pt}\ {\isachardoublequoteopen}j\ {\isachargreater}{\kern0pt}\ {\isadigit{0}}{\isachardoublequoteclose}\ \isacommand{by}\isamarkupfalse%
\ auto\isanewline
\ \ \ \ \ \isacommand{then}\isamarkupfalse%
\ \isacommand{show}\isamarkupfalse%
\ {\isachardoublequoteopen}T\ y\ i\ {\isacharequal}{\kern0pt}\ y\ j{\isachardoublequoteclose}\isanewline
\ \ \ \ \ \isacommand{proof}\isamarkupfalse%
\ cases\isanewline
\ \ \ \ \ \ \ \isacommand{case}\isamarkupfalse%
\ {\isadigit{1}}\isanewline
\ \ \ \ \ \ \ \isacommand{then}\isamarkupfalse%
\ \isacommand{have}\isamarkupfalse%
\ {\isachardoublequoteopen}i\ {\isasymin}\ BL\ {\isadigit{0}}{\isachardoublequoteclose}\ \isacommand{using}\isamarkupfalse%
\ i{\isacharunderscore}{\kern0pt}prop\ \isacommand{unfolding}\isamarkupfalse%
\ Bvar{\isacharunderscore}{\kern0pt}def\ \isacommand{by}\isamarkupfalse%
\ auto\isanewline
\ \ \ \ \ \ \ \isacommand{then}\isamarkupfalse%
\ \isacommand{have}\isamarkupfalse%
\ XY{\isacharcolon}{\kern0pt}\ {\isachardoublequoteopen}i\ {\isasymin}\ {\isacharbraceleft}{\kern0pt}{\isachardot}{\kern0pt}{\isachardot}{\kern0pt}{\isacharless}{\kern0pt}n{\isacharbraceright}{\kern0pt}{\isachardoublequoteclose}\ \isacommand{using}\isamarkupfalse%
\ {\isadigit{1}}\ BfL{\isacharunderscore}{\kern0pt}props{\isacharparenleft}{\kern0pt}{\isadigit{2}}{\isacharparenright}{\kern0pt}\ \isacommand{by}\isamarkupfalse%
\ blast\isanewline
\ \ \ \ \ \ \ \isacommand{have}\isamarkupfalse%
\ XX{\isacharcolon}{\kern0pt}\ {\isachardoublequoteopen}{\isacharparenleft}{\kern0pt}{\isasymlambda}z\ {\isasymin}\ {\isacharbraceleft}{\kern0pt}{\isachardot}{\kern0pt}{\isachardot}{\kern0pt}{\isacharless}{\kern0pt}{\isadigit{1}}{\isacharbraceright}{\kern0pt}{\isachardot}{\kern0pt}\ y\ z{\isacharparenright}{\kern0pt}\ \ {\isasymin}\ cube\ {\isadigit{1}}\ {\isacharparenleft}{\kern0pt}t{\isacharplus}{\kern0pt}{\isadigit{1}}{\isacharparenright}{\kern0pt}{\isachardoublequoteclose}\ \isacommand{using}\isamarkupfalse%
\ that\ split{\isacharunderscore}{\kern0pt}cube\ \isacommand{by}\isamarkupfalse%
\ simp\isanewline
\ \ \ \ \ \ \ \isacommand{have}\isamarkupfalse%
\ XZ{\isacharcolon}{\kern0pt}\ {\isachardoublequoteopen}y\ {\isadigit{0}}\ {\isasymin}\ {\isacharbraceleft}{\kern0pt}{\isachardot}{\kern0pt}{\isachardot}{\kern0pt}{\isacharless}{\kern0pt}t{\isacharplus}{\kern0pt}{\isadigit{1}}{\isacharbraceright}{\kern0pt}{\isachardoublequoteclose}\ \isacommand{using}\isamarkupfalse%
\ that\ \isacommand{unfolding}\isamarkupfalse%
\ cube{\isacharunderscore}{\kern0pt}def\ \isacommand{by}\isamarkupfalse%
\ auto\isanewline
\isanewline
\ \ \ \ \ \ \ \ \isacommand{have}\isamarkupfalse%
\ some{\isacharunderscore}{\kern0pt}eq{\isacharunderscore}{\kern0pt}restrict{\isacharcolon}{\kern0pt}\ {\isachardoublequoteopen}{\isacharparenleft}{\kern0pt}SOME\ p{\isachardot}{\kern0pt}\ p{\isasymin}cube\ {\isadigit{1}}\ {\isacharparenleft}{\kern0pt}t{\isacharplus}{\kern0pt}{\isadigit{1}}{\isacharparenright}{\kern0pt}\ {\isasymand}\ p\ {\isadigit{0}}\ {\isacharequal}{\kern0pt}\ {\isacharparenleft}{\kern0pt}{\isacharparenleft}{\kern0pt}{\isasymlambda}z\ {\isasymin}\ {\isacharbraceleft}{\kern0pt}{\isachardot}{\kern0pt}{\isachardot}{\kern0pt}{\isacharless}{\kern0pt}{\isadigit{1}}{\isacharbraceright}{\kern0pt}{\isachardot}{\kern0pt}\ y\ z{\isacharparenright}{\kern0pt}\ {\isadigit{0}}{\isacharparenright}{\kern0pt}{\isacharparenright}{\kern0pt}\ {\isacharequal}{\kern0pt}\ {\isacharparenleft}{\kern0pt}{\isasymlambda}z\ {\isasymin}\ {\isacharbraceleft}{\kern0pt}{\isachardot}{\kern0pt}{\isachardot}{\kern0pt}{\isacharless}{\kern0pt}{\isadigit{1}}{\isacharbraceright}{\kern0pt}{\isachardot}{\kern0pt}\ y\ z{\isacharparenright}{\kern0pt}{\isachardoublequoteclose}\isanewline
\ \ \ \ \ \ \ \ \isacommand{proof}\isamarkupfalse%
\ \isanewline
\ \ \ \ \ \ \ \ \ \ \isacommand{show}\isamarkupfalse%
\ {\isachardoublequoteopen}restrict\ y\ {\isacharbraceleft}{\kern0pt}{\isachardot}{\kern0pt}{\isachardot}{\kern0pt}{\isacharless}{\kern0pt}{\isadigit{1}}{\isacharbraceright}{\kern0pt}\ {\isasymin}\ cube\ {\isadigit{1}}\ {\isacharparenleft}{\kern0pt}t\ {\isacharplus}{\kern0pt}\ {\isadigit{1}}{\isacharparenright}{\kern0pt}\ {\isasymand}\ restrict\ y\ {\isacharbraceleft}{\kern0pt}{\isachardot}{\kern0pt}{\isachardot}{\kern0pt}{\isacharless}{\kern0pt}{\isadigit{1}}{\isacharbraceright}{\kern0pt}\ {\isadigit{0}}\ {\isacharequal}{\kern0pt}\ restrict\ y\ {\isacharbraceleft}{\kern0pt}{\isachardot}{\kern0pt}{\isachardot}{\kern0pt}{\isacharless}{\kern0pt}{\isadigit{1}}{\isacharbraceright}{\kern0pt}\ {\isadigit{0}}{\isachardoublequoteclose}\ \isacommand{using}\isamarkupfalse%
\ XX\ \isacommand{by}\isamarkupfalse%
\ simp\isanewline
\ \ \ \ \ \ \ \ \isacommand{next}\isamarkupfalse%
\isanewline
\ \ \ \ \ \ \ \ \ \ \isacommand{fix}\isamarkupfalse%
\ p\isanewline
\ \ \ \ \ \ \ \ \ \ \isacommand{assume}\isamarkupfalse%
\ {\isachardoublequoteopen}p\ {\isasymin}\ cube\ {\isadigit{1}}\ {\isacharparenleft}{\kern0pt}t{\isacharplus}{\kern0pt}{\isadigit{1}}{\isacharparenright}{\kern0pt}\ {\isasymand}\ p\ {\isadigit{0}}\ {\isacharequal}{\kern0pt}\ restrict\ y\ {\isacharbraceleft}{\kern0pt}{\isachardot}{\kern0pt}{\isachardot}{\kern0pt}{\isacharless}{\kern0pt}{\isadigit{1}}{\isacharbraceright}{\kern0pt}\ {\isadigit{0}}{\isachardoublequoteclose}\isanewline
\ \ \ \ \ \ \ \ \ \ \isacommand{moreover}\isamarkupfalse%
\ \isacommand{have}\isamarkupfalse%
\ {\isachardoublequoteopen}p\ u\ {\isacharequal}{\kern0pt}\ restrict\ y\ {\isacharbraceleft}{\kern0pt}{\isachardot}{\kern0pt}{\isachardot}{\kern0pt}{\isacharless}{\kern0pt}{\isadigit{1}}{\isacharbraceright}{\kern0pt}\ u{\isachardoublequoteclose}\ \isakeyword{if}\ {\isachardoublequoteopen}u\ {\isasymnotin}\ {\isacharbraceleft}{\kern0pt}{\isachardot}{\kern0pt}{\isachardot}{\kern0pt}{\isacharless}{\kern0pt}{\isadigit{1}}{\isacharbraceright}{\kern0pt}{\isachardoublequoteclose}\ \isakeyword{for}\ u\ \isacommand{using}\isamarkupfalse%
\ that\ calculation\ XX\ \isacommand{unfolding}\isamarkupfalse%
\ cube{\isacharunderscore}{\kern0pt}def\ \isacommand{using}\isamarkupfalse%
\ PiE{\isacharunderscore}{\kern0pt}arb{\isacharbrackleft}{\kern0pt}of\ {\isachardoublequoteopen}restrict\ y\ {\isacharbraceleft}{\kern0pt}{\isachardot}{\kern0pt}{\isachardot}{\kern0pt}{\isacharless}{\kern0pt}{\isadigit{1}}{\isacharbraceright}{\kern0pt}{\isachardoublequoteclose}\ {\isachardoublequoteopen}{\isacharbraceleft}{\kern0pt}{\isachardot}{\kern0pt}{\isachardot}{\kern0pt}{\isacharless}{\kern0pt}{\isadigit{1}}{\isacharbraceright}{\kern0pt}{\isachardoublequoteclose}\ {\isachardoublequoteopen}{\isasymlambda}x{\isachardot}{\kern0pt}\ {\isacharbraceleft}{\kern0pt}{\isachardot}{\kern0pt}{\isachardot}{\kern0pt}{\isacharless}{\kern0pt}t\ {\isacharplus}{\kern0pt}\ {\isadigit{1}}{\isacharbraceright}{\kern0pt}{\isachardoublequoteclose}\ u{\isacharbrackright}{\kern0pt}\ \ PiE{\isacharunderscore}{\kern0pt}arb{\isacharbrackleft}{\kern0pt}of\ p\ {\isachardoublequoteopen}{\isacharbraceleft}{\kern0pt}{\isachardot}{\kern0pt}{\isachardot}{\kern0pt}{\isacharless}{\kern0pt}{\isadigit{1}}{\isacharbraceright}{\kern0pt}{\isachardoublequoteclose}\ {\isachardoublequoteopen}{\isasymlambda}x{\isachardot}{\kern0pt}\ {\isacharbraceleft}{\kern0pt}{\isachardot}{\kern0pt}{\isachardot}{\kern0pt}{\isacharless}{\kern0pt}t\ {\isacharplus}{\kern0pt}\ {\isadigit{1}}{\isacharbraceright}{\kern0pt}{\isachardoublequoteclose}\ u{\isacharbrackright}{\kern0pt}\ \isacommand{by}\isamarkupfalse%
\ simp\isanewline
\ \ \ \ \ \ \ \ \ \ \isacommand{ultimately}\isamarkupfalse%
\ \isacommand{show}\isamarkupfalse%
\ {\isachardoublequoteopen}p\ {\isacharequal}{\kern0pt}\ restrict\ y\ {\isacharbraceleft}{\kern0pt}{\isachardot}{\kern0pt}{\isachardot}{\kern0pt}{\isacharless}{\kern0pt}{\isadigit{1}}{\isacharbraceright}{\kern0pt}{\isachardoublequoteclose}\ \isacommand{by}\isamarkupfalse%
\ auto\ \isanewline
\ \ \ \ \ \ \ \ \isacommand{qed}\isamarkupfalse%
\isanewline
\isanewline
\ \ \ \ \ \ \ \isacommand{from}\isamarkupfalse%
\ that\ \isacommand{have}\isamarkupfalse%
\ {\isachardoublequoteopen}T\ y\ i\ {\isacharequal}{\kern0pt}\ {\isacharparenleft}{\kern0pt}T{\isacharprime}{\kern0pt}\ {\isacharparenleft}{\kern0pt}{\isasymlambda}z\ {\isasymin}\ {\isacharbraceleft}{\kern0pt}{\isachardot}{\kern0pt}{\isachardot}{\kern0pt}{\isacharless}{\kern0pt}{\isadigit{1}}{\isacharbraceright}{\kern0pt}{\isachardot}{\kern0pt}\ y\ z{\isacharparenright}{\kern0pt}\ {\isacharparenleft}{\kern0pt}{\isasymlambda}z\ {\isasymin}\ {\isacharbraceleft}{\kern0pt}{\isachardot}{\kern0pt}{\isachardot}{\kern0pt}{\isacharless}{\kern0pt}k{\isacharbraceright}{\kern0pt}{\isachardot}{\kern0pt}\ y\ {\isacharparenleft}{\kern0pt}z\ {\isacharplus}{\kern0pt}\ {\isadigit{1}}{\isacharparenright}{\kern0pt}{\isacharparenright}{\kern0pt}{\isacharparenright}{\kern0pt}\ i{\isachardoublequoteclose}\ \isacommand{unfolding}\isamarkupfalse%
\ T{\isacharunderscore}{\kern0pt}def\ \isacommand{by}\isamarkupfalse%
\ auto\isanewline
\ \ \ \ \ \ \ \isacommand{also}\isamarkupfalse%
\ \isacommand{have}\isamarkupfalse%
\ {\isachardoublequoteopen}{\isachardot}{\kern0pt}{\isachardot}{\kern0pt}{\isachardot}{\kern0pt}\ {\isacharequal}{\kern0pt}\ {\isacharparenleft}{\kern0pt}join\ {\isacharparenleft}{\kern0pt}L{\isacharunderscore}{\kern0pt}line\ {\isacharparenleft}{\kern0pt}{\isacharparenleft}{\kern0pt}{\isasymlambda}z\ {\isasymin}\ {\isacharbraceleft}{\kern0pt}{\isachardot}{\kern0pt}{\isachardot}{\kern0pt}{\isacharless}{\kern0pt}{\isadigit{1}}{\isacharbraceright}{\kern0pt}{\isachardot}{\kern0pt}\ y\ z{\isacharparenright}{\kern0pt}\ {\isadigit{0}}{\isacharparenright}{\kern0pt}{\isacharparenright}{\kern0pt}\ {\isacharparenleft}{\kern0pt}S\ {\isacharparenleft}{\kern0pt}{\isasymlambda}z\ {\isasymin}\ {\isacharbraceleft}{\kern0pt}{\isachardot}{\kern0pt}{\isachardot}{\kern0pt}{\isacharless}{\kern0pt}k{\isacharbraceright}{\kern0pt}{\isachardot}{\kern0pt}\ y\ {\isacharparenleft}{\kern0pt}z\ {\isacharplus}{\kern0pt}\ {\isadigit{1}}{\isacharparenright}{\kern0pt}{\isacharparenright}{\kern0pt}{\isacharparenright}{\kern0pt}\ n\ m{\isacharparenright}{\kern0pt}\ i{\isachardoublequoteclose}\ \isacommand{using}\isamarkupfalse%
\ split{\isacharunderscore}{\kern0pt}cube\ that\ \isacommand{unfolding}\isamarkupfalse%
\ T{\isacharprime}{\kern0pt}{\isacharunderscore}{\kern0pt}def\ \isacommand{by}\isamarkupfalse%
\ simp\isanewline
\ \ \ \ \ \ \ \isacommand{also}\isamarkupfalse%
\ \isacommand{have}\isamarkupfalse%
\ {\isachardoublequoteopen}{\isachardot}{\kern0pt}{\isachardot}{\kern0pt}{\isachardot}{\kern0pt}\ {\isacharequal}{\kern0pt}\ {\isacharparenleft}{\kern0pt}L{\isacharunderscore}{\kern0pt}line\ {\isacharparenleft}{\kern0pt}{\isacharparenleft}{\kern0pt}{\isasymlambda}z\ {\isasymin}\ {\isacharbraceleft}{\kern0pt}{\isachardot}{\kern0pt}{\isachardot}{\kern0pt}{\isacharless}{\kern0pt}{\isadigit{1}}{\isacharbraceright}{\kern0pt}{\isachardot}{\kern0pt}\ y\ z{\isacharparenright}{\kern0pt}\ {\isadigit{0}}{\isacharparenright}{\kern0pt}{\isacharparenright}{\kern0pt}\ i{\isachardoublequoteclose}\ \isacommand{using}\isamarkupfalse%
\ XY\ \isacommand{unfolding}\isamarkupfalse%
\ join{\isacharunderscore}{\kern0pt}def\ \isacommand{by}\isamarkupfalse%
\ simp\isanewline
\ \ \ \ \ \ \ \isacommand{also}\isamarkupfalse%
\ \isacommand{have}\isamarkupfalse%
\ {\isachardoublequoteopen}{\isachardot}{\kern0pt}{\isachardot}{\kern0pt}{\isachardot}{\kern0pt}\ {\isacharequal}{\kern0pt}\ L\ {\isacharparenleft}{\kern0pt}SOME\ p{\isachardot}{\kern0pt}\ p{\isasymin}cube\ {\isadigit{1}}\ {\isacharparenleft}{\kern0pt}t{\isacharplus}{\kern0pt}{\isadigit{1}}{\isacharparenright}{\kern0pt}\ {\isasymand}\ p\ {\isadigit{0}}\ {\isacharequal}{\kern0pt}\ {\isacharparenleft}{\kern0pt}{\isacharparenleft}{\kern0pt}{\isasymlambda}z\ {\isasymin}\ {\isacharbraceleft}{\kern0pt}{\isachardot}{\kern0pt}{\isachardot}{\kern0pt}{\isacharless}{\kern0pt}{\isadigit{1}}{\isacharbraceright}{\kern0pt}{\isachardot}{\kern0pt}\ y\ z{\isacharparenright}{\kern0pt}\ {\isadigit{0}}{\isacharparenright}{\kern0pt}{\isacharparenright}{\kern0pt}\ i{\isachardoublequoteclose}\ \isacommand{using}\isamarkupfalse%
\ XZ\ \isacommand{unfolding}\isamarkupfalse%
\ L{\isacharunderscore}{\kern0pt}line{\isacharunderscore}{\kern0pt}def\ \isacommand{by}\isamarkupfalse%
\ auto\isanewline
\ \ \ \ \ \ \ \isacommand{also}\isamarkupfalse%
\ \isacommand{have}\isamarkupfalse%
\ {\isachardoublequoteopen}{\isachardot}{\kern0pt}{\isachardot}{\kern0pt}{\isachardot}{\kern0pt}\ {\isacharequal}{\kern0pt}\ L\ {\isacharparenleft}{\kern0pt}{\isasymlambda}z\ {\isasymin}\ {\isacharbraceleft}{\kern0pt}{\isachardot}{\kern0pt}{\isachardot}{\kern0pt}{\isacharless}{\kern0pt}{\isadigit{1}}{\isacharbraceright}{\kern0pt}{\isachardot}{\kern0pt}\ y\ z{\isacharparenright}{\kern0pt}\ i{\isachardoublequoteclose}\ \isacommand{using}\isamarkupfalse%
\ some{\isacharunderscore}{\kern0pt}eq{\isacharunderscore}{\kern0pt}restrict\ \isacommand{by}\isamarkupfalse%
\ simp\isanewline
\ \ \ \ \ \ \ \isacommand{also}\isamarkupfalse%
\ \isacommand{have}\isamarkupfalse%
\ {\isachardoublequoteopen}{\isachardot}{\kern0pt}{\isachardot}{\kern0pt}{\isachardot}{\kern0pt}\ {\isacharequal}{\kern0pt}\ \ {\isacharparenleft}{\kern0pt}{\isasymlambda}z\ {\isasymin}\ {\isacharbraceleft}{\kern0pt}{\isachardot}{\kern0pt}{\isachardot}{\kern0pt}{\isacharless}{\kern0pt}{\isadigit{1}}{\isacharbraceright}{\kern0pt}{\isachardot}{\kern0pt}\ y\ z{\isacharparenright}{\kern0pt}\ j{\isachardoublequoteclose}\ \isacommand{using}\isamarkupfalse%
\ BfL{\isacharunderscore}{\kern0pt}props{\isacharparenleft}{\kern0pt}{\isadigit{6}}{\isacharparenright}{\kern0pt}\ XX\ {\isadigit{1}}\ \ {\isacartoucheopen}i\ {\isasymin}\ BL\ {\isadigit{0}}{\isacartoucheclose}\ \isacommand{by}\isamarkupfalse%
\ blast\ \isanewline
\ \ \ \ \ \ \ \isacommand{also}\isamarkupfalse%
\ \isacommand{have}\isamarkupfalse%
\ {\isachardoublequoteopen}{\isachardot}{\kern0pt}{\isachardot}{\kern0pt}{\isachardot}{\kern0pt}\ {\isacharequal}{\kern0pt}\ {\isacharparenleft}{\kern0pt}{\isasymlambda}z\ {\isasymin}\ {\isacharbraceleft}{\kern0pt}{\isachardot}{\kern0pt}{\isachardot}{\kern0pt}{\isacharless}{\kern0pt}{\isadigit{1}}{\isacharbraceright}{\kern0pt}{\isachardot}{\kern0pt}\ y\ z{\isacharparenright}{\kern0pt}\ {\isadigit{0}}{\isachardoublequoteclose}\ \isacommand{using}\isamarkupfalse%
\ {\isadigit{1}}\ \isacommand{by}\isamarkupfalse%
\ blast\isanewline
\ \ \ \ \ \ \ \isacommand{also}\isamarkupfalse%
\ \isacommand{have}\isamarkupfalse%
\ {\isachardoublequoteopen}{\isachardot}{\kern0pt}{\isachardot}{\kern0pt}{\isachardot}{\kern0pt}\ {\isacharequal}{\kern0pt}\ y\ {\isadigit{0}}{\isachardoublequoteclose}\ \isacommand{by}\isamarkupfalse%
\ simp\isanewline
\ \ \ \ \ \ \ \isacommand{also}\isamarkupfalse%
\ \isacommand{have}\isamarkupfalse%
\ {\isachardoublequoteopen}{\isachardot}{\kern0pt}{\isachardot}{\kern0pt}{\isachardot}{\kern0pt}\ {\isacharequal}{\kern0pt}\ y\ j{\isachardoublequoteclose}\ \isacommand{using}\isamarkupfalse%
\ {\isadigit{1}}\ \isacommand{by}\isamarkupfalse%
\ simp\isanewline
\ \ \ \ \ \ \ \isacommand{finally}\isamarkupfalse%
\ \isacommand{show}\isamarkupfalse%
\ {\isacharquery}{\kern0pt}thesis\ \isacommand{{\isachardot}{\kern0pt}}\isamarkupfalse%
\isanewline
\ \ \ \ \ \isacommand{next}\isamarkupfalse%
\isanewline
\ \ \ \ \ \ \ \isacommand{case}\isamarkupfalse%
\ {\isadigit{2}}\isanewline
\ \ \ \ \ \ \ \isacommand{then}\isamarkupfalse%
\ \isacommand{have}\isamarkupfalse%
\ {\isachardoublequoteopen}i\ {\isasymin}\ shiftset\ n\ {\isacharparenleft}{\kern0pt}BS\ {\isacharparenleft}{\kern0pt}j\ {\isacharminus}{\kern0pt}\ {\isadigit{1}}{\isacharparenright}{\kern0pt}{\isacharparenright}{\kern0pt}{\isachardoublequoteclose}\ \isacommand{using}\isamarkupfalse%
\ i{\isacharunderscore}{\kern0pt}prop\ \isacommand{unfolding}\isamarkupfalse%
\ Bvar{\isacharunderscore}{\kern0pt}def\ \isacommand{by}\isamarkupfalse%
\ simp\isanewline
\ \ \ \ \ \ \ \isacommand{then}\isamarkupfalse%
\ \isacommand{have}\isamarkupfalse%
\ {\isachardoublequoteopen}{\isasymexists}s{\isacharless}{\kern0pt}m{\isachardot}{\kern0pt}\ n\ {\isacharplus}{\kern0pt}\ s\ {\isacharequal}{\kern0pt}\ i{\isachardoublequoteclose}\ \isacommand{using}\isamarkupfalse%
\ BfS{\isacharunderscore}{\kern0pt}props{\isacharparenleft}{\kern0pt}{\isadigit{2}}{\isacharparenright}{\kern0pt}\ {\isacartoucheopen}j\ {\isacharless}{\kern0pt}\ k\ {\isacharplus}{\kern0pt}\ {\isadigit{1}}{\isacartoucheclose}\ \isacommand{unfolding}\isamarkupfalse%
\ shiftset{\isacharunderscore}{\kern0pt}def\ \isacommand{by}\isamarkupfalse%
\ force\ \isanewline
\ \ \ \ \ \ \ \isacommand{then}\isamarkupfalse%
\ \isacommand{obtain}\isamarkupfalse%
\ s\ \isakeyword{where}\ s{\isacharunderscore}{\kern0pt}prop{\isacharcolon}{\kern0pt}\ {\isachardoublequoteopen}s\ {\isacharless}{\kern0pt}\ m{\isachardoublequoteclose}\ {\isachardoublequoteopen}i\ {\isacharequal}{\kern0pt}\ s\ {\isacharplus}{\kern0pt}\ n{\isachardoublequoteclose}\ \isacommand{by}\isamarkupfalse%
\ auto\isanewline
\ \ \ \ \ \ \ \isacommand{then}\isamarkupfalse%
\ \isacommand{have}\isamarkupfalse%
\ {\isacharasterisk}{\kern0pt}{\isacharcolon}{\kern0pt}\ {\isachardoublequoteopen}\ i\ {\isasymin}\ {\isacharbraceleft}{\kern0pt}n{\isachardot}{\kern0pt}{\isachardot}{\kern0pt}{\isacharless}{\kern0pt}n{\isacharplus}{\kern0pt}m{\isacharbraceright}{\kern0pt}{\isachardoublequoteclose}\ \isacommand{by}\isamarkupfalse%
\ simp\isanewline
\isanewline
\ \ \ \ \ \ \ \isacommand{have}\isamarkupfalse%
\ XX{\isacharcolon}{\kern0pt}\ {\isachardoublequoteopen}{\isacharparenleft}{\kern0pt}{\isasymlambda}z\ {\isasymin}\ {\isacharbraceleft}{\kern0pt}{\isachardot}{\kern0pt}{\isachardot}{\kern0pt}{\isacharless}{\kern0pt}k{\isacharbraceright}{\kern0pt}{\isachardot}{\kern0pt}\ y\ {\isacharparenleft}{\kern0pt}z\ {\isacharplus}{\kern0pt}\ {\isadigit{1}}{\isacharparenright}{\kern0pt}{\isacharparenright}{\kern0pt}\ {\isasymin}\ cube\ k\ {\isacharparenleft}{\kern0pt}t{\isacharplus}{\kern0pt}{\isadigit{1}}{\isacharparenright}{\kern0pt}{\isachardoublequoteclose}\ \isacommand{using}\isamarkupfalse%
\ split{\isacharunderscore}{\kern0pt}cube\ that\ \isacommand{by}\isamarkupfalse%
\ simp\isanewline
\ \ \ \ \ \ \ \isacommand{have}\isamarkupfalse%
\ XY{\isacharcolon}{\kern0pt}\ {\isachardoublequoteopen}s\ {\isasymin}\ BS\ {\isacharparenleft}{\kern0pt}j\ {\isacharminus}{\kern0pt}\ {\isadigit{1}}{\isacharparenright}{\kern0pt}{\isachardoublequoteclose}\ \isacommand{using}\isamarkupfalse%
\ s{\isacharunderscore}{\kern0pt}prop\ {\isadigit{2}}\ {\isacartoucheopen}i\ {\isasymin}\ shiftset\ n\ {\isacharparenleft}{\kern0pt}BS\ {\isacharparenleft}{\kern0pt}j\ {\isacharminus}{\kern0pt}\ {\isadigit{1}}{\isacharparenright}{\kern0pt}{\isacharparenright}{\kern0pt}{\isacartoucheclose}\ \isacommand{unfolding}\isamarkupfalse%
\ shiftset{\isacharunderscore}{\kern0pt}def\ \isacommand{by}\isamarkupfalse%
\ force\isanewline
\isanewline
\ \ \ \ \ \ \ \isacommand{from}\isamarkupfalse%
\ that\ \isacommand{have}\isamarkupfalse%
\ {\isachardoublequoteopen}T\ y\ i\ {\isacharequal}{\kern0pt}\ {\isacharparenleft}{\kern0pt}T{\isacharprime}{\kern0pt}\ {\isacharparenleft}{\kern0pt}{\isasymlambda}z\ {\isasymin}\ {\isacharbraceleft}{\kern0pt}{\isachardot}{\kern0pt}{\isachardot}{\kern0pt}{\isacharless}{\kern0pt}{\isadigit{1}}{\isacharbraceright}{\kern0pt}{\isachardot}{\kern0pt}\ y\ z{\isacharparenright}{\kern0pt}\ {\isacharparenleft}{\kern0pt}{\isasymlambda}z\ {\isasymin}\ {\isacharbraceleft}{\kern0pt}{\isachardot}{\kern0pt}{\isachardot}{\kern0pt}{\isacharless}{\kern0pt}k{\isacharbraceright}{\kern0pt}{\isachardot}{\kern0pt}\ y\ {\isacharparenleft}{\kern0pt}z\ {\isacharplus}{\kern0pt}\ {\isadigit{1}}{\isacharparenright}{\kern0pt}{\isacharparenright}{\kern0pt}{\isacharparenright}{\kern0pt}\ i{\isachardoublequoteclose}\ \isacommand{unfolding}\isamarkupfalse%
\ T{\isacharunderscore}{\kern0pt}def\ \isacommand{by}\isamarkupfalse%
\ auto\isanewline
\ \ \ \ \ \ \ \isacommand{also}\isamarkupfalse%
\ \isacommand{have}\isamarkupfalse%
\ {\isachardoublequoteopen}{\isachardot}{\kern0pt}{\isachardot}{\kern0pt}{\isachardot}{\kern0pt}\ {\isacharequal}{\kern0pt}\ {\isacharparenleft}{\kern0pt}join\ {\isacharparenleft}{\kern0pt}L{\isacharunderscore}{\kern0pt}line\ {\isacharparenleft}{\kern0pt}{\isacharparenleft}{\kern0pt}{\isasymlambda}z\ {\isasymin}\ {\isacharbraceleft}{\kern0pt}{\isachardot}{\kern0pt}{\isachardot}{\kern0pt}{\isacharless}{\kern0pt}{\isadigit{1}}{\isacharbraceright}{\kern0pt}{\isachardot}{\kern0pt}\ y\ z{\isacharparenright}{\kern0pt}\ {\isadigit{0}}{\isacharparenright}{\kern0pt}{\isacharparenright}{\kern0pt}\ {\isacharparenleft}{\kern0pt}S\ {\isacharparenleft}{\kern0pt}{\isasymlambda}z\ {\isasymin}\ {\isacharbraceleft}{\kern0pt}{\isachardot}{\kern0pt}{\isachardot}{\kern0pt}{\isacharless}{\kern0pt}k{\isacharbraceright}{\kern0pt}{\isachardot}{\kern0pt}\ y\ {\isacharparenleft}{\kern0pt}z\ {\isacharplus}{\kern0pt}\ {\isadigit{1}}{\isacharparenright}{\kern0pt}{\isacharparenright}{\kern0pt}{\isacharparenright}{\kern0pt}\ n\ m{\isacharparenright}{\kern0pt}\ i{\isachardoublequoteclose}\ \isacommand{using}\isamarkupfalse%
\ split{\isacharunderscore}{\kern0pt}cube\ that\ \isacommand{unfolding}\isamarkupfalse%
\ T{\isacharprime}{\kern0pt}{\isacharunderscore}{\kern0pt}def\ \isacommand{by}\isamarkupfalse%
\ simp\isanewline
\ \ \ \ \ \ \ \isacommand{also}\isamarkupfalse%
\ \isacommand{have}\isamarkupfalse%
\ {\isachardoublequoteopen}{\isachardot}{\kern0pt}{\isachardot}{\kern0pt}{\isachardot}{\kern0pt}\ {\isacharequal}{\kern0pt}\ {\isacharparenleft}{\kern0pt}join\ {\isacharparenleft}{\kern0pt}L{\isacharunderscore}{\kern0pt}line\ {\isacharparenleft}{\kern0pt}y\ {\isadigit{0}}{\isacharparenright}{\kern0pt}{\isacharparenright}{\kern0pt}\ {\isacharparenleft}{\kern0pt}S\ {\isacharparenleft}{\kern0pt}{\isasymlambda}z\ {\isasymin}\ {\isacharbraceleft}{\kern0pt}{\isachardot}{\kern0pt}{\isachardot}{\kern0pt}{\isacharless}{\kern0pt}k{\isacharbraceright}{\kern0pt}{\isachardot}{\kern0pt}\ y\ {\isacharparenleft}{\kern0pt}z\ {\isacharplus}{\kern0pt}\ {\isadigit{1}}{\isacharparenright}{\kern0pt}{\isacharparenright}{\kern0pt}{\isacharparenright}{\kern0pt}\ n\ m{\isacharparenright}{\kern0pt}\ i{\isachardoublequoteclose}\ \isacommand{by}\isamarkupfalse%
\ simp\isanewline
\ \ \ \ \ \ \ \isacommand{also}\isamarkupfalse%
\ \isacommand{have}\isamarkupfalse%
\ {\isachardoublequoteopen}{\isachardot}{\kern0pt}{\isachardot}{\kern0pt}{\isachardot}{\kern0pt}\ {\isacharequal}{\kern0pt}\ {\isacharparenleft}{\kern0pt}S\ {\isacharparenleft}{\kern0pt}{\isasymlambda}z\ {\isasymin}\ {\isacharbraceleft}{\kern0pt}{\isachardot}{\kern0pt}{\isachardot}{\kern0pt}{\isacharless}{\kern0pt}k{\isacharbraceright}{\kern0pt}{\isachardot}{\kern0pt}\ y\ {\isacharparenleft}{\kern0pt}z\ {\isacharplus}{\kern0pt}\ {\isadigit{1}}{\isacharparenright}{\kern0pt}{\isacharparenright}{\kern0pt}{\isacharparenright}{\kern0pt}\ s{\isachardoublequoteclose}\ \isacommand{using}\isamarkupfalse%
\ {\isacharasterisk}{\kern0pt}\ s{\isacharunderscore}{\kern0pt}prop\ \isacommand{unfolding}\isamarkupfalse%
\ join{\isacharunderscore}{\kern0pt}def\ \isacommand{by}\isamarkupfalse%
\ simp\isanewline
\ \ \ \ \ \ \ \isacommand{also}\isamarkupfalse%
\ \isacommand{have}\isamarkupfalse%
\ {\isachardoublequoteopen}{\isachardot}{\kern0pt}{\isachardot}{\kern0pt}{\isachardot}{\kern0pt}\ {\isacharequal}{\kern0pt}\ {\isacharparenleft}{\kern0pt}{\isasymlambda}z\ {\isasymin}\ {\isacharbraceleft}{\kern0pt}{\isachardot}{\kern0pt}{\isachardot}{\kern0pt}{\isacharless}{\kern0pt}k{\isacharbraceright}{\kern0pt}{\isachardot}{\kern0pt}\ y\ {\isacharparenleft}{\kern0pt}z\ {\isacharplus}{\kern0pt}\ {\isadigit{1}}{\isacharparenright}{\kern0pt}{\isacharparenright}{\kern0pt}\ {\isacharparenleft}{\kern0pt}j{\isacharminus}{\kern0pt}{\isadigit{1}}{\isacharparenright}{\kern0pt}{\isachardoublequoteclose}\ \isacommand{using}\isamarkupfalse%
\ XX\ XY\ BfS{\isacharunderscore}{\kern0pt}props{\isacharparenleft}{\kern0pt}{\isadigit{6}}{\isacharparenright}{\kern0pt}\ {\isadigit{2}}\ {\isacartoucheopen}j\ {\isacharless}{\kern0pt}\ k\ {\isacharplus}{\kern0pt}\ {\isadigit{1}}{\isacartoucheclose}\ \isacommand{by}\isamarkupfalse%
\ auto\isanewline
\ \ \ \ \ \ \ \isacommand{also}\isamarkupfalse%
\ \isacommand{have}\isamarkupfalse%
\ {\isachardoublequoteopen}{\isachardot}{\kern0pt}{\isachardot}{\kern0pt}{\isachardot}{\kern0pt}\ {\isacharequal}{\kern0pt}\ y\ j{\isachardoublequoteclose}\ \isacommand{using}\isamarkupfalse%
\ {\isadigit{2}}\ {\isacartoucheopen}j\ {\isacharless}{\kern0pt}\ k\ {\isacharplus}{\kern0pt}\ {\isadigit{1}}{\isacartoucheclose}\ \isacommand{by}\isamarkupfalse%
\ force\isanewline
\ \ \ \ \ \ \ \isacommand{finally}\isamarkupfalse%
\ \isacommand{show}\isamarkupfalse%
\ {\isacharquery}{\kern0pt}thesis\ \isacommand{{\isachardot}{\kern0pt}}\isamarkupfalse%
\isanewline
\ \ \ \ \ \isacommand{qed}\isamarkupfalse%
\isanewline
\ \ \ \isacommand{qed}\isamarkupfalse%
\isanewline
\isanewline
\isanewline
\isanewline
\ \ \ \isacommand{from}\isamarkupfalse%
\ F{\isadigit{1}}\ F{\isadigit{2}}\ F{\isadigit{3}}\ F{\isadigit{4}}\ F{\isadigit{5}}\ \isacommand{have}\isamarkupfalse%
\ subspace{\isacharunderscore}{\kern0pt}T{\isacharcolon}{\kern0pt}\ {\isachardoublequoteopen}is{\isacharunderscore}{\kern0pt}subspace\ T\ {\isacharparenleft}{\kern0pt}k{\isacharplus}{\kern0pt}{\isadigit{1}}{\isacharparenright}{\kern0pt}\ {\isacharparenleft}{\kern0pt}n{\isacharplus}{\kern0pt}m{\isacharparenright}{\kern0pt}\ {\isacharparenleft}{\kern0pt}t{\isacharplus}{\kern0pt}{\isadigit{1}}{\isacharparenright}{\kern0pt}{\isachardoublequoteclose}\ \isacommand{unfolding}\isamarkupfalse%
\ is{\isacharunderscore}{\kern0pt}subspace{\isacharunderscore}{\kern0pt}def\ \isacommand{using}\isamarkupfalse%
\ T{\isacharunderscore}{\kern0pt}prop\ \isacommand{by}\isamarkupfalse%
\ metis\isanewline
\isanewline
\isanewline
\isanewline
\isanewline
\isanewline
\isanewline
\isanewline
\isanewline
\ \ \ \ \isacommand{define}\isamarkupfalse%
\ T{\isacharunderscore}{\kern0pt}class\ \isakeyword{where}\ {\isachardoublequoteopen}T{\isacharunderscore}{\kern0pt}class\ {\isasymequiv}\ {\isacharparenleft}{\kern0pt}{\isasymlambda}j{\isasymin}{\isacharbraceleft}{\kern0pt}{\isachardot}{\kern0pt}{\isachardot}{\kern0pt}k{\isacharbraceright}{\kern0pt}{\isachardot}{\kern0pt}\ {\isacharbraceleft}{\kern0pt}join\ {\isacharparenleft}{\kern0pt}L{\isacharunderscore}{\kern0pt}line\ i{\isacharparenright}{\kern0pt}\ s\ n\ m\ {\isacharbar}{\kern0pt}\ i\ s\ {\isachardot}{\kern0pt}\ i\ {\isasymin}\ {\isacharbraceleft}{\kern0pt}{\isachardot}{\kern0pt}{\isachardot}{\kern0pt}{\isacharless}{\kern0pt}t{\isacharbraceright}{\kern0pt}\ {\isasymand}\ s\ {\isasymin}\ S\ {\isacharbackquote}{\kern0pt}\ {\isacharparenleft}{\kern0pt}classes\ k\ t\ j{\isacharparenright}{\kern0pt}{\isacharbraceright}{\kern0pt}{\isacharparenright}{\kern0pt}{\isacharparenleft}{\kern0pt}k{\isacharplus}{\kern0pt}{\isadigit{1}}{\isacharcolon}{\kern0pt}{\isacharequal}{\kern0pt}\ {\isacharbraceleft}{\kern0pt}join\ {\isacharparenleft}{\kern0pt}L{\isacharunderscore}{\kern0pt}line\ t{\isacharparenright}{\kern0pt}\ {\isacharparenleft}{\kern0pt}SOME\ s{\isachardot}{\kern0pt}\ s\ {\isasymin}\ S\ {\isacharbackquote}{\kern0pt}\ {\isacharparenleft}{\kern0pt}cube\ m\ {\isacharparenleft}{\kern0pt}t{\isacharplus}{\kern0pt}{\isadigit{1}}{\isacharparenright}{\kern0pt}{\isacharparenright}{\kern0pt}{\isacharparenright}{\kern0pt}\ n\ m{\isacharbraceright}{\kern0pt}{\isacharparenright}{\kern0pt}{\isachardoublequoteclose}\isanewline
\ \ \ \ \ \ \ \isanewline
\ \ \ \ \isacommand{have}\isamarkupfalse%
\ classprop{\isacharcolon}{\kern0pt}\ {\isachardoublequoteopen}T{\isacharunderscore}{\kern0pt}class\ j\ {\isacharequal}{\kern0pt}\ T\ {\isacharbackquote}{\kern0pt}\ classes\ {\isacharparenleft}{\kern0pt}k\ {\isacharplus}{\kern0pt}\ {\isadigit{1}}{\isacharparenright}{\kern0pt}\ t\ j{\isachardoublequoteclose}\ \isakeyword{if}\ j{\isacharunderscore}{\kern0pt}prop{\isacharcolon}{\kern0pt}\ {\isachardoublequoteopen}j\ {\isasymle}\ k{\isachardoublequoteclose}\ \isakeyword{for}\ j\isanewline
\ \ \ \ \isacommand{proof}\isamarkupfalse%
\isanewline
\ \ \ \ \ \ \isacommand{show}\isamarkupfalse%
\ {\isachardoublequoteopen}T{\isacharunderscore}{\kern0pt}class\ j\ {\isasymsubseteq}\ T\ {\isacharbackquote}{\kern0pt}\ classes\ {\isacharparenleft}{\kern0pt}k\ {\isacharplus}{\kern0pt}\ {\isadigit{1}}{\isacharparenright}{\kern0pt}\ t\ j{\isachardoublequoteclose}\isanewline
\ \ \ \ \ \ \isacommand{proof}\isamarkupfalse%
\isanewline
\ \ \ \ \ \ \ \ \isacommand{fix}\isamarkupfalse%
\ x\ \isacommand{assume}\isamarkupfalse%
\ {\isachardoublequoteopen}x\ {\isasymin}\ T{\isacharunderscore}{\kern0pt}class\ j{\isachardoublequoteclose}\isanewline
\ \ \ \ \ \ \ \ \isacommand{from}\isamarkupfalse%
\ that\ \isacommand{have}\isamarkupfalse%
\ {\isachardoublequoteopen}T{\isacharunderscore}{\kern0pt}class\ j\ {\isacharequal}{\kern0pt}\ {\isacharbraceleft}{\kern0pt}join\ {\isacharparenleft}{\kern0pt}L{\isacharunderscore}{\kern0pt}line\ i{\isacharparenright}{\kern0pt}\ s\ n\ m\ {\isacharbar}{\kern0pt}\ i\ s\ {\isachardot}{\kern0pt}\ i\ {\isasymin}\ {\isacharbraceleft}{\kern0pt}{\isachardot}{\kern0pt}{\isachardot}{\kern0pt}{\isacharless}{\kern0pt}t{\isacharbraceright}{\kern0pt}\ {\isasymand}\ s\ {\isasymin}\ S\ {\isacharbackquote}{\kern0pt}\ {\isacharparenleft}{\kern0pt}classes\ k\ t\ j{\isacharparenright}{\kern0pt}{\isacharbraceright}{\kern0pt}{\isachardoublequoteclose}\ \isacommand{unfolding}\isamarkupfalse%
\ T{\isacharunderscore}{\kern0pt}class{\isacharunderscore}{\kern0pt}def\ \isacommand{by}\isamarkupfalse%
\ simp\isanewline
\ \ \ \ \ \ \ \ \isacommand{then}\isamarkupfalse%
\ \isacommand{obtain}\isamarkupfalse%
\ i\ s\ \isakeyword{where}\ is{\isacharunderscore}{\kern0pt}defs{\isacharcolon}{\kern0pt}\ {\isachardoublequoteopen}x\ {\isacharequal}{\kern0pt}\ join\ {\isacharparenleft}{\kern0pt}L{\isacharunderscore}{\kern0pt}line\ i{\isacharparenright}{\kern0pt}\ s\ n\ m\ {\isasymand}\ i\ {\isacharless}{\kern0pt}\ t\ {\isasymand}\ s\ {\isasymin}\ S\ {\isacharbackquote}{\kern0pt}\ {\isacharparenleft}{\kern0pt}classes\ k\ t\ j{\isacharparenright}{\kern0pt}{\isachardoublequoteclose}\ \isacommand{using}\isamarkupfalse%
\ {\isacartoucheopen}x\ {\isasymin}\ T{\isacharunderscore}{\kern0pt}class\ j{\isacartoucheclose}\ \isacommand{unfolding}\isamarkupfalse%
\ T{\isacharunderscore}{\kern0pt}class{\isacharunderscore}{\kern0pt}def\ \isacommand{by}\isamarkupfalse%
\ auto\isanewline
\ \ \ \ \ \ \ \ \isacommand{moreover}\isamarkupfalse%
\ \isacommand{have}\isamarkupfalse%
\ {\isacharasterisk}{\kern0pt}{\isacharcolon}{\kern0pt}{\isachardoublequoteopen}classes\ k\ t\ j\ {\isasymsubseteq}\ cube\ k\ {\isacharparenleft}{\kern0pt}t{\isacharplus}{\kern0pt}{\isadigit{1}}{\isacharparenright}{\kern0pt}{\isachardoublequoteclose}\ \isacommand{unfolding}\isamarkupfalse%
\ classes{\isacharunderscore}{\kern0pt}def\ \isacommand{by}\isamarkupfalse%
\ simp\isanewline
\ \ \ \ \ \ \ \ \isacommand{moreover}\isamarkupfalse%
\ \isacommand{have}\isamarkupfalse%
\ {\isachardoublequoteopen}{\isasymexists}{\isacharbang}{\kern0pt}y{\isachardot}{\kern0pt}\ y\ {\isasymin}\ classes\ k\ t\ j\ {\isasymand}\ s\ {\isacharequal}{\kern0pt}\ S\ y{\isachardoublequoteclose}\ \isacommand{using}\isamarkupfalse%
\ subspace{\isacharunderscore}{\kern0pt}inj{\isacharunderscore}{\kern0pt}on{\isacharunderscore}{\kern0pt}cube{\isacharbrackleft}{\kern0pt}of\ S\ k\ m\ {\isachardoublequoteopen}t{\isacharplus}{\kern0pt}{\isadigit{1}}{\isachardoublequoteclose}{\isacharbrackright}{\kern0pt}\ S{\isacharunderscore}{\kern0pt}prop\ inj{\isacharunderscore}{\kern0pt}onD{\isacharbrackleft}{\kern0pt}of\ S\ {\isachardoublequoteopen}cube\ k\ {\isacharparenleft}{\kern0pt}t{\isacharplus}{\kern0pt}{\isadigit{1}}{\isacharparenright}{\kern0pt}{\isachardoublequoteclose}{\isacharbrackright}{\kern0pt}\ calculation\ \isacommand{unfolding}\isamarkupfalse%
\ layered{\isacharunderscore}{\kern0pt}subspace{\isacharunderscore}{\kern0pt}def\ inj{\isacharunderscore}{\kern0pt}on{\isacharunderscore}{\kern0pt}def\ \isacommand{by}\isamarkupfalse%
\ blast\isanewline
\ \ \ \ \ \ \ \ \isacommand{ultimately}\isamarkupfalse%
\ \isacommand{obtain}\isamarkupfalse%
\ y\ \isakeyword{where}\ y{\isacharunderscore}{\kern0pt}prop{\isacharcolon}{\kern0pt}\ {\isachardoublequoteopen}y\ {\isasymin}\ classes\ k\ t\ j\ {\isasymand}\ s\ {\isacharequal}{\kern0pt}\ S\ y\ {\isasymand}\ {\isacharparenleft}{\kern0pt}{\isasymforall}z{\isasymin}classes\ k\ t\ j{\isachardot}{\kern0pt}\ s\ {\isacharequal}{\kern0pt}\ S\ z\ {\isasymlongrightarrow}\ y\ {\isacharequal}{\kern0pt}\ z{\isacharparenright}{\kern0pt}{\isachardoublequoteclose}\ \isacommand{by}\isamarkupfalse%
\ auto\isanewline
\isanewline
\ \ \ \ \ \ \ \ \isacommand{define}\isamarkupfalse%
\ p\ \isakeyword{where}\ {\isachardoublequoteopen}p\ {\isasymequiv}\ join\ {\isacharparenleft}{\kern0pt}{\isasymlambda}g{\isasymin}{\isacharbraceleft}{\kern0pt}{\isachardot}{\kern0pt}{\isachardot}{\kern0pt}{\isacharless}{\kern0pt}{\isadigit{1}}{\isacharbraceright}{\kern0pt}{\isachardot}{\kern0pt}\ i{\isacharparenright}{\kern0pt}\ y\ {\isadigit{1}}\ k{\isachardoublequoteclose}\isanewline
\ \ \ \ \ \ \ \ \isacommand{have}\isamarkupfalse%
\ {\isachardoublequoteopen}{\isacharparenleft}{\kern0pt}{\isasymlambda}g{\isasymin}{\isacharbraceleft}{\kern0pt}{\isachardot}{\kern0pt}{\isachardot}{\kern0pt}{\isacharless}{\kern0pt}{\isadigit{1}}{\isacharbraceright}{\kern0pt}{\isachardot}{\kern0pt}\ i{\isacharparenright}{\kern0pt}\ {\isasymin}\ cube\ {\isadigit{1}}\ {\isacharparenleft}{\kern0pt}t{\isacharplus}{\kern0pt}{\isadigit{1}}{\isacharparenright}{\kern0pt}{\isachardoublequoteclose}\ \isacommand{using}\isamarkupfalse%
\ is{\isacharunderscore}{\kern0pt}defs\ \isacommand{unfolding}\isamarkupfalse%
\ cube{\isacharunderscore}{\kern0pt}def\ \isacommand{by}\isamarkupfalse%
\ simp\isanewline
\ \ \ \ \ \ \ \ \isacommand{then}\isamarkupfalse%
\ \isacommand{have}\isamarkupfalse%
\ p{\isacharunderscore}{\kern0pt}in{\isacharunderscore}{\kern0pt}cube{\isacharcolon}{\kern0pt}\ {\isachardoublequoteopen}p\ {\isasymin}\ cube\ {\isacharparenleft}{\kern0pt}k\ {\isacharplus}{\kern0pt}\ {\isadigit{1}}{\isacharparenright}{\kern0pt}\ {\isacharparenleft}{\kern0pt}t{\isacharplus}{\kern0pt}{\isadigit{1}}{\isacharparenright}{\kern0pt}{\isachardoublequoteclose}\ \isacommand{using}\isamarkupfalse%
\ join{\isacharunderscore}{\kern0pt}cubes{\isacharbrackleft}{\kern0pt}of\ {\isachardoublequoteopen}{\isacharparenleft}{\kern0pt}{\isasymlambda}g{\isasymin}{\isacharbraceleft}{\kern0pt}{\isachardot}{\kern0pt}{\isachardot}{\kern0pt}{\isacharless}{\kern0pt}{\isadigit{1}}{\isacharbraceright}{\kern0pt}{\isachardot}{\kern0pt}\ i{\isacharparenright}{\kern0pt}{\isachardoublequoteclose}\ {\isadigit{1}}\ t\ y\ k{\isacharbrackright}{\kern0pt}\ \ y{\isacharunderscore}{\kern0pt}prop\ {\isacharasterisk}{\kern0pt}\ \isacommand{unfolding}\isamarkupfalse%
\ p{\isacharunderscore}{\kern0pt}def\ \isacommand{by}\isamarkupfalse%
\ auto\isanewline
\ \ \ \ \ \ \ \ \isacommand{then}\isamarkupfalse%
\ \isacommand{have}\isamarkupfalse%
\ {\isacharasterisk}{\kern0pt}{\isacharasterisk}{\kern0pt}{\isacharcolon}{\kern0pt}\ {\isachardoublequoteopen}p\ {\isadigit{0}}\ {\isacharequal}{\kern0pt}\ i\ {\isasymand}\ {\isacharparenleft}{\kern0pt}{\isasymforall}l\ {\isacharless}{\kern0pt}\ k{\isachardot}{\kern0pt}\ p\ {\isacharparenleft}{\kern0pt}l\ {\isacharplus}{\kern0pt}\ {\isadigit{1}}{\isacharparenright}{\kern0pt}\ {\isacharequal}{\kern0pt}\ y\ l{\isacharparenright}{\kern0pt}{\isachardoublequoteclose}\ \isacommand{unfolding}\isamarkupfalse%
\ p{\isacharunderscore}{\kern0pt}def\ join{\isacharunderscore}{\kern0pt}def\ \isacommand{by}\isamarkupfalse%
\ simp\ \isanewline
\isanewline
\ \ \ \ \ \ \ \ \isacommand{have}\isamarkupfalse%
\ {\isachardoublequoteopen}t\ {\isasymnotin}\ y\ {\isacharbackquote}{\kern0pt}\ {\isacharbraceleft}{\kern0pt}{\isachardot}{\kern0pt}{\isachardot}{\kern0pt}{\isacharless}{\kern0pt}{\isacharparenleft}{\kern0pt}k\ {\isacharminus}{\kern0pt}\ j{\isacharparenright}{\kern0pt}{\isacharbraceright}{\kern0pt}{\isachardoublequoteclose}\ \isacommand{using}\isamarkupfalse%
\ y{\isacharunderscore}{\kern0pt}prop\ \isacommand{unfolding}\isamarkupfalse%
\ classes{\isacharunderscore}{\kern0pt}def\ \isacommand{by}\isamarkupfalse%
\ simp\isanewline
\ \ \ \ \ \ \ \ \isacommand{then}\isamarkupfalse%
\ \isacommand{have}\isamarkupfalse%
\ {\isachardoublequoteopen}{\isasymforall}u\ {\isacharless}{\kern0pt}\ k\ {\isacharminus}{\kern0pt}\ j{\isachardot}{\kern0pt}\ y\ u\ {\isasymnoteq}\ t{\isachardoublequoteclose}\ \isacommand{by}\isamarkupfalse%
\ auto\isanewline
\ \ \ \ \ \ \ \ \isacommand{then}\isamarkupfalse%
\ \isacommand{have}\isamarkupfalse%
\ {\isachardoublequoteopen}{\isasymforall}u\ {\isacharless}{\kern0pt}\ k\ {\isacharminus}{\kern0pt}\ j{\isachardot}{\kern0pt}\ p\ {\isacharparenleft}{\kern0pt}u\ {\isacharplus}{\kern0pt}\ {\isadigit{1}}{\isacharparenright}{\kern0pt}\ {\isasymnoteq}\ t{\isachardoublequoteclose}\ \isacommand{using}\isamarkupfalse%
\ {\isacharasterisk}{\kern0pt}{\isacharasterisk}{\kern0pt}\ \isacommand{by}\isamarkupfalse%
\ simp\isanewline
\ \ \ \ \ \ \ \ \isacommand{moreover}\isamarkupfalse%
\ \isacommand{have}\isamarkupfalse%
\ {\isachardoublequoteopen}p\ {\isadigit{0}}\ {\isasymnoteq}\ t{\isachardoublequoteclose}\ \isacommand{using}\isamarkupfalse%
\ is{\isacharunderscore}{\kern0pt}defs\ {\isacharasterisk}{\kern0pt}{\isacharasterisk}{\kern0pt}\ \isacommand{by}\isamarkupfalse%
\ simp\isanewline
\ \ \ \ \ \ \ \ \isacommand{moreover}\isamarkupfalse%
\ \isacommand{have}\isamarkupfalse%
\ {\isachardoublequoteopen}{\isasymforall}u\ {\isacharless}{\kern0pt}\ k\ {\isacharminus}{\kern0pt}\ j\ {\isacharplus}{\kern0pt}\ {\isadigit{1}}{\isachardot}{\kern0pt}\ p\ u\ {\isasymnoteq}\ t{\isachardoublequoteclose}\ \isacommand{using}\isamarkupfalse%
\ calculation\ \isacommand{by}\isamarkupfalse%
\ {\isacharparenleft}{\kern0pt}auto\ simp{\isacharcolon}{\kern0pt}\ algebra{\isacharunderscore}{\kern0pt}simps\ less{\isacharunderscore}{\kern0pt}Suc{\isacharunderscore}{\kern0pt}eq{\isacharunderscore}{\kern0pt}{\isadigit{0}}{\isacharunderscore}{\kern0pt}disj{\isacharparenright}{\kern0pt}\isanewline
\ \ \ \ \ \ \ \ \isacommand{ultimately}\isamarkupfalse%
\ \isacommand{have}\isamarkupfalse%
\ {\isachardoublequoteopen}{\isasymforall}u\ {\isacharless}{\kern0pt}\ {\isacharparenleft}{\kern0pt}k\ {\isacharplus}{\kern0pt}\ {\isadigit{1}}{\isacharparenright}{\kern0pt}\ {\isacharminus}{\kern0pt}\ j{\isachardot}{\kern0pt}\ p\ u\ {\isasymnoteq}\ t{\isachardoublequoteclose}\ \isacommand{using}\isamarkupfalse%
\ that\ \isacommand{by}\isamarkupfalse%
\ auto\isanewline
\ \ \ \ \ \ \ \ \isacommand{then}\isamarkupfalse%
\ \isacommand{have}\isamarkupfalse%
\ A{\isadigit{1}}{\isacharcolon}{\kern0pt}\ {\isachardoublequoteopen}t\ {\isasymnotin}\ p\ {\isacharbackquote}{\kern0pt}\ {\isacharbraceleft}{\kern0pt}{\isachardot}{\kern0pt}{\isachardot}{\kern0pt}{\isacharless}{\kern0pt}{\isacharparenleft}{\kern0pt}{\isacharparenleft}{\kern0pt}k{\isacharplus}{\kern0pt}{\isadigit{1}}{\isacharparenright}{\kern0pt}\ {\isacharminus}{\kern0pt}\ j{\isacharparenright}{\kern0pt}{\isacharbraceright}{\kern0pt}{\isachardoublequoteclose}\ \isacommand{by}\isamarkupfalse%
\ blast\isanewline
\isanewline
\isanewline
\ \ \ \ \ \ \ \ \isacommand{have}\isamarkupfalse%
\ {\isachardoublequoteopen}p\ u\ {\isacharequal}{\kern0pt}\ t{\isachardoublequoteclose}\ \isakeyword{if}\ {\isachardoublequoteopen}u\ {\isasymin}\ {\isacharbraceleft}{\kern0pt}k\ {\isacharminus}{\kern0pt}\ j\ {\isacharplus}{\kern0pt}\ {\isadigit{1}}{\isachardot}{\kern0pt}{\isachardot}{\kern0pt}{\isacharless}{\kern0pt}k{\isacharplus}{\kern0pt}{\isadigit{1}}{\isacharbraceright}{\kern0pt}{\isachardoublequoteclose}\ \isakeyword{for}\ u\ \isanewline
\ \ \ \ \ \ \ \ \isacommand{proof}\isamarkupfalse%
\ {\isacharminus}{\kern0pt}\isanewline
\ \ \ \ \ \ \ \ \ \ \isacommand{from}\isamarkupfalse%
\ that\ \isacommand{have}\isamarkupfalse%
\ {\isachardoublequoteopen}u\ {\isacharminus}{\kern0pt}\ {\isadigit{1}}\ {\isasymin}\ {\isacharbraceleft}{\kern0pt}k\ {\isacharminus}{\kern0pt}\ j{\isachardot}{\kern0pt}{\isachardot}{\kern0pt}{\isacharless}{\kern0pt}k{\isacharbraceright}{\kern0pt}{\isachardoublequoteclose}\ \isacommand{by}\isamarkupfalse%
\ auto\isanewline
\ \ \ \ \ \ \ \ \ \ \isacommand{then}\isamarkupfalse%
\ \isacommand{have}\isamarkupfalse%
\ {\isachardoublequoteopen}y\ {\isacharparenleft}{\kern0pt}u\ {\isacharminus}{\kern0pt}\ {\isadigit{1}}{\isacharparenright}{\kern0pt}\ {\isacharequal}{\kern0pt}\ t{\isachardoublequoteclose}\ \isacommand{using}\isamarkupfalse%
\ y{\isacharunderscore}{\kern0pt}prop\ \isacommand{unfolding}\isamarkupfalse%
\ classes{\isacharunderscore}{\kern0pt}def\ \isacommand{by}\isamarkupfalse%
\ blast\isanewline
\ \ \ \ \ \ \ \ \ \ \isacommand{then}\isamarkupfalse%
\ \isacommand{show}\isamarkupfalse%
\ {\isachardoublequoteopen}p\ u\ {\isacharequal}{\kern0pt}\ t{\isachardoublequoteclose}\ \isacommand{using}\isamarkupfalse%
\ {\isacharasterisk}{\kern0pt}{\isacharasterisk}{\kern0pt}\ that\ {\isacartoucheopen}u\ {\isacharminus}{\kern0pt}\ {\isadigit{1}}\ {\isasymin}\ {\isacharbraceleft}{\kern0pt}k\ {\isacharminus}{\kern0pt}\ j{\isachardot}{\kern0pt}{\isachardot}{\kern0pt}{\isacharless}{\kern0pt}k{\isacharbraceright}{\kern0pt}{\isacartoucheclose}\ \isacommand{by}\isamarkupfalse%
\ auto\isanewline
\ \ \ \ \ \ \ \ \isacommand{qed}\isamarkupfalse%
\isanewline
\ \ \ \ \ \ \ \ \isacommand{then}\isamarkupfalse%
\ \isacommand{have}\isamarkupfalse%
\ A{\isadigit{2}}{\isacharcolon}{\kern0pt}\ {\isachardoublequoteopen}{\isasymforall}u{\isasymin}{\isacharbraceleft}{\kern0pt}{\isacharparenleft}{\kern0pt}k{\isacharplus}{\kern0pt}{\isadigit{1}}{\isacharparenright}{\kern0pt}\ {\isacharminus}{\kern0pt}\ j{\isachardot}{\kern0pt}{\isachardot}{\kern0pt}{\isacharless}{\kern0pt}k{\isacharplus}{\kern0pt}{\isadigit{1}}{\isacharbraceright}{\kern0pt}{\isachardot}{\kern0pt}\ p\ u\ {\isacharequal}{\kern0pt}\ t{\isachardoublequoteclose}\ \isacommand{using}\isamarkupfalse%
\ that\ \isacommand{by}\isamarkupfalse%
\ auto\isanewline
\isanewline
\ \ \ \ \ \ \ \ \isacommand{from}\isamarkupfalse%
\ A{\isadigit{1}}\ A{\isadigit{2}}\ p{\isacharunderscore}{\kern0pt}in{\isacharunderscore}{\kern0pt}cube\ \isacommand{have}\isamarkupfalse%
\ {\isachardoublequoteopen}p\ {\isasymin}\ classes\ {\isacharparenleft}{\kern0pt}k{\isacharplus}{\kern0pt}{\isadigit{1}}{\isacharparenright}{\kern0pt}\ t\ j{\isachardoublequoteclose}\ \isacommand{unfolding}\isamarkupfalse%
\ classes{\isacharunderscore}{\kern0pt}def\ \isacommand{by}\isamarkupfalse%
\ blast\isanewline
\isanewline
\ \ \ \ \ \ \ \ \isacommand{moreover}\isamarkupfalse%
\ \isacommand{have}\isamarkupfalse%
\ {\isachardoublequoteopen}x\ {\isacharequal}{\kern0pt}\ T\ p{\isachardoublequoteclose}\isanewline
\ \ \ \ \ \ \ \ \isacommand{proof}\isamarkupfalse%
{\isacharminus}{\kern0pt}\isanewline
\ \ \ \ \ \ \ \ \ \ \isacommand{have}\isamarkupfalse%
\ loc{\isacharunderscore}{\kern0pt}useful{\isacharcolon}{\kern0pt}{\isachardoublequoteopen}{\isacharparenleft}{\kern0pt}{\isasymlambda}y\ {\isasymin}\ {\isacharbraceleft}{\kern0pt}{\isachardot}{\kern0pt}{\isachardot}{\kern0pt}{\isacharless}{\kern0pt}k{\isacharbraceright}{\kern0pt}{\isachardot}{\kern0pt}\ p\ {\isacharparenleft}{\kern0pt}y\ {\isacharplus}{\kern0pt}\ {\isadigit{1}}{\isacharparenright}{\kern0pt}{\isacharparenright}{\kern0pt}\ {\isacharequal}{\kern0pt}\ {\isacharparenleft}{\kern0pt}{\isasymlambda}z\ {\isasymin}\ {\isacharbraceleft}{\kern0pt}{\isachardot}{\kern0pt}{\isachardot}{\kern0pt}{\isacharless}{\kern0pt}k{\isacharbraceright}{\kern0pt}{\isachardot}{\kern0pt}\ y\ z{\isacharparenright}{\kern0pt}{\isachardoublequoteclose}\ \isacommand{using}\isamarkupfalse%
\ {\isacharasterisk}{\kern0pt}{\isacharasterisk}{\kern0pt}\ \isacommand{by}\isamarkupfalse%
\ auto\isanewline
\ \ \ \ \ \ \ \ \ \ \isacommand{have}\isamarkupfalse%
\ {\isachardoublequoteopen}T\ p\ {\isacharequal}{\kern0pt}\ T{\isacharprime}{\kern0pt}\ {\isacharparenleft}{\kern0pt}{\isasymlambda}y\ {\isasymin}\ {\isacharbraceleft}{\kern0pt}{\isachardot}{\kern0pt}{\isachardot}{\kern0pt}{\isacharless}{\kern0pt}{\isadigit{1}}{\isacharbraceright}{\kern0pt}{\isachardot}{\kern0pt}\ p\ y{\isacharparenright}{\kern0pt}\ {\isacharparenleft}{\kern0pt}{\isasymlambda}y\ {\isasymin}\ {\isacharbraceleft}{\kern0pt}{\isachardot}{\kern0pt}{\isachardot}{\kern0pt}{\isacharless}{\kern0pt}k{\isacharbraceright}{\kern0pt}{\isachardot}{\kern0pt}\ p\ {\isacharparenleft}{\kern0pt}y\ {\isacharplus}{\kern0pt}\ {\isadigit{1}}{\isacharparenright}{\kern0pt}{\isacharparenright}{\kern0pt}{\isachardoublequoteclose}\ \isacommand{using}\isamarkupfalse%
\ p{\isacharunderscore}{\kern0pt}in{\isacharunderscore}{\kern0pt}cube\ \isacommand{unfolding}\isamarkupfalse%
\ T{\isacharunderscore}{\kern0pt}def\ \isacommand{by}\isamarkupfalse%
\ auto\isanewline
\isanewline
\ \ \ \ \ \ \ \ \ \ \isacommand{have}\isamarkupfalse%
\ {\isachardoublequoteopen}T{\isacharprime}{\kern0pt}\ {\isacharparenleft}{\kern0pt}{\isasymlambda}y\ {\isasymin}\ {\isacharbraceleft}{\kern0pt}{\isachardot}{\kern0pt}{\isachardot}{\kern0pt}{\isacharless}{\kern0pt}{\isadigit{1}}{\isacharbraceright}{\kern0pt}{\isachardot}{\kern0pt}\ p\ y{\isacharparenright}{\kern0pt}\ {\isacharparenleft}{\kern0pt}{\isasymlambda}y\ {\isasymin}\ {\isacharbraceleft}{\kern0pt}{\isachardot}{\kern0pt}{\isachardot}{\kern0pt}{\isacharless}{\kern0pt}k{\isacharbraceright}{\kern0pt}{\isachardot}{\kern0pt}\ p\ {\isacharparenleft}{\kern0pt}y\ {\isacharplus}{\kern0pt}\ {\isadigit{1}}{\isacharparenright}{\kern0pt}{\isacharparenright}{\kern0pt}\ {\isacharequal}{\kern0pt}\ join\ {\isacharparenleft}{\kern0pt}L{\isacharunderscore}{\kern0pt}line\ {\isacharparenleft}{\kern0pt}{\isacharparenleft}{\kern0pt}{\isasymlambda}y\ {\isasymin}\ {\isacharbraceleft}{\kern0pt}{\isachardot}{\kern0pt}{\isachardot}{\kern0pt}{\isacharless}{\kern0pt}{\isadigit{1}}{\isacharbraceright}{\kern0pt}{\isachardot}{\kern0pt}\ p\ y{\isacharparenright}{\kern0pt}\ {\isadigit{0}}{\isacharparenright}{\kern0pt}{\isacharparenright}{\kern0pt}\ {\isacharparenleft}{\kern0pt}S\ {\isacharparenleft}{\kern0pt}{\isasymlambda}y\ {\isasymin}\ {\isacharbraceleft}{\kern0pt}{\isachardot}{\kern0pt}{\isachardot}{\kern0pt}{\isacharless}{\kern0pt}k{\isacharbraceright}{\kern0pt}{\isachardot}{\kern0pt}\ p\ {\isacharparenleft}{\kern0pt}y\ {\isacharplus}{\kern0pt}\ {\isadigit{1}}{\isacharparenright}{\kern0pt}{\isacharparenright}{\kern0pt}{\isacharparenright}{\kern0pt}\ n\ m{\isachardoublequoteclose}\ \isacommand{using}\isamarkupfalse%
\ split{\isacharunderscore}{\kern0pt}cube\ p{\isacharunderscore}{\kern0pt}in{\isacharunderscore}{\kern0pt}cube\ \isacommand{unfolding}\isamarkupfalse%
\ T{\isacharprime}{\kern0pt}{\isacharunderscore}{\kern0pt}def\ \isacommand{by}\isamarkupfalse%
\ simp\isanewline
\ \ \ \ \ \ \ \ \ \ \isacommand{also}\isamarkupfalse%
\ \isacommand{have}\isamarkupfalse%
\ {\isachardoublequoteopen}{\isachardot}{\kern0pt}{\isachardot}{\kern0pt}{\isachardot}{\kern0pt}\ {\isacharequal}{\kern0pt}\ join\ {\isacharparenleft}{\kern0pt}L{\isacharunderscore}{\kern0pt}line\ {\isacharparenleft}{\kern0pt}p\ {\isadigit{0}}{\isacharparenright}{\kern0pt}{\isacharparenright}{\kern0pt}\ {\isacharparenleft}{\kern0pt}S\ {\isacharparenleft}{\kern0pt}{\isasymlambda}y\ {\isasymin}\ {\isacharbraceleft}{\kern0pt}{\isachardot}{\kern0pt}{\isachardot}{\kern0pt}{\isacharless}{\kern0pt}k{\isacharbraceright}{\kern0pt}{\isachardot}{\kern0pt}\ p\ {\isacharparenleft}{\kern0pt}y\ {\isacharplus}{\kern0pt}\ {\isadigit{1}}{\isacharparenright}{\kern0pt}{\isacharparenright}{\kern0pt}{\isacharparenright}{\kern0pt}\ n\ m{\isachardoublequoteclose}\ \isacommand{by}\isamarkupfalse%
\ simp\isanewline
\ \ \ \ \ \ \ \ \ \ \isacommand{also}\isamarkupfalse%
\ \isacommand{have}\isamarkupfalse%
\ {\isachardoublequoteopen}{\isachardot}{\kern0pt}{\isachardot}{\kern0pt}{\isachardot}{\kern0pt}\ {\isacharequal}{\kern0pt}\ join\ {\isacharparenleft}{\kern0pt}L{\isacharunderscore}{\kern0pt}line\ i{\isacharparenright}{\kern0pt}\ {\isacharparenleft}{\kern0pt}S\ {\isacharparenleft}{\kern0pt}{\isasymlambda}y\ {\isasymin}\ {\isacharbraceleft}{\kern0pt}{\isachardot}{\kern0pt}{\isachardot}{\kern0pt}{\isacharless}{\kern0pt}k{\isacharbraceright}{\kern0pt}{\isachardot}{\kern0pt}\ p\ {\isacharparenleft}{\kern0pt}y\ {\isacharplus}{\kern0pt}\ {\isadigit{1}}{\isacharparenright}{\kern0pt}{\isacharparenright}{\kern0pt}{\isacharparenright}{\kern0pt}\ n\ m{\isachardoublequoteclose}\ \isacommand{by}\isamarkupfalse%
\ {\isacharparenleft}{\kern0pt}simp\ add{\isacharcolon}{\kern0pt}\ {\isacharasterisk}{\kern0pt}{\isacharasterisk}{\kern0pt}{\isacharparenright}{\kern0pt}\isanewline
\ \ \ \ \ \ \ \ \ \ \isacommand{also}\isamarkupfalse%
\ \isacommand{have}\isamarkupfalse%
\ {\isachardoublequoteopen}{\isachardot}{\kern0pt}{\isachardot}{\kern0pt}{\isachardot}{\kern0pt}\ {\isacharequal}{\kern0pt}\ join\ {\isacharparenleft}{\kern0pt}L{\isacharunderscore}{\kern0pt}line\ i{\isacharparenright}{\kern0pt}\ {\isacharparenleft}{\kern0pt}S\ {\isacharparenleft}{\kern0pt}{\isasymlambda}z\ {\isasymin}\ {\isacharbraceleft}{\kern0pt}{\isachardot}{\kern0pt}{\isachardot}{\kern0pt}{\isacharless}{\kern0pt}k{\isacharbraceright}{\kern0pt}{\isachardot}{\kern0pt}\ y\ z{\isacharparenright}{\kern0pt}{\isacharparenright}{\kern0pt}\ n\ m{\isachardoublequoteclose}\ \isacommand{using}\isamarkupfalse%
\ loc{\isacharunderscore}{\kern0pt}useful\ \isacommand{by}\isamarkupfalse%
\ simp\isanewline
\ \ \ \ \ \ \ \ \ \ \isacommand{also}\isamarkupfalse%
\ \isacommand{have}\isamarkupfalse%
\ {\isachardoublequoteopen}{\isachardot}{\kern0pt}{\isachardot}{\kern0pt}{\isachardot}{\kern0pt}\ {\isacharequal}{\kern0pt}\ join\ {\isacharparenleft}{\kern0pt}L{\isacharunderscore}{\kern0pt}line\ i{\isacharparenright}{\kern0pt}\ {\isacharparenleft}{\kern0pt}S\ y{\isacharparenright}{\kern0pt}\ n\ m{\isachardoublequoteclose}\ \isacommand{using}\isamarkupfalse%
\ y{\isacharunderscore}{\kern0pt}prop\ {\isacharasterisk}{\kern0pt}\ \isacommand{unfolding}\isamarkupfalse%
\ cube{\isacharunderscore}{\kern0pt}def\ \isacommand{by}\isamarkupfalse%
\ auto\isanewline
\ \ \ \ \ \ \ \ \ \ \isacommand{also}\isamarkupfalse%
\ \isacommand{have}\isamarkupfalse%
\ {\isachardoublequoteopen}{\isachardot}{\kern0pt}{\isachardot}{\kern0pt}{\isachardot}{\kern0pt}\ {\isacharequal}{\kern0pt}\ x{\isachardoublequoteclose}\ \isacommand{using}\isamarkupfalse%
\ is{\isacharunderscore}{\kern0pt}defs\ y{\isacharunderscore}{\kern0pt}prop\ \isacommand{by}\isamarkupfalse%
\ simp\isanewline
\ \ \ \ \ \ \ \ \ \ \isacommand{finally}\isamarkupfalse%
\ \isacommand{show}\isamarkupfalse%
\ {\isachardoublequoteopen}x\ {\isacharequal}{\kern0pt}\ T\ p{\isachardoublequoteclose}\ \isanewline
\ \ \ \ \ \ \ \ \ \ \ \ \isacommand{using}\isamarkupfalse%
\ {\isacartoucheopen}T\ p\ {\isacharequal}{\kern0pt}\ T{\isacharprime}{\kern0pt}\ {\isacharparenleft}{\kern0pt}restrict\ p\ {\isacharbraceleft}{\kern0pt}{\isachardot}{\kern0pt}{\isachardot}{\kern0pt}{\isacharless}{\kern0pt}{\isadigit{1}}{\isacharbraceright}{\kern0pt}{\isacharparenright}{\kern0pt}\ {\isacharparenleft}{\kern0pt}{\isasymlambda}y{\isasymin}{\isacharbraceleft}{\kern0pt}{\isachardot}{\kern0pt}{\isachardot}{\kern0pt}{\isacharless}{\kern0pt}k{\isacharbraceright}{\kern0pt}{\isachardot}{\kern0pt}\ p\ {\isacharparenleft}{\kern0pt}y\ {\isacharplus}{\kern0pt}\ {\isadigit{1}}{\isacharparenright}{\kern0pt}{\isacharparenright}{\kern0pt}{\isacartoucheclose}\ \isacommand{by}\isamarkupfalse%
\ presburger\isanewline
\ \ \ \ \ \ \ \ \isacommand{qed}\isamarkupfalse%
\isanewline
\ \ \ \ \ \ \ \ \isacommand{ultimately}\isamarkupfalse%
\ \isacommand{show}\isamarkupfalse%
\ {\isachardoublequoteopen}x\ {\isasymin}\ T\ {\isacharbackquote}{\kern0pt}\ classes\ {\isacharparenleft}{\kern0pt}k\ {\isacharplus}{\kern0pt}\ {\isadigit{1}}{\isacharparenright}{\kern0pt}\ t\ j{\isachardoublequoteclose}\ \isacommand{by}\isamarkupfalse%
\ blast\isanewline
\ \ \ \ \ \ \isacommand{qed}\isamarkupfalse%
\isanewline
\ \ \ \ \isacommand{next}\isamarkupfalse%
\isanewline
\ \ \ \ \ \ \isacommand{show}\isamarkupfalse%
\ {\isachardoublequoteopen}T\ {\isacharbackquote}{\kern0pt}\ classes\ {\isacharparenleft}{\kern0pt}k\ {\isacharplus}{\kern0pt}\ {\isadigit{1}}{\isacharparenright}{\kern0pt}\ t\ j\ {\isasymsubseteq}\ T{\isacharunderscore}{\kern0pt}class\ j{\isachardoublequoteclose}\isanewline
\ \ \ \ \ \ \isacommand{proof}\isamarkupfalse%
\isanewline
\ \ \ \ \ \ \ \ \isacommand{fix}\isamarkupfalse%
\ x\ \isacommand{assume}\isamarkupfalse%
\ {\isachardoublequoteopen}x\ {\isasymin}\ T\ {\isacharbackquote}{\kern0pt}\ classes\ {\isacharparenleft}{\kern0pt}k{\isacharplus}{\kern0pt}{\isadigit{1}}{\isacharparenright}{\kern0pt}\ t\ j{\isachardoublequoteclose}\isanewline
\ \ \ \ \ \ \ \ \isacommand{then}\isamarkupfalse%
\ \isacommand{obtain}\isamarkupfalse%
\ y\ \isakeyword{where}\ y{\isacharunderscore}{\kern0pt}prop{\isacharcolon}{\kern0pt}\ {\isachardoublequoteopen}y\ {\isasymin}\ classes\ {\isacharparenleft}{\kern0pt}k{\isacharplus}{\kern0pt}{\isadigit{1}}{\isacharparenright}{\kern0pt}\ t\ j\ {\isasymand}\ T\ y\ {\isacharequal}{\kern0pt}\ x{\isachardoublequoteclose}\ \isacommand{by}\isamarkupfalse%
\ blast\isanewline
\ \ \ \ \ \ \ \ \isacommand{then}\isamarkupfalse%
\ \isacommand{have}\isamarkupfalse%
\ y{\isacharunderscore}{\kern0pt}props{\isacharcolon}{\kern0pt}\ {\isachardoublequoteopen}{\isacharparenleft}{\kern0pt}{\isasymforall}u\ {\isasymin}\ {\isacharbraceleft}{\kern0pt}{\isacharparenleft}{\kern0pt}{\isacharparenleft}{\kern0pt}k{\isacharplus}{\kern0pt}{\isadigit{1}}{\isacharparenright}{\kern0pt}{\isacharminus}{\kern0pt}j{\isacharparenright}{\kern0pt}{\isachardot}{\kern0pt}{\isachardot}{\kern0pt}{\isacharless}{\kern0pt}k{\isacharplus}{\kern0pt}{\isadigit{1}}{\isacharbraceright}{\kern0pt}{\isachardot}{\kern0pt}\ y\ u\ {\isacharequal}{\kern0pt}\ t{\isacharparenright}{\kern0pt}\ {\isasymand}\ t\ {\isasymnotin}\ y\ {\isacharbackquote}{\kern0pt}\ {\isacharbraceleft}{\kern0pt}{\isachardot}{\kern0pt}{\isachardot}{\kern0pt}{\isacharless}{\kern0pt}{\isacharparenleft}{\kern0pt}k{\isacharplus}{\kern0pt}{\isadigit{1}}{\isacharparenright}{\kern0pt}\ {\isacharminus}{\kern0pt}\ j\ {\isacharbraceright}{\kern0pt}{\isachardoublequoteclose}\ \isacommand{unfolding}\isamarkupfalse%
\ classes{\isacharunderscore}{\kern0pt}def\ \isacommand{by}\isamarkupfalse%
\ blast\isanewline
\isanewline
\ \ \ \ \ \ \ \ \isacommand{define}\isamarkupfalse%
\ z\ \isakeyword{where}\ {\isachardoublequoteopen}z\ {\isasymequiv}\ {\isacharparenleft}{\kern0pt}{\isasymlambda}v\ {\isasymin}\ {\isacharbraceleft}{\kern0pt}{\isachardot}{\kern0pt}{\isachardot}{\kern0pt}{\isacharless}{\kern0pt}k{\isacharbraceright}{\kern0pt}{\isachardot}{\kern0pt}\ y\ {\isacharparenleft}{\kern0pt}v{\isacharplus}{\kern0pt}{\isadigit{1}}{\isacharparenright}{\kern0pt}{\isacharparenright}{\kern0pt}{\isachardoublequoteclose}\ \isanewline
\ \ \ \ \ \ \ \ \isacommand{have}\isamarkupfalse%
\ {\isachardoublequoteopen}z\ {\isasymin}\ cube\ k\ {\isacharparenleft}{\kern0pt}t{\isacharplus}{\kern0pt}{\isadigit{1}}{\isacharparenright}{\kern0pt}{\isachardoublequoteclose}\ \isacommand{using}\isamarkupfalse%
\ \ y{\isacharunderscore}{\kern0pt}prop\ classes{\isacharunderscore}{\kern0pt}subset{\isacharunderscore}{\kern0pt}cube{\isacharbrackleft}{\kern0pt}of\ {\isachardoublequoteopen}k{\isacharplus}{\kern0pt}{\isadigit{1}}{\isachardoublequoteclose}\ t\ j{\isacharbrackright}{\kern0pt}\ \isacommand{unfolding}\isamarkupfalse%
\ z{\isacharunderscore}{\kern0pt}def\ cube{\isacharunderscore}{\kern0pt}def\ \isacommand{by}\isamarkupfalse%
\ auto\isanewline
\ \ \ \ \ \ \ \ \isacommand{moreover}\isamarkupfalse%
\isanewline
\ \ \ \ \ \ \ \ \isacommand{{\isacharbraceleft}{\kern0pt}}\isamarkupfalse%
\isanewline
\ \ \ \ \ \ \ \ \ \ \isacommand{have}\isamarkupfalse%
\ {\isachardoublequoteopen}z\ {\isacharbackquote}{\kern0pt}\ {\isacharbraceleft}{\kern0pt}{\isachardot}{\kern0pt}{\isachardot}{\kern0pt}{\isacharless}{\kern0pt}k\ {\isacharminus}{\kern0pt}\ j{\isacharbraceright}{\kern0pt}\ {\isacharequal}{\kern0pt}\ y\ {\isacharbackquote}{\kern0pt}\ {\isacharparenleft}{\kern0pt}{\isacharparenleft}{\kern0pt}{\isacharplus}{\kern0pt}{\isacharparenright}{\kern0pt}\ {\isadigit{1}}\ {\isacharbackquote}{\kern0pt}\ {\isacharbraceleft}{\kern0pt}{\isachardot}{\kern0pt}{\isachardot}{\kern0pt}{\isacharless}{\kern0pt}k{\isacharminus}{\kern0pt}j{\isacharbraceright}{\kern0pt}{\isacharparenright}{\kern0pt}\ {\isachardoublequoteclose}\ \ \isacommand{unfolding}\isamarkupfalse%
\ z{\isacharunderscore}{\kern0pt}def\ \isacommand{by}\isamarkupfalse%
\ fastforce\isanewline
\ \ \ \ \ \ \ \ \ \ \isacommand{also}\isamarkupfalse%
\ \isacommand{have}\isamarkupfalse%
\ {\isachardoublequoteopen}{\isachardot}{\kern0pt}{\isachardot}{\kern0pt}{\isachardot}{\kern0pt}\ {\isacharequal}{\kern0pt}\ y\ {\isacharbackquote}{\kern0pt}\ {\isacharbraceleft}{\kern0pt}{\isadigit{1}}{\isachardot}{\kern0pt}{\isachardot}{\kern0pt}{\isacharless}{\kern0pt}k{\isacharminus}{\kern0pt}j{\isacharplus}{\kern0pt}{\isadigit{1}}{\isacharbraceright}{\kern0pt}{\isachardoublequoteclose}\ \isacommand{by}\isamarkupfalse%
\ {\isacharparenleft}{\kern0pt}simp\ add{\isacharcolon}{\kern0pt}\ atLeastLessThanSuc{\isacharunderscore}{\kern0pt}atLeastAtMost\ image{\isacharunderscore}{\kern0pt}Suc{\isacharunderscore}{\kern0pt}lessThan{\isacharparenright}{\kern0pt}\isanewline
\ \ \ \ \ \ \ \ \ \ \isacommand{also}\isamarkupfalse%
\ \isacommand{have}\isamarkupfalse%
\ {\isachardoublequoteopen}{\isachardot}{\kern0pt}{\isachardot}{\kern0pt}{\isachardot}{\kern0pt}\ {\isacharequal}{\kern0pt}\ y\ {\isacharbackquote}{\kern0pt}\ {\isacharbraceleft}{\kern0pt}{\isadigit{1}}{\isachardot}{\kern0pt}{\isachardot}{\kern0pt}{\isacharless}{\kern0pt}{\isacharparenleft}{\kern0pt}k{\isacharplus}{\kern0pt}{\isadigit{1}}{\isacharparenright}{\kern0pt}{\isacharminus}{\kern0pt}j{\isacharbraceright}{\kern0pt}{\isachardoublequoteclose}\ \isacommand{using}\isamarkupfalse%
\ j{\isacharunderscore}{\kern0pt}prop\ \isacommand{by}\isamarkupfalse%
\ auto\isanewline
\ \ \ \ \ \ \ \ \ \ \isacommand{finally}\isamarkupfalse%
\ \isacommand{have}\isamarkupfalse%
\ {\isachardoublequoteopen}z\ {\isacharbackquote}{\kern0pt}\ {\isacharbraceleft}{\kern0pt}{\isachardot}{\kern0pt}{\isachardot}{\kern0pt}{\isacharless}{\kern0pt}k\ {\isacharminus}{\kern0pt}\ j{\isacharbraceright}{\kern0pt}\ {\isasymsubseteq}\ y\ {\isacharbackquote}{\kern0pt}\ {\isacharbraceleft}{\kern0pt}{\isachardot}{\kern0pt}{\isachardot}{\kern0pt}{\isacharless}{\kern0pt}{\isacharparenleft}{\kern0pt}k{\isacharplus}{\kern0pt}{\isadigit{1}}{\isacharparenright}{\kern0pt}{\isacharminus}{\kern0pt}j{\isacharbraceright}{\kern0pt}{\isachardoublequoteclose}\ \isacommand{by}\isamarkupfalse%
\ auto\isanewline
\ \ \ \ \ \ \ \ \ \ \isacommand{then}\isamarkupfalse%
\ \isacommand{have}\isamarkupfalse%
\ {\isachardoublequoteopen}t\ {\isasymnotin}\ z\ {\isacharbackquote}{\kern0pt}\ {\isacharbraceleft}{\kern0pt}{\isachardot}{\kern0pt}{\isachardot}{\kern0pt}{\isacharless}{\kern0pt}k\ {\isacharminus}{\kern0pt}\ j{\isacharbraceright}{\kern0pt}{\isachardoublequoteclose}\ \isacommand{using}\isamarkupfalse%
\ y{\isacharunderscore}{\kern0pt}props\ \isacommand{by}\isamarkupfalse%
\ blast\isanewline
\isanewline
\ \ \ \ \ \ \ \ \isacommand{{\isacharbraceright}{\kern0pt}}\isamarkupfalse%
\isanewline
\ \ \ \ \ \ \ \ \isacommand{moreover}\isamarkupfalse%
\ \isacommand{have}\isamarkupfalse%
\ {\isachardoublequoteopen}{\isasymforall}u\ {\isasymin}\ {\isacharbraceleft}{\kern0pt}k{\isacharminus}{\kern0pt}j{\isachardot}{\kern0pt}{\isachardot}{\kern0pt}{\isacharless}{\kern0pt}k{\isacharbraceright}{\kern0pt}{\isachardot}{\kern0pt}\ z\ u\ {\isacharequal}{\kern0pt}\ t{\isachardoublequoteclose}\ \isacommand{unfolding}\isamarkupfalse%
\ z{\isacharunderscore}{\kern0pt}def\ \isacommand{using}\isamarkupfalse%
\ y{\isacharunderscore}{\kern0pt}props\ \isacommand{by}\isamarkupfalse%
\ auto\isanewline
\ \ \ \ \ \ \ \ \isacommand{ultimately}\isamarkupfalse%
\ \isacommand{have}\isamarkupfalse%
\ z{\isacharunderscore}{\kern0pt}in{\isacharunderscore}{\kern0pt}classes{\isacharcolon}{\kern0pt}\ {\isachardoublequoteopen}z\ {\isasymin}\ classes\ k\ t\ j{\isachardoublequoteclose}\ \isacommand{unfolding}\isamarkupfalse%
\ classes{\isacharunderscore}{\kern0pt}def\ \isacommand{by}\isamarkupfalse%
\ blast\isanewline
\isanewline
\ \ \ \ \ \ \ \ \isacommand{have}\isamarkupfalse%
\ {\isachardoublequoteopen}y\ {\isadigit{0}}\ {\isasymnoteq}\ t{\isachardoublequoteclose}\isanewline
\ \ \ \ \ \ \ \ \isacommand{proof}\isamarkupfalse%
{\isacharminus}{\kern0pt}\isanewline
\ \ \ \ \ \ \ \ \ \ \isacommand{from}\isamarkupfalse%
\ that\ \isacommand{have}\isamarkupfalse%
\ {\isachardoublequoteopen}{\isadigit{0}}\ {\isasymin}\ {\isacharbraceleft}{\kern0pt}{\isachardot}{\kern0pt}{\isachardot}{\kern0pt}{\isacharless}{\kern0pt}k\ {\isacharplus}{\kern0pt}\ {\isadigit{1}}\ {\isacharminus}{\kern0pt}\ j{\isacharbraceright}{\kern0pt}{\isachardoublequoteclose}\ \isacommand{by}\isamarkupfalse%
\ simp\isanewline
\ \ \ \ \ \ \ \ \ \ \isacommand{then}\isamarkupfalse%
\ \isacommand{show}\isamarkupfalse%
\ {\isachardoublequoteopen}y\ {\isadigit{0}}\ {\isasymnoteq}\ t{\isachardoublequoteclose}\ \isacommand{using}\isamarkupfalse%
\ y{\isacharunderscore}{\kern0pt}props\ \isacommand{by}\isamarkupfalse%
\ blast\isanewline
\ \ \ \ \ \ \ \ \isacommand{qed}\isamarkupfalse%
\isanewline
\ \ \ \ \ \ \ \ \isacommand{then}\isamarkupfalse%
\ \isacommand{have}\isamarkupfalse%
\ tr{\isacharcolon}{\kern0pt}\ {\isachardoublequoteopen}y\ {\isadigit{0}}\ {\isacharless}{\kern0pt}\ t{\isachardoublequoteclose}\ \isacommand{using}\isamarkupfalse%
\ y{\isacharunderscore}{\kern0pt}prop\ classes{\isacharunderscore}{\kern0pt}subset{\isacharunderscore}{\kern0pt}cube{\isacharbrackleft}{\kern0pt}of\ {\isachardoublequoteopen}k{\isacharplus}{\kern0pt}{\isadigit{1}}{\isachardoublequoteclose}\ t\ j{\isacharbrackright}{\kern0pt}\ \isacommand{unfolding}\isamarkupfalse%
\ cube{\isacharunderscore}{\kern0pt}def\ \isacommand{by}\isamarkupfalse%
\ fastforce\isanewline
\isanewline
\ \ \ \ \ \ \ \ \isacommand{have}\isamarkupfalse%
\ {\isachardoublequoteopen}{\isacharparenleft}{\kern0pt}{\isasymlambda}g\ {\isasymin}\ {\isacharbraceleft}{\kern0pt}{\isachardot}{\kern0pt}{\isachardot}{\kern0pt}{\isacharless}{\kern0pt}{\isadigit{1}}{\isacharbraceright}{\kern0pt}{\isachardot}{\kern0pt}\ y\ g{\isacharparenright}{\kern0pt}\ {\isasymin}\ cube\ {\isadigit{1}}\ {\isacharparenleft}{\kern0pt}t{\isacharplus}{\kern0pt}{\isadigit{1}}{\isacharparenright}{\kern0pt}{\isachardoublequoteclose}\ \isacommand{using}\isamarkupfalse%
\ y{\isacharunderscore}{\kern0pt}prop\ classes{\isacharunderscore}{\kern0pt}subset{\isacharunderscore}{\kern0pt}cube{\isacharbrackleft}{\kern0pt}of\ {\isachardoublequoteopen}k{\isacharplus}{\kern0pt}{\isadigit{1}}{\isachardoublequoteclose}\ t\ j{\isacharbrackright}{\kern0pt}\ cube{\isacharunderscore}{\kern0pt}restrict{\isacharbrackleft}{\kern0pt}of\ {\isadigit{1}}\ {\isachardoublequoteopen}{\isacharparenleft}{\kern0pt}k{\isacharplus}{\kern0pt}{\isadigit{1}}{\isacharparenright}{\kern0pt}{\isachardoublequoteclose}\ y\ {\isachardoublequoteopen}t{\isacharplus}{\kern0pt}{\isadigit{1}}{\isachardoublequoteclose}{\isacharbrackright}{\kern0pt}\ assms{\isacharparenleft}{\kern0pt}{\isadigit{2}}{\isacharparenright}{\kern0pt}\ \isacommand{by}\isamarkupfalse%
\ auto\isanewline
\ \ \ \ \ \ \ \ \isacommand{then}\isamarkupfalse%
\ \isacommand{have}\isamarkupfalse%
\ {\isachardoublequoteopen}T\ y\ {\isacharequal}{\kern0pt}\ T{\isacharprime}{\kern0pt}\ {\isacharparenleft}{\kern0pt}{\isasymlambda}g\ {\isasymin}\ {\isacharbraceleft}{\kern0pt}{\isachardot}{\kern0pt}{\isachardot}{\kern0pt}{\isacharless}{\kern0pt}{\isadigit{1}}{\isacharbraceright}{\kern0pt}{\isachardot}{\kern0pt}\ y\ g{\isacharparenright}{\kern0pt}\ z\ \ {\isachardoublequoteclose}\ \isacommand{using}\isamarkupfalse%
\ y{\isacharunderscore}{\kern0pt}prop\ classes{\isacharunderscore}{\kern0pt}subset{\isacharunderscore}{\kern0pt}cube{\isacharbrackleft}{\kern0pt}of\ {\isachardoublequoteopen}k{\isacharplus}{\kern0pt}{\isadigit{1}}{\isachardoublequoteclose}\ t\ j{\isacharbrackright}{\kern0pt}\ \isacommand{unfolding}\isamarkupfalse%
\ T{\isacharunderscore}{\kern0pt}def\ z{\isacharunderscore}{\kern0pt}def\ \isacommand{by}\isamarkupfalse%
\ auto\isanewline
\ \ \ \ \ \ \ \ \isacommand{also}\isamarkupfalse%
\ \isacommand{have}\isamarkupfalse%
\ {\isachardoublequoteopen}\ {\isachardot}{\kern0pt}{\isachardot}{\kern0pt}{\isachardot}{\kern0pt}\ {\isacharequal}{\kern0pt}\ join\ {\isacharparenleft}{\kern0pt}L{\isacharunderscore}{\kern0pt}line\ {\isacharparenleft}{\kern0pt}{\isacharparenleft}{\kern0pt}{\isasymlambda}g\ {\isasymin}\ {\isacharbraceleft}{\kern0pt}{\isachardot}{\kern0pt}{\isachardot}{\kern0pt}{\isacharless}{\kern0pt}{\isadigit{1}}{\isacharbraceright}{\kern0pt}{\isachardot}{\kern0pt}\ y\ g{\isacharparenright}{\kern0pt}\ {\isadigit{0}}{\isacharparenright}{\kern0pt}{\isacharparenright}{\kern0pt}\ {\isacharparenleft}{\kern0pt}S\ z{\isacharparenright}{\kern0pt}\ n\ m{\isachardoublequoteclose}\ \isacommand{unfolding}\isamarkupfalse%
\ T{\isacharprime}{\kern0pt}{\isacharunderscore}{\kern0pt}def\ \isacommand{using}\isamarkupfalse%
\ {\isacartoucheopen}{\isacharparenleft}{\kern0pt}{\isasymlambda}g\ {\isasymin}\ {\isacharbraceleft}{\kern0pt}{\isachardot}{\kern0pt}{\isachardot}{\kern0pt}{\isacharless}{\kern0pt}{\isadigit{1}}{\isacharbraceright}{\kern0pt}{\isachardot}{\kern0pt}\ y\ g{\isacharparenright}{\kern0pt}\ {\isasymin}\ cube\ {\isadigit{1}}\ {\isacharparenleft}{\kern0pt}t{\isacharplus}{\kern0pt}{\isadigit{1}}{\isacharparenright}{\kern0pt}{\isacartoucheclose}\ {\isacartoucheopen}z\ {\isasymin}\ cube\ k\ {\isacharparenleft}{\kern0pt}t{\isacharplus}{\kern0pt}{\isadigit{1}}{\isacharparenright}{\kern0pt}{\isacartoucheclose}\ \isacommand{by}\isamarkupfalse%
\ auto\isanewline
\ \ \ \ \ \ \ \ \isacommand{also}\isamarkupfalse%
\ \isacommand{have}\isamarkupfalse%
\ {\isachardoublequoteopen}\ {\isachardot}{\kern0pt}{\isachardot}{\kern0pt}{\isachardot}{\kern0pt}\ {\isacharequal}{\kern0pt}\ join\ {\isacharparenleft}{\kern0pt}L{\isacharunderscore}{\kern0pt}line\ {\isacharparenleft}{\kern0pt}y\ {\isadigit{0}}{\isacharparenright}{\kern0pt}{\isacharparenright}{\kern0pt}\ {\isacharparenleft}{\kern0pt}S\ z{\isacharparenright}{\kern0pt}\ n\ m{\isachardoublequoteclose}\ \isacommand{by}\isamarkupfalse%
\ simp\isanewline
\ \ \ \ \ \ \ \ \isacommand{also}\isamarkupfalse%
\ \isacommand{have}\isamarkupfalse%
\ {\isachardoublequoteopen}\ {\isachardot}{\kern0pt}{\isachardot}{\kern0pt}{\isachardot}{\kern0pt}\ {\isasymin}\ T{\isacharunderscore}{\kern0pt}class\ j{\isachardoublequoteclose}\ \isacommand{using}\isamarkupfalse%
\ tr\ z{\isacharunderscore}{\kern0pt}in{\isacharunderscore}{\kern0pt}classes\ that\ \isacommand{unfolding}\isamarkupfalse%
\ T{\isacharunderscore}{\kern0pt}class{\isacharunderscore}{\kern0pt}def\ \isacommand{by}\isamarkupfalse%
\ force\isanewline
\ \ \ \ \ \ \ \ \isacommand{finally}\isamarkupfalse%
\ \isacommand{show}\isamarkupfalse%
\ {\isachardoublequoteopen}x\ {\isasymin}\ T{\isacharunderscore}{\kern0pt}class\ j{\isachardoublequoteclose}\ \isacommand{using}\isamarkupfalse%
\ y{\isacharunderscore}{\kern0pt}prop\ \isacommand{by}\isamarkupfalse%
\ simp\isanewline
\ \ \ \ \ \ \isacommand{qed}\isamarkupfalse%
\isanewline
\ \ \ \ \isacommand{qed}\isamarkupfalse%
\isanewline
\ \ \ \ \ \ \isanewline
\ \ \ \ \isacommand{have}\isamarkupfalse%
\ {\isachardoublequoteopen}{\isasymforall}x\ {\isasymin}\ T\ {\isacharbackquote}{\kern0pt}\ classes\ {\isacharparenleft}{\kern0pt}k{\isacharplus}{\kern0pt}{\isadigit{1}}{\isacharparenright}{\kern0pt}\ t\ i{\isachardot}{\kern0pt}\ {\isasymforall}y\ {\isasymin}\ T\ {\isacharbackquote}{\kern0pt}\ classes\ {\isacharparenleft}{\kern0pt}k{\isacharplus}{\kern0pt}{\isadigit{1}}{\isacharparenright}{\kern0pt}\ t\ i{\isachardot}{\kern0pt}\ \ {\isasymchi}\ x\ {\isacharequal}{\kern0pt}\ {\isasymchi}\ y\ {\isasymand}\ {\isasymchi}\ x\ {\isacharless}{\kern0pt}\ r{\isachardoublequoteclose}\ \isakeyword{if}\ i{\isacharunderscore}{\kern0pt}assm{\isacharcolon}{\kern0pt}\ {\isachardoublequoteopen}i\ {\isasymle}\ k{\isachardoublequoteclose}\ \isakeyword{for}\ i\isanewline
\ \ \ \ \isacommand{proof}\isamarkupfalse%
\ {\isacharparenleft}{\kern0pt}intro\ ballI{\isacharparenright}{\kern0pt}\isanewline
\ \ \ \ \ \ \isacommand{fix}\isamarkupfalse%
\ x\ y\ \isacommand{assume}\isamarkupfalse%
\ a{\isacharcolon}{\kern0pt}\ {\isachardoublequoteopen}x\ {\isasymin}\ T\ {\isacharbackquote}{\kern0pt}\ classes\ {\isacharparenleft}{\kern0pt}k\ {\isacharplus}{\kern0pt}\ {\isadigit{1}}{\isacharparenright}{\kern0pt}\ t\ i{\isachardoublequoteclose}\ {\isachardoublequoteopen}y\ {\isasymin}\ T\ {\isacharbackquote}{\kern0pt}\ classes\ {\isacharparenleft}{\kern0pt}k\ {\isacharplus}{\kern0pt}\ {\isadigit{1}}{\isacharparenright}{\kern0pt}\ t\ i{\isachardoublequoteclose}\isanewline
\ \ \ \ \ \ \isacommand{from}\isamarkupfalse%
\ that\ \isacommand{have}\isamarkupfalse%
\ {\isacharasterisk}{\kern0pt}{\isacharcolon}{\kern0pt}\ {\isachardoublequoteopen}T\ {\isacharbackquote}{\kern0pt}\ classes\ {\isacharparenleft}{\kern0pt}k{\isacharplus}{\kern0pt}{\isadigit{1}}{\isacharparenright}{\kern0pt}\ t\ i\ {\isacharequal}{\kern0pt}\ T{\isacharunderscore}{\kern0pt}class\ i{\isachardoublequoteclose}\ \isacommand{by}\isamarkupfalse%
\ {\isacharparenleft}{\kern0pt}simp\ add{\isacharcolon}{\kern0pt}\ classprop{\isacharparenright}{\kern0pt}\isanewline
\ \ \ \ \ \ \isacommand{then}\isamarkupfalse%
\ \isacommand{have}\isamarkupfalse%
\ \ {\isachardoublequoteopen}x\ {\isasymin}\ T{\isacharunderscore}{\kern0pt}class\ i\ {\isachardoublequoteclose}\ \isacommand{using}\isamarkupfalse%
\ a\ \isacommand{by}\isamarkupfalse%
\ simp\isanewline
\ \ \ \ \ \ \isacommand{moreover}\isamarkupfalse%
\ \isacommand{have}\isamarkupfalse%
\ {\isacharasterisk}{\kern0pt}{\isacharasterisk}{\kern0pt}{\isacharcolon}{\kern0pt}\ {\isachardoublequoteopen}T{\isacharunderscore}{\kern0pt}class\ i\ {\isacharequal}{\kern0pt}\ {\isacharbraceleft}{\kern0pt}join\ {\isacharparenleft}{\kern0pt}L{\isacharunderscore}{\kern0pt}line\ l{\isacharparenright}{\kern0pt}\ s\ n\ m\ {\isacharbar}{\kern0pt}\ l\ s\ {\isachardot}{\kern0pt}\ l\ {\isasymin}\ {\isacharbraceleft}{\kern0pt}{\isachardot}{\kern0pt}{\isachardot}{\kern0pt}{\isacharless}{\kern0pt}t{\isacharbraceright}{\kern0pt}\ {\isasymand}\ s\ {\isasymin}\ S\ {\isacharbackquote}{\kern0pt}\ {\isacharparenleft}{\kern0pt}classes\ k\ t\ i{\isacharparenright}{\kern0pt}{\isacharbraceright}{\kern0pt}{\isachardoublequoteclose}\ \isacommand{using}\isamarkupfalse%
\ that\ \isacommand{unfolding}\isamarkupfalse%
\ T{\isacharunderscore}{\kern0pt}class{\isacharunderscore}{\kern0pt}def\ \isacommand{by}\isamarkupfalse%
\ simp\isanewline
\ \ \ \ \ \ \isacommand{ultimately}\isamarkupfalse%
\ \isacommand{obtain}\isamarkupfalse%
\ xs\ xi\ \isakeyword{where}\ xdefs{\isacharcolon}{\kern0pt}\ \ {\isachardoublequoteopen}x\ {\isacharequal}{\kern0pt}\ join\ {\isacharparenleft}{\kern0pt}L{\isacharunderscore}{\kern0pt}line\ xi{\isacharparenright}{\kern0pt}\ xs\ n\ m\ {\isasymand}\ xi\ {\isacharless}{\kern0pt}\ t\ {\isasymand}\ xs\ {\isasymin}\ S\ {\isacharbackquote}{\kern0pt}\ {\isacharparenleft}{\kern0pt}classes\ k\ t\ i{\isacharparenright}{\kern0pt}{\isachardoublequoteclose}\ \isacommand{by}\isamarkupfalse%
\ blast\isanewline
\isanewline
\ \ \ \ \ \ \isacommand{from}\isamarkupfalse%
\ {\isacharasterisk}{\kern0pt}\ {\isacharasterisk}{\kern0pt}{\isacharasterisk}{\kern0pt}\ \isacommand{obtain}\isamarkupfalse%
\ ys\ yi\ \isakeyword{where}\ ydefs{\isacharcolon}{\kern0pt}\ {\isachardoublequoteopen}y\ {\isacharequal}{\kern0pt}\ join\ {\isacharparenleft}{\kern0pt}L{\isacharunderscore}{\kern0pt}line\ yi{\isacharparenright}{\kern0pt}\ ys\ n\ m\ {\isasymand}\ yi\ {\isacharless}{\kern0pt}\ t\ {\isasymand}\ ys\ {\isasymin}\ S\ {\isacharbackquote}{\kern0pt}\ {\isacharparenleft}{\kern0pt}classes\ k\ t\ i{\isacharparenright}{\kern0pt}{\isachardoublequoteclose}\ \isacommand{using}\isamarkupfalse%
\ a\ \isacommand{by}\isamarkupfalse%
\ auto\isanewline
\isanewline
\ \ \ \ \ \ \isacommand{have}\isamarkupfalse%
\ {\isachardoublequoteopen}{\isacharparenleft}{\kern0pt}L{\isacharunderscore}{\kern0pt}line\ xi{\isacharparenright}{\kern0pt}\ {\isasymin}\ cube\ n\ {\isacharparenleft}{\kern0pt}t{\isacharplus}{\kern0pt}{\isadigit{1}}{\isacharparenright}{\kern0pt}{\isachardoublequoteclose}\ \isacommand{using}\isamarkupfalse%
\ L{\isacharunderscore}{\kern0pt}line{\isacharunderscore}{\kern0pt}base{\isacharunderscore}{\kern0pt}prop\ xdefs\ \isacommand{by}\isamarkupfalse%
\ simp\isanewline
\ \ \ \ \ \ \isacommand{moreover}\isamarkupfalse%
\ \isacommand{have}\isamarkupfalse%
\ {\isachardoublequoteopen}xs\ {\isasymin}\ cube\ m\ {\isacharparenleft}{\kern0pt}t{\isacharplus}{\kern0pt}{\isadigit{1}}{\isacharparenright}{\kern0pt}{\isachardoublequoteclose}\ \isacommand{using}\isamarkupfalse%
\ xdefs\ S{\isacharunderscore}{\kern0pt}prop\ subspace{\isacharunderscore}{\kern0pt}elems{\isacharunderscore}{\kern0pt}embed\ imageE\ image{\isacharunderscore}{\kern0pt}subset{\isacharunderscore}{\kern0pt}iff\ mem{\isacharunderscore}{\kern0pt}Collect{\isacharunderscore}{\kern0pt}eq\ \isacommand{unfolding}\isamarkupfalse%
\ layered{\isacharunderscore}{\kern0pt}subspace{\isacharunderscore}{\kern0pt}def\ classes{\isacharunderscore}{\kern0pt}def\ \ \isacommand{by}\isamarkupfalse%
\ blast\isanewline
\ \ \ \ \ \ \isacommand{ultimately}\isamarkupfalse%
\ \isacommand{have}\isamarkupfalse%
\ AA{\isadigit{1}}{\isacharcolon}{\kern0pt}\ {\isachardoublequoteopen}{\isasymchi}\ x\ {\isacharequal}{\kern0pt}\ {\isasymchi}L\ {\isacharparenleft}{\kern0pt}L{\isacharunderscore}{\kern0pt}line\ xi{\isacharparenright}{\kern0pt}\ xs{\isachardoublequoteclose}\ \isacommand{using}\isamarkupfalse%
\ xdefs\ \isacommand{unfolding}\isamarkupfalse%
\ {\isasymchi}L{\isacharunderscore}{\kern0pt}def\ \isacommand{by}\isamarkupfalse%
\ simp\isanewline
\isanewline
\ \ \ \ \ \ \isacommand{have}\isamarkupfalse%
\ {\isachardoublequoteopen}{\isacharparenleft}{\kern0pt}L{\isacharunderscore}{\kern0pt}line\ yi{\isacharparenright}{\kern0pt}\ {\isasymin}\ cube\ n\ {\isacharparenleft}{\kern0pt}t{\isacharplus}{\kern0pt}{\isadigit{1}}{\isacharparenright}{\kern0pt}{\isachardoublequoteclose}\ \isacommand{using}\isamarkupfalse%
\ L{\isacharunderscore}{\kern0pt}line{\isacharunderscore}{\kern0pt}base{\isacharunderscore}{\kern0pt}prop\ ydefs\ \isacommand{by}\isamarkupfalse%
\ simp\isanewline
\ \ \ \ \ \ \isacommand{moreover}\isamarkupfalse%
\ \isacommand{have}\isamarkupfalse%
\ {\isachardoublequoteopen}ys\ {\isasymin}\ cube\ m\ {\isacharparenleft}{\kern0pt}t{\isacharplus}{\kern0pt}{\isadigit{1}}{\isacharparenright}{\kern0pt}{\isachardoublequoteclose}\ \isacommand{using}\isamarkupfalse%
\ ydefs\ S{\isacharunderscore}{\kern0pt}prop\ subspace{\isacharunderscore}{\kern0pt}elems{\isacharunderscore}{\kern0pt}embed\ imageE\ image{\isacharunderscore}{\kern0pt}subset{\isacharunderscore}{\kern0pt}iff\ mem{\isacharunderscore}{\kern0pt}Collect{\isacharunderscore}{\kern0pt}eq\ \isacommand{unfolding}\isamarkupfalse%
\ layered{\isacharunderscore}{\kern0pt}subspace{\isacharunderscore}{\kern0pt}def\ classes{\isacharunderscore}{\kern0pt}def\ \ \isacommand{by}\isamarkupfalse%
\ blast\isanewline
\ \ \ \ \ \ \isacommand{ultimately}\isamarkupfalse%
\ \isacommand{have}\isamarkupfalse%
\ AA{\isadigit{2}}{\isacharcolon}{\kern0pt}\ {\isachardoublequoteopen}{\isasymchi}\ y\ {\isacharequal}{\kern0pt}\ {\isasymchi}L\ {\isacharparenleft}{\kern0pt}L{\isacharunderscore}{\kern0pt}line\ yi{\isacharparenright}{\kern0pt}\ ys{\isachardoublequoteclose}\ \isacommand{using}\isamarkupfalse%
\ ydefs\ \isacommand{unfolding}\isamarkupfalse%
\ {\isasymchi}L{\isacharunderscore}{\kern0pt}def\ \isacommand{by}\isamarkupfalse%
\ simp\isanewline
\isanewline
\ \ \ \ \ \ \isacommand{have}\isamarkupfalse%
\ {\isachardoublequoteopen}{\isasymforall}s{\isacharless}{\kern0pt}t{\isachardot}{\kern0pt}\ {\isasymforall}l\ {\isacharless}{\kern0pt}\ t{\isachardot}{\kern0pt}\ {\isasymchi}L{\isacharunderscore}{\kern0pt}s\ {\isacharparenleft}{\kern0pt}L\ {\isacharparenleft}{\kern0pt}SOME\ p{\isachardot}{\kern0pt}\ p{\isasymin}cube\ {\isadigit{1}}\ {\isacharparenleft}{\kern0pt}t{\isacharplus}{\kern0pt}{\isadigit{1}}{\isacharparenright}{\kern0pt}\ {\isasymand}\ p\ {\isadigit{0}}\ {\isacharequal}{\kern0pt}\ s{\isacharparenright}{\kern0pt}{\isacharparenright}{\kern0pt}\ {\isacharequal}{\kern0pt}\ {\isasymchi}L{\isacharunderscore}{\kern0pt}s\ {\isacharparenleft}{\kern0pt}L\ {\isacharparenleft}{\kern0pt}SOME\ p{\isachardot}{\kern0pt}\ p{\isasymin}cube\ {\isadigit{1}}\ {\isacharparenleft}{\kern0pt}t{\isacharplus}{\kern0pt}{\isadigit{1}}{\isacharparenright}{\kern0pt}\ {\isasymand}\ p\ {\isadigit{0}}\ {\isacharequal}{\kern0pt}\ l{\isacharparenright}{\kern0pt}{\isacharparenright}{\kern0pt}{\isachardoublequoteclose}\ \isacommand{using}\isamarkupfalse%
\ dim{\isadigit{1}}{\isacharunderscore}{\kern0pt}layered{\isacharunderscore}{\kern0pt}subspace{\isacharunderscore}{\kern0pt}mono{\isacharunderscore}{\kern0pt}line{\isacharbrackleft}{\kern0pt}of\ t\ L\ n\ s\ {\isasymchi}L{\isacharunderscore}{\kern0pt}s{\isacharbrackright}{\kern0pt}\ L{\isacharunderscore}{\kern0pt}prop\ assms{\isacharparenleft}{\kern0pt}{\isadigit{1}}{\isacharparenright}{\kern0pt}\ \isacommand{by}\isamarkupfalse%
\ blast\isanewline
\ \ \ \ \ \ \isacommand{then}\isamarkupfalse%
\ \isacommand{have}\isamarkupfalse%
\ mykey{\isacharcolon}{\kern0pt}\ {\isachardoublequoteopen}{\isasymchi}L{\isacharunderscore}{\kern0pt}s\ {\isacharparenleft}{\kern0pt}L{\isacharunderscore}{\kern0pt}line\ s{\isacharparenright}{\kern0pt}\ {\isacharequal}{\kern0pt}\ {\isasymchi}L{\isacharunderscore}{\kern0pt}s\ {\isacharparenleft}{\kern0pt}L{\isacharunderscore}{\kern0pt}line\ l{\isacharparenright}{\kern0pt}{\isachardoublequoteclose}\ \isakeyword{if}\ {\isachardoublequoteopen}s\ {\isasymin}\ {\isacharbraceleft}{\kern0pt}{\isachardot}{\kern0pt}{\isachardot}{\kern0pt}{\isacharless}{\kern0pt}t{\isacharbraceright}{\kern0pt}{\isachardoublequoteclose}\ {\isachardoublequoteopen}l\ {\isasymin}\ {\isacharbraceleft}{\kern0pt}{\isachardot}{\kern0pt}{\isachardot}{\kern0pt}{\isacharless}{\kern0pt}t{\isacharbraceright}{\kern0pt}{\isachardoublequoteclose}\ \isakeyword{for}\ s\ l\ \isacommand{using}\isamarkupfalse%
\ that\ \ \isacommand{unfolding}\isamarkupfalse%
\ L{\isacharunderscore}{\kern0pt}line{\isacharunderscore}{\kern0pt}def\ \isanewline
\ \ \ \ \ \ \ \ \isacommand{by}\isamarkupfalse%
\ {\isacharparenleft}{\kern0pt}metis\ {\isacharparenleft}{\kern0pt}no{\isacharunderscore}{\kern0pt}types{\isacharcomma}{\kern0pt}\ lifting{\isacharparenright}{\kern0pt}\ add{\isachardot}{\kern0pt}commute\ lessThan{\isacharunderscore}{\kern0pt}iff\ less{\isacharunderscore}{\kern0pt}Suc{\isacharunderscore}{\kern0pt}eq\ plus{\isacharunderscore}{\kern0pt}{\isadigit{1}}{\isacharunderscore}{\kern0pt}eq{\isacharunderscore}{\kern0pt}Suc\ restrict{\isacharunderscore}{\kern0pt}apply{\isacharparenright}{\kern0pt}\isanewline
\ \ \ \ \ \ \isacommand{have}\isamarkupfalse%
\ BIGKEY{\isacharcolon}{\kern0pt}\ {\isachardoublequoteopen}{\isasymforall}s{\isacharless}{\kern0pt}t{\isachardot}{\kern0pt}\ {\isasymforall}l{\isacharless}{\kern0pt}t{\isachardot}{\kern0pt}\ {\isasymchi}L\ {\isacharparenleft}{\kern0pt}L{\isacharunderscore}{\kern0pt}line\ s{\isacharparenright}{\kern0pt}\ {\isacharequal}{\kern0pt}\ {\isasymchi}L\ {\isacharparenleft}{\kern0pt}L{\isacharunderscore}{\kern0pt}line\ l{\isacharparenright}{\kern0pt}{\isachardoublequoteclose}\isanewline
\ \ \ \ \ \ \isacommand{proof}\isamarkupfalse%
\ {\isacharparenleft}{\kern0pt}intro\ allI\ impI{\isacharparenright}{\kern0pt}\isanewline
\ \ \ \ \ \ \ \ \isacommand{fix}\isamarkupfalse%
\ s\ l\ \isacommand{assume}\isamarkupfalse%
\ {\isachardoublequoteopen}s\ {\isacharless}{\kern0pt}\ t{\isachardoublequoteclose}\ {\isachardoublequoteopen}l\ {\isacharless}{\kern0pt}\ t{\isachardoublequoteclose}\isanewline
\ \ \ \ \ \ \ \ \isacommand{have}\isamarkupfalse%
\ L{\isadigit{1}}{\isacharcolon}{\kern0pt}\ {\isachardoublequoteopen}{\isasymchi}L\ {\isacharparenleft}{\kern0pt}L{\isacharunderscore}{\kern0pt}line\ s{\isacharparenright}{\kern0pt}\ {\isasymin}\ cube\ m\ {\isacharparenleft}{\kern0pt}t\ {\isacharplus}{\kern0pt}\ {\isadigit{1}}{\isacharparenright}{\kern0pt}\ {\isasymrightarrow}\isactrlsub E\ {\isacharbraceleft}{\kern0pt}{\isachardot}{\kern0pt}{\isachardot}{\kern0pt}{\isacharless}{\kern0pt}r{\isacharbraceright}{\kern0pt}{\isachardoublequoteclose}\ \isacommand{unfolding}\isamarkupfalse%
\ {\isasymchi}L{\isacharunderscore}{\kern0pt}def\ \isacommand{using}\isamarkupfalse%
\ A\ L{\isacharunderscore}{\kern0pt}line{\isacharunderscore}{\kern0pt}base{\isacharunderscore}{\kern0pt}prop\ {\isacartoucheopen}s\ {\isacharless}{\kern0pt}\ t{\isacartoucheclose}\ \isacommand{by}\isamarkupfalse%
\ simp\isanewline
\ \ \ \ \ \ \ \ \isacommand{have}\isamarkupfalse%
\ L{\isadigit{2}}{\isacharcolon}{\kern0pt}\ {\isachardoublequoteopen}{\isasymchi}L\ {\isacharparenleft}{\kern0pt}L{\isacharunderscore}{\kern0pt}line\ l{\isacharparenright}{\kern0pt}\ {\isasymin}\ cube\ m\ {\isacharparenleft}{\kern0pt}t\ {\isacharplus}{\kern0pt}\ {\isadigit{1}}{\isacharparenright}{\kern0pt}\ {\isasymrightarrow}\isactrlsub E\ {\isacharbraceleft}{\kern0pt}{\isachardot}{\kern0pt}{\isachardot}{\kern0pt}{\isacharless}{\kern0pt}r{\isacharbraceright}{\kern0pt}{\isachardoublequoteclose}\ \isacommand{unfolding}\isamarkupfalse%
\ {\isasymchi}L{\isacharunderscore}{\kern0pt}def\ \isacommand{using}\isamarkupfalse%
\ A\ L{\isacharunderscore}{\kern0pt}line{\isacharunderscore}{\kern0pt}base{\isacharunderscore}{\kern0pt}prop\ {\isacartoucheopen}l\ {\isacharless}{\kern0pt}\ t{\isacartoucheclose}\ \isacommand{by}\isamarkupfalse%
\ simp\isanewline
\ \ \ \ \ \ \ \ \isacommand{have}\isamarkupfalse%
\ {\isachardoublequoteopen}{\isasymphi}\ {\isacharparenleft}{\kern0pt}{\isasymchi}L\ {\isacharparenleft}{\kern0pt}L{\isacharunderscore}{\kern0pt}line\ s{\isacharparenright}{\kern0pt}{\isacharparenright}{\kern0pt}\ {\isacharequal}{\kern0pt}\ {\isasymchi}L{\isacharunderscore}{\kern0pt}s\ {\isacharparenleft}{\kern0pt}L{\isacharunderscore}{\kern0pt}line\ s{\isacharparenright}{\kern0pt}{\isachardoublequoteclose}\ \isacommand{unfolding}\isamarkupfalse%
\ {\isasymchi}L{\isacharunderscore}{\kern0pt}s{\isacharunderscore}{\kern0pt}def\ \isacommand{using}\isamarkupfalse%
\ {\isacartoucheopen}s\ {\isacharless}{\kern0pt}\ t{\isacartoucheclose}\ L{\isacharunderscore}{\kern0pt}line{\isacharunderscore}{\kern0pt}base{\isacharunderscore}{\kern0pt}prop\ \isacommand{by}\isamarkupfalse%
\ simp\isanewline
\ \ \ \ \ \ \ \ \isacommand{also}\isamarkupfalse%
\ \isacommand{have}\isamarkupfalse%
\ {\isachardoublequoteopen}\ {\isachardot}{\kern0pt}{\isachardot}{\kern0pt}{\isachardot}{\kern0pt}\ {\isacharequal}{\kern0pt}\ \ {\isasymchi}L{\isacharunderscore}{\kern0pt}s\ {\isacharparenleft}{\kern0pt}L{\isacharunderscore}{\kern0pt}line\ l{\isacharparenright}{\kern0pt}{\isachardoublequoteclose}\ \isacommand{using}\isamarkupfalse%
\ mykey\ {\isacartoucheopen}s\ {\isacharless}{\kern0pt}t{\isacartoucheclose}\ {\isacartoucheopen}l\ {\isacharless}{\kern0pt}\ t{\isacartoucheclose}\ \isacommand{by}\isamarkupfalse%
\ blast\isanewline
\ \ \ \ \ \ \ \ \isacommand{also}\isamarkupfalse%
\ \isacommand{have}\isamarkupfalse%
\ {\isachardoublequoteopen}\ {\isachardot}{\kern0pt}{\isachardot}{\kern0pt}{\isachardot}{\kern0pt}\ {\isacharequal}{\kern0pt}\ {\isasymphi}\ {\isacharparenleft}{\kern0pt}{\isasymchi}L\ {\isacharparenleft}{\kern0pt}L{\isacharunderscore}{\kern0pt}line\ l{\isacharparenright}{\kern0pt}{\isacharparenright}{\kern0pt}{\isachardoublequoteclose}\ \isacommand{unfolding}\isamarkupfalse%
\ {\isasymchi}L{\isacharunderscore}{\kern0pt}s{\isacharunderscore}{\kern0pt}def\ \isacommand{using}\isamarkupfalse%
\ L{\isacharunderscore}{\kern0pt}line{\isacharunderscore}{\kern0pt}base{\isacharunderscore}{\kern0pt}prop\ {\isacartoucheopen}l{\isacharless}{\kern0pt}t{\isacartoucheclose}\ \isacommand{by}\isamarkupfalse%
\ simp\isanewline
\ \ \ \ \ \ \ \ \isacommand{finally}\isamarkupfalse%
\ \isacommand{have}\isamarkupfalse%
\ {\isachardoublequoteopen}{\isasymphi}\ {\isacharparenleft}{\kern0pt}{\isasymchi}L\ {\isacharparenleft}{\kern0pt}L{\isacharunderscore}{\kern0pt}line\ s{\isacharparenright}{\kern0pt}{\isacharparenright}{\kern0pt}\ {\isacharequal}{\kern0pt}\ {\isasymphi}\ {\isacharparenleft}{\kern0pt}{\isasymchi}L\ {\isacharparenleft}{\kern0pt}L{\isacharunderscore}{\kern0pt}line\ l{\isacharparenright}{\kern0pt}{\isacharparenright}{\kern0pt}{\isachardoublequoteclose}\ \isacommand{by}\isamarkupfalse%
\ simp\isanewline
\ \ \ \ \ \ \ \ \isacommand{then}\isamarkupfalse%
\ \isacommand{show}\isamarkupfalse%
\ {\isachardoublequoteopen}{\isasymchi}L\ {\isacharparenleft}{\kern0pt}L{\isacharunderscore}{\kern0pt}line\ s{\isacharparenright}{\kern0pt}\ {\isacharequal}{\kern0pt}\ {\isasymchi}L\ {\isacharparenleft}{\kern0pt}L{\isacharunderscore}{\kern0pt}line\ l{\isacharparenright}{\kern0pt}{\isachardoublequoteclose}\ \isacommand{using}\isamarkupfalse%
\ {\isasymphi}{\isacharunderscore}{\kern0pt}prop\ L{\isacharunderscore}{\kern0pt}line{\isacharunderscore}{\kern0pt}base{\isacharunderscore}{\kern0pt}prop\ L{\isadigit{1}}\ L{\isadigit{2}}\ \isacommand{unfolding}\isamarkupfalse%
\ bij{\isacharunderscore}{\kern0pt}betw{\isacharunderscore}{\kern0pt}def\ inj{\isacharunderscore}{\kern0pt}on{\isacharunderscore}{\kern0pt}def\ \isacommand{by}\isamarkupfalse%
\ blast\isanewline
\ \ \ \ \ \ \isacommand{qed}\isamarkupfalse%
\isanewline
\ \ \ \ \ \ \isacommand{then}\isamarkupfalse%
\ \isacommand{have}\isamarkupfalse%
\ {\isachardoublequoteopen}{\isasymchi}L\ {\isacharparenleft}{\kern0pt}L{\isacharunderscore}{\kern0pt}line\ xi{\isacharparenright}{\kern0pt}\ xs\ {\isacharequal}{\kern0pt}\ {\isasymchi}L\ {\isacharparenleft}{\kern0pt}L{\isacharunderscore}{\kern0pt}line\ {\isadigit{0}}{\isacharparenright}{\kern0pt}\ xs{\isachardoublequoteclose}\ \isacommand{using}\isamarkupfalse%
\ xdefs\ assms{\isacharparenleft}{\kern0pt}{\isadigit{1}}{\isacharparenright}{\kern0pt}\ \isacommand{by}\isamarkupfalse%
\ metis\isanewline
\ \ \ \ \ \ \isacommand{also}\isamarkupfalse%
\ \isacommand{have}\isamarkupfalse%
\ {\isachardoublequoteopen}\ {\isachardot}{\kern0pt}{\isachardot}{\kern0pt}{\isachardot}{\kern0pt}\ {\isacharequal}{\kern0pt}\ \ {\isasymchi}S\ xs{\isachardoublequoteclose}\ \isacommand{unfolding}\isamarkupfalse%
\ {\isasymchi}S{\isacharunderscore}{\kern0pt}def\ {\isasymchi}L{\isacharunderscore}{\kern0pt}def\ \isacommand{using}\isamarkupfalse%
\ xdefs\ L{\isacharunderscore}{\kern0pt}line{\isacharunderscore}{\kern0pt}base{\isacharunderscore}{\kern0pt}prop\ \isacommand{by}\isamarkupfalse%
\ auto\isanewline
\ \ \ \ \ \ \isacommand{also}\isamarkupfalse%
\ \isacommand{have}\isamarkupfalse%
\ {\isachardoublequoteopen}\ {\isachardot}{\kern0pt}{\isachardot}{\kern0pt}{\isachardot}{\kern0pt}\ {\isacharequal}{\kern0pt}\ {\isasymchi}S\ ys{\isachardoublequoteclose}\ \isacommand{using}\isamarkupfalse%
\ xdefs\ ydefs\ layered{\isacharunderscore}{\kern0pt}eq{\isacharunderscore}{\kern0pt}classes{\isacharbrackleft}{\kern0pt}of\ S\ k\ m\ t\ r\ {\isasymchi}S{\isacharbrackright}{\kern0pt}\ S{\isacharunderscore}{\kern0pt}prop\ i{\isacharunderscore}{\kern0pt}assm\ \isacommand{by}\isamarkupfalse%
\ blast\isanewline
\ \ \ \ \ \ \isacommand{also}\isamarkupfalse%
\ \isacommand{have}\isamarkupfalse%
\ {\isachardoublequoteopen}\ {\isachardot}{\kern0pt}{\isachardot}{\kern0pt}{\isachardot}{\kern0pt}\ {\isacharequal}{\kern0pt}\ {\isasymchi}L\ {\isacharparenleft}{\kern0pt}L{\isacharunderscore}{\kern0pt}line\ {\isadigit{0}}{\isacharparenright}{\kern0pt}\ ys{\isachardoublequoteclose}\ \ \isacommand{unfolding}\isamarkupfalse%
\ {\isasymchi}S{\isacharunderscore}{\kern0pt}def\ {\isasymchi}L{\isacharunderscore}{\kern0pt}def\ \isacommand{using}\isamarkupfalse%
\ xdefs\ L{\isacharunderscore}{\kern0pt}line{\isacharunderscore}{\kern0pt}base{\isacharunderscore}{\kern0pt}prop\ \isacommand{by}\isamarkupfalse%
\ auto\isanewline
\ \ \ \ \ \ \isacommand{also}\isamarkupfalse%
\ \isacommand{have}\isamarkupfalse%
\ {\isachardoublequoteopen}\ {\isachardot}{\kern0pt}{\isachardot}{\kern0pt}{\isachardot}{\kern0pt}\ {\isacharequal}{\kern0pt}\ {\isasymchi}L\ {\isacharparenleft}{\kern0pt}L{\isacharunderscore}{\kern0pt}line\ yi{\isacharparenright}{\kern0pt}\ ys{\isachardoublequoteclose}\ \isacommand{using}\isamarkupfalse%
\ ydefs\ BIGKEY\ assms{\isacharparenleft}{\kern0pt}{\isadigit{1}}{\isacharparenright}{\kern0pt}\ \isacommand{by}\isamarkupfalse%
\ metis\isanewline
\ \ \ \ \ \ \isacommand{finally}\isamarkupfalse%
\ \isacommand{have}\isamarkupfalse%
\ CORE{\isacharcolon}{\kern0pt}\ {\isachardoublequoteopen}{\isasymchi}L\ {\isacharparenleft}{\kern0pt}L{\isacharunderscore}{\kern0pt}line\ xi{\isacharparenright}{\kern0pt}\ xs\ {\isacharequal}{\kern0pt}\ \ {\isasymchi}L\ {\isacharparenleft}{\kern0pt}L{\isacharunderscore}{\kern0pt}line\ yi{\isacharparenright}{\kern0pt}\ ys{\isachardoublequoteclose}\ \isacommand{by}\isamarkupfalse%
\ simp\isanewline
\isanewline
\isanewline
\ \ \ \ \ \ \isacommand{then}\isamarkupfalse%
\ \isacommand{have}\isamarkupfalse%
\ {\isachardoublequoteopen}{\isasymchi}\ x\ {\isacharequal}{\kern0pt}\ {\isasymchi}\ y{\isachardoublequoteclose}\ \isacommand{using}\isamarkupfalse%
\ AA{\isadigit{1}}\ AA{\isadigit{2}}\ \isacommand{by}\isamarkupfalse%
\ simp\ \ \ \ \ \ \isanewline
\ \ \ \ \ \ \isacommand{then}\isamarkupfalse%
\ \isacommand{show}\isamarkupfalse%
\ {\isachardoublequoteopen}\ {\isasymchi}\ x\ {\isacharequal}{\kern0pt}\ {\isasymchi}\ y\ {\isasymand}\ {\isasymchi}\ x\ {\isacharless}{\kern0pt}\ r{\isachardoublequoteclose}\ \isacommand{using}\isamarkupfalse%
\ xdefs\ AA{\isadigit{1}}\ BIGKEY\ assms{\isacharparenleft}{\kern0pt}{\isadigit{1}}{\isacharparenright}{\kern0pt}\ A\ {\isacartoucheopen}L{\isacharunderscore}{\kern0pt}line\ xi\ {\isasymin}\ cube\ n\ {\isacharparenleft}{\kern0pt}t\ {\isacharplus}{\kern0pt}\ {\isadigit{1}}{\isacharparenright}{\kern0pt}{\isacartoucheclose}\ {\isacartoucheopen}xs\ {\isasymin}\ cube\ m\ {\isacharparenleft}{\kern0pt}t\ {\isacharplus}{\kern0pt}\ {\isadigit{1}}{\isacharparenright}{\kern0pt}{\isacartoucheclose}\ \isacommand{by}\isamarkupfalse%
\ blast\isanewline
\ \ \ \ \isacommand{qed}\isamarkupfalse%
\isanewline
\ \ \ \ \isacommand{then}\isamarkupfalse%
\ \isacommand{have}\isamarkupfalse%
\ {\isachardoublequoteopen}{\isasymforall}i{\isasymle}k{\isachardot}{\kern0pt}\ {\isasymexists}c{\isacharless}{\kern0pt}r{\isachardot}{\kern0pt}\ {\isasymforall}x\ {\isasymin}\ T\ {\isacharbackquote}{\kern0pt}\ classes\ {\isacharparenleft}{\kern0pt}k{\isacharplus}{\kern0pt}{\isadigit{1}}{\isacharparenright}{\kern0pt}\ t\ i{\isachardot}{\kern0pt}\ {\isasymchi}\ x\ {\isacharequal}{\kern0pt}\ c{\isachardoublequoteclose}\ \isanewline
\ \ \ \ \ \ \isacommand{by}\isamarkupfalse%
\ {\isacharparenleft}{\kern0pt}meson\ assms{\isacharparenleft}{\kern0pt}{\isadigit{5}}{\isacharparenright}{\kern0pt}{\isacharparenright}{\kern0pt}\isanewline
\isanewline
\ \ \ \ \isacommand{have}\isamarkupfalse%
\ {\isachardoublequoteopen}{\isasymexists}c{\isacharless}{\kern0pt}r{\isachardot}{\kern0pt}\ {\isasymforall}x\ {\isasymin}\ T\ {\isacharbackquote}{\kern0pt}\ classes\ {\isacharparenleft}{\kern0pt}k{\isacharplus}{\kern0pt}{\isadigit{1}}{\isacharparenright}{\kern0pt}\ t\ {\isacharparenleft}{\kern0pt}k{\isacharplus}{\kern0pt}{\isadigit{1}}{\isacharparenright}{\kern0pt}{\isachardot}{\kern0pt}\ {\isasymchi}\ x\ {\isacharequal}{\kern0pt}\ c{\isachardoublequoteclose}\isanewline
\ \ \ \ \isacommand{proof}\isamarkupfalse%
\ {\isacharminus}{\kern0pt}\isanewline
\ \ \ \ \ \ \isacommand{have}\isamarkupfalse%
\ {\isachardoublequoteopen}{\isasymforall}x\ {\isasymin}\ classes\ {\isacharparenleft}{\kern0pt}k{\isacharplus}{\kern0pt}{\isadigit{1}}{\isacharparenright}{\kern0pt}\ t\ {\isacharparenleft}{\kern0pt}k{\isacharplus}{\kern0pt}{\isadigit{1}}{\isacharparenright}{\kern0pt}{\isachardot}{\kern0pt}\ {\isasymforall}u\ {\isacharless}{\kern0pt}\ k\ {\isacharplus}{\kern0pt}\ {\isadigit{1}}{\isachardot}{\kern0pt}\ x\ u\ {\isacharequal}{\kern0pt}\ t{\isachardoublequoteclose}\ \isacommand{unfolding}\isamarkupfalse%
\ classes{\isacharunderscore}{\kern0pt}def\ \isacommand{by}\isamarkupfalse%
\ auto\isanewline
\ \ \ \ \ \ \isacommand{have}\isamarkupfalse%
\ {\isachardoublequoteopen}{\isacharparenleft}{\kern0pt}{\isasymlambda}u{\isachardot}{\kern0pt}\ t{\isacharparenright}{\kern0pt}\ {\isacharbackquote}{\kern0pt}\ {\isacharbraceleft}{\kern0pt}{\isachardot}{\kern0pt}{\isachardot}{\kern0pt}{\isacharless}{\kern0pt}k\ {\isacharplus}{\kern0pt}\ {\isadigit{1}}{\isacharbraceright}{\kern0pt}\ {\isasymsubseteq}\ {\isacharbraceleft}{\kern0pt}{\isachardot}{\kern0pt}{\isachardot}{\kern0pt}{\isacharless}{\kern0pt}t\ {\isacharplus}{\kern0pt}\ {\isadigit{1}}{\isacharbraceright}{\kern0pt}{\isachardoublequoteclose}\ \isacommand{by}\isamarkupfalse%
\ auto\isanewline
\ \ \ \ \ \ \isacommand{then}\isamarkupfalse%
\ \isacommand{have}\isamarkupfalse%
\ {\isachardoublequoteopen}{\isasymexists}{\isacharbang}{\kern0pt}y\ {\isasymin}\ cube\ {\isacharparenleft}{\kern0pt}k{\isacharplus}{\kern0pt}{\isadigit{1}}{\isacharparenright}{\kern0pt}\ {\isacharparenleft}{\kern0pt}t{\isacharplus}{\kern0pt}{\isadigit{1}}{\isacharparenright}{\kern0pt}{\isachardot}{\kern0pt}\ {\isacharparenleft}{\kern0pt}{\isasymforall}u\ {\isacharless}{\kern0pt}\ k\ {\isacharplus}{\kern0pt}\ {\isadigit{1}}{\isachardot}{\kern0pt}\ y\ u\ {\isacharequal}{\kern0pt}\ t{\isacharparenright}{\kern0pt}{\isachardoublequoteclose}\ \isacommand{using}\isamarkupfalse%
\ PiE{\isacharunderscore}{\kern0pt}uniqueness{\isacharbrackleft}{\kern0pt}of\ {\isachardoublequoteopen}{\isacharparenleft}{\kern0pt}{\isasymlambda}u{\isachardot}{\kern0pt}\ t{\isacharparenright}{\kern0pt}{\isachardoublequoteclose}\ {\isachardoublequoteopen}{\isacharbraceleft}{\kern0pt}{\isachardot}{\kern0pt}{\isachardot}{\kern0pt}{\isacharless}{\kern0pt}k{\isacharplus}{\kern0pt}{\isadigit{1}}{\isacharbraceright}{\kern0pt}{\isachardoublequoteclose}\ {\isachardoublequoteopen}{\isacharbraceleft}{\kern0pt}{\isachardot}{\kern0pt}{\isachardot}{\kern0pt}{\isacharless}{\kern0pt}t{\isacharplus}{\kern0pt}{\isadigit{1}}{\isacharbraceright}{\kern0pt}{\isachardoublequoteclose}{\isacharbrackright}{\kern0pt}\ \isacommand{unfolding}\isamarkupfalse%
\ cube{\isacharunderscore}{\kern0pt}def\ \isacommand{by}\isamarkupfalse%
\ auto\isanewline
\ \ \ \ \ \ \isacommand{then}\isamarkupfalse%
\ \isacommand{have}\isamarkupfalse%
\ {\isachardoublequoteopen}{\isasymexists}{\isacharbang}{\kern0pt}y\ {\isasymin}\ classes\ {\isacharparenleft}{\kern0pt}k{\isacharplus}{\kern0pt}{\isadigit{1}}{\isacharparenright}{\kern0pt}\ t\ {\isacharparenleft}{\kern0pt}k{\isacharplus}{\kern0pt}{\isadigit{1}}{\isacharparenright}{\kern0pt}{\isachardot}{\kern0pt}\ {\isacharparenleft}{\kern0pt}{\isasymforall}u\ {\isacharless}{\kern0pt}\ k\ {\isacharplus}{\kern0pt}\ {\isadigit{1}}{\isachardot}{\kern0pt}\ y\ u\ {\isacharequal}{\kern0pt}\ t{\isacharparenright}{\kern0pt}{\isachardoublequoteclose}\ \isacommand{unfolding}\isamarkupfalse%
\ classes{\isacharunderscore}{\kern0pt}def\ \isacommand{using}\isamarkupfalse%
\ classes{\isacharunderscore}{\kern0pt}subset{\isacharunderscore}{\kern0pt}cube{\isacharbrackleft}{\kern0pt}of\ {\isachardoublequoteopen}k{\isacharplus}{\kern0pt}{\isadigit{1}}{\isachardoublequoteclose}\ t\ {\isachardoublequoteopen}k{\isacharplus}{\kern0pt}{\isadigit{1}}{\isachardoublequoteclose}{\isacharbrackright}{\kern0pt}\ \isacommand{by}\isamarkupfalse%
\ auto\isanewline
\ \ \ \ \ \ \isacommand{then}\isamarkupfalse%
\ \isacommand{have}\isamarkupfalse%
\ {\isachardoublequoteopen}{\isasymexists}{\isacharbang}{\kern0pt}y{\isachardot}{\kern0pt}\ y\ {\isasymin}\ classes\ {\isacharparenleft}{\kern0pt}k{\isacharplus}{\kern0pt}{\isadigit{1}}{\isacharparenright}{\kern0pt}\ t\ {\isacharparenleft}{\kern0pt}k{\isacharplus}{\kern0pt}{\isadigit{1}}{\isacharparenright}{\kern0pt}{\isachardoublequoteclose}\ \isacommand{using}\isamarkupfalse%
\ {\isacartoucheopen}{\isasymforall}x\ {\isasymin}\ classes\ {\isacharparenleft}{\kern0pt}k{\isacharplus}{\kern0pt}{\isadigit{1}}{\isacharparenright}{\kern0pt}\ t\ {\isacharparenleft}{\kern0pt}k{\isacharplus}{\kern0pt}{\isadigit{1}}{\isacharparenright}{\kern0pt}{\isachardot}{\kern0pt}\ {\isasymforall}u\ {\isacharless}{\kern0pt}\ k\ {\isacharplus}{\kern0pt}\ {\isadigit{1}}{\isachardot}{\kern0pt}\ x\ u\ {\isacharequal}{\kern0pt}\ t{\isacartoucheclose}\ \isacommand{by}\isamarkupfalse%
\ auto\isanewline
\ \ \ \ \ \ \isacommand{have}\isamarkupfalse%
\ {\isachardoublequoteopen}{\isasymexists}c{\isacharless}{\kern0pt}r{\isachardot}{\kern0pt}\ {\isasymforall}y\ {\isasymin}\ classes\ {\isacharparenleft}{\kern0pt}k{\isacharplus}{\kern0pt}{\isadigit{1}}{\isacharparenright}{\kern0pt}\ t\ {\isacharparenleft}{\kern0pt}k{\isacharplus}{\kern0pt}{\isadigit{1}}{\isacharparenright}{\kern0pt}{\isachardot}{\kern0pt}\ {\isasymchi}\ {\isacharparenleft}{\kern0pt}T\ y{\isacharparenright}{\kern0pt}\ {\isacharequal}{\kern0pt}\ c{\isachardoublequoteclose}\ \isanewline
\ \ \ \ \ \ \isacommand{proof}\isamarkupfalse%
\ {\isacharminus}{\kern0pt}\isanewline
\ \ \ \ \ \ \ \ \isacommand{have}\isamarkupfalse%
\ {\isachardoublequoteopen}{\isasymforall}y\ {\isasymin}\ classes\ {\isacharparenleft}{\kern0pt}k{\isacharplus}{\kern0pt}{\isadigit{1}}{\isacharparenright}{\kern0pt}\ t\ {\isacharparenleft}{\kern0pt}k{\isacharplus}{\kern0pt}{\isadigit{1}}{\isacharparenright}{\kern0pt}{\isachardot}{\kern0pt}\ T\ y\ {\isasymin}\ cube\ {\isacharparenleft}{\kern0pt}n{\isacharplus}{\kern0pt}m{\isacharparenright}{\kern0pt}\ {\isacharparenleft}{\kern0pt}t{\isacharplus}{\kern0pt}{\isadigit{1}}{\isacharparenright}{\kern0pt}{\isachardoublequoteclose}\ \isacommand{using}\isamarkupfalse%
\ T{\isacharunderscore}{\kern0pt}prop\ classes{\isacharunderscore}{\kern0pt}subset{\isacharunderscore}{\kern0pt}cube\ \isacommand{by}\isamarkupfalse%
\ blast\isanewline
\ \ \ \ \ \ \ \ \isacommand{then}\isamarkupfalse%
\ \isacommand{have}\isamarkupfalse%
\ {\isachardoublequoteopen}{\isasymforall}y\ {\isasymin}\ classes\ {\isacharparenleft}{\kern0pt}k{\isacharplus}{\kern0pt}{\isadigit{1}}{\isacharparenright}{\kern0pt}\ t\ {\isacharparenleft}{\kern0pt}k{\isacharplus}{\kern0pt}{\isadigit{1}}{\isacharparenright}{\kern0pt}{\isachardot}{\kern0pt}\ {\isasymchi}\ {\isacharparenleft}{\kern0pt}T\ y{\isacharparenright}{\kern0pt}\ {\isacharless}{\kern0pt}\ r{\isachardoublequoteclose}\ \isacommand{using}\isamarkupfalse%
\ {\isasymchi}{\isacharunderscore}{\kern0pt}prop\ \isanewline
\ \ \ \ \ \ \ \ \ \ \isacommand{unfolding}\isamarkupfalse%
\ n{\isacharunderscore}{\kern0pt}def\ d{\isacharunderscore}{\kern0pt}def\ \isacommand{using}\isamarkupfalse%
\ M{\isacharprime}{\kern0pt}{\isacharunderscore}{\kern0pt}prop\ \isacommand{by}\isamarkupfalse%
\ auto\ \isanewline
\ \ \ \ \ \ \ \ \isacommand{then}\isamarkupfalse%
\ \isacommand{show}\isamarkupfalse%
\ {\isachardoublequoteopen}{\isasymexists}c{\isacharless}{\kern0pt}r{\isachardot}{\kern0pt}\ {\isasymforall}y\ {\isasymin}\ classes\ {\isacharparenleft}{\kern0pt}k{\isacharplus}{\kern0pt}{\isadigit{1}}{\isacharparenright}{\kern0pt}\ t\ {\isacharparenleft}{\kern0pt}k{\isacharplus}{\kern0pt}{\isadigit{1}}{\isacharparenright}{\kern0pt}{\isachardot}{\kern0pt}\ {\isasymchi}\ {\isacharparenleft}{\kern0pt}T\ y{\isacharparenright}{\kern0pt}\ {\isacharequal}{\kern0pt}\ c{\isachardoublequoteclose}\ \isacommand{using}\isamarkupfalse%
\ {\isacartoucheopen}{\isasymexists}{\isacharbang}{\kern0pt}y{\isachardot}{\kern0pt}\ y\ {\isasymin}\ classes\ {\isacharparenleft}{\kern0pt}k{\isacharplus}{\kern0pt}{\isadigit{1}}{\isacharparenright}{\kern0pt}\ t\ {\isacharparenleft}{\kern0pt}k{\isacharplus}{\kern0pt}{\isadigit{1}}{\isacharparenright}{\kern0pt}{\isacartoucheclose}\ \isacommand{by}\isamarkupfalse%
\ blast\isanewline
\ \ \ \ \ \ \isacommand{qed}\isamarkupfalse%
\isanewline
\ \ \ \ \ \ \isacommand{then}\isamarkupfalse%
\ \isacommand{show}\isamarkupfalse%
\ {\isachardoublequoteopen}{\isasymexists}c{\isacharless}{\kern0pt}r{\isachardot}{\kern0pt}\ {\isasymforall}x\ {\isasymin}\ T\ {\isacharbackquote}{\kern0pt}\ classes\ {\isacharparenleft}{\kern0pt}k{\isacharplus}{\kern0pt}{\isadigit{1}}{\isacharparenright}{\kern0pt}\ t\ {\isacharparenleft}{\kern0pt}k{\isacharplus}{\kern0pt}{\isadigit{1}}{\isacharparenright}{\kern0pt}{\isachardot}{\kern0pt}\ {\isasymchi}\ x\ {\isacharequal}{\kern0pt}\ c{\isachardoublequoteclose}\ \isacommand{by}\isamarkupfalse%
\ blast\isanewline
\ \ \ \ \isacommand{qed}\isamarkupfalse%
\isanewline
\ \ \ \ \isacommand{then}\isamarkupfalse%
\ \isacommand{have}\isamarkupfalse%
\ {\isachardoublequoteopen}\ {\isacharparenleft}{\kern0pt}{\isasymforall}i\ {\isasymin}\ {\isacharbraceleft}{\kern0pt}{\isachardot}{\kern0pt}{\isachardot}{\kern0pt}k{\isacharplus}{\kern0pt}{\isadigit{1}}{\isacharbraceright}{\kern0pt}{\isachardot}{\kern0pt}\ {\isasymexists}c{\isacharless}{\kern0pt}r{\isachardot}{\kern0pt}\ {\isasymforall}x\ {\isasymin}\ T\ {\isacharbackquote}{\kern0pt}\ classes\ {\isacharparenleft}{\kern0pt}k{\isacharplus}{\kern0pt}{\isadigit{1}}{\isacharparenright}{\kern0pt}\ t\ i{\isachardot}{\kern0pt}\ {\isasymchi}\ x\ {\isacharequal}{\kern0pt}\ c{\isacharparenright}{\kern0pt}{\isachardoublequoteclose}\ \isacommand{using}\isamarkupfalse%
\ {\isacartoucheopen}{\isasymforall}i{\isasymle}k{\isachardot}{\kern0pt}\ {\isasymexists}c{\isacharless}{\kern0pt}r{\isachardot}{\kern0pt}\ {\isasymforall}x\ {\isasymin}\ T\ {\isacharbackquote}{\kern0pt}\ classes\ {\isacharparenleft}{\kern0pt}k{\isacharplus}{\kern0pt}{\isadigit{1}}{\isacharparenright}{\kern0pt}\ t\ i{\isachardot}{\kern0pt}\ {\isasymchi}\ x\ {\isacharequal}{\kern0pt}\ c{\isacartoucheclose}\ \isacommand{by}\isamarkupfalse%
\ {\isacharparenleft}{\kern0pt}auto\ simp{\isacharcolon}{\kern0pt}\ algebra{\isacharunderscore}{\kern0pt}simps\ le{\isacharunderscore}{\kern0pt}Suc{\isacharunderscore}{\kern0pt}eq{\isacharparenright}{\kern0pt}\ \isanewline
\ \ \ \ \isacommand{then}\isamarkupfalse%
\ \isacommand{have}\isamarkupfalse%
\ {\isachardoublequoteopen}{\isacharparenleft}{\kern0pt}{\isasymforall}i\ {\isasymin}\ {\isacharbraceleft}{\kern0pt}{\isachardot}{\kern0pt}{\isachardot}{\kern0pt}k{\isacharplus}{\kern0pt}{\isadigit{1}}{\isacharbraceright}{\kern0pt}{\isachardot}{\kern0pt}\ {\isasymexists}c{\isacharless}{\kern0pt}r{\isachardot}{\kern0pt}\ {\isasymforall}x\ {\isasymin}\ classes\ {\isacharparenleft}{\kern0pt}k{\isacharplus}{\kern0pt}{\isadigit{1}}{\isacharparenright}{\kern0pt}\ t\ i{\isachardot}{\kern0pt}\ {\isasymchi}\ {\isacharparenleft}{\kern0pt}T\ x{\isacharparenright}{\kern0pt}\ {\isacharequal}{\kern0pt}\ c{\isacharparenright}{\kern0pt}{\isachardoublequoteclose}\ \isacommand{by}\isamarkupfalse%
\ simp\isanewline
\ \ \ \ \isacommand{then}\isamarkupfalse%
\ \isacommand{have}\isamarkupfalse%
\ {\isachardoublequoteopen}layered{\isacharunderscore}{\kern0pt}subspace\ T\ {\isacharparenleft}{\kern0pt}k{\isacharplus}{\kern0pt}{\isadigit{1}}{\isacharparenright}{\kern0pt}\ {\isacharparenleft}{\kern0pt}n\ {\isacharplus}{\kern0pt}\ m{\isacharparenright}{\kern0pt}\ t\ r\ {\isasymchi}{\isachardoublequoteclose}\ \ \isacommand{using}\isamarkupfalse%
\ subspace{\isacharunderscore}{\kern0pt}T\ that{\isacharparenleft}{\kern0pt}{\isadigit{1}}{\isacharparenright}{\kern0pt}\ {\isacartoucheopen}n\ {\isacharplus}{\kern0pt}\ m\ {\isacharequal}{\kern0pt}\ M{\isacharprime}{\kern0pt}{\isacartoucheclose}\ \isacommand{unfolding}\isamarkupfalse%
\ layered{\isacharunderscore}{\kern0pt}subspace{\isacharunderscore}{\kern0pt}def\ \isacommand{by}\isamarkupfalse%
\ blast\isanewline
\ \ \ \isacommand{then}\isamarkupfalse%
\ \isacommand{show}\isamarkupfalse%
\ {\isacharquery}{\kern0pt}thesis\ \isacommand{using}\isamarkupfalse%
\ {\isacartoucheopen}n\ {\isacharplus}{\kern0pt}\ m\ {\isacharequal}{\kern0pt}\ M{\isacharprime}{\kern0pt}{\isacartoucheclose}\ \isacommand{by}\isamarkupfalse%
\ blast\ \isanewline
\ \ \isacommand{qed}\isamarkupfalse%
\isanewline
\ \ \isacommand{then}\isamarkupfalse%
\ \isacommand{show}\isamarkupfalse%
\ {\isacharquery}{\kern0pt}thesis\ \isacommand{unfolding}\isamarkupfalse%
\ lhj{\isacharunderscore}{\kern0pt}def\ \isacommand{using}\isamarkupfalse%
\ m{\isacharunderscore}{\kern0pt}props\ exI{\isacharbrackleft}{\kern0pt}of\ {\isachardoublequoteopen}{\isasymlambda}M{\isachardot}{\kern0pt}\ {\isasymforall}M{\isacharprime}{\kern0pt}{\isasymge}M{\isachardot}{\kern0pt}\ {\isasymforall}{\isasymchi}{\isachardot}{\kern0pt}\ {\isasymchi}\ {\isasymin}\ cube\ M{\isacharprime}{\kern0pt}\ {\isacharparenleft}{\kern0pt}t\ {\isacharplus}{\kern0pt}\ {\isadigit{1}}{\isacharparenright}{\kern0pt}\ {\isasymrightarrow}\isactrlsub E\ {\isacharbraceleft}{\kern0pt}{\isachardot}{\kern0pt}{\isachardot}{\kern0pt}{\isacharless}{\kern0pt}r{\isacharbraceright}{\kern0pt}\ {\isasymlongrightarrow}\ {\isacharparenleft}{\kern0pt}{\isasymexists}S{\isachardot}{\kern0pt}\ layered{\isacharunderscore}{\kern0pt}subspace\ S\ {\isacharparenleft}{\kern0pt}k\ {\isacharplus}{\kern0pt}\ {\isadigit{1}}{\isacharparenright}{\kern0pt}\ M{\isacharprime}{\kern0pt}\ t\ r\ {\isasymchi}{\isacharparenright}{\kern0pt}{\isachardoublequoteclose}\ m{\isacharbrackright}{\kern0pt}\isanewline
\ \ \ \ \isacommand{by}\isamarkupfalse%
\ blast\isanewline
\isacommand{qed}\isamarkupfalse%
%
\endisatagproof
{\isafoldproof}%
%
\isadelimproof
\isanewline
%
\endisadelimproof
\isanewline
\isanewline
\isanewline
\isacommand{theorem}\isamarkupfalse%
\ theorem{\isadigit{4}}{\isacharcolon}{\kern0pt}\ \isakeyword{fixes}\ k\ \isakeyword{assumes}\ {\isachardoublequoteopen}{\isasymAnd}r{\isacharprime}{\kern0pt}{\isachardot}{\kern0pt}\ hj\ r{\isacharprime}{\kern0pt}\ t{\isachardoublequoteclose}\ \isakeyword{shows}\ {\isachardoublequoteopen}lhj\ r\ t\ k{\isachardoublequoteclose}\isanewline
%
\isadelimproof
%
\endisadelimproof
%
\isatagproof
\isacommand{proof}\isamarkupfalse%
\ {\isacharparenleft}{\kern0pt}induction\ k\ arbitrary{\isacharcolon}{\kern0pt}\ r\ rule{\isacharcolon}{\kern0pt}\ less{\isacharunderscore}{\kern0pt}induct{\isacharparenright}{\kern0pt}\isanewline
\ \ \isacommand{case}\isamarkupfalse%
\ {\isacharparenleft}{\kern0pt}less\ k{\isacharparenright}{\kern0pt}\isanewline
\ \ \isacommand{consider}\isamarkupfalse%
\ {\isachardoublequoteopen}k\ {\isacharequal}{\kern0pt}\ {\isadigit{0}}{\isachardoublequoteclose}\ {\isacharbar}{\kern0pt}\ {\isachardoublequoteopen}k\ {\isacharequal}{\kern0pt}\ {\isadigit{1}}{\isachardoublequoteclose}\ {\isacharbar}{\kern0pt}\ {\isachardoublequoteopen}k\ {\isasymge}\ {\isadigit{2}}{\isachardoublequoteclose}\ \isacommand{by}\isamarkupfalse%
\ linarith\isanewline
\ \ \isacommand{then}\isamarkupfalse%
\ \isacommand{show}\isamarkupfalse%
\ {\isacharquery}{\kern0pt}case\isanewline
\ \ \isacommand{proof}\isamarkupfalse%
\ {\isacharparenleft}{\kern0pt}cases{\isacharparenright}{\kern0pt}\isanewline
\ \ \ \ \isacommand{case}\isamarkupfalse%
\ {\isadigit{1}}\isanewline
\ \ \ \ \isacommand{then}\isamarkupfalse%
\ \isacommand{show}\isamarkupfalse%
\ {\isacharquery}{\kern0pt}thesis\ \isacommand{using}\isamarkupfalse%
\ dim{\isadigit{0}}{\isacharunderscore}{\kern0pt}layered{\isacharunderscore}{\kern0pt}subspace{\isacharunderscore}{\kern0pt}ex\ \isacommand{unfolding}\isamarkupfalse%
\ lhj{\isacharunderscore}{\kern0pt}def\ \isacommand{by}\isamarkupfalse%
\ auto\isanewline
\ \ \isacommand{next}\isamarkupfalse%
\isanewline
\ \ \ \ \isacommand{case}\isamarkupfalse%
\ {\isadigit{2}}\isanewline
\ \ \ \ \isacommand{then}\isamarkupfalse%
\ \isacommand{show}\isamarkupfalse%
\ {\isacharquery}{\kern0pt}thesis\isanewline
\ \ \ \ \isacommand{proof}\isamarkupfalse%
\ {\isacharparenleft}{\kern0pt}cases\ {\isachardoublequoteopen}t\ {\isachargreater}{\kern0pt}\ {\isadigit{0}}{\isachardoublequoteclose}{\isacharparenright}{\kern0pt}\isanewline
\ \ \ \ \ \ \isacommand{case}\isamarkupfalse%
\ True\isanewline
\ \ \ \ \ \ \isacommand{then}\isamarkupfalse%
\ \isacommand{show}\isamarkupfalse%
\ {\isacharquery}{\kern0pt}thesis\ \isacommand{using}\isamarkupfalse%
\ thm{\isadigit{4}}{\isacharunderscore}{\kern0pt}k{\isacharunderscore}{\kern0pt}{\isadigit{1}}{\isacharbrackleft}{\kern0pt}of\ {\isachardoublequoteopen}t{\isachardoublequoteclose}{\isacharbrackright}{\kern0pt}\ assms\ {\isadigit{2}}\ \isacommand{by}\isamarkupfalse%
\ blast\isanewline
\ \ \ \ \isacommand{next}\isamarkupfalse%
\isanewline
\ \ \ \ \ \ \isacommand{case}\isamarkupfalse%
\ False\isanewline
\ \ \ \ \ \ \isacommand{then}\isamarkupfalse%
\ \isacommand{show}\isamarkupfalse%
\ {\isacharquery}{\kern0pt}thesis\ \isacommand{using}\isamarkupfalse%
\ assms\ \isacommand{unfolding}\isamarkupfalse%
\ hj{\isacharunderscore}{\kern0pt}def\ lhj{\isacharunderscore}{\kern0pt}def\ cube{\isacharunderscore}{\kern0pt}def\ \isacommand{by}\isamarkupfalse%
\ fastforce\isanewline
\ \ \ \ \isacommand{qed}\isamarkupfalse%
\isanewline
\ \ \isacommand{next}\isamarkupfalse%
\isanewline
\ \ \ \ \isacommand{case}\isamarkupfalse%
\ {\isadigit{3}}\isanewline
\ \ \ \ \isacommand{note}\isamarkupfalse%
\ less\isanewline
\ \ \ \ \isacommand{then}\isamarkupfalse%
\ \isacommand{show}\isamarkupfalse%
\ {\isacharquery}{\kern0pt}thesis\isanewline
\ \ \ \ \isacommand{proof}\isamarkupfalse%
\ {\isacharparenleft}{\kern0pt}cases\ {\isachardoublequoteopen}t\ {\isachargreater}{\kern0pt}\ {\isadigit{0}}\ {\isasymand}\ r\ {\isachargreater}{\kern0pt}\ {\isadigit{0}}{\isachardoublequoteclose}{\isacharparenright}{\kern0pt}\isanewline
\ \ \ \ \ \isacommand{case}\isamarkupfalse%
\ True\isanewline
\ \ \ \ \ \isacommand{then}\isamarkupfalse%
\ \isacommand{show}\isamarkupfalse%
\ {\isacharquery}{\kern0pt}thesis\ \ \isacommand{using}\isamarkupfalse%
\ thm{\isadigit{4}}{\isacharunderscore}{\kern0pt}step{\isacharbrackleft}{\kern0pt}of\ t\ {\isachardoublequoteopen}k{\isacharminus}{\kern0pt}{\isadigit{1}}{\isachardoublequoteclose}\ r{\isacharbrackright}{\kern0pt}\isanewline
\ \ \ \ \ \ \ \isacommand{using}\isamarkupfalse%
\ assms\ less{\isachardot}{\kern0pt}IH\ {\isadigit{3}}\ One{\isacharunderscore}{\kern0pt}nat{\isacharunderscore}{\kern0pt}def\ Suc{\isacharunderscore}{\kern0pt}pred\ \isacommand{by}\isamarkupfalse%
\ fastforce\isanewline
\ \ \ \ \isacommand{next}\isamarkupfalse%
\isanewline
\ \ \ \ \ \ \isacommand{case}\isamarkupfalse%
\ False\isanewline
\ \ \ \ \ \ \isacommand{then}\isamarkupfalse%
\ \isacommand{consider}\isamarkupfalse%
\ {\isachardoublequoteopen}t\ {\isacharequal}{\kern0pt}\ {\isadigit{0}}{\isachardoublequoteclose}\ {\isacharbar}{\kern0pt}\ {\isachardoublequoteopen}t\ {\isachargreater}{\kern0pt}\ {\isadigit{0}}\ {\isasymand}\ r\ {\isacharequal}{\kern0pt}\ {\isadigit{0}}{\isachardoublequoteclose}\ {\isacharbar}{\kern0pt}\ {\isachardoublequoteopen}t\ {\isacharequal}{\kern0pt}\ {\isadigit{0}}\ {\isasymand}\ r\ {\isacharequal}{\kern0pt}\ {\isadigit{0}}{\isachardoublequoteclose}\ \isacommand{by}\isamarkupfalse%
\ fastforce\isanewline
\ \ \ \ \ \ \isacommand{then}\isamarkupfalse%
\ \isacommand{show}\isamarkupfalse%
\ {\isacharquery}{\kern0pt}thesis\isanewline
\ \ \ \ \ \ \isacommand{proof}\isamarkupfalse%
\ cases\isanewline
\ \ \ \ \ \ \ \ \isacommand{case}\isamarkupfalse%
\ {\isadigit{1}}\isanewline
\ \ \ \ \ \ \ \ \isacommand{then}\isamarkupfalse%
\ \isacommand{show}\isamarkupfalse%
\ {\isacharquery}{\kern0pt}thesis\ \isacommand{using}\isamarkupfalse%
\ assms\ \isacommand{unfolding}\isamarkupfalse%
\ hj{\isacharunderscore}{\kern0pt}def\ lhj{\isacharunderscore}{\kern0pt}def\ cube{\isacharunderscore}{\kern0pt}def\ \isacommand{by}\isamarkupfalse%
\ fastforce\isanewline
\ \ \ \ \ \ \isacommand{next}\isamarkupfalse%
\isanewline
\ \ \ \ \ \ \ \ \isacommand{case}\isamarkupfalse%
\ {\isadigit{2}}\isanewline
\ \ \ \ \ \ \ \ \isacommand{then}\isamarkupfalse%
\ \isacommand{obtain}\isamarkupfalse%
\ N\ \isakeyword{where}\ N{\isacharunderscore}{\kern0pt}prop{\isacharcolon}{\kern0pt}\ {\isachardoublequoteopen}N\ {\isachargreater}{\kern0pt}\ {\isadigit{0}}{\isachardoublequoteclose}\ {\isachardoublequoteopen}{\isacharparenleft}{\kern0pt}{\isasymforall}N{\isacharprime}{\kern0pt}{\isasymge}N{\isachardot}{\kern0pt}\ {\isasymforall}{\isasymchi}{\isachardot}{\kern0pt}\ {\isasymchi}\ {\isasymin}\ cube\ N{\isacharprime}{\kern0pt}\ t\ {\isasymrightarrow}\isactrlsub E\ {\isacharbraceleft}{\kern0pt}{\isachardot}{\kern0pt}{\isachardot}{\kern0pt}{\isacharless}{\kern0pt}r{\isacharbraceright}{\kern0pt}\ {\isasymlongrightarrow}\ {\isacharparenleft}{\kern0pt}{\isasymexists}L\ c{\isachardot}{\kern0pt}\ c\ {\isacharless}{\kern0pt}\ r\ {\isasymand}\ is{\isacharunderscore}{\kern0pt}line\ L\ N{\isacharprime}{\kern0pt}\ t\ {\isasymand}\ {\isacharparenleft}{\kern0pt}{\isasymforall}y\ {\isasymin}\ L\ {\isacharbackquote}{\kern0pt}\ {\isacharbraceleft}{\kern0pt}{\isachardot}{\kern0pt}{\isachardot}{\kern0pt}{\isacharless}{\kern0pt}t{\isacharbraceright}{\kern0pt}{\isachardot}{\kern0pt}\ {\isasymchi}\ y\ {\isacharequal}{\kern0pt}\ c{\isacharparenright}{\kern0pt}{\isacharparenright}{\kern0pt}{\isacharparenright}{\kern0pt}{\isachardoublequoteclose}\ \isacommand{using}\isamarkupfalse%
\ assms\ \isacommand{unfolding}\isamarkupfalse%
\ hj{\isacharunderscore}{\kern0pt}def\ \isacommand{by}\isamarkupfalse%
\ blast\isanewline
\ \ \ \ \ \ \ \ \isacommand{have}\isamarkupfalse%
\ {\isachardoublequoteopen}cube\ N{\isacharprime}{\kern0pt}\ t\ {\isasymrightarrow}\isactrlsub E\ {\isacharbraceleft}{\kern0pt}{\isachardot}{\kern0pt}{\isachardot}{\kern0pt}{\isacharless}{\kern0pt}r{\isacharbraceright}{\kern0pt}\ {\isacharequal}{\kern0pt}\ {\isacharbraceleft}{\kern0pt}{\isacharbraceright}{\kern0pt}{\isachardoublequoteclose}\ \isakeyword{if}\ {\isachardoublequoteopen}N{\isacharprime}{\kern0pt}{\isasymge}N{\isachardoublequoteclose}\ \isakeyword{for}\ N{\isacharprime}{\kern0pt}\isanewline
\ \ \ \ \ \ \ \ \isacommand{proof}\isamarkupfalse%
{\isacharminus}{\kern0pt}\isanewline
\ \ \ \ \ \ \ \ \ \ \isacommand{have}\isamarkupfalse%
\ {\isachardoublequoteopen}cube\ N{\isacharprime}{\kern0pt}\ t\ {\isasymnoteq}\ {\isacharbraceleft}{\kern0pt}{\isacharbraceright}{\kern0pt}{\isachardoublequoteclose}\ \isacommand{using}\isamarkupfalse%
\ N{\isacharunderscore}{\kern0pt}prop{\isacharparenleft}{\kern0pt}{\isadigit{2}}{\isacharparenright}{\kern0pt}\ that\ {\isadigit{2}}\ \isacommand{by}\isamarkupfalse%
\ auto\ \ \ \isanewline
\ \ \ \ \ \ \ \ \ \ \isacommand{then}\isamarkupfalse%
\ \isacommand{show}\isamarkupfalse%
\ {\isacharquery}{\kern0pt}thesis\ \isacommand{using}\isamarkupfalse%
\ {\isadigit{2}}\ \isacommand{by}\isamarkupfalse%
\ blast\isanewline
\ \ \ \ \ \ \ \ \isacommand{qed}\isamarkupfalse%
\isanewline
\ \ \ \ \ \ \ \ \isacommand{then}\isamarkupfalse%
\ \isacommand{show}\isamarkupfalse%
\ {\isacharquery}{\kern0pt}thesis\ \isacommand{using}\isamarkupfalse%
\ N{\isacharunderscore}{\kern0pt}prop\ \ \isacommand{unfolding}\isamarkupfalse%
\ lhj{\isacharunderscore}{\kern0pt}def\ cube{\isacharunderscore}{\kern0pt}def\ \isanewline
\ \ \ \ \ \ \ \ \ \ \isacommand{by}\isamarkupfalse%
\ {\isacharparenleft}{\kern0pt}metis\ PiE{\isacharunderscore}{\kern0pt}eq{\isacharunderscore}{\kern0pt}empty{\isacharunderscore}{\kern0pt}iff\ all{\isacharunderscore}{\kern0pt}not{\isacharunderscore}{\kern0pt}in{\isacharunderscore}{\kern0pt}conv\ lessThan{\isacharunderscore}{\kern0pt}iff\ trans{\isacharunderscore}{\kern0pt}less{\isacharunderscore}{\kern0pt}add{\isadigit{1}}{\isacharparenright}{\kern0pt}\ \isanewline
\ \ \ \ \ \ \isacommand{next}\isamarkupfalse%
\isanewline
\ \ \ \ \ \ \ \ \isacommand{case}\isamarkupfalse%
\ {\isadigit{3}}\isanewline
\ \ \ \ \ \ \ \ \isacommand{then}\isamarkupfalse%
\ \isacommand{have}\isamarkupfalse%
\ {\isachardoublequoteopen}{\isacharparenleft}{\kern0pt}{\isasymexists}L\ c{\isachardot}{\kern0pt}\ c\ {\isacharless}{\kern0pt}\ r\ {\isasymand}\ is{\isacharunderscore}{\kern0pt}line\ L\ N{\isacharprime}{\kern0pt}\ t\ {\isasymand}\ {\isacharparenleft}{\kern0pt}{\isasymforall}y\ {\isasymin}\ L\ {\isacharbackquote}{\kern0pt}\ {\isacharbraceleft}{\kern0pt}{\isachardot}{\kern0pt}{\isachardot}{\kern0pt}{\isacharless}{\kern0pt}t{\isacharbraceright}{\kern0pt}{\isachardot}{\kern0pt}\ {\isasymchi}\ y\ {\isacharequal}{\kern0pt}\ c{\isacharparenright}{\kern0pt}{\isacharparenright}{\kern0pt}\ {\isasymLongrightarrow}\ False{\isachardoublequoteclose}\ \isakeyword{for}\ N{\isacharprime}{\kern0pt}\ {\isasymchi}\ \isacommand{by}\isamarkupfalse%
\ blast\isanewline
\ \ \ \ \ \ \ \ \isacommand{then}\isamarkupfalse%
\ \isacommand{have}\isamarkupfalse%
\ False\ \isacommand{using}\isamarkupfalse%
\ assms\ {\isadigit{3}}\ \isacommand{unfolding}\isamarkupfalse%
\ hj{\isacharunderscore}{\kern0pt}def\ cube{\isacharunderscore}{\kern0pt}def\ \isacommand{by}\isamarkupfalse%
\ fastforce\isanewline
\ \ \ \ \ \ \ \ \isacommand{then}\isamarkupfalse%
\ \isacommand{show}\isamarkupfalse%
\ {\isacharquery}{\kern0pt}thesis\ \isacommand{by}\isamarkupfalse%
\ blast\isanewline
\ \ \ \ \ \ \isacommand{qed}\isamarkupfalse%
\isanewline
\isanewline
\ \ \ \ \isacommand{qed}\isamarkupfalse%
\isanewline
\ \ \isacommand{qed}\isamarkupfalse%
\isanewline
\isacommand{qed}\isamarkupfalse%
%
\endisatagproof
{\isafoldproof}%
%
\isadelimproof
%
\endisadelimproof
%
\begin{isamarkuptext}%
We provide a way to construct a monochromatic line in C(n, t + 1) from a k-dimensional k-coloured layered subspace S in C(n, t + 1).
The idea is to rely on the fact that there are k+1 classes in S, but only k colours. It thus follows by the Pigeonhole Principle that two classes must share the same colour. The way classes are defined allows for a straightforward construction of a line that contains points in both classes. Thus we have our monochromatic line.%
\end{isamarkuptext}\isamarkuptrue%
\isacommand{theorem}\isamarkupfalse%
\ thm{\isadigit{5}}{\isacharcolon}{\kern0pt}\ \isakeyword{assumes}\ {\isachardoublequoteopen}layered{\isacharunderscore}{\kern0pt}subspace\ S\ k\ n\ t\ k\ {\isasymchi}{\isachardoublequoteclose}\ \isakeyword{and}\ {\isachardoublequoteopen}t\ {\isachargreater}{\kern0pt}\ {\isadigit{0}}{\isachardoublequoteclose}\ \ \isakeyword{shows}\ {\isachardoublequoteopen}{\isacharparenleft}{\kern0pt}{\isasymexists}L{\isachardot}{\kern0pt}\ {\isasymexists}c{\isacharless}{\kern0pt}k{\isachardot}{\kern0pt}\ is{\isacharunderscore}{\kern0pt}line\ L\ n\ {\isacharparenleft}{\kern0pt}t{\isacharplus}{\kern0pt}{\isadigit{1}}{\isacharparenright}{\kern0pt}\ {\isasymand}\ {\isacharparenleft}{\kern0pt}{\isasymforall}y\ {\isasymin}\ L\ {\isacharbackquote}{\kern0pt}\ {\isacharbraceleft}{\kern0pt}{\isachardot}{\kern0pt}{\isachardot}{\kern0pt}{\isacharless}{\kern0pt}t{\isacharplus}{\kern0pt}{\isadigit{1}}{\isacharbraceright}{\kern0pt}{\isachardot}{\kern0pt}\ {\isasymchi}\ y\ {\isacharequal}{\kern0pt}\ c{\isacharparenright}{\kern0pt}{\isacharparenright}{\kern0pt}{\isachardoublequoteclose}\isanewline
%
\isadelimproof
%
\endisadelimproof
%
\isatagproof
\isacommand{proof}\isamarkupfalse%
{\isacharminus}{\kern0pt}\isanewline
\ \ \isacommand{define}\isamarkupfalse%
\ x\ \isakeyword{where}\ {\isachardoublequoteopen}x\ {\isasymequiv}\ {\isacharparenleft}{\kern0pt}{\isasymlambda}i{\isasymin}{\isacharbraceleft}{\kern0pt}{\isachardot}{\kern0pt}{\isachardot}{\kern0pt}k{\isacharbraceright}{\kern0pt}{\isachardot}{\kern0pt}\ {\isasymlambda}j{\isasymin}{\isacharbraceleft}{\kern0pt}{\isachardot}{\kern0pt}{\isachardot}{\kern0pt}{\isacharless}{\kern0pt}k{\isacharbraceright}{\kern0pt}{\isachardot}{\kern0pt}\ {\isacharparenleft}{\kern0pt}if\ j\ {\isacharless}{\kern0pt}\ k\ {\isacharminus}{\kern0pt}\ i\ then\ {\isadigit{0}}\ else\ t{\isacharparenright}{\kern0pt}{\isacharparenright}{\kern0pt}{\isachardoublequoteclose}\isanewline
\isanewline
\ \ \isacommand{have}\isamarkupfalse%
\ A{\isacharcolon}{\kern0pt}\ {\isachardoublequoteopen}x\ i\ {\isasymin}\ cube\ k\ {\isacharparenleft}{\kern0pt}t\ {\isacharplus}{\kern0pt}\ {\isadigit{1}}{\isacharparenright}{\kern0pt}{\isachardoublequoteclose}\ \isakeyword{if}\ {\isachardoublequoteopen}i\ {\isasymle}\ k{\isachardoublequoteclose}\ \isakeyword{for}\ i\ \isacommand{using}\isamarkupfalse%
\ that\ \isacommand{unfolding}\isamarkupfalse%
\ cube{\isacharunderscore}{\kern0pt}def\ x{\isacharunderscore}{\kern0pt}def\ \isacommand{by}\isamarkupfalse%
\ simp\isanewline
\ \ \isacommand{then}\isamarkupfalse%
\ \isacommand{have}\isamarkupfalse%
\ {\isachardoublequoteopen}S\ {\isacharparenleft}{\kern0pt}x\ i{\isacharparenright}{\kern0pt}\ {\isasymin}\ cube\ n\ {\isacharparenleft}{\kern0pt}t{\isacharplus}{\kern0pt}{\isadigit{1}}{\isacharparenright}{\kern0pt}{\isachardoublequoteclose}\ \isakeyword{if}\ {\isachardoublequoteopen}i\ {\isasymle}\ k{\isachardoublequoteclose}\ \isakeyword{for}\ i\ \isacommand{using}\isamarkupfalse%
\ that\ assms{\isacharparenleft}{\kern0pt}{\isadigit{1}}{\isacharparenright}{\kern0pt}\ \isacommand{unfolding}\isamarkupfalse%
\ layered{\isacharunderscore}{\kern0pt}subspace{\isacharunderscore}{\kern0pt}def\ is{\isacharunderscore}{\kern0pt}subspace{\isacharunderscore}{\kern0pt}def\ \isacommand{by}\isamarkupfalse%
\ fast\isanewline
\isanewline
\ \ \isacommand{have}\isamarkupfalse%
\ {\isachardoublequoteopen}{\isasymchi}\ {\isasymin}\ cube\ n\ {\isacharparenleft}{\kern0pt}t\ {\isacharplus}{\kern0pt}\ {\isadigit{1}}{\isacharparenright}{\kern0pt}\ {\isasymrightarrow}\isactrlsub E\ {\isacharbraceleft}{\kern0pt}{\isachardot}{\kern0pt}{\isachardot}{\kern0pt}{\isacharless}{\kern0pt}k{\isacharbraceright}{\kern0pt}{\isachardoublequoteclose}\ \isacommand{using}\isamarkupfalse%
\ assms\ \isacommand{unfolding}\isamarkupfalse%
\ layered{\isacharunderscore}{\kern0pt}subspace{\isacharunderscore}{\kern0pt}def\ \isacommand{by}\isamarkupfalse%
\ linarith\isanewline
\ \ \isacommand{then}\isamarkupfalse%
\ \isacommand{have}\isamarkupfalse%
\ {\isachardoublequoteopen}{\isasymchi}\ {\isacharbackquote}{\kern0pt}\ {\isacharparenleft}{\kern0pt}cube\ n\ {\isacharparenleft}{\kern0pt}t{\isacharplus}{\kern0pt}{\isadigit{1}}{\isacharparenright}{\kern0pt}{\isacharparenright}{\kern0pt}\ {\isasymsubseteq}\ {\isacharbraceleft}{\kern0pt}{\isachardot}{\kern0pt}{\isachardot}{\kern0pt}{\isacharless}{\kern0pt}k{\isacharbraceright}{\kern0pt}{\isachardoublequoteclose}\ \isacommand{by}\isamarkupfalse%
\ blast\isanewline
\ \ \isacommand{then}\isamarkupfalse%
\ \isacommand{have}\isamarkupfalse%
\ {\isachardoublequoteopen}card\ {\isacharparenleft}{\kern0pt}{\isasymchi}\ {\isacharbackquote}{\kern0pt}\ {\isacharparenleft}{\kern0pt}cube\ n\ {\isacharparenleft}{\kern0pt}t{\isacharplus}{\kern0pt}{\isadigit{1}}{\isacharparenright}{\kern0pt}{\isacharparenright}{\kern0pt}{\isacharparenright}{\kern0pt}\ {\isasymle}\ card\ {\isacharbraceleft}{\kern0pt}{\isachardot}{\kern0pt}{\isachardot}{\kern0pt}{\isacharless}{\kern0pt}k{\isacharbraceright}{\kern0pt}{\isachardoublequoteclose}\ \isanewline
\ \ \ \ \isacommand{by}\isamarkupfalse%
\ {\isacharparenleft}{\kern0pt}meson\ card{\isacharunderscore}{\kern0pt}mono\ finite{\isacharunderscore}{\kern0pt}lessThan{\isacharparenright}{\kern0pt}\isanewline
\ \ \isacommand{then}\isamarkupfalse%
\ \isacommand{have}\isamarkupfalse%
\ {\isacharasterisk}{\kern0pt}{\isacharcolon}{\kern0pt}\ {\isachardoublequoteopen}card\ {\isacharparenleft}{\kern0pt}{\isasymchi}\ {\isacharbackquote}{\kern0pt}\ {\isacharparenleft}{\kern0pt}cube\ n\ {\isacharparenleft}{\kern0pt}t{\isacharplus}{\kern0pt}{\isadigit{1}}{\isacharparenright}{\kern0pt}{\isacharparenright}{\kern0pt}{\isacharparenright}{\kern0pt}\ {\isasymle}\ k{\isachardoublequoteclose}\ \isacommand{by}\isamarkupfalse%
\ auto\isanewline
\ \ \isacommand{have}\isamarkupfalse%
\ {\isachardoublequoteopen}k\ {\isachargreater}{\kern0pt}\ {\isadigit{0}}{\isachardoublequoteclose}\ \isacommand{using}\isamarkupfalse%
\ assms{\isacharparenleft}{\kern0pt}{\isadigit{1}}{\isacharparenright}{\kern0pt}\ \isacommand{unfolding}\isamarkupfalse%
\ layered{\isacharunderscore}{\kern0pt}subspace{\isacharunderscore}{\kern0pt}def\ \isacommand{by}\isamarkupfalse%
\ auto\isanewline
\ \ \isacommand{have}\isamarkupfalse%
\ {\isachardoublequoteopen}inj{\isacharunderscore}{\kern0pt}on\ x\ {\isacharbraceleft}{\kern0pt}{\isachardot}{\kern0pt}{\isachardot}{\kern0pt}k{\isacharbraceright}{\kern0pt}{\isachardoublequoteclose}\isanewline
\ \ \isacommand{proof}\isamarkupfalse%
\ {\isacharminus}{\kern0pt}\isanewline
\ \ \ \ \isacommand{have}\isamarkupfalse%
\ {\isacharasterisk}{\kern0pt}{\isacharcolon}{\kern0pt}{\isachardoublequoteopen}x\ i{\isadigit{1}}\ {\isacharparenleft}{\kern0pt}k\ {\isacharminus}{\kern0pt}\ i{\isadigit{2}}{\isacharparenright}{\kern0pt}\ {\isasymnoteq}\ x\ i{\isadigit{2}}\ {\isacharparenleft}{\kern0pt}k\ {\isacharminus}{\kern0pt}\ i{\isadigit{2}}{\isacharparenright}{\kern0pt}{\isachardoublequoteclose}\ \isakeyword{if}\ {\isachardoublequoteopen}i{\isadigit{1}}\ {\isasymle}\ k{\isachardoublequoteclose}\ {\isachardoublequoteopen}i{\isadigit{2}}\ {\isasymle}\ k{\isachardoublequoteclose}\ {\isachardoublequoteopen}i{\isadigit{1}}\ {\isasymnoteq}\ i{\isadigit{2}}{\isachardoublequoteclose}\ {\isachardoublequoteopen}i{\isadigit{1}}\ {\isacharless}{\kern0pt}\ i{\isadigit{2}}{\isachardoublequoteclose}\ \isakeyword{for}\ i{\isadigit{1}}\ i{\isadigit{2}}\ \isacommand{using}\isamarkupfalse%
\ that\ assms{\isacharparenleft}{\kern0pt}{\isadigit{2}}{\isacharparenright}{\kern0pt}\ \isacommand{unfolding}\isamarkupfalse%
\ x{\isacharunderscore}{\kern0pt}def\ \isacommand{by}\isamarkupfalse%
\ auto\ \isanewline
\ \ \ \ \isacommand{have}\isamarkupfalse%
\ {\isachardoublequoteopen}{\isasymexists}j{\isacharless}{\kern0pt}k{\isachardot}{\kern0pt}\ x\ i{\isadigit{1}}\ j\ {\isasymnoteq}\ x\ i{\isadigit{2}}\ j{\isachardoublequoteclose}\ \isakeyword{if}\ {\isachardoublequoteopen}i{\isadigit{1}}\ {\isasymle}\ k{\isachardoublequoteclose}\ {\isachardoublequoteopen}i{\isadigit{2}}\ {\isasymle}\ k{\isachardoublequoteclose}\ {\isachardoublequoteopen}i{\isadigit{1}}\ {\isasymnoteq}\ i{\isadigit{2}}{\isachardoublequoteclose}\ \isakeyword{for}\ i{\isadigit{1}}\ i{\isadigit{2}}\isanewline
\ \ \ \ \isacommand{proof}\isamarkupfalse%
\ {\isacharparenleft}{\kern0pt}cases\ {\isachardoublequoteopen}i{\isadigit{1}}\ {\isasymle}\ i{\isadigit{2}}{\isachardoublequoteclose}{\isacharparenright}{\kern0pt}\isanewline
\ \ \ \ \ \ \isacommand{case}\isamarkupfalse%
\ True\isanewline
\ \ \ \ \ \ \isacommand{then}\isamarkupfalse%
\ \isacommand{have}\isamarkupfalse%
\ {\isachardoublequoteopen}k\ {\isacharminus}{\kern0pt}\ i{\isadigit{2}}\ {\isacharless}{\kern0pt}\ k{\isachardoublequoteclose}\ \isanewline
\ \ \ \ \ \ \ \ \isacommand{using}\isamarkupfalse%
\ {\isacartoucheopen}{\isadigit{0}}\ {\isacharless}{\kern0pt}\ k{\isacartoucheclose}\ that{\isacharparenleft}{\kern0pt}{\isadigit{3}}{\isacharparenright}{\kern0pt}\ \isacommand{by}\isamarkupfalse%
\ linarith\isanewline
\ \ \ \ \ \ \isacommand{then}\isamarkupfalse%
\ \isacommand{show}\isamarkupfalse%
\ {\isacharquery}{\kern0pt}thesis\ \isacommand{using}\isamarkupfalse%
\ that\ {\isacharasterisk}{\kern0pt}\ \isanewline
\ \ \ \ \ \ \ \ \isacommand{by}\isamarkupfalse%
\ {\isacharparenleft}{\kern0pt}meson\ True\ nat{\isacharunderscore}{\kern0pt}less{\isacharunderscore}{\kern0pt}le{\isacharparenright}{\kern0pt}\isanewline
\ \ \ \ \isacommand{next}\isamarkupfalse%
\isanewline
\ \ \ \ \ \ \isacommand{case}\isamarkupfalse%
\ False\isanewline
\ \ \ \ \ \ \isacommand{then}\isamarkupfalse%
\ \isacommand{have}\isamarkupfalse%
\ {\isachardoublequoteopen}i{\isadigit{2}}\ {\isacharless}{\kern0pt}\ i{\isadigit{1}}{\isachardoublequoteclose}\ \isacommand{by}\isamarkupfalse%
\ simp\isanewline
\ \ \ \ \ \ \isacommand{then}\isamarkupfalse%
\ \isacommand{show}\isamarkupfalse%
\ {\isacharquery}{\kern0pt}thesis\ \isacommand{using}\isamarkupfalse%
\ that\ {\isacharasterisk}{\kern0pt}{\isacharbrackleft}{\kern0pt}of\ i{\isadigit{2}}\ i{\isadigit{1}}{\isacharbrackright}{\kern0pt}\ {\isacartoucheopen}k\ {\isachargreater}{\kern0pt}\ {\isadigit{0}}{\isacartoucheclose}\ \ \isanewline
\ \ \ \ \ \ \ \ \isacommand{by}\isamarkupfalse%
\ {\isacharparenleft}{\kern0pt}metis\ diff{\isacharunderscore}{\kern0pt}less\ gr{\isacharunderscore}{\kern0pt}implies{\isacharunderscore}{\kern0pt}not{\isadigit{0}}\ le{\isadigit{0}}\ nat{\isacharunderscore}{\kern0pt}less{\isacharunderscore}{\kern0pt}le{\isacharparenright}{\kern0pt}\isanewline
\ \ \ \ \isacommand{qed}\isamarkupfalse%
\isanewline
\ \ \ \ \isacommand{then}\isamarkupfalse%
\ \isacommand{have}\isamarkupfalse%
\ {\isachardoublequoteopen}x\ i{\isadigit{1}}\ {\isasymnoteq}\ x\ i{\isadigit{2}}{\isachardoublequoteclose}\ \isakeyword{if}\ {\isachardoublequoteopen}i{\isadigit{1}}\ {\isasymle}\ k{\isachardoublequoteclose}\ {\isachardoublequoteopen}i{\isadigit{2}}\ {\isasymle}\ k{\isachardoublequoteclose}\ {\isachardoublequoteopen}i{\isadigit{1}}\ {\isasymnoteq}\ i{\isadigit{2}}{\isachardoublequoteclose}\ {\isachardoublequoteopen}i{\isadigit{1}}\ {\isacharless}{\kern0pt}\ i{\isadigit{2}}{\isachardoublequoteclose}\ \isakeyword{for}\ i{\isadigit{1}}\ i{\isadigit{2}}\ \isacommand{using}\isamarkupfalse%
\ that\ \isacommand{by}\isamarkupfalse%
\ fastforce\isanewline
\ \ \ \ \isacommand{then}\isamarkupfalse%
\ \isacommand{show}\isamarkupfalse%
\ {\isacharquery}{\kern0pt}thesis\ \isacommand{unfolding}\isamarkupfalse%
\ inj{\isacharunderscore}{\kern0pt}on{\isacharunderscore}{\kern0pt}def\ \ \isacommand{by}\isamarkupfalse%
\ {\isacharparenleft}{\kern0pt}metis\ atMost{\isacharunderscore}{\kern0pt}iff\ linorder{\isacharunderscore}{\kern0pt}cases{\isacharparenright}{\kern0pt}\isanewline
\ \ \isacommand{qed}\isamarkupfalse%
\isanewline
\ \ \isacommand{then}\isamarkupfalse%
\ \isacommand{have}\isamarkupfalse%
\ {\isachardoublequoteopen}card\ {\isacharparenleft}{\kern0pt}x\ {\isacharbackquote}{\kern0pt}\ {\isacharbraceleft}{\kern0pt}{\isachardot}{\kern0pt}{\isachardot}{\kern0pt}k{\isacharbraceright}{\kern0pt}{\isacharparenright}{\kern0pt}\ {\isacharequal}{\kern0pt}\ card\ {\isacharbraceleft}{\kern0pt}{\isachardot}{\kern0pt}{\isachardot}{\kern0pt}k{\isacharbraceright}{\kern0pt}{\isachardoublequoteclose}\ \isacommand{using}\isamarkupfalse%
\ card{\isacharunderscore}{\kern0pt}image\ \isacommand{by}\isamarkupfalse%
\ blast\isanewline
\ \ \isacommand{then}\isamarkupfalse%
\ \isacommand{have}\isamarkupfalse%
\ B{\isacharcolon}{\kern0pt}\ {\isachardoublequoteopen}card\ {\isacharparenleft}{\kern0pt}x\ {\isacharbackquote}{\kern0pt}\ {\isacharbraceleft}{\kern0pt}{\isachardot}{\kern0pt}{\isachardot}{\kern0pt}k{\isacharbraceright}{\kern0pt}{\isacharparenright}{\kern0pt}\ {\isacharequal}{\kern0pt}\ k{\isacharplus}{\kern0pt}{\isadigit{1}}{\isachardoublequoteclose}\ \isacommand{by}\isamarkupfalse%
\ simp\isanewline
\ \ \isacommand{have}\isamarkupfalse%
\ {\isachardoublequoteopen}x\ {\isacharbackquote}{\kern0pt}\ {\isacharbraceleft}{\kern0pt}{\isachardot}{\kern0pt}{\isachardot}{\kern0pt}k{\isacharbraceright}{\kern0pt}\ {\isasymsubseteq}\ cube\ k\ {\isacharparenleft}{\kern0pt}t{\isacharplus}{\kern0pt}{\isadigit{1}}{\isacharparenright}{\kern0pt}{\isachardoublequoteclose}\ \isacommand{using}\isamarkupfalse%
\ A\ \isacommand{by}\isamarkupfalse%
\ blast\isanewline
\ \ \isacommand{then}\isamarkupfalse%
\ \isacommand{have}\isamarkupfalse%
\ {\isachardoublequoteopen}S\ {\isacharbackquote}{\kern0pt}\ x\ {\isacharbackquote}{\kern0pt}\ {\isacharbraceleft}{\kern0pt}{\isachardot}{\kern0pt}{\isachardot}{\kern0pt}k{\isacharbraceright}{\kern0pt}\ {\isasymsubseteq}\ S\ {\isacharbackquote}{\kern0pt}\ cube\ k\ {\isacharparenleft}{\kern0pt}t{\isacharplus}{\kern0pt}{\isadigit{1}}{\isacharparenright}{\kern0pt}{\isachardoublequoteclose}\ \isacommand{by}\isamarkupfalse%
\ fast\isanewline
\ \ \isacommand{also}\isamarkupfalse%
\ \isacommand{have}\isamarkupfalse%
\ {\isachardoublequoteopen}{\isachardot}{\kern0pt}{\isachardot}{\kern0pt}{\isachardot}{\kern0pt}\ {\isasymsubseteq}\ cube\ n\ {\isacharparenleft}{\kern0pt}t{\isacharplus}{\kern0pt}{\isadigit{1}}{\isacharparenright}{\kern0pt}{\isachardoublequoteclose}\ \isanewline
\ \ \ \ \isacommand{by}\isamarkupfalse%
\ {\isacharparenleft}{\kern0pt}meson\ assms{\isacharparenleft}{\kern0pt}{\isadigit{1}}{\isacharparenright}{\kern0pt}\ layered{\isacharunderscore}{\kern0pt}subspace{\isacharunderscore}{\kern0pt}def\ subspace{\isacharunderscore}{\kern0pt}elems{\isacharunderscore}{\kern0pt}embed{\isacharparenright}{\kern0pt}\isanewline
\ \ \isacommand{finally}\isamarkupfalse%
\ \isacommand{have}\isamarkupfalse%
\ {\isachardoublequoteopen}S\ {\isacharbackquote}{\kern0pt}\ x\ {\isacharbackquote}{\kern0pt}\ {\isacharbraceleft}{\kern0pt}{\isachardot}{\kern0pt}{\isachardot}{\kern0pt}k{\isacharbraceright}{\kern0pt}\ {\isasymsubseteq}\ cube\ n\ {\isacharparenleft}{\kern0pt}t{\isacharplus}{\kern0pt}{\isadigit{1}}{\isacharparenright}{\kern0pt}{\isachardoublequoteclose}\ \isacommand{by}\isamarkupfalse%
\ blast\isanewline
\ \ \isacommand{then}\isamarkupfalse%
\ \isacommand{have}\isamarkupfalse%
\ {\isachardoublequoteopen}{\isasymchi}\ {\isacharbackquote}{\kern0pt}\ S\ {\isacharbackquote}{\kern0pt}\ x\ {\isacharbackquote}{\kern0pt}\ {\isacharbraceleft}{\kern0pt}{\isachardot}{\kern0pt}{\isachardot}{\kern0pt}k{\isacharbraceright}{\kern0pt}\ {\isasymsubseteq}\ {\isasymchi}\ {\isacharbackquote}{\kern0pt}\ cube\ n\ {\isacharparenleft}{\kern0pt}t{\isacharplus}{\kern0pt}{\isadigit{1}}{\isacharparenright}{\kern0pt}{\isachardoublequoteclose}\ \isacommand{by}\isamarkupfalse%
\ auto\isanewline
\ \ \isacommand{then}\isamarkupfalse%
\ \isacommand{have}\isamarkupfalse%
\ {\isachardoublequoteopen}card\ {\isacharparenleft}{\kern0pt}{\isasymchi}\ {\isacharbackquote}{\kern0pt}\ S\ {\isacharbackquote}{\kern0pt}\ x\ {\isacharbackquote}{\kern0pt}\ {\isacharbraceleft}{\kern0pt}{\isachardot}{\kern0pt}{\isachardot}{\kern0pt}k{\isacharbraceright}{\kern0pt}{\isacharparenright}{\kern0pt}\ {\isasymle}\ card\ {\isacharparenleft}{\kern0pt}{\isasymchi}\ {\isacharbackquote}{\kern0pt}\ cube\ n\ {\isacharparenleft}{\kern0pt}t{\isacharplus}{\kern0pt}{\isadigit{1}}{\isacharparenright}{\kern0pt}{\isacharparenright}{\kern0pt}{\isachardoublequoteclose}\ \isanewline
\ \ \ \ \isacommand{by}\isamarkupfalse%
\ {\isacharparenleft}{\kern0pt}simp\ add{\isacharcolon}{\kern0pt}\ card{\isacharunderscore}{\kern0pt}mono\ cube{\isacharunderscore}{\kern0pt}def\ finite{\isacharunderscore}{\kern0pt}PiE{\isacharparenright}{\kern0pt}\isanewline
\ \ \isacommand{also}\isamarkupfalse%
\ \isacommand{have}\isamarkupfalse%
\ {\isachardoublequoteopen}\ {\isachardot}{\kern0pt}{\isachardot}{\kern0pt}{\isachardot}{\kern0pt}\ {\isasymle}\ k{\isachardoublequoteclose}\ \isacommand{using}\isamarkupfalse%
\ {\isacharasterisk}{\kern0pt}\ \isacommand{by}\isamarkupfalse%
\ blast\isanewline
\ \ \isacommand{also}\isamarkupfalse%
\ \isacommand{have}\isamarkupfalse%
\ {\isachardoublequoteopen}\ {\isachardot}{\kern0pt}{\isachardot}{\kern0pt}{\isachardot}{\kern0pt}\ {\isacharless}{\kern0pt}\ k\ {\isacharplus}{\kern0pt}\ {\isadigit{1}}{\isachardoublequoteclose}\ \isacommand{by}\isamarkupfalse%
\ auto\isanewline
\ \ \isacommand{also}\isamarkupfalse%
\ \isacommand{have}\isamarkupfalse%
\ {\isachardoublequoteopen}\ {\isachardot}{\kern0pt}{\isachardot}{\kern0pt}{\isachardot}{\kern0pt}\ {\isacharequal}{\kern0pt}\ card\ {\isacharbraceleft}{\kern0pt}{\isachardot}{\kern0pt}{\isachardot}{\kern0pt}k{\isacharbraceright}{\kern0pt}{\isachardoublequoteclose}\ \isacommand{by}\isamarkupfalse%
\ simp\isanewline
\ \ \isacommand{also}\isamarkupfalse%
\ \isacommand{have}\isamarkupfalse%
\ {\isachardoublequoteopen}\ {\isachardot}{\kern0pt}{\isachardot}{\kern0pt}{\isachardot}{\kern0pt}\ {\isacharequal}{\kern0pt}\ card\ {\isacharparenleft}{\kern0pt}x\ {\isacharbackquote}{\kern0pt}\ {\isacharbraceleft}{\kern0pt}{\isachardot}{\kern0pt}{\isachardot}{\kern0pt}k{\isacharbraceright}{\kern0pt}{\isacharparenright}{\kern0pt}{\isachardoublequoteclose}\ \isacommand{using}\isamarkupfalse%
\ B\ \isacommand{by}\isamarkupfalse%
\ auto\isanewline
\ \ \isacommand{also}\isamarkupfalse%
\ \isacommand{have}\isamarkupfalse%
\ {\isachardoublequoteopen}\ {\isachardot}{\kern0pt}{\isachardot}{\kern0pt}{\isachardot}{\kern0pt}\ {\isacharequal}{\kern0pt}\ card\ {\isacharparenleft}{\kern0pt}S\ {\isacharbackquote}{\kern0pt}\ x\ {\isacharbackquote}{\kern0pt}\ {\isacharbraceleft}{\kern0pt}{\isachardot}{\kern0pt}{\isachardot}{\kern0pt}k{\isacharbraceright}{\kern0pt}{\isacharparenright}{\kern0pt}{\isachardoublequoteclose}\ \isacommand{using}\isamarkupfalse%
\ subspace{\isacharunderscore}{\kern0pt}inj{\isacharunderscore}{\kern0pt}on{\isacharunderscore}{\kern0pt}cube{\isacharbrackleft}{\kern0pt}of\ S\ k\ n\ {\isachardoublequoteopen}t{\isacharplus}{\kern0pt}{\isadigit{1}}{\isachardoublequoteclose}{\isacharbrackright}{\kern0pt}\ card{\isacharunderscore}{\kern0pt}image{\isacharbrackleft}{\kern0pt}of\ S\ {\isachardoublequoteopen}x\ {\isacharbackquote}{\kern0pt}\ {\isacharbraceleft}{\kern0pt}{\isachardot}{\kern0pt}{\isachardot}{\kern0pt}k{\isacharbraceright}{\kern0pt}{\isachardoublequoteclose}{\isacharbrackright}{\kern0pt}\ inj{\isacharunderscore}{\kern0pt}on{\isacharunderscore}{\kern0pt}subset{\isacharbrackleft}{\kern0pt}of\ S\ {\isachardoublequoteopen}cube\ k\ {\isacharparenleft}{\kern0pt}t{\isacharplus}{\kern0pt}{\isadigit{1}}{\isacharparenright}{\kern0pt}{\isachardoublequoteclose}\ {\isachardoublequoteopen}x\ {\isacharbackquote}{\kern0pt}\ {\isacharbraceleft}{\kern0pt}{\isachardot}{\kern0pt}{\isachardot}{\kern0pt}k{\isacharbraceright}{\kern0pt}{\isachardoublequoteclose}{\isacharbrackright}{\kern0pt}\ \ assms{\isacharparenleft}{\kern0pt}{\isadigit{1}}{\isacharparenright}{\kern0pt}\ {\isacartoucheopen}x\ {\isacharbackquote}{\kern0pt}\ {\isacharbraceleft}{\kern0pt}{\isachardot}{\kern0pt}{\isachardot}{\kern0pt}k{\isacharbraceright}{\kern0pt}\ {\isasymsubseteq}\ cube\ k\ {\isacharparenleft}{\kern0pt}t\ {\isacharplus}{\kern0pt}\ {\isadigit{1}}{\isacharparenright}{\kern0pt}{\isacartoucheclose}\ \isacommand{unfolding}\isamarkupfalse%
\ layered{\isacharunderscore}{\kern0pt}subspace{\isacharunderscore}{\kern0pt}def\ \isacommand{by}\isamarkupfalse%
\ simp\isanewline
\ \ \isacommand{finally}\isamarkupfalse%
\ \isacommand{have}\isamarkupfalse%
\ {\isachardoublequoteopen}card\ {\isacharparenleft}{\kern0pt}{\isasymchi}\ {\isacharbackquote}{\kern0pt}\ S\ {\isacharbackquote}{\kern0pt}\ x\ {\isacharbackquote}{\kern0pt}\ {\isacharbraceleft}{\kern0pt}{\isachardot}{\kern0pt}{\isachardot}{\kern0pt}k{\isacharbraceright}{\kern0pt}{\isacharparenright}{\kern0pt}\ {\isacharless}{\kern0pt}\ card\ {\isacharparenleft}{\kern0pt}S\ {\isacharbackquote}{\kern0pt}\ x\ {\isacharbackquote}{\kern0pt}\ {\isacharbraceleft}{\kern0pt}{\isachardot}{\kern0pt}{\isachardot}{\kern0pt}k{\isacharbraceright}{\kern0pt}{\isacharparenright}{\kern0pt}{\isachardoublequoteclose}\ \isacommand{by}\isamarkupfalse%
\ blast\isanewline
\ \ \isacommand{then}\isamarkupfalse%
\ \isacommand{have}\isamarkupfalse%
\ {\isachardoublequoteopen}{\isasymnot}inj{\isacharunderscore}{\kern0pt}on\ {\isasymchi}\ {\isacharparenleft}{\kern0pt}S\ {\isacharbackquote}{\kern0pt}\ x\ {\isacharbackquote}{\kern0pt}\ {\isacharbraceleft}{\kern0pt}{\isachardot}{\kern0pt}{\isachardot}{\kern0pt}k{\isacharbraceright}{\kern0pt}{\isacharparenright}{\kern0pt}{\isachardoublequoteclose}\ \isacommand{using}\isamarkupfalse%
\ pigeonhole{\isacharbrackleft}{\kern0pt}of\ {\isasymchi}\ {\isachardoublequoteopen}S\ {\isacharbackquote}{\kern0pt}\ x\ {\isacharbackquote}{\kern0pt}\ {\isacharbraceleft}{\kern0pt}{\isachardot}{\kern0pt}{\isachardot}{\kern0pt}k{\isacharbraceright}{\kern0pt}{\isachardoublequoteclose}{\isacharbrackright}{\kern0pt}\ \isacommand{by}\isamarkupfalse%
\ blast\isanewline
\ \ \isacommand{then}\isamarkupfalse%
\ \isacommand{have}\isamarkupfalse%
\ {\isachardoublequoteopen}{\isasymexists}a\ b{\isachardot}{\kern0pt}\ a\ {\isasymin}\ S\ {\isacharbackquote}{\kern0pt}\ x\ {\isacharbackquote}{\kern0pt}\ {\isacharbraceleft}{\kern0pt}{\isachardot}{\kern0pt}{\isachardot}{\kern0pt}k{\isacharbraceright}{\kern0pt}\ {\isasymand}\ b\ {\isasymin}\ S\ {\isacharbackquote}{\kern0pt}\ x\ {\isacharbackquote}{\kern0pt}\ {\isacharbraceleft}{\kern0pt}{\isachardot}{\kern0pt}{\isachardot}{\kern0pt}k{\isacharbraceright}{\kern0pt}\ {\isasymand}\ a\ {\isasymnoteq}\ b\ {\isasymand}\ {\isasymchi}\ a\ {\isacharequal}{\kern0pt}\ {\isasymchi}\ b{\isachardoublequoteclose}\ \isacommand{unfolding}\isamarkupfalse%
\ inj{\isacharunderscore}{\kern0pt}on{\isacharunderscore}{\kern0pt}def\ \isacommand{by}\isamarkupfalse%
\ auto\isanewline
\ \ \isacommand{then}\isamarkupfalse%
\ \isacommand{obtain}\isamarkupfalse%
\ ax\ bx\ \isakeyword{where}\ ab{\isacharunderscore}{\kern0pt}props{\isacharcolon}{\kern0pt}\ {\isachardoublequoteopen}ax\ {\isasymin}\ S\ {\isacharbackquote}{\kern0pt}\ x\ {\isacharbackquote}{\kern0pt}\ {\isacharbraceleft}{\kern0pt}{\isachardot}{\kern0pt}{\isachardot}{\kern0pt}k{\isacharbraceright}{\kern0pt}\ {\isasymand}\ bx\ {\isasymin}\ S\ {\isacharbackquote}{\kern0pt}\ x\ {\isacharbackquote}{\kern0pt}\ {\isacharbraceleft}{\kern0pt}{\isachardot}{\kern0pt}{\isachardot}{\kern0pt}k{\isacharbraceright}{\kern0pt}\ {\isasymand}\ ax\ {\isasymnoteq}\ bx\ {\isasymand}\ {\isasymchi}\ ax\ {\isacharequal}{\kern0pt}\ {\isasymchi}\ bx{\isachardoublequoteclose}\ \isacommand{by}\isamarkupfalse%
\ blast\isanewline
\ \ \isacommand{then}\isamarkupfalse%
\ \isacommand{have}\isamarkupfalse%
\ {\isachardoublequoteopen}{\isasymexists}u\ v{\isachardot}{\kern0pt}\ u\ {\isasymin}\ {\isacharbraceleft}{\kern0pt}{\isachardot}{\kern0pt}{\isachardot}{\kern0pt}k{\isacharbraceright}{\kern0pt}\ {\isasymand}\ v\ {\isasymin}\ {\isacharbraceleft}{\kern0pt}{\isachardot}{\kern0pt}{\isachardot}{\kern0pt}k{\isacharbraceright}{\kern0pt}\ {\isasymand}\ u\ {\isasymnoteq}\ v\ {\isasymand}\ {\isasymchi}\ {\isacharparenleft}{\kern0pt}S\ {\isacharparenleft}{\kern0pt}x\ u{\isacharparenright}{\kern0pt}{\isacharparenright}{\kern0pt}\ {\isacharequal}{\kern0pt}\ {\isasymchi}\ {\isacharparenleft}{\kern0pt}S\ {\isacharparenleft}{\kern0pt}x\ v{\isacharparenright}{\kern0pt}{\isacharparenright}{\kern0pt}{\isachardoublequoteclose}\ \isacommand{by}\isamarkupfalse%
\ blast\isanewline
\ \ \isacommand{then}\isamarkupfalse%
\ \isacommand{obtain}\isamarkupfalse%
\ u\ v\ \isakeyword{where}\ uv{\isacharunderscore}{\kern0pt}props{\isacharcolon}{\kern0pt}\ {\isachardoublequoteopen}u\ {\isasymin}\ {\isacharbraceleft}{\kern0pt}{\isachardot}{\kern0pt}{\isachardot}{\kern0pt}k{\isacharbraceright}{\kern0pt}\ {\isasymand}\ v\ {\isasymin}\ {\isacharbraceleft}{\kern0pt}{\isachardot}{\kern0pt}{\isachardot}{\kern0pt}k{\isacharbraceright}{\kern0pt}\ {\isasymand}\ u\ {\isacharless}{\kern0pt}\ v\ {\isasymand}\ {\isasymchi}\ {\isacharparenleft}{\kern0pt}S\ {\isacharparenleft}{\kern0pt}x\ u{\isacharparenright}{\kern0pt}{\isacharparenright}{\kern0pt}\ {\isacharequal}{\kern0pt}\ {\isasymchi}\ {\isacharparenleft}{\kern0pt}S\ {\isacharparenleft}{\kern0pt}x\ v{\isacharparenright}{\kern0pt}{\isacharparenright}{\kern0pt}{\isachardoublequoteclose}\ \isacommand{by}\isamarkupfalse%
\ {\isacharparenleft}{\kern0pt}metis\ linorder{\isacharunderscore}{\kern0pt}cases{\isacharparenright}{\kern0pt}\isanewline
\isanewline
\ \ \isacommand{let}\isamarkupfalse%
\ {\isacharquery}{\kern0pt}f\ {\isacharequal}{\kern0pt}\ {\isachardoublequoteopen}{\isasymlambda}s{\isachardot}{\kern0pt}\ {\isacharparenleft}{\kern0pt}{\isasymlambda}i\ {\isasymin}\ {\isacharbraceleft}{\kern0pt}{\isachardot}{\kern0pt}{\isachardot}{\kern0pt}{\isacharless}{\kern0pt}k{\isacharbraceright}{\kern0pt}{\isachardot}{\kern0pt}\ if\ i\ {\isacharless}{\kern0pt}\ k\ {\isacharminus}{\kern0pt}\ v\ then\ {\isadigit{0}}\ else\ {\isacharparenleft}{\kern0pt}if\ i\ {\isacharless}{\kern0pt}\ k\ {\isacharminus}{\kern0pt}\ u\ then\ s\ else\ t{\isacharparenright}{\kern0pt}{\isacharparenright}{\kern0pt}{\isachardoublequoteclose}\isanewline
\ \ \isacommand{define}\isamarkupfalse%
\ y\ \isakeyword{where}\ {\isachardoublequoteopen}y\ {\isasymequiv}\ {\isacharparenleft}{\kern0pt}{\isasymlambda}s\ {\isasymin}\ {\isacharbraceleft}{\kern0pt}{\isachardot}{\kern0pt}{\isachardot}{\kern0pt}t{\isacharbraceright}{\kern0pt}{\isachardot}{\kern0pt}\ S\ {\isacharparenleft}{\kern0pt}{\isacharquery}{\kern0pt}f\ s{\isacharparenright}{\kern0pt}{\isacharparenright}{\kern0pt}{\isachardoublequoteclose}\isanewline
\isanewline
\ \ \isacommand{have}\isamarkupfalse%
\ line{\isadigit{1}}{\isacharcolon}{\kern0pt}\ {\isachardoublequoteopen}{\isacharquery}{\kern0pt}f\ s\ {\isasymin}\ cube\ k\ {\isacharparenleft}{\kern0pt}t{\isacharplus}{\kern0pt}{\isadigit{1}}{\isacharparenright}{\kern0pt}{\isachardoublequoteclose}\ \isakeyword{if}\ {\isachardoublequoteopen}s\ {\isasymle}\ t{\isachardoublequoteclose}\ \isakeyword{for}\ s\ \isacommand{unfolding}\isamarkupfalse%
\ cube{\isacharunderscore}{\kern0pt}def\ \isacommand{using}\isamarkupfalse%
\ that\ \isacommand{by}\isamarkupfalse%
\ auto\isanewline
\isanewline
\ \ \isacommand{have}\isamarkupfalse%
\ f{\isacharunderscore}{\kern0pt}cube{\isacharcolon}{\kern0pt}\ {\isachardoublequoteopen}{\isacharquery}{\kern0pt}f\ j\ {\isasymin}\ cube\ k\ {\isacharparenleft}{\kern0pt}t{\isacharplus}{\kern0pt}{\isadigit{1}}{\isacharparenright}{\kern0pt}{\isachardoublequoteclose}\ \isakeyword{if}\ {\isachardoublequoteopen}j\ {\isacharless}{\kern0pt}\ t{\isacharplus}{\kern0pt}{\isadigit{1}}{\isachardoublequoteclose}\ \isakeyword{for}\ j\ \isacommand{using}\isamarkupfalse%
\ line{\isadigit{1}}\ that\ \isacommand{by}\isamarkupfalse%
\ simp\isanewline
\ \ \isacommand{have}\isamarkupfalse%
\ f{\isacharunderscore}{\kern0pt}classes{\isacharunderscore}{\kern0pt}u{\isacharcolon}{\kern0pt}\ {\isachardoublequoteopen}{\isacharquery}{\kern0pt}f\ j\ {\isasymin}\ classes\ k\ t\ u{\isachardoublequoteclose}\ \isakeyword{if}\ j{\isacharunderscore}{\kern0pt}prop{\isacharcolon}{\kern0pt}\ {\isachardoublequoteopen}j\ {\isacharless}{\kern0pt}\ t{\isachardoublequoteclose}\ \isakeyword{for}\ j\isanewline
\ \ \ \ \isacommand{using}\isamarkupfalse%
\ that\ j{\isacharunderscore}{\kern0pt}prop\ uv{\isacharunderscore}{\kern0pt}props\ f{\isacharunderscore}{\kern0pt}cube\ \isacommand{unfolding}\isamarkupfalse%
\ classes{\isacharunderscore}{\kern0pt}def\ \isacommand{by}\isamarkupfalse%
\ auto\isanewline
\ \ \isacommand{have}\isamarkupfalse%
\ f{\isacharunderscore}{\kern0pt}classes{\isacharunderscore}{\kern0pt}v{\isacharcolon}{\kern0pt}\ {\isachardoublequoteopen}{\isacharquery}{\kern0pt}f\ j\ {\isasymin}\ classes\ k\ t\ v{\isachardoublequoteclose}\ \isakeyword{if}\ j{\isacharunderscore}{\kern0pt}prop{\isacharcolon}{\kern0pt}\ {\isachardoublequoteopen}j\ {\isacharequal}{\kern0pt}\ t{\isachardoublequoteclose}\ \isakeyword{for}\ j\isanewline
\ \ \ \ \isacommand{using}\isamarkupfalse%
\ that\ j{\isacharunderscore}{\kern0pt}prop\ uv{\isacharunderscore}{\kern0pt}props\ assms{\isacharparenleft}{\kern0pt}{\isadigit{2}}{\isacharparenright}{\kern0pt}\ f{\isacharunderscore}{\kern0pt}cube\ \isacommand{unfolding}\isamarkupfalse%
\ classes{\isacharunderscore}{\kern0pt}def\ \isacommand{by}\isamarkupfalse%
\ auto\isanewline
\isanewline
\ \ \isacommand{obtain}\isamarkupfalse%
\ B\ f\ \isakeyword{where}\ Bf{\isacharunderscore}{\kern0pt}props{\isacharcolon}{\kern0pt}\ {\isachardoublequoteopen}disjoint{\isacharunderscore}{\kern0pt}family{\isacharunderscore}{\kern0pt}on\ B\ {\isacharbraceleft}{\kern0pt}{\isachardot}{\kern0pt}{\isachardot}{\kern0pt}k{\isacharbraceright}{\kern0pt}{\isachardoublequoteclose}\ {\isachardoublequoteopen}{\isasymUnion}{\isacharparenleft}{\kern0pt}B\ {\isacharbackquote}{\kern0pt}\ {\isacharbraceleft}{\kern0pt}{\isachardot}{\kern0pt}{\isachardot}{\kern0pt}k{\isacharbraceright}{\kern0pt}{\isacharparenright}{\kern0pt}\ {\isacharequal}{\kern0pt}\ {\isacharbraceleft}{\kern0pt}{\isachardot}{\kern0pt}{\isachardot}{\kern0pt}{\isacharless}{\kern0pt}n{\isacharbraceright}{\kern0pt}{\isachardoublequoteclose}\ {\isachardoublequoteopen}{\isacharparenleft}{\kern0pt}{\isacharbraceleft}{\kern0pt}{\isacharbraceright}{\kern0pt}\ {\isasymnotin}\ B\ {\isacharbackquote}{\kern0pt}\ {\isacharbraceleft}{\kern0pt}{\isachardot}{\kern0pt}{\isachardot}{\kern0pt}{\isacharless}{\kern0pt}k{\isacharbraceright}{\kern0pt}{\isacharparenright}{\kern0pt}{\isachardoublequoteclose}\ {\isachardoublequoteopen}f\ {\isasymin}\ {\isacharparenleft}{\kern0pt}B\ k{\isacharparenright}{\kern0pt}\ {\isasymrightarrow}\isactrlsub E\ {\isacharbraceleft}{\kern0pt}{\isachardot}{\kern0pt}{\isachardot}{\kern0pt}{\isacharless}{\kern0pt}t{\isacharplus}{\kern0pt}{\isadigit{1}}{\isacharbraceright}{\kern0pt}{\isachardoublequoteclose}\ {\isachardoublequoteopen}S\ {\isasymin}\ {\isacharparenleft}{\kern0pt}cube\ k\ {\isacharparenleft}{\kern0pt}t{\isacharplus}{\kern0pt}{\isadigit{1}}{\isacharparenright}{\kern0pt}{\isacharparenright}{\kern0pt}\ {\isasymrightarrow}\isactrlsub E\ {\isacharparenleft}{\kern0pt}cube\ n\ {\isacharparenleft}{\kern0pt}t{\isacharplus}{\kern0pt}{\isadigit{1}}{\isacharparenright}{\kern0pt}{\isacharparenright}{\kern0pt}{\isachardoublequoteclose}\ {\isachardoublequoteopen}{\isacharparenleft}{\kern0pt}{\isasymforall}y\ {\isasymin}\ cube\ k\ {\isacharparenleft}{\kern0pt}t{\isacharplus}{\kern0pt}{\isadigit{1}}{\isacharparenright}{\kern0pt}{\isachardot}{\kern0pt}\ {\isacharparenleft}{\kern0pt}{\isasymforall}i\ {\isasymin}\ B\ k{\isachardot}{\kern0pt}\ S\ y\ i\ {\isacharequal}{\kern0pt}\ f\ i{\isacharparenright}{\kern0pt}\ {\isasymand}\ {\isacharparenleft}{\kern0pt}{\isasymforall}j{\isacharless}{\kern0pt}k{\isachardot}{\kern0pt}\ {\isasymforall}i\ {\isasymin}\ B\ j{\isachardot}{\kern0pt}\ {\isacharparenleft}{\kern0pt}S\ y{\isacharparenright}{\kern0pt}\ i\ {\isacharequal}{\kern0pt}\ y\ j{\isacharparenright}{\kern0pt}{\isacharparenright}{\kern0pt}{\isachardoublequoteclose}\ \isacommand{using}\isamarkupfalse%
\ assms{\isacharparenleft}{\kern0pt}{\isadigit{1}}{\isacharparenright}{\kern0pt}\ \isacommand{unfolding}\isamarkupfalse%
\ layered{\isacharunderscore}{\kern0pt}subspace{\isacharunderscore}{\kern0pt}def\ is{\isacharunderscore}{\kern0pt}subspace{\isacharunderscore}{\kern0pt}def\ \isacommand{by}\isamarkupfalse%
\ auto\isanewline
\isanewline
\ \ \isacommand{have}\isamarkupfalse%
\ {\isachardoublequoteopen}y\ {\isasymin}\ {\isacharbraceleft}{\kern0pt}{\isachardot}{\kern0pt}{\isachardot}{\kern0pt}{\isacharless}{\kern0pt}t{\isacharplus}{\kern0pt}{\isadigit{1}}{\isacharbraceright}{\kern0pt}\ {\isasymrightarrow}\isactrlsub E\ cube\ n\ {\isacharparenleft}{\kern0pt}t{\isacharplus}{\kern0pt}{\isadigit{1}}{\isacharparenright}{\kern0pt}{\isachardoublequoteclose}\ \isacommand{unfolding}\isamarkupfalse%
\ y{\isacharunderscore}{\kern0pt}def\ \isacommand{using}\isamarkupfalse%
\ line{\isadigit{1}}\ {\isacartoucheopen}S\ {\isacharbackquote}{\kern0pt}\ cube\ k\ {\isacharparenleft}{\kern0pt}t\ {\isacharplus}{\kern0pt}\ {\isadigit{1}}{\isacharparenright}{\kern0pt}\ {\isasymsubseteq}\ cube\ n\ {\isacharparenleft}{\kern0pt}t\ {\isacharplus}{\kern0pt}\ {\isadigit{1}}{\isacharparenright}{\kern0pt}{\isacartoucheclose}\ \isacommand{by}\isamarkupfalse%
\ auto\isanewline
\ \ \isacommand{moreover}\isamarkupfalse%
\ \isacommand{have}\isamarkupfalse%
\ {\isachardoublequoteopen}{\isacharparenleft}{\kern0pt}{\isasymforall}u{\isacharless}{\kern0pt}t{\isacharplus}{\kern0pt}{\isadigit{1}}{\isachardot}{\kern0pt}\ {\isasymforall}v{\isacharless}{\kern0pt}t{\isacharplus}{\kern0pt}{\isadigit{1}}{\isachardot}{\kern0pt}\ y\ u\ j\ {\isacharequal}{\kern0pt}\ y\ v\ j{\isacharparenright}{\kern0pt}\ {\isasymor}\ {\isacharparenleft}{\kern0pt}{\isasymforall}s{\isacharless}{\kern0pt}t{\isacharplus}{\kern0pt}{\isadigit{1}}{\isachardot}{\kern0pt}\ y\ s\ j\ {\isacharequal}{\kern0pt}\ s{\isacharparenright}{\kern0pt}{\isachardoublequoteclose}\ \isakeyword{if}\ j{\isacharunderscore}{\kern0pt}prop{\isacharcolon}{\kern0pt}\ {\isachardoublequoteopen}j{\isacharless}{\kern0pt}n{\isachardoublequoteclose}\ \isakeyword{for}\ j\ \isanewline
\ \ \isacommand{proof}\isamarkupfalse%
{\isacharminus}{\kern0pt}\isanewline
\ \ \ \ \isacommand{show}\isamarkupfalse%
\ {\isachardoublequoteopen}{\isacharparenleft}{\kern0pt}{\isasymforall}u{\isacharless}{\kern0pt}t{\isacharplus}{\kern0pt}{\isadigit{1}}{\isachardot}{\kern0pt}\ {\isasymforall}v{\isacharless}{\kern0pt}t{\isacharplus}{\kern0pt}{\isadigit{1}}{\isachardot}{\kern0pt}\ y\ u\ j\ {\isacharequal}{\kern0pt}\ y\ v\ j{\isacharparenright}{\kern0pt}\ {\isasymor}\ {\isacharparenleft}{\kern0pt}{\isasymforall}s{\isacharless}{\kern0pt}t{\isacharplus}{\kern0pt}{\isadigit{1}}{\isachardot}{\kern0pt}\ y\ s\ j\ {\isacharequal}{\kern0pt}\ s{\isacharparenright}{\kern0pt}{\isachardoublequoteclose}\isanewline
\ \ \ \ \isacommand{proof}\isamarkupfalse%
\ {\isacharminus}{\kern0pt}\isanewline
\ \ \ \ \ \ \isacommand{consider}\isamarkupfalse%
\ {\isachardoublequoteopen}j\ {\isasymin}\ B\ k{\isachardoublequoteclose}\ {\isacharbar}{\kern0pt}\ {\isachardoublequoteopen}{\isasymexists}ii{\isacharless}{\kern0pt}k{\isachardot}{\kern0pt}\ j\ {\isasymin}\ B\ ii{\isachardoublequoteclose}\ \isacommand{using}\isamarkupfalse%
\ Bf{\isacharunderscore}{\kern0pt}props{\isacharparenleft}{\kern0pt}{\isadigit{2}}{\isacharparenright}{\kern0pt}\ j{\isacharunderscore}{\kern0pt}prop\ \isanewline
\ \ \ \ \ \ \ \ \isacommand{by}\isamarkupfalse%
\ {\isacharparenleft}{\kern0pt}metis\ UN{\isacharunderscore}{\kern0pt}E\ atMost{\isacharunderscore}{\kern0pt}iff\ le{\isacharunderscore}{\kern0pt}neq{\isacharunderscore}{\kern0pt}implies{\isacharunderscore}{\kern0pt}less\ lessThan{\isacharunderscore}{\kern0pt}iff{\isacharparenright}{\kern0pt}\isanewline
\ \ \ \ \ \ \isacommand{then}\isamarkupfalse%
\ \isacommand{have}\isamarkupfalse%
\ {\isachardoublequoteopen}y\ a\ j\ {\isacharequal}{\kern0pt}\ y\ b\ j\ {\isasymor}\ y\ s\ j\ {\isacharequal}{\kern0pt}\ s{\isachardoublequoteclose}\ \isakeyword{if}\ {\isachardoublequoteopen}a\ {\isacharless}{\kern0pt}\ t\ {\isacharplus}{\kern0pt}\ {\isadigit{1}}{\isachardoublequoteclose}\ {\isachardoublequoteopen}b\ {\isacharless}{\kern0pt}\ t\ {\isacharplus}{\kern0pt}{\isadigit{1}}{\isachardoublequoteclose}\ {\isachardoublequoteopen}s\ {\isacharless}{\kern0pt}\ t\ {\isacharplus}{\kern0pt}{\isadigit{1}}{\isachardoublequoteclose}\ \isakeyword{for}\ a\ b\ s\isanewline
\ \ \ \ \ \ \isacommand{proof}\isamarkupfalse%
\ cases\isanewline
\ \ \ \ \ \ \ \ \isacommand{case}\isamarkupfalse%
\ {\isadigit{1}}\isanewline
\ \ \ \ \ \ \ \ \isacommand{then}\isamarkupfalse%
\ \isacommand{have}\isamarkupfalse%
\ {\isachardoublequoteopen}y\ a\ j\ {\isacharequal}{\kern0pt}\ S\ {\isacharparenleft}{\kern0pt}{\isacharquery}{\kern0pt}f\ a{\isacharparenright}{\kern0pt}\ j{\isachardoublequoteclose}\ \isacommand{using}\isamarkupfalse%
\ that{\isacharparenleft}{\kern0pt}{\isadigit{1}}{\isacharparenright}{\kern0pt}\ \isacommand{unfolding}\isamarkupfalse%
\ y{\isacharunderscore}{\kern0pt}def\ \isacommand{by}\isamarkupfalse%
\ auto\isanewline
\ \ \ \ \ \ \ \ \isacommand{also}\isamarkupfalse%
\ \isacommand{have}\isamarkupfalse%
\ {\isachardoublequoteopen}\ {\isachardot}{\kern0pt}{\isachardot}{\kern0pt}{\isachardot}{\kern0pt}\ {\isacharequal}{\kern0pt}\ f\ j{\isachardoublequoteclose}\ \isacommand{using}\isamarkupfalse%
\ Bf{\isacharunderscore}{\kern0pt}props{\isacharparenleft}{\kern0pt}{\isadigit{6}}{\isacharparenright}{\kern0pt}\ f{\isacharunderscore}{\kern0pt}cube\ {\isadigit{1}}\ that{\isacharparenleft}{\kern0pt}{\isadigit{1}}{\isacharparenright}{\kern0pt}\ \isacommand{by}\isamarkupfalse%
\ auto\isanewline
\ \ \ \ \ \ \ \ \isacommand{also}\isamarkupfalse%
\ \isacommand{have}\isamarkupfalse%
\ {\isachardoublequoteopen}\ {\isachardot}{\kern0pt}{\isachardot}{\kern0pt}{\isachardot}{\kern0pt}\ {\isacharequal}{\kern0pt}\ S\ {\isacharparenleft}{\kern0pt}{\isacharquery}{\kern0pt}f\ b{\isacharparenright}{\kern0pt}\ j{\isachardoublequoteclose}\ \isacommand{using}\isamarkupfalse%
\ Bf{\isacharunderscore}{\kern0pt}props{\isacharparenleft}{\kern0pt}{\isadigit{6}}{\isacharparenright}{\kern0pt}\ f{\isacharunderscore}{\kern0pt}cube\ {\isadigit{1}}\ that{\isacharparenleft}{\kern0pt}{\isadigit{2}}{\isacharparenright}{\kern0pt}\ \isacommand{by}\isamarkupfalse%
\ auto\isanewline
\ \ \ \ \ \ \ \ \isacommand{also}\isamarkupfalse%
\ \isacommand{have}\isamarkupfalse%
\ {\isachardoublequoteopen}\ {\isachardot}{\kern0pt}{\isachardot}{\kern0pt}{\isachardot}{\kern0pt}\ {\isacharequal}{\kern0pt}\ y\ b\ j{\isachardoublequoteclose}\ \isacommand{using}\isamarkupfalse%
\ that{\isacharparenleft}{\kern0pt}{\isadigit{2}}{\isacharparenright}{\kern0pt}\ \isacommand{unfolding}\isamarkupfalse%
\ y{\isacharunderscore}{\kern0pt}def\ \isacommand{by}\isamarkupfalse%
\ simp\isanewline
\ \ \ \ \ \ \ \ \isacommand{finally}\isamarkupfalse%
\ \isacommand{show}\isamarkupfalse%
\ {\isacharquery}{\kern0pt}thesis\ \isacommand{by}\isamarkupfalse%
\ simp\isanewline
\ \ \ \ \ \ \isacommand{next}\isamarkupfalse%
\isanewline
\ \ \ \ \ \ \ \ \isacommand{case}\isamarkupfalse%
\ {\isadigit{2}}\isanewline
\ \ \ \ \ \ \ \ \isacommand{then}\isamarkupfalse%
\ \isacommand{obtain}\isamarkupfalse%
\ ii\ \isakeyword{where}\ ii{\isacharunderscore}{\kern0pt}prop{\isacharcolon}{\kern0pt}{\isachardoublequoteopen}\ ii\ {\isacharless}{\kern0pt}\ k\ {\isasymand}\ j\ {\isasymin}\ B\ ii{\isachardoublequoteclose}\ \isacommand{by}\isamarkupfalse%
\ blast\isanewline
\ \ \ \ \ \ \ \ \isacommand{then}\isamarkupfalse%
\ \isacommand{consider}\isamarkupfalse%
\ {\isachardoublequoteopen}ii\ {\isacharless}{\kern0pt}\ k\ {\isacharminus}{\kern0pt}\ v{\isachardoublequoteclose}\ {\isacharbar}{\kern0pt}\ {\isachardoublequoteopen}ii\ {\isasymge}\ k\ {\isacharminus}{\kern0pt}\ v\ {\isasymand}\ ii\ {\isacharless}{\kern0pt}\ k\ {\isacharminus}{\kern0pt}\ u{\isachardoublequoteclose}\ {\isacharbar}{\kern0pt}\ {\isachardoublequoteopen}ii\ {\isasymge}\ k\ {\isacharminus}{\kern0pt}\ u\ {\isasymand}\ ii\ {\isacharless}{\kern0pt}\ k{\isachardoublequoteclose}\ \isacommand{using}\isamarkupfalse%
\ not{\isacharunderscore}{\kern0pt}less\ \ \isacommand{by}\isamarkupfalse%
\ blast\isanewline
\ \ \ \ \ \ \ \ \isacommand{then}\isamarkupfalse%
\ \isacommand{show}\isamarkupfalse%
\ {\isacharquery}{\kern0pt}thesis\isanewline
\ \ \ \ \ \ \ \ \isacommand{proof}\isamarkupfalse%
\ cases\isanewline
\ \ \ \ \ \ \ \ \ \ \isacommand{case}\isamarkupfalse%
\ {\isadigit{1}}\isanewline
\ \ \ \ \ \ \ \ \ \ \isacommand{then}\isamarkupfalse%
\ \isacommand{have}\isamarkupfalse%
\ {\isachardoublequoteopen}y\ a\ j\ {\isacharequal}{\kern0pt}\ S\ {\isacharparenleft}{\kern0pt}{\isacharquery}{\kern0pt}f\ a{\isacharparenright}{\kern0pt}\ j{\isachardoublequoteclose}\ \isacommand{using}\isamarkupfalse%
\ that{\isacharparenleft}{\kern0pt}{\isadigit{1}}{\isacharparenright}{\kern0pt}\ \isacommand{unfolding}\isamarkupfalse%
\ y{\isacharunderscore}{\kern0pt}def\ \isacommand{by}\isamarkupfalse%
\ auto\isanewline
\ \ \ \ \ \ \ \ \ \ \isacommand{also}\isamarkupfalse%
\ \isacommand{have}\isamarkupfalse%
\ {\isachardoublequoteopen}\ {\isachardot}{\kern0pt}{\isachardot}{\kern0pt}{\isachardot}{\kern0pt}\ {\isacharequal}{\kern0pt}\ {\isacharparenleft}{\kern0pt}{\isacharquery}{\kern0pt}f\ a{\isacharparenright}{\kern0pt}\ ii{\isachardoublequoteclose}\ \isacommand{using}\isamarkupfalse%
\ Bf{\isacharunderscore}{\kern0pt}props{\isacharparenleft}{\kern0pt}{\isadigit{6}}{\isacharparenright}{\kern0pt}\ f{\isacharunderscore}{\kern0pt}cube\ that{\isacharparenleft}{\kern0pt}{\isadigit{1}}{\isacharparenright}{\kern0pt}\ ii{\isacharunderscore}{\kern0pt}prop\ \isacommand{by}\isamarkupfalse%
\ auto\isanewline
\ \ \ \ \ \ \ \ \ \ \isacommand{also}\isamarkupfalse%
\ \isacommand{have}\isamarkupfalse%
\ {\isachardoublequoteopen}\ {\isachardot}{\kern0pt}{\isachardot}{\kern0pt}{\isachardot}{\kern0pt}\ {\isacharequal}{\kern0pt}\ {\isadigit{0}}{\isachardoublequoteclose}\ \isacommand{using}\isamarkupfalse%
\ {\isadigit{1}}\ \isacommand{by}\isamarkupfalse%
\ {\isacharparenleft}{\kern0pt}simp\ add{\isacharcolon}{\kern0pt}\ ii{\isacharunderscore}{\kern0pt}prop{\isacharparenright}{\kern0pt}\isanewline
\ \ \ \ \ \ \ \ \ \ \isacommand{also}\isamarkupfalse%
\ \isacommand{have}\isamarkupfalse%
\ {\isachardoublequoteopen}\ {\isachardot}{\kern0pt}{\isachardot}{\kern0pt}{\isachardot}{\kern0pt}\ {\isacharequal}{\kern0pt}\ {\isacharparenleft}{\kern0pt}{\isacharquery}{\kern0pt}f\ b{\isacharparenright}{\kern0pt}\ ii{\isachardoublequoteclose}\ \isacommand{using}\isamarkupfalse%
\ {\isadigit{1}}\ \isacommand{by}\isamarkupfalse%
\ {\isacharparenleft}{\kern0pt}simp\ add{\isacharcolon}{\kern0pt}\ ii{\isacharunderscore}{\kern0pt}prop{\isacharparenright}{\kern0pt}\isanewline
\ \ \ \ \ \ \ \ \ \ \isacommand{also}\isamarkupfalse%
\ \isacommand{have}\isamarkupfalse%
\ {\isachardoublequoteopen}\ {\isachardot}{\kern0pt}{\isachardot}{\kern0pt}{\isachardot}{\kern0pt}\ {\isacharequal}{\kern0pt}\ S\ {\isacharparenleft}{\kern0pt}{\isacharquery}{\kern0pt}f\ b{\isacharparenright}{\kern0pt}\ j{\isachardoublequoteclose}\ \isacommand{using}\isamarkupfalse%
\ Bf{\isacharunderscore}{\kern0pt}props{\isacharparenleft}{\kern0pt}{\isadigit{6}}{\isacharparenright}{\kern0pt}\ f{\isacharunderscore}{\kern0pt}cube\ that{\isacharparenleft}{\kern0pt}{\isadigit{2}}{\isacharparenright}{\kern0pt}\ ii{\isacharunderscore}{\kern0pt}prop\ \isacommand{by}\isamarkupfalse%
\ auto\isanewline
\ \ \ \ \ \ \ \ \ \ \isacommand{also}\isamarkupfalse%
\ \isacommand{have}\isamarkupfalse%
\ {\isachardoublequoteopen}\ {\isachardot}{\kern0pt}{\isachardot}{\kern0pt}{\isachardot}{\kern0pt}\ {\isacharequal}{\kern0pt}\ y\ b\ j{\isachardoublequoteclose}\ \isacommand{using}\isamarkupfalse%
\ that{\isacharparenleft}{\kern0pt}{\isadigit{2}}{\isacharparenright}{\kern0pt}\ \isacommand{unfolding}\isamarkupfalse%
\ y{\isacharunderscore}{\kern0pt}def\ \isacommand{by}\isamarkupfalse%
\ auto\isanewline
\ \ \ \ \ \ \ \ \ \ \isacommand{finally}\isamarkupfalse%
\ \isacommand{show}\isamarkupfalse%
\ {\isacharquery}{\kern0pt}thesis\ \isacommand{by}\isamarkupfalse%
\ simp\isanewline
\ \ \ \ \ \ \ \ \isacommand{next}\isamarkupfalse%
\isanewline
\ \ \ \ \ \ \ \ \ \ \isacommand{case}\isamarkupfalse%
\ {\isadigit{2}}\isanewline
\ \ \ \ \ \ \ \ \ \ \isacommand{then}\isamarkupfalse%
\ \isacommand{have}\isamarkupfalse%
\ {\isachardoublequoteopen}y\ s\ j\ {\isacharequal}{\kern0pt}\ S\ {\isacharparenleft}{\kern0pt}{\isacharquery}{\kern0pt}f\ s{\isacharparenright}{\kern0pt}\ j{\isachardoublequoteclose}\ \isacommand{using}\isamarkupfalse%
\ that{\isacharparenleft}{\kern0pt}{\isadigit{3}}{\isacharparenright}{\kern0pt}\ \isacommand{unfolding}\isamarkupfalse%
\ y{\isacharunderscore}{\kern0pt}def\ \isacommand{by}\isamarkupfalse%
\ auto\isanewline
\ \ \ \ \ \ \ \ \ \ \isacommand{also}\isamarkupfalse%
\ \isacommand{have}\isamarkupfalse%
\ {\isachardoublequoteopen}\ {\isachardot}{\kern0pt}{\isachardot}{\kern0pt}{\isachardot}{\kern0pt}\ {\isacharequal}{\kern0pt}\ {\isacharparenleft}{\kern0pt}{\isacharquery}{\kern0pt}f\ s{\isacharparenright}{\kern0pt}\ ii{\isachardoublequoteclose}\ \isacommand{using}\isamarkupfalse%
\ Bf{\isacharunderscore}{\kern0pt}props{\isacharparenleft}{\kern0pt}{\isadigit{6}}{\isacharparenright}{\kern0pt}\ f{\isacharunderscore}{\kern0pt}cube\ that{\isacharparenleft}{\kern0pt}{\isadigit{3}}{\isacharparenright}{\kern0pt}\ ii{\isacharunderscore}{\kern0pt}prop\ \isacommand{by}\isamarkupfalse%
\ auto\isanewline
\ \ \ \ \ \ \ \ \ \ \isacommand{also}\isamarkupfalse%
\ \isacommand{have}\isamarkupfalse%
\ {\isachardoublequoteopen}\ {\isachardot}{\kern0pt}{\isachardot}{\kern0pt}{\isachardot}{\kern0pt}\ {\isacharequal}{\kern0pt}\ s{\isachardoublequoteclose}\ \isacommand{using}\isamarkupfalse%
\ {\isadigit{2}}\ \isacommand{by}\isamarkupfalse%
\ {\isacharparenleft}{\kern0pt}simp\ add{\isacharcolon}{\kern0pt}\ ii{\isacharunderscore}{\kern0pt}prop{\isacharparenright}{\kern0pt}\isanewline
\ \ \ \ \ \ \ \ \ \ \isacommand{finally}\isamarkupfalse%
\ \isacommand{show}\isamarkupfalse%
\ {\isacharquery}{\kern0pt}thesis\ \isacommand{by}\isamarkupfalse%
\ simp\isanewline
\ \ \ \ \ \ \ \ \isacommand{next}\isamarkupfalse%
\isanewline
\ \ \ \ \ \ \ \ \ \ \isacommand{case}\isamarkupfalse%
\ {\isadigit{3}}\isanewline
\ \ \ \ \ \ \ \ \ \ \isacommand{then}\isamarkupfalse%
\ \isacommand{have}\isamarkupfalse%
\ {\isachardoublequoteopen}y\ a\ j\ {\isacharequal}{\kern0pt}\ S\ {\isacharparenleft}{\kern0pt}{\isacharquery}{\kern0pt}f\ a{\isacharparenright}{\kern0pt}\ j{\isachardoublequoteclose}\ \isacommand{using}\isamarkupfalse%
\ that{\isacharparenleft}{\kern0pt}{\isadigit{1}}{\isacharparenright}{\kern0pt}\ \isacommand{unfolding}\isamarkupfalse%
\ y{\isacharunderscore}{\kern0pt}def\ \isacommand{by}\isamarkupfalse%
\ auto\isanewline
\ \ \ \ \ \ \ \ \ \ \isacommand{also}\isamarkupfalse%
\ \isacommand{have}\isamarkupfalse%
\ {\isachardoublequoteopen}\ {\isachardot}{\kern0pt}{\isachardot}{\kern0pt}{\isachardot}{\kern0pt}\ {\isacharequal}{\kern0pt}\ {\isacharparenleft}{\kern0pt}{\isacharquery}{\kern0pt}f\ a{\isacharparenright}{\kern0pt}\ ii{\isachardoublequoteclose}\ \isacommand{using}\isamarkupfalse%
\ Bf{\isacharunderscore}{\kern0pt}props{\isacharparenleft}{\kern0pt}{\isadigit{6}}{\isacharparenright}{\kern0pt}\ f{\isacharunderscore}{\kern0pt}cube\ that{\isacharparenleft}{\kern0pt}{\isadigit{1}}{\isacharparenright}{\kern0pt}\ ii{\isacharunderscore}{\kern0pt}prop\ \isacommand{by}\isamarkupfalse%
\ auto\isanewline
\ \ \ \ \ \ \ \ \ \ \isacommand{also}\isamarkupfalse%
\ \isacommand{have}\isamarkupfalse%
\ {\isachardoublequoteopen}\ {\isachardot}{\kern0pt}{\isachardot}{\kern0pt}{\isachardot}{\kern0pt}\ {\isacharequal}{\kern0pt}\ t{\isachardoublequoteclose}\ \isacommand{using}\isamarkupfalse%
\ {\isadigit{3}}\ uv{\isacharunderscore}{\kern0pt}props\ \isacommand{by}\isamarkupfalse%
\ auto\isanewline
\ \ \ \ \ \ \ \ \ \ \isacommand{also}\isamarkupfalse%
\ \isacommand{have}\isamarkupfalse%
\ {\isachardoublequoteopen}\ {\isachardot}{\kern0pt}{\isachardot}{\kern0pt}{\isachardot}{\kern0pt}\ {\isacharequal}{\kern0pt}\ {\isacharparenleft}{\kern0pt}{\isacharquery}{\kern0pt}f\ b{\isacharparenright}{\kern0pt}\ ii{\isachardoublequoteclose}\ \isacommand{using}\isamarkupfalse%
\ {\isadigit{3}}\ uv{\isacharunderscore}{\kern0pt}props\ \isacommand{by}\isamarkupfalse%
\ auto\isanewline
\ \ \ \ \ \ \ \ \ \ \isacommand{also}\isamarkupfalse%
\ \isacommand{have}\isamarkupfalse%
\ {\isachardoublequoteopen}\ {\isachardot}{\kern0pt}{\isachardot}{\kern0pt}{\isachardot}{\kern0pt}\ {\isacharequal}{\kern0pt}\ S\ {\isacharparenleft}{\kern0pt}{\isacharquery}{\kern0pt}f\ b{\isacharparenright}{\kern0pt}\ j{\isachardoublequoteclose}\ \isacommand{using}\isamarkupfalse%
\ Bf{\isacharunderscore}{\kern0pt}props{\isacharparenleft}{\kern0pt}{\isadigit{6}}{\isacharparenright}{\kern0pt}\ f{\isacharunderscore}{\kern0pt}cube\ that{\isacharparenleft}{\kern0pt}{\isadigit{2}}{\isacharparenright}{\kern0pt}\ ii{\isacharunderscore}{\kern0pt}prop\ \isacommand{by}\isamarkupfalse%
\ auto\isanewline
\ \ \ \ \ \ \ \ \ \ \isacommand{also}\isamarkupfalse%
\ \isacommand{have}\isamarkupfalse%
\ {\isachardoublequoteopen}\ {\isachardot}{\kern0pt}{\isachardot}{\kern0pt}{\isachardot}{\kern0pt}\ {\isacharequal}{\kern0pt}\ y\ b\ j{\isachardoublequoteclose}\ \isacommand{using}\isamarkupfalse%
\ that{\isacharparenleft}{\kern0pt}{\isadigit{2}}{\isacharparenright}{\kern0pt}\ \isacommand{unfolding}\isamarkupfalse%
\ y{\isacharunderscore}{\kern0pt}def\ \isacommand{by}\isamarkupfalse%
\ auto\isanewline
\ \ \ \ \ \ \ \ \ \ \isacommand{finally}\isamarkupfalse%
\ \isacommand{show}\isamarkupfalse%
\ {\isacharquery}{\kern0pt}thesis\ \isacommand{by}\isamarkupfalse%
\ simp\isanewline
\ \ \ \ \ \ \ \ \isacommand{qed}\isamarkupfalse%
\isanewline
\ \ \ \ \ \ \isacommand{qed}\isamarkupfalse%
\isanewline
\ \ \ \ \ \ \isacommand{then}\isamarkupfalse%
\ \isacommand{show}\isamarkupfalse%
\ {\isacharquery}{\kern0pt}thesis\ \isacommand{by}\isamarkupfalse%
\ blast\isanewline
\ \ \ \ \isacommand{qed}\isamarkupfalse%
\isanewline
\ \ \isacommand{qed}\isamarkupfalse%
\isanewline
\ \ \isacommand{moreover}\isamarkupfalse%
\ \isacommand{have}\isamarkupfalse%
\ {\isachardoublequoteopen}{\isasymexists}j\ {\isacharless}{\kern0pt}\ n{\isachardot}{\kern0pt}\ {\isasymforall}s{\isacharless}{\kern0pt}t{\isacharplus}{\kern0pt}{\isadigit{1}}{\isachardot}{\kern0pt}\ y\ s\ j\ {\isacharequal}{\kern0pt}\ s{\isachardoublequoteclose}\isanewline
\ \ \isacommand{proof}\isamarkupfalse%
\ {\isacharminus}{\kern0pt}\isanewline
\ \ \ \ \isacommand{have}\isamarkupfalse%
\ {\isachardoublequoteopen}k\ {\isachargreater}{\kern0pt}\ {\isadigit{0}}{\isachardoublequoteclose}\ \isacommand{using}\isamarkupfalse%
\ uv{\isacharunderscore}{\kern0pt}props\ \isacommand{by}\isamarkupfalse%
\ simp\isanewline
\ \ \ \ \isacommand{have}\isamarkupfalse%
\ {\isachardoublequoteopen}k\ {\isacharminus}{\kern0pt}\ v\ {\isacharless}{\kern0pt}\ k{\isachardoublequoteclose}\ \isacommand{using}\isamarkupfalse%
\ uv{\isacharunderscore}{\kern0pt}props\ \isacommand{by}\isamarkupfalse%
\ auto\isanewline
\ \ \ \ \isacommand{have}\isamarkupfalse%
\ {\isachardoublequoteopen}k\ {\isacharminus}{\kern0pt}\ v\ {\isacharless}{\kern0pt}\ k\ {\isacharminus}{\kern0pt}\ u{\isachardoublequoteclose}\ \isacommand{using}\isamarkupfalse%
\ uv{\isacharunderscore}{\kern0pt}props\ \isacommand{by}\isamarkupfalse%
\ auto\isanewline
\ \ \ \ \isacommand{then}\isamarkupfalse%
\ \isacommand{have}\isamarkupfalse%
\ {\isachardoublequoteopen}B\ {\isacharparenleft}{\kern0pt}k\ {\isacharminus}{\kern0pt}\ v{\isacharparenright}{\kern0pt}\ {\isasymnoteq}\ {\isacharbraceleft}{\kern0pt}{\isacharbraceright}{\kern0pt}{\isachardoublequoteclose}\ \isacommand{using}\isamarkupfalse%
\ Bf{\isacharunderscore}{\kern0pt}props{\isacharparenleft}{\kern0pt}{\isadigit{3}}{\isacharparenright}{\kern0pt}\ uv{\isacharunderscore}{\kern0pt}props\ \isacommand{by}\isamarkupfalse%
\ auto\isanewline
\ \ \ \ \isacommand{then}\isamarkupfalse%
\ \isacommand{obtain}\isamarkupfalse%
\ j\ \isakeyword{where}\ j{\isacharunderscore}{\kern0pt}prop{\isacharcolon}{\kern0pt}\ {\isachardoublequoteopen}j\ {\isasymin}\ B\ {\isacharparenleft}{\kern0pt}k\ {\isacharminus}{\kern0pt}\ v{\isacharparenright}{\kern0pt}\ {\isasymand}\ j\ {\isacharless}{\kern0pt}\ n{\isachardoublequoteclose}\ \isacommand{using}\isamarkupfalse%
\ Bf{\isacharunderscore}{\kern0pt}props{\isacharparenleft}{\kern0pt}{\isadigit{2}}{\isacharparenright}{\kern0pt}\ uv{\isacharunderscore}{\kern0pt}props\ \isacommand{by}\isamarkupfalse%
\ force\isanewline
\ \ \ \ \isacommand{then}\isamarkupfalse%
\ \isacommand{have}\isamarkupfalse%
\ {\isachardoublequoteopen}y\ s\ j\ {\isacharequal}{\kern0pt}\ s{\isachardoublequoteclose}\ \isakeyword{if}\ {\isachardoublequoteopen}s{\isacharless}{\kern0pt}t{\isacharplus}{\kern0pt}{\isadigit{1}}{\isachardoublequoteclose}\ \isakeyword{for}\ s\isanewline
\ \ \ \ \isacommand{proof}\isamarkupfalse%
\isanewline
\ \ \ \ \ \ \isacommand{have}\isamarkupfalse%
\ {\isachardoublequoteopen}y\ s\ j\ {\isacharequal}{\kern0pt}\ S\ {\isacharparenleft}{\kern0pt}{\isacharquery}{\kern0pt}f\ s{\isacharparenright}{\kern0pt}\ j{\isachardoublequoteclose}\ \isacommand{using}\isamarkupfalse%
\ that\ \isacommand{unfolding}\isamarkupfalse%
\ y{\isacharunderscore}{\kern0pt}def\ \isacommand{by}\isamarkupfalse%
\ auto\isanewline
\ \ \ \ \ \ \isacommand{also}\isamarkupfalse%
\ \isacommand{have}\isamarkupfalse%
\ {\isachardoublequoteopen}\ {\isachardot}{\kern0pt}{\isachardot}{\kern0pt}{\isachardot}{\kern0pt}\ {\isacharequal}{\kern0pt}\ {\isacharparenleft}{\kern0pt}{\isacharquery}{\kern0pt}f\ s{\isacharparenright}{\kern0pt}\ {\isacharparenleft}{\kern0pt}k\ {\isacharminus}{\kern0pt}\ v{\isacharparenright}{\kern0pt}{\isachardoublequoteclose}\ \isacommand{using}\isamarkupfalse%
\ Bf{\isacharunderscore}{\kern0pt}props{\isacharparenleft}{\kern0pt}{\isadigit{6}}{\isacharparenright}{\kern0pt}\ f{\isacharunderscore}{\kern0pt}cube\ that\ j{\isacharunderscore}{\kern0pt}prop\ {\isacartoucheopen}k\ {\isacharminus}{\kern0pt}\ v\ {\isacharless}{\kern0pt}\ k{\isacartoucheclose}\ \isacommand{by}\isamarkupfalse%
\ fast\isanewline
\ \ \ \ \ \ \isacommand{also}\isamarkupfalse%
\ \isacommand{have}\isamarkupfalse%
\ {\isachardoublequoteopen}\ {\isachardot}{\kern0pt}{\isachardot}{\kern0pt}{\isachardot}{\kern0pt}\ {\isacharequal}{\kern0pt}\ s{\isachardoublequoteclose}\ \isacommand{using}\isamarkupfalse%
\ that\ j{\isacharunderscore}{\kern0pt}prop\ {\isacartoucheopen}k\ {\isacharminus}{\kern0pt}\ v\ {\isacharless}{\kern0pt}\ k\ {\isacharminus}{\kern0pt}\ u{\isacartoucheclose}\ \isacommand{by}\isamarkupfalse%
\ simp\isanewline
\ \ \ \ \ \ \isacommand{finally}\isamarkupfalse%
\ \isacommand{show}\isamarkupfalse%
\ {\isacharquery}{\kern0pt}thesis\ \isacommand{{\isachardot}{\kern0pt}}\isamarkupfalse%
\isanewline
\ \ \ \ \isacommand{qed}\isamarkupfalse%
\isanewline
\ \ \ \ \isacommand{then}\isamarkupfalse%
\ \isacommand{show}\isamarkupfalse%
\ {\isachardoublequoteopen}{\isasymexists}j\ {\isacharless}{\kern0pt}\ n{\isachardot}{\kern0pt}\ {\isasymforall}s{\isacharless}{\kern0pt}t{\isacharplus}{\kern0pt}{\isadigit{1}}{\isachardot}{\kern0pt}\ y\ s\ j\ {\isacharequal}{\kern0pt}\ s{\isachardoublequoteclose}\ \isacommand{using}\isamarkupfalse%
\ j{\isacharunderscore}{\kern0pt}prop\ \isacommand{by}\isamarkupfalse%
\ blast\isanewline
\ \ \isacommand{qed}\isamarkupfalse%
\isanewline
\ \ \isacommand{ultimately}\isamarkupfalse%
\ \isacommand{have}\isamarkupfalse%
\ Z{\isadigit{1}}{\isacharcolon}{\kern0pt}\ {\isachardoublequoteopen}is{\isacharunderscore}{\kern0pt}line\ y\ n\ {\isacharparenleft}{\kern0pt}t{\isacharplus}{\kern0pt}{\isadigit{1}}{\isacharparenright}{\kern0pt}{\isachardoublequoteclose}\ \isacommand{unfolding}\isamarkupfalse%
\ is{\isacharunderscore}{\kern0pt}line{\isacharunderscore}{\kern0pt}def\ \isacommand{by}\isamarkupfalse%
\ blast\isanewline
\isanewline
\ \ \isacommand{have}\isamarkupfalse%
\ k{\isacharunderscore}{\kern0pt}color{\isacharcolon}{\kern0pt}\ {\isachardoublequoteopen}{\isasymchi}\ e\ {\isacharless}{\kern0pt}\ k{\isachardoublequoteclose}\ \isakeyword{if}\ {\isachardoublequoteopen}e\ {\isasymin}\ y\ {\isacharbackquote}{\kern0pt}\ {\isacharbraceleft}{\kern0pt}{\isachardot}{\kern0pt}{\isachardot}{\kern0pt}{\isacharless}{\kern0pt}t{\isacharplus}{\kern0pt}{\isadigit{1}}{\isacharbraceright}{\kern0pt}{\isachardoublequoteclose}\ \isakeyword{for}\ e\ \isacommand{using}\isamarkupfalse%
\ {\isacartoucheopen}y\ {\isasymin}\ {\isacharbraceleft}{\kern0pt}{\isachardot}{\kern0pt}{\isachardot}{\kern0pt}{\isacharless}{\kern0pt}t{\isacharplus}{\kern0pt}{\isadigit{1}}{\isacharbraceright}{\kern0pt}\ {\isasymrightarrow}\isactrlsub E\ cube\ n\ {\isacharparenleft}{\kern0pt}t\ {\isacharplus}{\kern0pt}\ {\isadigit{1}}{\isacharparenright}{\kern0pt}{\isacartoucheclose}\ {\isacartoucheopen}{\isasymchi}\ {\isasymin}\ cube\ n\ {\isacharparenleft}{\kern0pt}t\ {\isacharplus}{\kern0pt}\ {\isadigit{1}}{\isacharparenright}{\kern0pt}\ {\isasymrightarrow}\isactrlsub E\ {\isacharbraceleft}{\kern0pt}{\isachardot}{\kern0pt}{\isachardot}{\kern0pt}{\isacharless}{\kern0pt}k{\isacharbraceright}{\kern0pt}{\isacartoucheclose}\ that\ \isacommand{by}\isamarkupfalse%
\ auto\isanewline
\ \ \isacommand{have}\isamarkupfalse%
\ {\isachardoublequoteopen}{\isasymchi}\ e{\isadigit{1}}\ {\isacharequal}{\kern0pt}\ {\isasymchi}\ e{\isadigit{2}}\ {\isasymand}\ {\isasymchi}\ e{\isadigit{1}}\ {\isacharless}{\kern0pt}\ k{\isachardoublequoteclose}\ \isakeyword{if}\ {\isachardoublequoteopen}e{\isadigit{1}}\ {\isasymin}\ y\ {\isacharbackquote}{\kern0pt}\ {\isacharbraceleft}{\kern0pt}{\isachardot}{\kern0pt}{\isachardot}{\kern0pt}{\isacharless}{\kern0pt}t{\isacharplus}{\kern0pt}{\isadigit{1}}{\isacharbraceright}{\kern0pt}{\isachardoublequoteclose}\ {\isachardoublequoteopen}e{\isadigit{2}}\ {\isasymin}\ y\ {\isacharbackquote}{\kern0pt}\ {\isacharbraceleft}{\kern0pt}{\isachardot}{\kern0pt}{\isachardot}{\kern0pt}{\isacharless}{\kern0pt}t{\isacharplus}{\kern0pt}{\isadigit{1}}{\isacharbraceright}{\kern0pt}{\isachardoublequoteclose}\ \isakeyword{for}\ e{\isadigit{1}}\ e{\isadigit{2}}\ \isanewline
\ \ \isacommand{proof}\isamarkupfalse%
\ \ \isanewline
\ \ \ \ \isacommand{from}\isamarkupfalse%
\ that\ \isacommand{obtain}\isamarkupfalse%
\ i{\isadigit{1}}\ i{\isadigit{2}}\ \isakeyword{where}\ i{\isacharunderscore}{\kern0pt}props{\isacharcolon}{\kern0pt}\ {\isachardoublequoteopen}i{\isadigit{1}}\ {\isacharless}{\kern0pt}\ t\ {\isacharplus}{\kern0pt}\ {\isadigit{1}}{\isachardoublequoteclose}\ {\isachardoublequoteopen}i{\isadigit{2}}\ {\isacharless}{\kern0pt}\ t\ {\isacharplus}{\kern0pt}\ {\isadigit{1}}{\isachardoublequoteclose}\ {\isachardoublequoteopen}e{\isadigit{1}}\ {\isacharequal}{\kern0pt}\ y\ i{\isadigit{1}}{\isachardoublequoteclose}\ {\isachardoublequoteopen}e{\isadigit{2}}\ {\isacharequal}{\kern0pt}\ y\ i{\isadigit{2}}{\isachardoublequoteclose}\ \isacommand{by}\isamarkupfalse%
\ blast\ \isanewline
\ \ \ \ \isacommand{from}\isamarkupfalse%
\ i{\isacharunderscore}{\kern0pt}props{\isacharparenleft}{\kern0pt}{\isadigit{1}}{\isacharcomma}{\kern0pt}{\isadigit{2}}{\isacharparenright}{\kern0pt}\ \isacommand{have}\isamarkupfalse%
\ {\isachardoublequoteopen}{\isasymchi}\ {\isacharparenleft}{\kern0pt}y\ i{\isadigit{1}}{\isacharparenright}{\kern0pt}\ {\isacharequal}{\kern0pt}\ {\isasymchi}\ {\isacharparenleft}{\kern0pt}y\ i{\isadigit{2}}{\isacharparenright}{\kern0pt}{\isachardoublequoteclose}\isanewline
\ \ \ \ \isacommand{proof}\isamarkupfalse%
\ {\isacharparenleft}{\kern0pt}induction\ i{\isadigit{1}}\ i{\isadigit{2}}\ rule{\isacharcolon}{\kern0pt}\ linorder{\isacharunderscore}{\kern0pt}wlog{\isacharparenright}{\kern0pt}\isanewline
\ \ \ \ \ \ \isacommand{case}\isamarkupfalse%
\ {\isacharparenleft}{\kern0pt}le\ a\ b{\isacharparenright}{\kern0pt}\isanewline
\ \ \ \ \ \ \isacommand{then}\isamarkupfalse%
\ \isacommand{show}\isamarkupfalse%
\ {\isacharquery}{\kern0pt}case\isanewline
\ \ \ \ \ \ \isacommand{proof}\isamarkupfalse%
\ {\isacharparenleft}{\kern0pt}cases\ {\isachardoublequoteopen}a\ {\isacharequal}{\kern0pt}\ b{\isachardoublequoteclose}{\isacharparenright}{\kern0pt}\isanewline
\ \ \ \ \ \ \ \ \isacommand{case}\isamarkupfalse%
\ True\isanewline
\ \ \ \ \ \ \ \ \isacommand{then}\isamarkupfalse%
\ \isacommand{show}\isamarkupfalse%
\ {\isacharquery}{\kern0pt}thesis\ \isacommand{by}\isamarkupfalse%
\ blast\isanewline
\ \ \ \ \ \ \isacommand{next}\isamarkupfalse%
\isanewline
\ \ \ \ \ \ \ \ \isacommand{case}\isamarkupfalse%
\ False\isanewline
\ \ \ \ \ \ \ \ \isacommand{then}\isamarkupfalse%
\ \isacommand{have}\isamarkupfalse%
\ {\isachardoublequoteopen}a\ {\isacharless}{\kern0pt}\ b{\isachardoublequoteclose}\ \isacommand{using}\isamarkupfalse%
\ le\ \isacommand{by}\isamarkupfalse%
\ linarith\isanewline
\ \ \ \ \ \ \ \ \isacommand{then}\isamarkupfalse%
\ \isacommand{consider}\isamarkupfalse%
\ {\isachardoublequoteopen}b\ {\isacharequal}{\kern0pt}\ t{\isachardoublequoteclose}\ {\isacharbar}{\kern0pt}\ {\isachardoublequoteopen}b\ {\isacharless}{\kern0pt}\ t{\isachardoublequoteclose}\ \isacommand{using}\isamarkupfalse%
\ le{\isachardot}{\kern0pt}prems{\isacharparenleft}{\kern0pt}{\isadigit{2}}{\isacharparenright}{\kern0pt}\ \isacommand{by}\isamarkupfalse%
\ linarith\isanewline
\ \ \ \ \ \ \ \ \isacommand{then}\isamarkupfalse%
\ \isacommand{show}\isamarkupfalse%
\ {\isacharquery}{\kern0pt}thesis\isanewline
\ \ \ \ \ \ \ \ \isacommand{proof}\isamarkupfalse%
\ cases\isanewline
\ \ \ \ \ \ \ \ \ \ \isacommand{case}\isamarkupfalse%
\ {\isadigit{1}}\isanewline
\ \ \ \ \ \ \ \ \ \ \isacommand{then}\isamarkupfalse%
\ \isacommand{have}\isamarkupfalse%
\ {\isachardoublequoteopen}y\ b\ {\isasymin}\ S\ {\isacharbackquote}{\kern0pt}\ classes\ k\ t\ v{\isachardoublequoteclose}\ \isanewline
\ \ \ \ \ \ \ \ \ \ \isacommand{proof}\isamarkupfalse%
\ {\isacharminus}{\kern0pt}\isanewline
\ \ \ \ \ \ \ \ \ \ \ \ \isacommand{have}\isamarkupfalse%
\ {\isachardoublequoteopen}y\ b\ {\isacharequal}{\kern0pt}\ S\ {\isacharparenleft}{\kern0pt}{\isacharquery}{\kern0pt}f\ b{\isacharparenright}{\kern0pt}{\isachardoublequoteclose}\ \isacommand{unfolding}\isamarkupfalse%
\ y{\isacharunderscore}{\kern0pt}def\ \isacommand{using}\isamarkupfalse%
\ {\isacartoucheopen}b\ {\isacharequal}{\kern0pt}\ t{\isacartoucheclose}\ \isacommand{by}\isamarkupfalse%
\ auto\isanewline
\ \ \ \ \ \ \ \ \ \ \ \ \isacommand{moreover}\isamarkupfalse%
\ \isacommand{have}\isamarkupfalse%
\ {\isachardoublequoteopen}{\isacharquery}{\kern0pt}f\ b\ {\isasymin}\ classes\ k\ t\ v{\isachardoublequoteclose}\ \isacommand{using}\isamarkupfalse%
\ {\isacartoucheopen}b\ {\isacharequal}{\kern0pt}\ t{\isacartoucheclose}\ f{\isacharunderscore}{\kern0pt}classes{\isacharunderscore}{\kern0pt}v\ \isacommand{by}\isamarkupfalse%
\ blast\isanewline
\ \ \ \ \ \ \ \ \ \ \ \ \isacommand{ultimately}\isamarkupfalse%
\ \isacommand{show}\isamarkupfalse%
\ {\isachardoublequoteopen}y\ b\ {\isasymin}\ S\ {\isacharbackquote}{\kern0pt}\ classes\ k\ t\ v{\isachardoublequoteclose}\ \isacommand{by}\isamarkupfalse%
\ blast\isanewline
\ \ \ \ \ \ \ \ \ \ \isacommand{qed}\isamarkupfalse%
\isanewline
\ \ \ \ \ \ \ \ \ \ \isacommand{moreover}\isamarkupfalse%
\ \isacommand{have}\isamarkupfalse%
\ {\isachardoublequoteopen}x\ u\ {\isasymin}\ classes\ k\ t\ u{\isachardoublequoteclose}\isanewline
\ \ \ \ \ \ \ \ \ \ \isacommand{proof}\isamarkupfalse%
\ {\isacharminus}{\kern0pt}\isanewline
\ \ \ \ \ \ \ \ \ \ \ \ \isacommand{have}\isamarkupfalse%
\ {\isachardoublequoteopen}x\ u\ cord\ {\isacharequal}{\kern0pt}\ t{\isachardoublequoteclose}\ \isakeyword{if}\ {\isachardoublequoteopen}cord\ {\isasymin}\ {\isacharbraceleft}{\kern0pt}k\ {\isacharminus}{\kern0pt}\ u{\isachardot}{\kern0pt}{\isachardot}{\kern0pt}{\isacharless}{\kern0pt}k{\isacharbraceright}{\kern0pt}{\isachardoublequoteclose}\ \isakeyword{for}\ cord\ \isacommand{using}\isamarkupfalse%
\ uv{\isacharunderscore}{\kern0pt}props\ that\ \isacommand{unfolding}\isamarkupfalse%
\ x{\isacharunderscore}{\kern0pt}def\ \isacommand{by}\isamarkupfalse%
\ simp\ \isanewline
\ \ \ \ \ \ \ \ \ \ \ \ \isacommand{moreover}\isamarkupfalse%
\ \isanewline
\ \ \ \ \ \ \ \ \ \ \ \ \isacommand{{\isacharbraceleft}{\kern0pt}}\isamarkupfalse%
\ \ \isanewline
\ \ \ \ \ \ \ \ \ \ \ \ \ \ \isacommand{have}\isamarkupfalse%
\ {\isachardoublequoteopen}x\ u\ cord\ {\isasymnoteq}\ t{\isachardoublequoteclose}\ \isakeyword{if}\ {\isachardoublequoteopen}cord\ {\isasymin}\ {\isacharbraceleft}{\kern0pt}{\isachardot}{\kern0pt}{\isachardot}{\kern0pt}{\isacharless}{\kern0pt}k\ {\isacharminus}{\kern0pt}\ u{\isacharbraceright}{\kern0pt}{\isachardoublequoteclose}\ \isakeyword{for}\ cord\ \isacommand{using}\isamarkupfalse%
\ uv{\isacharunderscore}{\kern0pt}props\ that\ assms{\isacharparenleft}{\kern0pt}{\isadigit{2}}{\isacharparenright}{\kern0pt}\ \isacommand{unfolding}\isamarkupfalse%
\ x{\isacharunderscore}{\kern0pt}def\ \isacommand{by}\isamarkupfalse%
\ auto\isanewline
\ \ \ \ \ \ \ \ \ \ \ \ \ \ \isacommand{then}\isamarkupfalse%
\ \isacommand{have}\isamarkupfalse%
\ {\isachardoublequoteopen}t\ {\isasymnotin}\ x\ u\ {\isacharbackquote}{\kern0pt}\ {\isacharbraceleft}{\kern0pt}{\isachardot}{\kern0pt}{\isachardot}{\kern0pt}{\isacharless}{\kern0pt}k\ {\isacharminus}{\kern0pt}\ u{\isacharbraceright}{\kern0pt}{\isachardoublequoteclose}\ \isacommand{by}\isamarkupfalse%
\ blast\isanewline
\ \ \ \ \ \ \ \ \ \ \ \ \isacommand{{\isacharbraceright}{\kern0pt}}\isamarkupfalse%
\isanewline
\ \ \ \ \ \ \ \ \ \ \ \ \isacommand{ultimately}\isamarkupfalse%
\ \isacommand{show}\isamarkupfalse%
\ {\isachardoublequoteopen}x\ u\ {\isasymin}\ classes\ k\ t\ u{\isachardoublequoteclose}\ \isacommand{unfolding}\isamarkupfalse%
\ classes{\isacharunderscore}{\kern0pt}def\ \isanewline
\ \ \ \ \ \ \ \ \ \ \ \ \ \ \isacommand{using}\isamarkupfalse%
\ {\isacartoucheopen}x\ {\isacharbackquote}{\kern0pt}\ {\isacharbraceleft}{\kern0pt}{\isachardot}{\kern0pt}{\isachardot}{\kern0pt}k{\isacharbraceright}{\kern0pt}\ {\isasymsubseteq}\ cube\ k\ {\isacharparenleft}{\kern0pt}t\ {\isacharplus}{\kern0pt}\ {\isadigit{1}}{\isacharparenright}{\kern0pt}{\isacartoucheclose}\ uv{\isacharunderscore}{\kern0pt}props\ \isacommand{by}\isamarkupfalse%
\ blast\isanewline
\ \ \ \ \ \ \ \ \ \ \isacommand{qed}\isamarkupfalse%
\isanewline
\ \ \ \ \ \ \ \ \ \ \isacommand{moreover}\isamarkupfalse%
\ \isacommand{have}\isamarkupfalse%
\ {\isachardoublequoteopen}x\ v\ {\isasymin}\ classes\ k\ t\ v{\isachardoublequoteclose}\isanewline
\ \ \ \ \ \ \ \ \ \ \isacommand{proof}\isamarkupfalse%
\ {\isacharminus}{\kern0pt}\isanewline
\ \ \ \ \ \ \ \ \ \ \ \ \isacommand{have}\isamarkupfalse%
\ {\isachardoublequoteopen}x\ v\ cord\ {\isacharequal}{\kern0pt}\ t{\isachardoublequoteclose}\ \isakeyword{if}\ {\isachardoublequoteopen}cord\ {\isasymin}\ {\isacharbraceleft}{\kern0pt}k\ {\isacharminus}{\kern0pt}\ v{\isachardot}{\kern0pt}{\isachardot}{\kern0pt}{\isacharless}{\kern0pt}k{\isacharbraceright}{\kern0pt}{\isachardoublequoteclose}\ \isakeyword{for}\ cord\ \isacommand{using}\isamarkupfalse%
\ uv{\isacharunderscore}{\kern0pt}props\ that\ \isacommand{unfolding}\isamarkupfalse%
\ x{\isacharunderscore}{\kern0pt}def\ \isacommand{by}\isamarkupfalse%
\ simp\ \isanewline
\ \ \ \ \ \ \ \ \ \ \ \ \isacommand{moreover}\isamarkupfalse%
\ \isanewline
\ \ \ \ \ \ \ \ \ \ \ \ \isacommand{{\isacharbraceleft}{\kern0pt}}\isamarkupfalse%
\ \ \isanewline
\ \ \ \ \ \ \ \ \ \ \ \ \ \ \isacommand{have}\isamarkupfalse%
\ {\isachardoublequoteopen}x\ v\ cord\ {\isasymnoteq}\ t{\isachardoublequoteclose}\ \isakeyword{if}\ {\isachardoublequoteopen}cord\ {\isasymin}\ {\isacharbraceleft}{\kern0pt}{\isachardot}{\kern0pt}{\isachardot}{\kern0pt}{\isacharless}{\kern0pt}k\ {\isacharminus}{\kern0pt}\ v{\isacharbraceright}{\kern0pt}{\isachardoublequoteclose}\ \isakeyword{for}\ cord\ \isacommand{using}\isamarkupfalse%
\ uv{\isacharunderscore}{\kern0pt}props\ that\ assms{\isacharparenleft}{\kern0pt}{\isadigit{2}}{\isacharparenright}{\kern0pt}\ \isacommand{unfolding}\isamarkupfalse%
\ x{\isacharunderscore}{\kern0pt}def\ \isacommand{by}\isamarkupfalse%
\ auto\isanewline
\ \ \ \ \ \ \ \ \ \ \ \ \ \ \isacommand{then}\isamarkupfalse%
\ \isacommand{have}\isamarkupfalse%
\ {\isachardoublequoteopen}t\ {\isasymnotin}\ x\ v\ {\isacharbackquote}{\kern0pt}\ {\isacharbraceleft}{\kern0pt}{\isachardot}{\kern0pt}{\isachardot}{\kern0pt}{\isacharless}{\kern0pt}k\ {\isacharminus}{\kern0pt}\ v{\isacharbraceright}{\kern0pt}{\isachardoublequoteclose}\ \isacommand{by}\isamarkupfalse%
\ blast\isanewline
\ \ \ \ \ \ \ \ \ \ \ \ \isacommand{{\isacharbraceright}{\kern0pt}}\isamarkupfalse%
\isanewline
\ \ \ \ \ \ \ \ \ \ \ \ \isacommand{ultimately}\isamarkupfalse%
\ \isacommand{show}\isamarkupfalse%
\ {\isachardoublequoteopen}x\ v\ {\isasymin}\ classes\ k\ t\ v{\isachardoublequoteclose}\ \isacommand{unfolding}\isamarkupfalse%
\ classes{\isacharunderscore}{\kern0pt}def\ \isanewline
\ \ \ \ \ \ \ \ \ \ \ \ \ \ \isacommand{using}\isamarkupfalse%
\ {\isacartoucheopen}x\ {\isacharbackquote}{\kern0pt}\ {\isacharbraceleft}{\kern0pt}{\isachardot}{\kern0pt}{\isachardot}{\kern0pt}k{\isacharbraceright}{\kern0pt}\ {\isasymsubseteq}\ cube\ k\ {\isacharparenleft}{\kern0pt}t\ {\isacharplus}{\kern0pt}\ {\isadigit{1}}{\isacharparenright}{\kern0pt}{\isacartoucheclose}\ uv{\isacharunderscore}{\kern0pt}props\ \isacommand{by}\isamarkupfalse%
\ blast\isanewline
\ \ \ \ \ \ \ \ \ \ \isacommand{qed}\isamarkupfalse%
\isanewline
\ \ \ \ \ \ \ \ \ \ \isacommand{moreover}\isamarkupfalse%
\ \isacommand{have}\isamarkupfalse%
\ {\isachardoublequoteopen}{\isasymchi}\ {\isacharparenleft}{\kern0pt}y\ b{\isacharparenright}{\kern0pt}\ {\isacharequal}{\kern0pt}\ {\isasymchi}\ {\isacharparenleft}{\kern0pt}S\ {\isacharparenleft}{\kern0pt}x\ v{\isacharparenright}{\kern0pt}{\isacharparenright}{\kern0pt}{\isachardoublequoteclose}\ \isacommand{using}\isamarkupfalse%
\ assms{\isacharparenleft}{\kern0pt}{\isadigit{1}}{\isacharparenright}{\kern0pt}\ calculation{\isacharparenleft}{\kern0pt}{\isadigit{1}}{\isacharcomma}{\kern0pt}\ {\isadigit{3}}{\isacharparenright}{\kern0pt}\ \isacommand{unfolding}\isamarkupfalse%
\ layered{\isacharunderscore}{\kern0pt}subspace{\isacharunderscore}{\kern0pt}def\ \isanewline
\ \ \ \ \ \ \ \ \ \ \ \ \isacommand{by}\isamarkupfalse%
\ {\isacharparenleft}{\kern0pt}metis\ imageE\ uv{\isacharunderscore}{\kern0pt}props{\isacharparenright}{\kern0pt}\isanewline
\ \ \ \ \ \ \ \ \ \ \isacommand{moreover}\isamarkupfalse%
\ \isacommand{have}\isamarkupfalse%
\ {\isachardoublequoteopen}y\ a\ {\isasymin}\ S\ {\isacharbackquote}{\kern0pt}\ classes\ k\ t\ u{\isachardoublequoteclose}\ \isanewline
\ \ \ \ \ \ \ \ \ \ \isacommand{proof}\isamarkupfalse%
\ {\isacharminus}{\kern0pt}\isanewline
\ \ \ \ \ \ \ \ \ \ \ \ \isacommand{have}\isamarkupfalse%
\ {\isachardoublequoteopen}y\ a\ {\isacharequal}{\kern0pt}\ S\ {\isacharparenleft}{\kern0pt}{\isacharquery}{\kern0pt}f\ a{\isacharparenright}{\kern0pt}{\isachardoublequoteclose}\ \isacommand{unfolding}\isamarkupfalse%
\ y{\isacharunderscore}{\kern0pt}def\ \isacommand{using}\isamarkupfalse%
\ {\isacartoucheopen}a\ {\isacharless}{\kern0pt}\ b{\isacartoucheclose}\ {\isadigit{1}}\ \isacommand{by}\isamarkupfalse%
\ simp\isanewline
\ \ \ \ \ \ \ \ \ \ \ \ \isacommand{moreover}\isamarkupfalse%
\ \isacommand{have}\isamarkupfalse%
\ {\isachardoublequoteopen}{\isacharquery}{\kern0pt}f\ a\ {\isasymin}\ classes\ k\ t\ u{\isachardoublequoteclose}\ \isacommand{using}\isamarkupfalse%
\ {\isacartoucheopen}a\ {\isacharless}{\kern0pt}\ b{\isacartoucheclose}\ {\isadigit{1}}\ f{\isacharunderscore}{\kern0pt}classes{\isacharunderscore}{\kern0pt}u\ \isacommand{by}\isamarkupfalse%
\ blast\isanewline
\ \ \ \ \ \ \ \ \ \ \ \ \isacommand{ultimately}\isamarkupfalse%
\ \isacommand{show}\isamarkupfalse%
\ {\isachardoublequoteopen}y\ a\ {\isasymin}\ S\ {\isacharbackquote}{\kern0pt}\ classes\ k\ t\ u{\isachardoublequoteclose}\ \isacommand{by}\isamarkupfalse%
\ blast\isanewline
\ \ \ \ \ \ \ \ \ \ \isacommand{qed}\isamarkupfalse%
\isanewline
\ \ \ \ \ \ \ \ \ \ \isacommand{moreover}\isamarkupfalse%
\ \isacommand{have}\isamarkupfalse%
\ {\isachardoublequoteopen}{\isasymchi}\ {\isacharparenleft}{\kern0pt}y\ a{\isacharparenright}{\kern0pt}\ {\isacharequal}{\kern0pt}\ {\isasymchi}\ {\isacharparenleft}{\kern0pt}S\ {\isacharparenleft}{\kern0pt}x\ u{\isacharparenright}{\kern0pt}{\isacharparenright}{\kern0pt}{\isachardoublequoteclose}\ \isacommand{using}\isamarkupfalse%
\ assms{\isacharparenleft}{\kern0pt}{\isadigit{1}}{\isacharparenright}{\kern0pt}\ calculation{\isacharparenleft}{\kern0pt}{\isadigit{2}}{\isacharcomma}{\kern0pt}\ {\isadigit{5}}{\isacharparenright}{\kern0pt}\ \isacommand{unfolding}\isamarkupfalse%
\ layered{\isacharunderscore}{\kern0pt}subspace{\isacharunderscore}{\kern0pt}def\ \isanewline
\ \ \ \ \ \ \ \ \ \ \ \ \isacommand{by}\isamarkupfalse%
\ {\isacharparenleft}{\kern0pt}metis\ imageE\ uv{\isacharunderscore}{\kern0pt}props{\isacharparenright}{\kern0pt}\isanewline
\ \ \ \ \ \ \ \ \ \ \isacommand{ultimately}\isamarkupfalse%
\ \isacommand{have}\isamarkupfalse%
\ {\isachardoublequoteopen}{\isasymchi}\ {\isacharparenleft}{\kern0pt}y\ a{\isacharparenright}{\kern0pt}\ {\isacharequal}{\kern0pt}\ {\isasymchi}\ {\isacharparenleft}{\kern0pt}y\ b{\isacharparenright}{\kern0pt}{\isachardoublequoteclose}\ \isacommand{using}\isamarkupfalse%
\ uv{\isacharunderscore}{\kern0pt}props\ \isacommand{by}\isamarkupfalse%
\ simp\isanewline
\ \ \ \ \ \ \ \ \ \ \isacommand{then}\isamarkupfalse%
\ \isacommand{show}\isamarkupfalse%
\ {\isacharquery}{\kern0pt}thesis\ \isacommand{by}\isamarkupfalse%
\ blast\isanewline
\ \ \ \ \ \ \ \ \isacommand{next}\isamarkupfalse%
\isanewline
\ \ \ \ \ \ \ \ \ \ \isacommand{case}\isamarkupfalse%
\ {\isadigit{2}}\isanewline
\ \ \ \ \ \ \ \ \ \ \isacommand{then}\isamarkupfalse%
\ \isacommand{have}\isamarkupfalse%
\ {\isachardoublequoteopen}a\ {\isacharless}{\kern0pt}\ t{\isachardoublequoteclose}\ \isacommand{using}\isamarkupfalse%
\ {\isacartoucheopen}a\ {\isacharless}{\kern0pt}\ b{\isacartoucheclose}\ less{\isacharunderscore}{\kern0pt}trans\ \isacommand{by}\isamarkupfalse%
\ blast\isanewline
\ \ \ \ \ \ \ \ \ \ \isacommand{then}\isamarkupfalse%
\ \isacommand{have}\isamarkupfalse%
\ {\isachardoublequoteopen}y\ a\ {\isasymin}\ S\ {\isacharbackquote}{\kern0pt}\ classes\ k\ t\ u{\isachardoublequoteclose}\isanewline
\ \ \ \ \ \ \ \ \ \ \isacommand{proof}\isamarkupfalse%
\ {\isacharminus}{\kern0pt}\isanewline
\ \ \ \ \ \ \ \ \ \ \ \ \isacommand{have}\isamarkupfalse%
\ {\isachardoublequoteopen}y\ a\ {\isacharequal}{\kern0pt}\ S\ {\isacharparenleft}{\kern0pt}{\isacharquery}{\kern0pt}f\ a{\isacharparenright}{\kern0pt}{\isachardoublequoteclose}\ \isacommand{unfolding}\isamarkupfalse%
\ y{\isacharunderscore}{\kern0pt}def\ \isacommand{using}\isamarkupfalse%
\ {\isacartoucheopen}a\ {\isacharless}{\kern0pt}\ t{\isacartoucheclose}\ \isacommand{by}\isamarkupfalse%
\ auto\isanewline
\ \ \ \ \ \ \ \ \ \ \ \ \isacommand{moreover}\isamarkupfalse%
\ \isacommand{have}\isamarkupfalse%
\ {\isachardoublequoteopen}{\isacharquery}{\kern0pt}f\ a\ {\isasymin}\ classes\ k\ t\ u{\isachardoublequoteclose}\ \isacommand{using}\isamarkupfalse%
\ {\isacartoucheopen}a\ {\isacharless}{\kern0pt}\ t{\isacartoucheclose}\ f{\isacharunderscore}{\kern0pt}classes{\isacharunderscore}{\kern0pt}u\ \isacommand{by}\isamarkupfalse%
\ blast\isanewline
\ \ \ \ \ \ \ \ \ \ \ \ \isacommand{ultimately}\isamarkupfalse%
\ \isacommand{show}\isamarkupfalse%
\ {\isachardoublequoteopen}y\ a\ {\isasymin}\ S\ {\isacharbackquote}{\kern0pt}\ classes\ k\ t\ u{\isachardoublequoteclose}\ \isacommand{by}\isamarkupfalse%
\ blast\isanewline
\ \ \ \ \ \ \ \ \ \ \isacommand{qed}\isamarkupfalse%
\isanewline
\ \ \ \ \ \ \ \ \ \ \isacommand{moreover}\isamarkupfalse%
\ \isacommand{have}\isamarkupfalse%
\ {\isachardoublequoteopen}y\ b\ {\isasymin}\ S\ {\isacharbackquote}{\kern0pt}\ classes\ k\ t\ u{\isachardoublequoteclose}\isanewline
\ \ \ \ \ \ \ \ \ \ \isacommand{proof}\isamarkupfalse%
\ {\isacharminus}{\kern0pt}\isanewline
\ \ \ \ \ \ \ \ \ \ \ \ \isacommand{have}\isamarkupfalse%
\ {\isachardoublequoteopen}y\ b\ {\isacharequal}{\kern0pt}\ S\ {\isacharparenleft}{\kern0pt}{\isacharquery}{\kern0pt}f\ b{\isacharparenright}{\kern0pt}{\isachardoublequoteclose}\ \isacommand{unfolding}\isamarkupfalse%
\ y{\isacharunderscore}{\kern0pt}def\ \isacommand{using}\isamarkupfalse%
\ {\isacartoucheopen}b\ {\isacharless}{\kern0pt}\ t{\isacartoucheclose}\ \isacommand{by}\isamarkupfalse%
\ auto\isanewline
\ \ \ \ \ \ \ \ \ \ \ \ \isacommand{moreover}\isamarkupfalse%
\ \isacommand{have}\isamarkupfalse%
\ {\isachardoublequoteopen}{\isacharquery}{\kern0pt}f\ b\ {\isasymin}\ classes\ k\ t\ u{\isachardoublequoteclose}\ \isacommand{using}\isamarkupfalse%
\ {\isacartoucheopen}b\ {\isacharless}{\kern0pt}\ t{\isacartoucheclose}\ f{\isacharunderscore}{\kern0pt}classes{\isacharunderscore}{\kern0pt}u\ \isacommand{by}\isamarkupfalse%
\ blast\isanewline
\ \ \ \ \ \ \ \ \ \ \ \ \isacommand{ultimately}\isamarkupfalse%
\ \isacommand{show}\isamarkupfalse%
\ {\isachardoublequoteopen}y\ b\ {\isasymin}\ S\ {\isacharbackquote}{\kern0pt}\ classes\ k\ t\ u{\isachardoublequoteclose}\ \isacommand{by}\isamarkupfalse%
\ blast\isanewline
\ \ \ \ \ \ \ \ \ \ \isacommand{qed}\isamarkupfalse%
\isanewline
\ \ \ \ \ \ \ \ \ \ \isacommand{ultimately}\isamarkupfalse%
\ \isacommand{have}\isamarkupfalse%
\ {\isachardoublequoteopen}{\isasymchi}\ {\isacharparenleft}{\kern0pt}y\ a{\isacharparenright}{\kern0pt}\ {\isacharequal}{\kern0pt}\ {\isasymchi}\ {\isacharparenleft}{\kern0pt}y\ b{\isacharparenright}{\kern0pt}{\isachardoublequoteclose}\ \isacommand{using}\isamarkupfalse%
\ assms{\isacharparenleft}{\kern0pt}{\isadigit{1}}{\isacharparenright}{\kern0pt}\ uv{\isacharunderscore}{\kern0pt}props\ \isacommand{unfolding}\isamarkupfalse%
\ layered{\isacharunderscore}{\kern0pt}subspace{\isacharunderscore}{\kern0pt}def\ \isacommand{by}\isamarkupfalse%
\ {\isacharparenleft}{\kern0pt}metis\ imageE{\isacharparenright}{\kern0pt}\isanewline
\ \ \ \ \ \ \ \ \ \ \isacommand{then}\isamarkupfalse%
\ \isacommand{show}\isamarkupfalse%
\ {\isacharquery}{\kern0pt}thesis\ \isacommand{by}\isamarkupfalse%
\ blast\isanewline
\ \ \ \ \ \ \ \ \isacommand{qed}\isamarkupfalse%
\isanewline
\ \ \ \ \ \ \isacommand{qed}\isamarkupfalse%
\isanewline
\ \ \ \ \isacommand{next}\isamarkupfalse%
\isanewline
\ \ \ \ \ \ \isacommand{case}\isamarkupfalse%
\ {\isacharparenleft}{\kern0pt}sym\ a\ b{\isacharparenright}{\kern0pt}\isanewline
\ \ \ \ \ \ \isacommand{then}\isamarkupfalse%
\ \isacommand{show}\isamarkupfalse%
\ {\isacharquery}{\kern0pt}case\ \isacommand{by}\isamarkupfalse%
\ presburger\isanewline
\ \ \ \ \isacommand{qed}\isamarkupfalse%
\isanewline
\ \ \ \ \isacommand{then}\isamarkupfalse%
\ \isacommand{show}\isamarkupfalse%
\ {\isachardoublequoteopen}{\isasymchi}\ e{\isadigit{1}}\ {\isacharequal}{\kern0pt}\ {\isasymchi}\ e{\isadigit{2}}{\isachardoublequoteclose}\ \isacommand{using}\isamarkupfalse%
\ i{\isacharunderscore}{\kern0pt}props{\isacharparenleft}{\kern0pt}{\isadigit{3}}{\isacharcomma}{\kern0pt}{\isadigit{4}}{\isacharparenright}{\kern0pt}\ \isacommand{by}\isamarkupfalse%
\ blast\isanewline
\ \ \isacommand{qed}\isamarkupfalse%
\ {\isacharparenleft}{\kern0pt}use\ that{\isacharparenleft}{\kern0pt}{\isadigit{1}}{\isacharparenright}{\kern0pt}\ k{\isacharunderscore}{\kern0pt}color\ \isakeyword{in}\ blast{\isacharparenright}{\kern0pt}\isanewline
\ \ \isacommand{then}\isamarkupfalse%
\ \isacommand{have}\isamarkupfalse%
\ Z{\isadigit{2}}{\isacharcolon}{\kern0pt}\ {\isachardoublequoteopen}{\isasymexists}c\ {\isacharless}{\kern0pt}\ k{\isachardot}{\kern0pt}\ {\isasymforall}e\ {\isasymin}\ y\ {\isacharbackquote}{\kern0pt}\ {\isacharbraceleft}{\kern0pt}{\isachardot}{\kern0pt}{\isachardot}{\kern0pt}{\isacharless}{\kern0pt}t{\isacharplus}{\kern0pt}{\isadigit{1}}{\isacharbraceright}{\kern0pt}{\isachardot}{\kern0pt}\ {\isasymchi}\ e\ {\isacharequal}{\kern0pt}\ c{\isachardoublequoteclose}\isanewline
\ \ \ \ \isacommand{by}\isamarkupfalse%
\ {\isacharparenleft}{\kern0pt}meson\ image{\isacharunderscore}{\kern0pt}eqI\ lessThan{\isacharunderscore}{\kern0pt}iff\ less{\isacharunderscore}{\kern0pt}add{\isacharunderscore}{\kern0pt}one{\isacharparenright}{\kern0pt}\isanewline
\isanewline
\ \ \isacommand{from}\isamarkupfalse%
\ Z{\isadigit{1}}\ Z{\isadigit{2}}\ \isacommand{show}\isamarkupfalse%
\ {\isachardoublequoteopen}{\isasymexists}L\ c{\isachardot}{\kern0pt}\ c\ {\isacharless}{\kern0pt}\ k\ {\isasymand}\ is{\isacharunderscore}{\kern0pt}line\ L\ n\ {\isacharparenleft}{\kern0pt}t\ {\isacharplus}{\kern0pt}\ {\isadigit{1}}{\isacharparenright}{\kern0pt}\ {\isasymand}\ {\isacharparenleft}{\kern0pt}{\isasymforall}y{\isasymin}L\ {\isacharbackquote}{\kern0pt}\ {\isacharbraceleft}{\kern0pt}{\isachardot}{\kern0pt}{\isachardot}{\kern0pt}{\isacharless}{\kern0pt}t\ {\isacharplus}{\kern0pt}\ {\isadigit{1}}{\isacharbraceright}{\kern0pt}{\isachardot}{\kern0pt}\ {\isasymchi}\ y\ {\isacharequal}{\kern0pt}\ c{\isacharparenright}{\kern0pt}{\isachardoublequoteclose}\ \isacommand{by}\isamarkupfalse%
\ blast\isanewline
\isanewline
\isacommand{qed}\isamarkupfalse%
%
\endisatagproof
{\isafoldproof}%
%
\isadelimproof
\isanewline
%
\endisadelimproof
\isanewline
\isacommand{corollary}\isamarkupfalse%
\ corollary{\isadigit{6}}{\isacharcolon}{\kern0pt}\ \isakeyword{assumes}\ {\isachardoublequoteopen}{\isacharparenleft}{\kern0pt}{\isasymAnd}r\ k{\isachardot}{\kern0pt}\ lhj\ r\ t\ k{\isacharparenright}{\kern0pt}{\isachardoublequoteclose}\ {\isachardoublequoteopen}t{\isachargreater}{\kern0pt}{\isadigit{0}}{\isachardoublequoteclose}\ \isakeyword{shows}\ {\isachardoublequoteopen}{\isacharparenleft}{\kern0pt}hj\ r\ {\isacharparenleft}{\kern0pt}t{\isacharplus}{\kern0pt}{\isadigit{1}}{\isacharparenright}{\kern0pt}{\isacharparenright}{\kern0pt}{\isachardoublequoteclose}\isanewline
%
\isadelimproof
\ \ %
\endisadelimproof
%
\isatagproof
\isacommand{using}\isamarkupfalse%
\ assms{\isacharparenleft}{\kern0pt}{\isadigit{1}}{\isacharparenright}{\kern0pt}{\isacharbrackleft}{\kern0pt}of\ r\ r{\isacharbrackright}{\kern0pt}\ assms{\isacharparenleft}{\kern0pt}{\isadigit{2}}{\isacharparenright}{\kern0pt}\ \ \isacommand{unfolding}\isamarkupfalse%
\ lhj{\isacharunderscore}{\kern0pt}def\ hj{\isacharunderscore}{\kern0pt}def\ \isacommand{using}\isamarkupfalse%
\ thm{\isadigit{5}}{\isacharbrackleft}{\kern0pt}of\ {\isacharunderscore}{\kern0pt}\ r\ {\isacharunderscore}{\kern0pt}\ t{\isacharbrackright}{\kern0pt}\ \isacommand{by}\isamarkupfalse%
\ metis%
\endisatagproof
{\isafoldproof}%
%
\isadelimproof
\isanewline
%
\endisadelimproof
\isanewline
\isanewline
\isanewline
\isacommand{lemma}\isamarkupfalse%
\ hj{\isacharunderscore}{\kern0pt}r{\isacharunderscore}{\kern0pt}nonzero{\isacharunderscore}{\kern0pt}t{\isacharunderscore}{\kern0pt}{\isadigit{0}}{\isacharcolon}{\kern0pt}\ \isakeyword{assumes}\ {\isachardoublequoteopen}r\ {\isachargreater}{\kern0pt}\ {\isadigit{0}}{\isachardoublequoteclose}\ \isakeyword{shows}\ {\isachardoublequoteopen}hj\ r\ {\isadigit{0}}{\isachardoublequoteclose}\isanewline
%
\isadelimproof
%
\endisadelimproof
%
\isatagproof
\isacommand{proof}\isamarkupfalse%
{\isacharminus}{\kern0pt}\isanewline
\ \ \isacommand{have}\isamarkupfalse%
\ {\isachardoublequoteopen}{\isacharparenleft}{\kern0pt}{\isasymexists}L\ c{\isachardot}{\kern0pt}\ c\ {\isacharless}{\kern0pt}\ r\ {\isasymand}\ is{\isacharunderscore}{\kern0pt}line\ L\ N{\isacharprime}{\kern0pt}\ {\isadigit{0}}\ {\isasymand}\ {\isacharparenleft}{\kern0pt}{\isasymforall}y\ {\isasymin}\ L\ {\isacharbackquote}{\kern0pt}\ {\isacharbraceleft}{\kern0pt}{\isachardot}{\kern0pt}{\isachardot}{\kern0pt}{\isacharless}{\kern0pt}{\isadigit{0}}{\isacharcolon}{\kern0pt}{\isacharcolon}{\kern0pt}nat{\isacharbraceright}{\kern0pt}{\isachardot}{\kern0pt}\ {\isasymchi}\ y\ {\isacharequal}{\kern0pt}\ c{\isacharparenright}{\kern0pt}{\isacharparenright}{\kern0pt}{\isachardoublequoteclose}\ \isakeyword{if}\ {\isachardoublequoteopen}N{\isacharprime}{\kern0pt}\ {\isasymge}\ {\isadigit{1}}{\isachardoublequoteclose}\ {\isachardoublequoteopen}{\isasymchi}\ {\isasymin}\ cube\ N{\isacharprime}{\kern0pt}\ {\isadigit{0}}\ {\isasymrightarrow}\isactrlsub E\ {\isacharbraceleft}{\kern0pt}{\isachardot}{\kern0pt}{\isachardot}{\kern0pt}{\isacharless}{\kern0pt}r{\isacharbraceright}{\kern0pt}{\isachardoublequoteclose}\ \isakeyword{for}\ N{\isacharprime}{\kern0pt}\ {\isasymchi}\isanewline
\ \ \ \ \isacommand{using}\isamarkupfalse%
\ assms\ is{\isacharunderscore}{\kern0pt}line{\isacharunderscore}{\kern0pt}def\ that{\isacharparenleft}{\kern0pt}{\isadigit{1}}{\isacharparenright}{\kern0pt}\ \isacommand{by}\isamarkupfalse%
\ fastforce\isanewline
\ \ \isacommand{then}\isamarkupfalse%
\ \isacommand{show}\isamarkupfalse%
\ {\isacharquery}{\kern0pt}thesis\ \isacommand{unfolding}\isamarkupfalse%
\ hj{\isacharunderscore}{\kern0pt}def\ \isacommand{by}\isamarkupfalse%
\ auto\isanewline
\isacommand{qed}\isamarkupfalse%
%
\endisatagproof
{\isafoldproof}%
%
\isadelimproof
\isanewline
%
\endisadelimproof
\isanewline
\isacommand{lemma}\isamarkupfalse%
\ single{\isacharunderscore}{\kern0pt}point{\isacharunderscore}{\kern0pt}line{\isacharcolon}{\kern0pt}\ \isakeyword{assumes}\ {\isachardoublequoteopen}N\ {\isachargreater}{\kern0pt}\ {\isadigit{0}}{\isachardoublequoteclose}\ \isakeyword{shows}\ {\isachardoublequoteopen}is{\isacharunderscore}{\kern0pt}line\ {\isacharparenleft}{\kern0pt}{\isasymlambda}s{\isasymin}{\isacharbraceleft}{\kern0pt}{\isachardot}{\kern0pt}{\isachardot}{\kern0pt}{\isacharless}{\kern0pt}{\isadigit{1}}{\isacharbraceright}{\kern0pt}{\isachardot}{\kern0pt}\ {\isasymlambda}a{\isasymin}{\isacharbraceleft}{\kern0pt}{\isachardot}{\kern0pt}{\isachardot}{\kern0pt}{\isacharless}{\kern0pt}N{\isacharbraceright}{\kern0pt}{\isachardot}{\kern0pt}\ {\isadigit{0}}{\isacharparenright}{\kern0pt}\ N\ {\isadigit{1}}{\isachardoublequoteclose}\isanewline
%
\isadelimproof
\ \ %
\endisadelimproof
%
\isatagproof
\isacommand{using}\isamarkupfalse%
\ assms\ \isacommand{unfolding}\isamarkupfalse%
\ is{\isacharunderscore}{\kern0pt}line{\isacharunderscore}{\kern0pt}def\ cube{\isacharunderscore}{\kern0pt}def\ \isacommand{by}\isamarkupfalse%
\ auto%
\endisatagproof
{\isafoldproof}%
%
\isadelimproof
\isanewline
%
\endisadelimproof
\isanewline
\isacommand{lemma}\isamarkupfalse%
\ single{\isacharunderscore}{\kern0pt}point{\isacharunderscore}{\kern0pt}line{\isacharunderscore}{\kern0pt}is{\isacharunderscore}{\kern0pt}monochromatic{\isacharcolon}{\kern0pt}\ \isakeyword{assumes}\ {\isachardoublequoteopen}{\isasymchi}\ {\isasymin}\ cube\ N\ {\isadigit{1}}\ {\isasymrightarrow}\isactrlsub E\ {\isacharbraceleft}{\kern0pt}{\isachardot}{\kern0pt}{\isachardot}{\kern0pt}{\isacharless}{\kern0pt}r{\isacharbraceright}{\kern0pt}{\isachardoublequoteclose}\ {\isachardoublequoteopen}N\ {\isachargreater}{\kern0pt}\ {\isadigit{0}}{\isachardoublequoteclose}\ \isakeyword{shows}\ {\isachardoublequoteopen}{\isacharparenleft}{\kern0pt}{\isasymexists}c\ {\isacharless}{\kern0pt}\ r{\isachardot}{\kern0pt}\ is{\isacharunderscore}{\kern0pt}line\ {\isacharparenleft}{\kern0pt}{\isasymlambda}s{\isasymin}{\isacharbraceleft}{\kern0pt}{\isachardot}{\kern0pt}{\isachardot}{\kern0pt}{\isacharless}{\kern0pt}{\isadigit{1}}{\isacharbraceright}{\kern0pt}{\isachardot}{\kern0pt}\ {\isasymlambda}a{\isasymin}{\isacharbraceleft}{\kern0pt}{\isachardot}{\kern0pt}{\isachardot}{\kern0pt}{\isacharless}{\kern0pt}N{\isacharbraceright}{\kern0pt}{\isachardot}{\kern0pt}\ {\isadigit{0}}{\isacharparenright}{\kern0pt}\ N\ {\isadigit{1}}\ {\isasymand}\ {\isacharparenleft}{\kern0pt}{\isasymforall}i\ {\isasymin}\ \ {\isacharparenleft}{\kern0pt}{\isasymlambda}s{\isasymin}{\isacharbraceleft}{\kern0pt}{\isachardot}{\kern0pt}{\isachardot}{\kern0pt}{\isacharless}{\kern0pt}{\isadigit{1}}{\isacharbraceright}{\kern0pt}{\isachardot}{\kern0pt}\ {\isasymlambda}a{\isasymin}{\isacharbraceleft}{\kern0pt}{\isachardot}{\kern0pt}{\isachardot}{\kern0pt}{\isacharless}{\kern0pt}N{\isacharbraceright}{\kern0pt}{\isachardot}{\kern0pt}\ {\isadigit{0}}{\isacharparenright}{\kern0pt}\ {\isacharbackquote}{\kern0pt}\ {\isacharbraceleft}{\kern0pt}{\isachardot}{\kern0pt}{\isachardot}{\kern0pt}{\isacharless}{\kern0pt}{\isadigit{1}}{\isacharbraceright}{\kern0pt}{\isachardot}{\kern0pt}\ {\isasymchi}\ i\ {\isacharequal}{\kern0pt}\ c{\isacharparenright}{\kern0pt}{\isacharparenright}{\kern0pt}{\isachardoublequoteclose}\isanewline
%
\isadelimproof
%
\endisadelimproof
%
\isatagproof
\isacommand{proof}\isamarkupfalse%
\ {\isacharminus}{\kern0pt}\isanewline
\ \ \isacommand{have}\isamarkupfalse%
\ {\isachardoublequoteopen}is{\isacharunderscore}{\kern0pt}line\ {\isacharparenleft}{\kern0pt}{\isasymlambda}s{\isasymin}{\isacharbraceleft}{\kern0pt}{\isachardot}{\kern0pt}{\isachardot}{\kern0pt}{\isacharless}{\kern0pt}{\isadigit{1}}{\isacharbraceright}{\kern0pt}{\isachardot}{\kern0pt}\ {\isasymlambda}a{\isasymin}{\isacharbraceleft}{\kern0pt}{\isachardot}{\kern0pt}{\isachardot}{\kern0pt}{\isacharless}{\kern0pt}N{\isacharbraceright}{\kern0pt}{\isachardot}{\kern0pt}\ {\isadigit{0}}{\isacharparenright}{\kern0pt}\ N\ {\isadigit{1}}{\isachardoublequoteclose}\ \isacommand{using}\isamarkupfalse%
\ assms{\isacharparenleft}{\kern0pt}{\isadigit{2}}{\isacharparenright}{\kern0pt}\ single{\isacharunderscore}{\kern0pt}point{\isacharunderscore}{\kern0pt}line\ \isacommand{by}\isamarkupfalse%
\ blast\isanewline
\ \ \isacommand{moreover}\isamarkupfalse%
\ \isacommand{have}\isamarkupfalse%
\ {\isachardoublequoteopen}{\isasymexists}c\ {\isacharless}{\kern0pt}\ r{\isachardot}{\kern0pt}\ {\isasymchi}\ {\isacharparenleft}{\kern0pt}{\isacharparenleft}{\kern0pt}{\isasymlambda}s{\isasymin}{\isacharbraceleft}{\kern0pt}{\isachardot}{\kern0pt}{\isachardot}{\kern0pt}{\isacharless}{\kern0pt}{\isadigit{1}}{\isacharbraceright}{\kern0pt}{\isachardot}{\kern0pt}\ {\isasymlambda}a{\isasymin}{\isacharbraceleft}{\kern0pt}{\isachardot}{\kern0pt}{\isachardot}{\kern0pt}{\isacharless}{\kern0pt}N{\isacharbraceright}{\kern0pt}{\isachardot}{\kern0pt}\ {\isadigit{0}}{\isacharparenright}{\kern0pt}\ j{\isacharparenright}{\kern0pt}\ {\isacharequal}{\kern0pt}\ c{\isachardoublequoteclose}\ \isakeyword{if}\ {\isachardoublequoteopen}{\isacharparenleft}{\kern0pt}j{\isacharcolon}{\kern0pt}{\isacharcolon}{\kern0pt}nat{\isacharparenright}{\kern0pt}\ {\isacharless}{\kern0pt}\ {\isadigit{1}}{\isachardoublequoteclose}\ \isakeyword{for}\ j\ \isacommand{using}\isamarkupfalse%
\ assms\ line{\isacharunderscore}{\kern0pt}points{\isacharunderscore}{\kern0pt}in{\isacharunderscore}{\kern0pt}cube\ calculation\ that\ \isacommand{unfolding}\isamarkupfalse%
\ cube{\isacharunderscore}{\kern0pt}def\ \isacommand{by}\isamarkupfalse%
\ blast\isanewline
\ \ \isacommand{ultimately}\isamarkupfalse%
\ \isacommand{show}\isamarkupfalse%
\ {\isacharquery}{\kern0pt}thesis\ \isacommand{by}\isamarkupfalse%
\ auto\isanewline
\isacommand{qed}\isamarkupfalse%
%
\endisatagproof
{\isafoldproof}%
%
\isadelimproof
\isanewline
%
\endisadelimproof
\isanewline
\isacommand{lemma}\isamarkupfalse%
\ hj{\isacharunderscore}{\kern0pt}t{\isacharunderscore}{\kern0pt}{\isadigit{1}}{\isacharcolon}{\kern0pt}\ {\isachardoublequoteopen}hj\ r\ {\isadigit{1}}{\isachardoublequoteclose}\isanewline
%
\isadelimproof
\ \ %
\endisadelimproof
%
\isatagproof
\isacommand{unfolding}\isamarkupfalse%
\ hj{\isacharunderscore}{\kern0pt}def\ \isacommand{using}\isamarkupfalse%
\ single{\isacharunderscore}{\kern0pt}point{\isacharunderscore}{\kern0pt}line{\isacharunderscore}{\kern0pt}is{\isacharunderscore}{\kern0pt}monochromatic\ le{\isacharunderscore}{\kern0pt}zero{\isacharunderscore}{\kern0pt}eq\ not{\isacharunderscore}{\kern0pt}le\isanewline
\ \ \isacommand{by}\isamarkupfalse%
\ {\isacharparenleft}{\kern0pt}metis\ less{\isacharunderscore}{\kern0pt}numeral{\isacharunderscore}{\kern0pt}extra{\isacharparenleft}{\kern0pt}{\isadigit{1}}{\isacharparenright}{\kern0pt}{\isacharparenright}{\kern0pt}%
\endisatagproof
{\isafoldproof}%
%
\isadelimproof
\isanewline
%
\endisadelimproof
\isanewline
\isacommand{lemma}\isamarkupfalse%
\ hales{\isacharunderscore}{\kern0pt}jewett{\isacharcolon}{\kern0pt}\ {\isachardoublequoteopen}{\isasymnot}{\isacharparenleft}{\kern0pt}r\ {\isacharequal}{\kern0pt}\ {\isadigit{0}}\ {\isasymand}\ t\ {\isacharequal}{\kern0pt}\ {\isadigit{0}}{\isacharparenright}{\kern0pt}\ {\isasymLongrightarrow}\ hj\ r\ t{\isachardoublequoteclose}\isanewline
%
\isadelimproof
%
\endisadelimproof
%
\isatagproof
\isacommand{proof}\isamarkupfalse%
\ {\isacharparenleft}{\kern0pt}induction\ t\ arbitrary{\isacharcolon}{\kern0pt}\ r{\isacharparenright}{\kern0pt}\isanewline
\ \ \isacommand{case}\isamarkupfalse%
\ {\isadigit{0}}\isanewline
\ \ \isacommand{then}\isamarkupfalse%
\ \isacommand{show}\isamarkupfalse%
\ {\isacharquery}{\kern0pt}case\ \isacommand{using}\isamarkupfalse%
\ hj{\isacharunderscore}{\kern0pt}r{\isacharunderscore}{\kern0pt}nonzero{\isacharunderscore}{\kern0pt}t{\isacharunderscore}{\kern0pt}{\isadigit{0}}\ \isacommand{by}\isamarkupfalse%
\ blast\isanewline
\isacommand{next}\isamarkupfalse%
\isanewline
\ \ \isacommand{case}\isamarkupfalse%
\ {\isacharparenleft}{\kern0pt}Suc\ t{\isacharparenright}{\kern0pt}\isanewline
\ \ \isacommand{then}\isamarkupfalse%
\ \isacommand{show}\isamarkupfalse%
\ {\isacharquery}{\kern0pt}case\ \isacommand{using}\isamarkupfalse%
\ hj{\isacharunderscore}{\kern0pt}t{\isacharunderscore}{\kern0pt}{\isadigit{1}}\ theorem{\isadigit{4}}\ corollary{\isadigit{6}}\ \isacommand{by}\isamarkupfalse%
\ {\isacharparenleft}{\kern0pt}metis\ One{\isacharunderscore}{\kern0pt}nat{\isacharunderscore}{\kern0pt}def\ Suc{\isacharunderscore}{\kern0pt}eq{\isacharunderscore}{\kern0pt}plus{\isadigit{1}}\ neq{\isadigit{0}}{\isacharunderscore}{\kern0pt}conv{\isacharparenright}{\kern0pt}\isanewline
\isacommand{qed}\isamarkupfalse%
%
\endisatagproof
{\isafoldproof}%
%
\isadelimproof
\isanewline
%
\endisadelimproof
\isanewline
\isacommand{unused{\isacharunderscore}{\kern0pt}thms}\isamarkupfalse%
\isanewline
%
\isadelimtheory
\isanewline
%
\endisadelimtheory
%
\isatagtheory
\isacommand{end}\isamarkupfalse%
%
\endisatagtheory
{\isafoldtheory}%
%
\isadelimtheory
%
\endisadelimtheory
%
\end{isabellebody}%
\endinput
%:%file=~/Desktop/uni/bachelor/informatik/7sem/bthesis/Hales-Jewett.thy%:%
%:%10=1%:%
%:%11=1%:%
%:%12=2%:%
%:%13=3%:%
%:%27=5%:%
%:%39=7%:%
%:%40=8%:%
%:%41=9%:%
%:%42=10%:%
%:%51=13%:%
%:%63=15%:%
%:%67=17%:%
%:%68=18%:%
%:%69=19%:%
%:%70=20%:%
%:%71=21%:%
%:%72=22%:%
%:%74=23%:%
%:%75=23%:%
%:%76=24%:%
%:%77=25%:%
%:%78=26%:%
%:%79=26%:%
%:%82=27%:%
%:%86=27%:%
%:%87=27%:%
%:%88=27%:%
%:%97=29%:%
%:%98=30%:%
%:%100=31%:%
%:%101=31%:%
%:%104=32%:%
%:%108=32%:%
%:%109=32%:%
%:%110=32%:%
%:%115=32%:%
%:%118=33%:%
%:%119=34%:%
%:%120=35%:%
%:%121=35%:%
%:%123=35%:%
%:%127=35%:%
%:%128=35%:%
%:%129=35%:%
%:%130=35%:%
%:%139=37%:%
%:%140=38%:%
%:%141=39%:%
%:%142=40%:%
%:%143=41%:%
%:%144=42%:%
%:%148=43%:%
%:%150=44%:%
%:%151=44%:%
%:%154=45%:%
%:%158=45%:%
%:%159=45%:%
%:%160=45%:%
%:%161=46%:%
%:%162=46%:%
%:%171=48%:%
%:%173=49%:%
%:%174=49%:%
%:%177=50%:%
%:%181=50%:%
%:%182=50%:%
%:%183=50%:%
%:%192=52%:%
%:%194=53%:%
%:%195=53%:%
%:%198=54%:%
%:%202=54%:%
%:%203=54%:%
%:%212=56%:%
%:%214=57%:%
%:%215=57%:%
%:%217=57%:%
%:%221=57%:%
%:%222=57%:%
%:%223=57%:%
%:%237=59%:%
%:%249=61%:%
%:%250=62%:%
%:%251=63%:%
%:%252=64%:%
%:%253=65%:%
%:%254=66%:%
%:%255=67%:%
%:%257=68%:%
%:%258=68%:%
%:%259=69%:%
%:%261=72%:%
%:%263=73%:%
%:%264=73%:%
%:%265=74%:%
%:%266=75%:%
%:%267=76%:%
%:%274=77%:%
%:%275=77%:%
%:%276=78%:%
%:%277=78%:%
%:%278=79%:%
%:%279=79%:%
%:%280=80%:%
%:%281=80%:%
%:%282=80%:%
%:%283=80%:%
%:%284=81%:%
%:%285=81%:%
%:%286=81%:%
%:%287=81%:%
%:%288=81%:%
%:%289=82%:%
%:%290=82%:%
%:%291=82%:%
%:%292=82%:%
%:%293=82%:%
%:%294=82%:%
%:%295=83%:%
%:%296=83%:%
%:%297=83%:%
%:%298=83%:%
%:%299=83%:%
%:%300=84%:%
%:%301=84%:%
%:%302=84%:%
%:%303=84%:%
%:%304=84%:%
%:%305=84%:%
%:%306=85%:%
%:%307=85%:%
%:%308=85%:%
%:%309=85%:%
%:%310=85%:%
%:%311=86%:%
%:%321=89%:%
%:%323=90%:%
%:%324=90%:%
%:%327=91%:%
%:%331=91%:%
%:%332=91%:%
%:%333=91%:%
%:%334=92%:%
%:%335=92%:%
%:%340=92%:%
%:%343=93%:%
%:%344=94%:%
%:%345=94%:%
%:%348=95%:%
%:%352=95%:%
%:%353=95%:%
%:%354=95%:%
%:%355=95%:%
%:%360=95%:%
%:%363=96%:%
%:%364=97%:%
%:%365=98%:%
%:%366=98%:%
%:%367=99%:%
%:%368=100%:%
%:%369=101%:%
%:%370=102%:%
%:%371=103%:%
%:%372=103%:%
%:%375=104%:%
%:%379=104%:%
%:%380=104%:%
%:%381=104%:%
%:%386=104%:%
%:%389=105%:%
%:%390=105%:%
%:%392=105%:%
%:%396=105%:%
%:%397=105%:%
%:%398=105%:%
%:%399=106%:%
%:%400=106%:%
%:%407=106%:%
%:%408=107%:%
%:%409=108%:%
%:%410=109%:%
%:%411=109%:%
%:%414=110%:%
%:%418=110%:%
%:%419=110%:%
%:%424=110%:%
%:%427=111%:%
%:%428=111%:%
%:%429=112%:%
%:%430=113%:%
%:%433=114%:%
%:%437=114%:%
%:%438=114%:%
%:%439=114%:%
%:%448=117%:%
%:%450=118%:%
%:%451=118%:%
%:%454=119%:%
%:%458=119%:%
%:%459=119%:%
%:%460=119%:%
%:%461=119%:%
%:%475=121%:%
%:%487=123%:%
%:%488=124%:%
%:%489=125%:%
%:%490=126%:%
%:%491=127%:%
%:%492=128%:%
%:%493=129%:%
%:%494=130%:%
%:%496=131%:%
%:%497=131%:%
%:%498=132%:%
%:%501=135%:%
%:%502=136%:%
%:%504=137%:%
%:%505=137%:%
%:%512=138%:%
%:%513=138%:%
%:%514=139%:%
%:%515=139%:%
%:%516=140%:%
%:%517=140%:%
%:%518=141%:%
%:%519=141%:%
%:%520=141%:%
%:%521=141%:%
%:%522=141%:%
%:%523=142%:%
%:%524=142%:%
%:%525=143%:%
%:%526=143%:%
%:%527=144%:%
%:%528=144%:%
%:%529=144%:%
%:%530=145%:%
%:%531=145%:%
%:%532=145%:%
%:%533=145%:%
%:%534=145%:%
%:%535=146%:%
%:%536=146%:%
%:%537=146%:%
%:%538=146%:%
%:%539=147%:%
%:%540=147%:%
%:%541=147%:%
%:%542=147%:%
%:%543=147%:%
%:%544=148%:%
%:%545=148%:%
%:%546=148%:%
%:%547=148%:%
%:%548=148%:%
%:%549=149%:%
%:%550=149%:%
%:%551=149%:%
%:%552=149%:%
%:%553=149%:%
%:%554=150%:%
%:%555=150%:%
%:%556=150%:%
%:%557=150%:%
%:%558=151%:%
%:%559=151%:%
%:%560=152%:%
%:%561=152%:%
%:%562=152%:%
%:%563=152%:%
%:%564=152%:%
%:%565=152%:%
%:%566=153%:%
%:%576=155%:%
%:%578=156%:%
%:%579=156%:%
%:%586=157%:%
%:%587=157%:%
%:%588=158%:%
%:%589=158%:%
%:%590=159%:%
%:%591=160%:%
%:%592=160%:%
%:%593=160%:%
%:%594=160%:%
%:%595=161%:%
%:%596=161%:%
%:%597=161%:%
%:%598=162%:%
%:%599=162%:%
%:%600=163%:%
%:%601=163%:%
%:%602=163%:%
%:%603=163%:%
%:%604=164%:%
%:%605=164%:%
%:%606=165%:%
%:%607=166%:%
%:%608=166%:%
%:%609=166%:%
%:%610=166%:%
%:%611=167%:%
%:%612=167%:%
%:%613=167%:%
%:%614=167%:%
%:%615=167%:%
%:%616=168%:%
%:%617=168%:%
%:%618=168%:%
%:%619=168%:%
%:%620=169%:%
%:%621=169%:%
%:%622=169%:%
%:%623=170%:%
%:%624=170%:%
%:%625=170%:%
%:%626=170%:%
%:%627=171%:%
%:%628=171%:%
%:%629=171%:%
%:%630=171%:%
%:%631=171%:%
%:%632=172%:%
%:%633=172%:%
%:%634=172%:%
%:%635=172%:%
%:%636=172%:%
%:%637=172%:%
%:%638=173%:%
%:%639=173%:%
%:%640=173%:%
%:%641=173%:%
%:%642=174%:%
%:%657=176%:%
%:%669=177%:%
%:%671=178%:%
%:%672=178%:%
%:%673=179%:%
%:%674=180%:%
%:%675=181%:%
%:%676=181%:%
%:%678=181%:%
%:%682=181%:%
%:%683=181%:%
%:%684=181%:%
%:%691=181%:%
%:%692=182%:%
%:%693=183%:%
%:%694=183%:%
%:%695=184%:%
%:%696=185%:%
%:%697=186%:%
%:%698=187%:%
%:%699=187%:%
%:%706=188%:%
%:%707=188%:%
%:%708=189%:%
%:%709=189%:%
%:%710=190%:%
%:%711=190%:%
%:%712=191%:%
%:%713=191%:%
%:%714=191%:%
%:%715=191%:%
%:%716=191%:%
%:%717=191%:%
%:%718=192%:%
%:%719=192%:%
%:%720=192%:%
%:%721=192%:%
%:%722=193%:%
%:%728=193%:%
%:%731=194%:%
%:%732=195%:%
%:%733=195%:%
%:%740=196%:%
%:%741=196%:%
%:%742=197%:%
%:%743=197%:%
%:%744=197%:%
%:%745=197%:%
%:%746=198%:%
%:%747=198%:%
%:%748=198%:%
%:%749=198%:%
%:%750=199%:%
%:%751=199%:%
%:%752=199%:%
%:%753=200%:%
%:%754=200%:%
%:%755=201%:%
%:%756=201%:%
%:%757=202%:%
%:%758=202%:%
%:%759=202%:%
%:%760=203%:%
%:%761=203%:%
%:%762=203%:%
%:%763=204%:%
%:%764=204%:%
%:%765=204%:%
%:%766=204%:%
%:%767=204%:%
%:%768=204%:%
%:%769=205%:%
%:%770=205%:%
%:%771=205%:%
%:%772=205%:%
%:%773=205%:%
%:%774=206%:%
%:%775=206%:%
%:%776=206%:%
%:%777=206%:%
%:%778=207%:%
%:%779=207%:%
%:%780=208%:%
%:%781=208%:%
%:%782=208%:%
%:%783=208%:%
%:%784=208%:%
%:%785=208%:%
%:%786=209%:%
%:%787=209%:%
%:%788=209%:%
%:%789=209%:%
%:%790=210%:%
%:%800=212%:%
%:%802=214%:%
%:%803=214%:%
%:%804=215%:%
%:%805=216%:%
%:%808=217%:%
%:%812=217%:%
%:%813=217%:%
%:%814=217%:%
%:%819=217%:%
%:%822=218%:%
%:%823=219%:%
%:%824=219%:%
%:%831=220%:%
%:%832=220%:%
%:%833=221%:%
%:%834=221%:%
%:%835=222%:%
%:%836=222%:%
%:%837=222%:%
%:%838=222%:%
%:%839=223%:%
%:%840=223%:%
%:%841=223%:%
%:%842=223%:%
%:%843=223%:%
%:%844=224%:%
%:%845=224%:%
%:%846=224%:%
%:%847=224%:%
%:%848=225%:%
%:%854=225%:%
%:%857=226%:%
%:%858=227%:%
%:%859=227%:%
%:%866=228%:%
%:%867=228%:%
%:%868=229%:%
%:%869=229%:%
%:%870=230%:%
%:%871=230%:%
%:%872=231%:%
%:%873=231%:%
%:%874=232%:%
%:%875=232%:%
%:%876=233%:%
%:%877=233%:%
%:%878=233%:%
%:%879=233%:%
%:%880=233%:%
%:%881=234%:%
%:%882=234%:%
%:%883=235%:%
%:%884=235%:%
%:%885=236%:%
%:%886=236%:%
%:%887=237%:%
%:%888=237%:%
%:%889=237%:%
%:%890=237%:%
%:%891=238%:%
%:%892=238%:%
%:%893=238%:%
%:%894=238%:%
%:%895=239%:%
%:%896=239%:%
%:%897=239%:%
%:%898=239%:%
%:%899=239%:%
%:%900=240%:%
%:%901=240%:%
%:%902=240%:%
%:%903=240%:%
%:%904=240%:%
%:%905=241%:%
%:%906=241%:%
%:%907=242%:%
%:%908=242%:%
%:%909=243%:%
%:%910=243%:%
%:%911=244%:%
%:%912=244%:%
%:%913=244%:%
%:%914=244%:%
%:%915=245%:%
%:%916=245%:%
%:%917=245%:%
%:%918=245%:%
%:%919=246%:%
%:%920=246%:%
%:%921=246%:%
%:%922=246%:%
%:%923=246%:%
%:%924=247%:%
%:%925=247%:%
%:%926=248%:%
%:%927=248%:%
%:%928=248%:%
%:%929=248%:%
%:%930=249%:%
%:%940=251%:%
%:%942=252%:%
%:%943=252%:%
%:%946=253%:%
%:%950=253%:%
%:%951=253%:%
%:%956=253%:%
%:%959=254%:%
%:%960=254%:%
%:%967=255%:%
%:%968=255%:%
%:%969=256%:%
%:%970=256%:%
%:%971=257%:%
%:%972=257%:%
%:%973=258%:%
%:%974=258%:%
%:%975=259%:%
%:%976=259%:%
%:%977=260%:%
%:%978=260%:%
%:%979=260%:%
%:%980=260%:%
%:%981=260%:%
%:%982=261%:%
%:%983=261%:%
%:%984=262%:%
%:%985=262%:%
%:%986=263%:%
%:%987=263%:%
%:%988=264%:%
%:%989=264%:%
%:%990=265%:%
%:%991=265%:%
%:%992=266%:%
%:%993=266%:%
%:%994=267%:%
%:%995=267%:%
%:%996=267%:%
%:%997=268%:%
%:%998=268%:%
%:%999=268%:%
%:%1000=268%:%
%:%1001=268%:%
%:%1002=268%:%
%:%1003=269%:%
%:%1004=269%:%
%:%1005=270%:%
%:%1006=270%:%
%:%1007=270%:%
%:%1008=270%:%
%:%1009=270%:%
%:%1010=271%:%
%:%1011=271%:%
%:%1012=271%:%
%:%1013=271%:%
%:%1014=272%:%
%:%1015=272%:%
%:%1016=273%:%
%:%1017=273%:%
%:%1018=274%:%
%:%1019=274%:%
%:%1020=275%:%
%:%1021=275%:%
%:%1022=275%:%
%:%1023=275%:%
%:%1024=276%:%
%:%1025=276%:%
%:%1026=276%:%
%:%1027=276%:%
%:%1028=277%:%
%:%1029=277%:%
%:%1030=277%:%
%:%1031=277%:%
%:%1032=277%:%
%:%1033=278%:%
%:%1034=278%:%
%:%1035=279%:%
%:%1036=279%:%
%:%1037=279%:%
%:%1038=279%:%
%:%1039=280%:%
%:%1049=282%:%
%:%1051=283%:%
%:%1052=283%:%
%:%1053=284%:%
%:%1054=285%:%
%:%1055=286%:%
%:%1058=287%:%
%:%1062=287%:%
%:%1063=287%:%
%:%1064=287%:%
%:%1073=289%:%
%:%1075=290%:%
%:%1076=290%:%
%:%1083=291%:%
%:%1084=291%:%
%:%1085=292%:%
%:%1086=292%:%
%:%1087=292%:%
%:%1088=292%:%
%:%1089=293%:%
%:%1090=294%:%
%:%1091=294%:%
%:%1092=295%:%
%:%1093=295%:%
%:%1094=296%:%
%:%1095=296%:%
%:%1096=296%:%
%:%1097=296%:%
%:%1098=296%:%
%:%1099=297%:%
%:%1100=297%:%
%:%1101=297%:%
%:%1102=297%:%
%:%1103=298%:%
%:%1104=298%:%
%:%1105=299%:%
%:%1106=299%:%
%:%1107=299%:%
%:%1108=299%:%
%:%1109=299%:%
%:%1110=299%:%
%:%1111=300%:%
%:%1112=300%:%
%:%1113=300%:%
%:%1114=300%:%
%:%1115=300%:%
%:%1116=300%:%
%:%1117=301%:%
%:%1118=301%:%
%:%1119=301%:%
%:%1120=301%:%
%:%1121=301%:%
%:%1122=301%:%
%:%1123=302%:%
%:%1124=302%:%
%:%1125=302%:%
%:%1126=302%:%
%:%1127=302%:%
%:%1128=302%:%
%:%1129=303%:%
%:%1130=303%:%
%:%1131=303%:%
%:%1132=303%:%
%:%1133=303%:%
%:%1134=303%:%
%:%1135=304%:%
%:%1136=304%:%
%:%1137=304%:%
%:%1138=304%:%
%:%1139=304%:%
%:%1140=305%:%
%:%1146=305%:%
%:%1149=306%:%
%:%1150=307%:%
%:%1151=307%:%
%:%1158=308%:%
%:%1159=308%:%
%:%1160=309%:%
%:%1161=309%:%
%:%1162=310%:%
%:%1163=310%:%
%:%1164=311%:%
%:%1165=311%:%
%:%1166=312%:%
%:%1167=312%:%
%:%1168=313%:%
%:%1169=313%:%
%:%1170=314%:%
%:%1171=314%:%
%:%1172=314%:%
%:%1173=314%:%
%:%1174=314%:%
%:%1175=315%:%
%:%1176=316%:%
%:%1177=316%:%
%:%1178=316%:%
%:%1179=316%:%
%:%1180=316%:%
%:%1181=317%:%
%:%1182=317%:%
%:%1183=318%:%
%:%1184=318%:%
%:%1185=318%:%
%:%1186=319%:%
%:%1187=319%:%
%:%1188=319%:%
%:%1189=320%:%
%:%1190=320%:%
%:%1191=320%:%
%:%1192=320%:%
%:%1193=320%:%
%:%1194=321%:%
%:%1195=321%:%
%:%1196=322%:%
%:%1197=322%:%
%:%1198=323%:%
%:%1199=323%:%
%:%1200=324%:%
%:%1201=324%:%
%:%1202=325%:%
%:%1203=325%:%
%:%1204=325%:%
%:%1205=326%:%
%:%1206=326%:%
%:%1207=326%:%
%:%1208=327%:%
%:%1209=327%:%
%:%1210=328%:%
%:%1211=328%:%
%:%1212=329%:%
%:%1213=329%:%
%:%1214=329%:%
%:%1215=329%:%
%:%1216=330%:%
%:%1217=331%:%
%:%1218=331%:%
%:%1219=332%:%
%:%1220=332%:%
%:%1221=333%:%
%:%1222=333%:%
%:%1223=333%:%
%:%1224=334%:%
%:%1225=334%:%
%:%1226=334%:%
%:%1227=334%:%
%:%1228=335%:%
%:%1229=335%:%
%:%1230=335%:%
%:%1231=335%:%
%:%1232=335%:%
%:%1233=336%:%
%:%1234=336%:%
%:%1235=336%:%
%:%1236=336%:%
%:%1237=337%:%
%:%1238=337%:%
%:%1239=338%:%
%:%1240=338%:%
%:%1241=339%:%
%:%1242=339%:%
%:%1243=340%:%
%:%1244=340%:%
%:%1245=341%:%
%:%1246=341%:%
%:%1247=341%:%
%:%1248=341%:%
%:%1249=342%:%
%:%1250=342%:%
%:%1251=342%:%
%:%1252=342%:%
%:%1253=342%:%
%:%1254=342%:%
%:%1255=343%:%
%:%1256=343%:%
%:%1257=344%:%
%:%1258=344%:%
%:%1259=345%:%
%:%1260=345%:%
%:%1261=346%:%
%:%1262=346%:%
%:%1263=347%:%
%:%1264=347%:%
%:%1265=347%:%
%:%1266=347%:%
%:%1267=348%:%
%:%1268=348%:%
%:%1269=348%:%
%:%1270=348%:%
%:%1271=349%:%
%:%1272=349%:%
%:%1273=350%:%
%:%1274=350%:%
%:%1275=351%:%
%:%1276=351%:%
%:%1277=352%:%
%:%1278=352%:%
%:%1279=353%:%
%:%1280=353%:%
%:%1281=354%:%
%:%1282=354%:%
%:%1283=355%:%
%:%1284=355%:%
%:%1285=356%:%
%:%1286=356%:%
%:%1287=356%:%
%:%1288=356%:%
%:%1289=356%:%
%:%1290=357%:%
%:%1291=357%:%
%:%1292=357%:%
%:%1293=357%:%
%:%1294=357%:%
%:%1295=358%:%
%:%1296=358%:%
%:%1297=358%:%
%:%1298=358%:%
%:%1299=358%:%
%:%1300=359%:%
%:%1301=359%:%
%:%1302=360%:%
%:%1303=360%:%
%:%1304=361%:%
%:%1305=362%:%
%:%1306=362%:%
%:%1307=363%:%
%:%1308=363%:%
%:%1309=364%:%
%:%1310=364%:%
%:%1311=364%:%
%:%1312=364%:%
%:%1313=365%:%
%:%1314=365%:%
%:%1315=365%:%
%:%1316=366%:%
%:%1317=366%:%
%:%1318=366%:%
%:%1319=367%:%
%:%1320=367%:%
%:%1321=368%:%
%:%1322=368%:%
%:%1323=369%:%
%:%1324=369%:%
%:%1325=370%:%
%:%1326=370%:%
%:%1327=371%:%
%:%1328=371%:%
%:%1329=372%:%
%:%1330=372%:%
%:%1331=373%:%
%:%1332=373%:%
%:%1333=374%:%
%:%1334=374%:%
%:%1335=374%:%
%:%1336=374%:%
%:%1337=374%:%
%:%1338=375%:%
%:%1339=376%:%
%:%1340=376%:%
%:%1341=377%:%
%:%1342=377%:%
%:%1343=378%:%
%:%1344=378%:%
%:%1345=379%:%
%:%1346=379%:%
%:%1347=380%:%
%:%1348=380%:%
%:%1349=380%:%
%:%1350=381%:%
%:%1351=381%:%
%:%1352=382%:%
%:%1353=382%:%
%:%1354=382%:%
%:%1355=382%:%
%:%1356=382%:%
%:%1357=383%:%
%:%1358=383%:%
%:%1359=383%:%
%:%1360=384%:%
%:%1361=384%:%
%:%1362=385%:%
%:%1363=385%:%
%:%1364=385%:%
%:%1365=385%:%
%:%1366=385%:%
%:%1367=386%:%
%:%1368=386%:%
%:%1369=386%:%
%:%1370=386%:%
%:%1371=387%:%
%:%1372=387%:%
%:%1373=387%:%
%:%1374=387%:%
%:%1375=387%:%
%:%1376=388%:%
%:%1377=388%:%
%:%1378=389%:%
%:%1379=389%:%
%:%1380=389%:%
%:%1381=389%:%
%:%1382=390%:%
%:%1383=390%:%
%:%1384=391%:%
%:%1385=392%:%
%:%1386=392%:%
%:%1387=392%:%
%:%1388=393%:%
%:%1389=393%:%
%:%1390=394%:%
%:%1391=394%:%
%:%1392=394%:%
%:%1393=394%:%
%:%1394=395%:%
%:%1395=395%:%
%:%1396=395%:%
%:%1397=395%:%
%:%1398=395%:%
%:%1399=396%:%
%:%1400=396%:%
%:%1401=396%:%
%:%1402=396%:%
%:%1403=397%:%
%:%1404=397%:%
%:%1405=397%:%
%:%1406=397%:%
%:%1407=397%:%
%:%1408=397%:%
%:%1409=398%:%
%:%1410=398%:%
%:%1411=398%:%
%:%1412=398%:%
%:%1413=399%:%
%:%1414=399%:%
%:%1415=399%:%
%:%1416=399%:%
%:%1417=399%:%
%:%1418=399%:%
%:%1419=400%:%
%:%1420=400%:%
%:%1421=401%:%
%:%1422=402%:%
%:%1423=402%:%
%:%1424=402%:%
%:%1425=403%:%
%:%1426=403%:%
%:%1427=404%:%
%:%1428=404%:%
%:%1429=405%:%
%:%1430=405%:%
%:%1431=406%:%
%:%1432=406%:%
%:%1433=406%:%
%:%1434=406%:%
%:%1435=406%:%
%:%1436=407%:%
%:%1442=407%:%
%:%1445=408%:%
%:%1446=409%:%
%:%1447=409%:%
%:%1450=410%:%
%:%1454=410%:%
%:%1455=410%:%
%:%1456=410%:%
%:%1461=410%:%
%:%1464=411%:%
%:%1465=412%:%
%:%1466=412%:%
%:%1467=413%:%
%:%1468=414%:%
%:%1469=415%:%
%:%1470=415%:%
%:%1471=416%:%
%:%1473=419%:%
%:%1475=421%:%
%:%1476=421%:%
%:%1477=422%:%
%:%1478=423%:%
%:%1479=424%:%
%:%1480=425%:%
%:%1483=426%:%
%:%1488=427%:%
%:%1489=427%:%
%:%1490=428%:%
%:%1491=428%:%
%:%1492=428%:%
%:%1493=428%:%
%:%1494=428%:%
%:%1495=429%:%
%:%1496=430%:%
%:%1497=430%:%
%:%1498=431%:%
%:%1499=431%:%
%:%1500=432%:%
%:%1501=432%:%
%:%1502=433%:%
%:%1503=433%:%
%:%1504=434%:%
%:%1505=434%:%
%:%1506=434%:%
%:%1507=434%:%
%:%1508=434%:%
%:%1509=435%:%
%:%1510=435%:%
%:%1511=436%:%
%:%1512=436%:%
%:%1513=436%:%
%:%1514=436%:%
%:%1515=437%:%
%:%1516=437%:%
%:%1517=437%:%
%:%1518=437%:%
%:%1519=437%:%
%:%1520=438%:%
%:%1521=439%:%
%:%1522=439%:%
%:%1523=439%:%
%:%1524=440%:%
%:%1525=440%:%
%:%1526=440%:%
%:%1527=441%:%
%:%1528=441%:%
%:%1529=441%:%
%:%1530=441%:%
%:%1531=441%:%
%:%1532=442%:%
%:%1533=443%:%
%:%1534=443%:%
%:%1535=443%:%
%:%1536=444%:%
%:%1537=444%:%
%:%1538=444%:%
%:%1539=444%:%
%:%1540=444%:%
%:%1541=444%:%
%:%1542=445%:%
%:%1543=445%:%
%:%1544=446%:%
%:%1545=446%:%
%:%1546=447%:%
%:%1547=447%:%
%:%1548=448%:%
%:%1549=448%:%
%:%1550=449%:%
%:%1551=449%:%
%:%1552=450%:%
%:%1553=450%:%
%:%1554=451%:%
%:%1555=451%:%
%:%1556=452%:%
%:%1557=452%:%
%:%1558=452%:%
%:%1559=453%:%
%:%1560=453%:%
%:%1561=454%:%
%:%1562=454%:%
%:%1563=455%:%
%:%1564=455%:%
%:%1565=455%:%
%:%1566=455%:%
%:%1567=456%:%
%:%1568=456%:%
%:%1569=456%:%
%:%1570=456%:%
%:%1571=456%:%
%:%1572=456%:%
%:%1573=457%:%
%:%1574=457%:%
%:%1575=457%:%
%:%1576=457%:%
%:%1577=457%:%
%:%1578=458%:%
%:%1579=458%:%
%:%1580=459%:%
%:%1581=459%:%
%:%1582=460%:%
%:%1583=460%:%
%:%1584=460%:%
%:%1585=460%:%
%:%1586=460%:%
%:%1587=461%:%
%:%1588=461%:%
%:%1589=462%:%
%:%1590=462%:%
%:%1591=463%:%
%:%1592=463%:%
%:%1593=464%:%
%:%1594=464%:%
%:%1595=465%:%
%:%1596=465%:%
%:%1597=465%:%
%:%1598=466%:%
%:%1599=466%:%
%:%1600=466%:%
%:%1601=467%:%
%:%1602=467%:%
%:%1603=468%:%
%:%1604=468%:%
%:%1605=469%:%
%:%1606=469%:%
%:%1607=469%:%
%:%1608=470%:%
%:%1609=470%:%
%:%1610=471%:%
%:%1611=471%:%
%:%1612=471%:%
%:%1613=472%:%
%:%1614=472%:%
%:%1615=473%:%
%:%1616=473%:%
%:%1617=473%:%
%:%1618=473%:%
%:%1619=473%:%
%:%1620=473%:%
%:%1621=474%:%
%:%1622=474%:%
%:%1623=474%:%
%:%1624=474%:%
%:%1625=474%:%
%:%1626=475%:%
%:%1627=475%:%
%:%1628=476%:%
%:%1629=476%:%
%:%1630=477%:%
%:%1631=477%:%
%:%1632=477%:%
%:%1633=477%:%
%:%1634=477%:%
%:%1635=477%:%
%:%1636=478%:%
%:%1637=478%:%
%:%1638=478%:%
%:%1639=478%:%
%:%1640=479%:%
%:%1641=479%:%
%:%1642=479%:%
%:%1643=479%:%
%:%1644=480%:%
%:%1645=480%:%
%:%1646=481%:%
%:%1647=481%:%
%:%1648=482%:%
%:%1649=482%:%
%:%1650=482%:%
%:%1651=483%:%
%:%1652=483%:%
%:%1653=483%:%
%:%1654=483%:%
%:%1655=484%:%
%:%1656=484%:%
%:%1657=484%:%
%:%1658=484%:%
%:%1659=484%:%
%:%1660=484%:%
%:%1661=485%:%
%:%1662=485%:%
%:%1663=485%:%
%:%1664=485%:%
%:%1665=485%:%
%:%1666=486%:%
%:%1667=486%:%
%:%1668=486%:%
%:%1669=486%:%
%:%1670=486%:%
%:%1671=487%:%
%:%1672=487%:%
%:%1673=487%:%
%:%1674=487%:%
%:%1675=488%:%
%:%1676=488%:%
%:%1677=489%:%
%:%1678=489%:%
%:%1679=490%:%
%:%1680=490%:%
%:%1681=491%:%
%:%1682=491%:%
%:%1683=492%:%
%:%1684=492%:%
%:%1685=492%:%
%:%1686=492%:%
%:%1687=492%:%
%:%1688=492%:%
%:%1689=493%:%
%:%1690=493%:%
%:%1691=493%:%
%:%1692=493%:%
%:%1693=493%:%
%:%1694=494%:%
%:%1695=494%:%
%:%1696=495%:%
%:%1697=495%:%
%:%1698=496%:%
%:%1699=496%:%
%:%1700=497%:%
%:%1701=497%:%
%:%1702=498%:%
%:%1703=498%:%
%:%1704=499%:%
%:%1705=499%:%
%:%1706=499%:%
%:%1707=499%:%
%:%1708=500%:%
%:%1709=500%:%
%:%1710=500%:%
%:%1711=500%:%
%:%1712=500%:%
%:%1713=500%:%
%:%1714=501%:%
%:%1715=501%:%
%:%1716=502%:%
%:%1717=503%:%
%:%1718=504%:%
%:%1719=504%:%
%:%1720=505%:%
%:%1721=505%:%
%:%1722=506%:%
%:%1723=506%:%
%:%1724=507%:%
%:%1725=507%:%
%:%1726=507%:%
%:%1727=507%:%
%:%1728=508%:%
%:%1729=508%:%
%:%1730=508%:%
%:%1731=508%:%
%:%1732=508%:%
%:%1733=509%:%
%:%1734=509%:%
%:%1735=509%:%
%:%1736=509%:%
%:%1737=509%:%
%:%1738=510%:%
%:%1739=510%:%
%:%1740=510%:%
%:%1741=510%:%
%:%1742=510%:%
%:%1743=511%:%
%:%1744=511%:%
%:%1745=511%:%
%:%1746=511%:%
%:%1747=511%:%
%:%1748=512%:%
%:%1749=512%:%
%:%1750=512%:%
%:%1751=512%:%
%:%1752=512%:%
%:%1753=512%:%
%:%1754=513%:%
%:%1755=513%:%
%:%1756=514%:%
%:%1757=514%:%
%:%1758=515%:%
%:%1759=515%:%
%:%1760=515%:%
%:%1761=516%:%
%:%1762=516%:%
%:%1763=517%:%
%:%1764=517%:%
%:%1765=517%:%
%:%1766=517%:%
%:%1767=517%:%
%:%1768=518%:%
%:%1769=518%:%
%:%1770=518%:%
%:%1771=518%:%
%:%1772=519%:%
%:%1773=519%:%
%:%1774=520%:%
%:%1775=520%:%
%:%1776=521%:%
%:%1777=521%:%
%:%1778=522%:%
%:%1779=522%:%
%:%1780=522%:%
%:%1781=522%:%
%:%1782=522%:%
%:%1783=523%:%
%:%1784=523%:%
%:%1785=523%:%
%:%1786=523%:%
%:%1787=523%:%
%:%1788=524%:%
%:%1789=524%:%
%:%1790=524%:%
%:%1791=524%:%
%:%1792=524%:%
%:%1793=524%:%
%:%1794=525%:%
%:%1795=525%:%
%:%1796=525%:%
%:%1797=526%:%
%:%1798=526%:%
%:%1799=527%:%
%:%1800=527%:%
%:%1801=528%:%
%:%1802=528%:%
%:%1803=528%:%
%:%1804=528%:%
%:%1805=528%:%
%:%1806=529%:%
%:%1807=529%:%
%:%1808=529%:%
%:%1809=529%:%
%:%1810=529%:%
%:%1811=530%:%
%:%1812=530%:%
%:%1813=530%:%
%:%1814=531%:%
%:%1815=531%:%
%:%1816=532%:%
%:%1817=532%:%
%:%1818=533%:%
%:%1819=533%:%
%:%1820=533%:%
%:%1821=533%:%
%:%1822=534%:%
%:%1823=534%:%
%:%1824=534%:%
%:%1825=535%:%
%:%1826=535%:%
%:%1827=535%:%
%:%1828=536%:%
%:%1829=536%:%
%:%1830=537%:%
%:%1831=537%:%
%:%1832=538%:%
%:%1833=538%:%
%:%1834=538%:%
%:%1835=538%:%
%:%1836=538%:%
%:%1837=538%:%
%:%1838=539%:%
%:%1839=539%:%
%:%1840=540%:%
%:%1841=540%:%
%:%1842=541%:%
%:%1843=541%:%
%:%1844=541%:%
%:%1845=541%:%
%:%1846=541%:%
%:%1847=541%:%
%:%1848=542%:%
%:%1849=542%:%
%:%1850=542%:%
%:%1851=542%:%
%:%1852=542%:%
%:%1853=542%:%
%:%1854=543%:%
%:%1855=543%:%
%:%1856=543%:%
%:%1857=543%:%
%:%1858=543%:%
%:%1859=544%:%
%:%1860=544%:%
%:%1861=545%:%
%:%1862=545%:%
%:%1863=546%:%
%:%1864=546%:%
%:%1865=546%:%
%:%1866=546%:%
%:%1867=547%:%
%:%1868=547%:%
%:%1869=548%:%
%:%1870=549%:%
%:%1871=549%:%
%:%1872=549%:%
%:%1873=549%:%
%:%1874=549%:%
%:%1875=550%:%
%:%1876=550%:%
%:%1877=550%:%
%:%1878=550%:%
%:%1879=550%:%
%:%1880=551%:%
%:%1881=551%:%
%:%1882=551%:%
%:%1883=551%:%
%:%1884=552%:%
%:%1885=552%:%
%:%1886=553%:%
%:%1887=553%:%
%:%1888=554%:%
%:%1889=554%:%
%:%1890=554%:%
%:%1891=554%:%
%:%1892=554%:%
%:%1893=555%:%
%:%1894=556%:%
%:%1895=556%:%
%:%1896=556%:%
%:%1897=556%:%
%:%1898=556%:%
%:%1899=557%:%
%:%1900=557%:%
%:%1901=557%:%
%:%1902=557%:%
%:%1903=557%:%
%:%1904=558%:%
%:%1905=558%:%
%:%1906=558%:%
%:%1907=558%:%
%:%1908=558%:%
%:%1909=559%:%
%:%1910=559%:%
%:%1911=560%:%
%:%1912=560%:%
%:%1913=561%:%
%:%1914=562%:%
%:%1915=563%:%
%:%1916=563%:%
%:%1917=564%:%
%:%1918=565%:%
%:%1919=565%:%
%:%1920=566%:%
%:%1921=566%:%
%:%1922=567%:%
%:%1923=567%:%
%:%1924=568%:%
%:%1925=568%:%
%:%1926=569%:%
%:%1927=569%:%
%:%1928=569%:%
%:%1929=569%:%
%:%1930=569%:%
%:%1931=570%:%
%:%1932=570%:%
%:%1933=570%:%
%:%1934=570%:%
%:%1935=570%:%
%:%1936=570%:%
%:%1937=571%:%
%:%1938=571%:%
%:%1939=571%:%
%:%1940=572%:%
%:%1941=572%:%
%:%1942=573%:%
%:%1943=573%:%
%:%1944=574%:%
%:%1945=574%:%
%:%1946=574%:%
%:%1947=574%:%
%:%1948=574%:%
%:%1949=575%:%
%:%1950=575%:%
%:%1951=575%:%
%:%1952=575%:%
%:%1953=575%:%
%:%1954=576%:%
%:%1955=576%:%
%:%1956=576%:%
%:%1957=577%:%
%:%1958=577%:%
%:%1959=578%:%
%:%1960=578%:%
%:%1961=579%:%
%:%1962=579%:%
%:%1963=579%:%
%:%1964=579%:%
%:%1965=580%:%
%:%1966=580%:%
%:%1967=580%:%
%:%1968=581%:%
%:%1969=581%:%
%:%1970=581%:%
%:%1971=582%:%
%:%1972=582%:%
%:%1973=583%:%
%:%1974=583%:%
%:%1975=584%:%
%:%1976=584%:%
%:%1977=584%:%
%:%1978=584%:%
%:%1979=584%:%
%:%1980=584%:%
%:%1981=585%:%
%:%1982=585%:%
%:%1983=586%:%
%:%1984=586%:%
%:%1985=587%:%
%:%1986=587%:%
%:%1987=587%:%
%:%1988=587%:%
%:%1989=587%:%
%:%1990=587%:%
%:%1991=588%:%
%:%1992=588%:%
%:%1993=588%:%
%:%1994=588%:%
%:%1995=588%:%
%:%1996=588%:%
%:%1997=589%:%
%:%1998=589%:%
%:%1999=589%:%
%:%2000=589%:%
%:%2001=589%:%
%:%2002=590%:%
%:%2003=590%:%
%:%2004=591%:%
%:%2005=591%:%
%:%2006=592%:%
%:%2007=592%:%
%:%2008=592%:%
%:%2009=592%:%
%:%2010=593%:%
%:%2011=594%:%
%:%2012=594%:%
%:%2013=595%:%
%:%2014=595%:%
%:%2015=595%:%
%:%2016=595%:%
%:%2017=595%:%
%:%2018=596%:%
%:%2019=596%:%
%:%2020=596%:%
%:%2021=596%:%
%:%2022=596%:%
%:%2023=597%:%
%:%2024=597%:%
%:%2025=597%:%
%:%2026=597%:%
%:%2027=597%:%
%:%2028=597%:%
%:%2029=598%:%
%:%2030=598%:%
%:%2031=598%:%
%:%2032=598%:%
%:%2033=598%:%
%:%2034=599%:%
%:%2035=600%:%
%:%2036=600%:%
%:%2037=600%:%
%:%2038=600%:%
%:%2039=601%:%
%:%2040=601%:%
%:%2041=601%:%
%:%2042=601%:%
%:%2043=601%:%
%:%2044=602%:%
%:%2045=602%:%
%:%2046=602%:%
%:%2047=602%:%
%:%2048=602%:%
%:%2049=603%:%
%:%2050=603%:%
%:%2051=604%:%
%:%2052=604%:%
%:%2053=605%:%
%:%2054=605%:%
%:%2055=605%:%
%:%2056=605%:%
%:%2057=606%:%
%:%2058=606%:%
%:%2059=606%:%
%:%2060=607%:%
%:%2061=607%:%
%:%2062=608%:%
%:%2063=608%:%
%:%2064=609%:%
%:%2065=609%:%
%:%2066=609%:%
%:%2067=609%:%
%:%2068=609%:%
%:%2069=610%:%
%:%2070=610%:%
%:%2071=611%:%
%:%2072=611%:%
%:%2073=612%:%
%:%2074=612%:%
%:%2075=612%:%
%:%2076=612%:%
%:%2077=612%:%
%:%2078=613%:%
%:%2079=613%:%
%:%2080=614%:%
%:%2081=614%:%
%:%2082=615%:%
%:%2083=615%:%
%:%2084=615%:%
%:%2085=615%:%
%:%2086=616%:%
%:%2087=616%:%
%:%2088=617%:%
%:%2089=617%:%
%:%2090=618%:%
%:%2091=618%:%
%:%2092=618%:%
%:%2093=618%:%
%:%2094=618%:%
%:%2095=619%:%
%:%2096=619%:%
%:%2097=620%:%
%:%2098=620%:%
%:%2099=621%:%
%:%2100=621%:%
%:%2101=621%:%
%:%2102=621%:%
%:%2103=622%:%
%:%2104=622%:%
%:%2105=623%:%
%:%2106=623%:%
%:%2107=624%:%
%:%2108=624%:%
%:%2109=624%:%
%:%2110=624%:%
%:%2111=624%:%
%:%2112=625%:%
%:%2113=625%:%
%:%2114=625%:%
%:%2115=625%:%
%:%2116=625%:%
%:%2117=626%:%
%:%2118=626%:%
%:%2119=626%:%
%:%2120=626%:%
%:%2121=627%:%
%:%2122=627%:%
%:%2123=627%:%
%:%2124=628%:%
%:%2125=628%:%
%:%2126=629%:%
%:%2127=629%:%
%:%2128=630%:%
%:%2129=630%:%
%:%2130=630%:%
%:%2131=630%:%
%:%2132=630%:%
%:%2133=631%:%
%:%2134=631%:%
%:%2135=631%:%
%:%2136=631%:%
%:%2137=631%:%
%:%2138=632%:%
%:%2139=632%:%
%:%2140=632%:%
%:%2141=633%:%
%:%2142=633%:%
%:%2143=634%:%
%:%2144=634%:%
%:%2145=635%:%
%:%2146=635%:%
%:%2147=635%:%
%:%2148=635%:%
%:%2149=636%:%
%:%2150=636%:%
%:%2151=636%:%
%:%2152=637%:%
%:%2153=637%:%
%:%2154=637%:%
%:%2155=638%:%
%:%2156=638%:%
%:%2157=639%:%
%:%2158=639%:%
%:%2159=640%:%
%:%2160=640%:%
%:%2161=640%:%
%:%2162=640%:%
%:%2163=640%:%
%:%2164=640%:%
%:%2165=641%:%
%:%2166=641%:%
%:%2167=642%:%
%:%2168=642%:%
%:%2169=643%:%
%:%2170=643%:%
%:%2171=643%:%
%:%2172=643%:%
%:%2173=643%:%
%:%2174=643%:%
%:%2175=644%:%
%:%2176=644%:%
%:%2177=644%:%
%:%2178=644%:%
%:%2179=644%:%
%:%2180=644%:%
%:%2181=645%:%
%:%2182=645%:%
%:%2183=645%:%
%:%2184=645%:%
%:%2185=645%:%
%:%2186=646%:%
%:%2187=646%:%
%:%2188=647%:%
%:%2189=647%:%
%:%2190=648%:%
%:%2191=648%:%
%:%2192=648%:%
%:%2193=648%:%
%:%2194=649%:%
%:%2195=650%:%
%:%2196=650%:%
%:%2197=651%:%
%:%2198=651%:%
%:%2199=651%:%
%:%2200=651%:%
%:%2201=651%:%
%:%2202=652%:%
%:%2203=652%:%
%:%2204=652%:%
%:%2205=652%:%
%:%2206=652%:%
%:%2207=653%:%
%:%2208=653%:%
%:%2209=653%:%
%:%2210=653%:%
%:%2211=653%:%
%:%2212=654%:%
%:%2213=654%:%
%:%2214=654%:%
%:%2215=654%:%
%:%2216=654%:%
%:%2217=655%:%
%:%2218=655%:%
%:%2219=655%:%
%:%2220=655%:%
%:%2221=656%:%
%:%2222=656%:%
%:%2223=657%:%
%:%2224=658%:%
%:%2225=659%:%
%:%2226=660%:%
%:%2227=661%:%
%:%2228=661%:%
%:%2229=662%:%
%:%2230=662%:%
%:%2231=662%:%
%:%2232=662%:%
%:%2233=662%:%
%:%2234=663%:%
%:%2235=664%:%
%:%2236=665%:%
%:%2237=665%:%
%:%2238=666%:%
%:%2239=666%:%
%:%2240=666%:%
%:%2241=666%:%
%:%2242=666%:%
%:%2243=667%:%
%:%2244=668%:%
%:%2245=668%:%
%:%2246=669%:%
%:%2247=669%:%
%:%2248=670%:%
%:%2249=670%:%
%:%2250=671%:%
%:%2251=671%:%
%:%2252=671%:%
%:%2253=671%:%
%:%2254=671%:%
%:%2255=672%:%
%:%2256=672%:%
%:%2257=672%:%
%:%2258=673%:%
%:%2259=673%:%
%:%2260=673%:%
%:%2261=674%:%
%:%2262=674%:%
%:%2263=674%:%
%:%2264=674%:%
%:%2265=675%:%
%:%2266=675%:%
%:%2267=675%:%
%:%2268=675%:%
%:%2269=676%:%
%:%2270=676%:%
%:%2271=676%:%
%:%2272=676%:%
%:%2273=677%:%
%:%2274=677%:%
%:%2275=677%:%
%:%2276=677%:%
%:%2277=677%:%
%:%2278=678%:%
%:%2279=678%:%
%:%2280=678%:%
%:%2281=678%:%
%:%2282=679%:%
%:%2283=679%:%
%:%2284=679%:%
%:%2285=679%:%
%:%2286=680%:%
%:%2287=680%:%
%:%2288=680%:%
%:%2289=681%:%
%:%2290=681%:%
%:%2291=682%:%
%:%2292=682%:%
%:%2293=683%:%
%:%2294=683%:%
%:%2295=684%:%
%:%2296=684%:%
%:%2297=684%:%
%:%2298=685%:%
%:%2299=685%:%
%:%2300=685%:%
%:%2301=685%:%
%:%2302=685%:%
%:%2303=686%:%
%:%2304=686%:%
%:%2305=686%:%
%:%2306=686%:%
%:%2307=687%:%
%:%2308=687%:%
%:%2309=687%:%
%:%2310=687%:%
%:%2311=688%:%
%:%2312=688%:%
%:%2313=688%:%
%:%2314=688%:%
%:%2315=688%:%
%:%2316=689%:%
%:%2317=689%:%
%:%2318=689%:%
%:%2319=689%:%
%:%2320=689%:%
%:%2321=690%:%
%:%2322=690%:%
%:%2323=690%:%
%:%2324=690%:%
%:%2325=691%:%
%:%2326=691%:%
%:%2327=691%:%
%:%2328=691%:%
%:%2329=692%:%
%:%2330=692%:%
%:%2331=692%:%
%:%2332=693%:%
%:%2333=693%:%
%:%2334=694%:%
%:%2335=694%:%
%:%2336=694%:%
%:%2337=695%:%
%:%2338=695%:%
%:%2339=696%:%
%:%2340=696%:%
%:%2341=696%:%
%:%2342=697%:%
%:%2343=697%:%
%:%2344=697%:%
%:%2345=698%:%
%:%2346=698%:%
%:%2347=698%:%
%:%2348=698%:%
%:%2349=699%:%
%:%2350=699%:%
%:%2351=700%:%
%:%2352=700%:%
%:%2353=700%:%
%:%2354=700%:%
%:%2355=701%:%
%:%2356=702%:%
%:%2357=702%:%
%:%2358=703%:%
%:%2359=703%:%
%:%2360=704%:%
%:%2361=704%:%
%:%2362=705%:%
%:%2363=705%:%
%:%2364=706%:%
%:%2365=706%:%
%:%2366=706%:%
%:%2367=706%:%
%:%2368=706%:%
%:%2369=707%:%
%:%2370=707%:%
%:%2371=707%:%
%:%2372=707%:%
%:%2373=707%:%
%:%2374=708%:%
%:%2375=708%:%
%:%2376=708%:%
%:%2377=708%:%
%:%2378=708%:%
%:%2379=708%:%
%:%2380=709%:%
%:%2381=709%:%
%:%2382=709%:%
%:%2383=709%:%
%:%2384=710%:%
%:%2385=710%:%
%:%2386=710%:%
%:%2387=711%:%
%:%2388=711%:%
%:%2389=712%:%
%:%2390=712%:%
%:%2391=712%:%
%:%2392=712%:%
%:%2393=712%:%
%:%2394=713%:%
%:%2395=713%:%
%:%2396=714%:%
%:%2397=714%:%
%:%2398=715%:%
%:%2399=715%:%
%:%2400=716%:%
%:%2401=716%:%
%:%2402=716%:%
%:%2403=716%:%
%:%2404=717%:%
%:%2405=717%:%
%:%2406=717%:%
%:%2407=717%:%
%:%2408=718%:%
%:%2409=718%:%
%:%2410=718%:%
%:%2411=718%:%
%:%2412=719%:%
%:%2413=719%:%
%:%2414=719%:%
%:%2415=719%:%
%:%2416=719%:%
%:%2417=720%:%
%:%2418=720%:%
%:%2419=720%:%
%:%2420=720%:%
%:%2421=721%:%
%:%2422=721%:%
%:%2423=721%:%
%:%2424=721%:%
%:%2425=722%:%
%:%2426=722%:%
%:%2427=722%:%
%:%2428=722%:%
%:%2429=723%:%
%:%2430=723%:%
%:%2431=723%:%
%:%2432=724%:%
%:%2433=724%:%
%:%2434=724%:%
%:%2435=725%:%
%:%2436=726%:%
%:%2437=726%:%
%:%2438=726%:%
%:%2439=727%:%
%:%2440=727%:%
%:%2441=728%:%
%:%2442=728%:%
%:%2443=729%:%
%:%2444=729%:%
%:%2445=729%:%
%:%2446=729%:%
%:%2447=730%:%
%:%2448=730%:%
%:%2449=730%:%
%:%2450=730%:%
%:%2451=730%:%
%:%2452=731%:%
%:%2453=731%:%
%:%2454=731%:%
%:%2455=732%:%
%:%2456=732%:%
%:%2457=733%:%
%:%2458=733%:%
%:%2459=734%:%
%:%2460=734%:%
%:%2461=734%:%
%:%2462=734%:%
%:%2463=735%:%
%:%2464=735%:%
%:%2465=735%:%
%:%2466=736%:%
%:%2467=736%:%
%:%2468=736%:%
%:%2469=737%:%
%:%2470=737%:%
%:%2471=738%:%
%:%2472=738%:%
%:%2473=739%:%
%:%2474=739%:%
%:%2475=739%:%
%:%2476=739%:%
%:%2477=739%:%
%:%2478=739%:%
%:%2479=740%:%
%:%2480=740%:%
%:%2481=741%:%
%:%2482=741%:%
%:%2483=742%:%
%:%2484=742%:%
%:%2485=742%:%
%:%2486=742%:%
%:%2487=742%:%
%:%2488=742%:%
%:%2489=743%:%
%:%2490=743%:%
%:%2491=743%:%
%:%2492=743%:%
%:%2493=743%:%
%:%2494=743%:%
%:%2495=744%:%
%:%2496=744%:%
%:%2497=744%:%
%:%2498=744%:%
%:%2499=744%:%
%:%2500=745%:%
%:%2501=745%:%
%:%2502=746%:%
%:%2503=746%:%
%:%2504=747%:%
%:%2505=747%:%
%:%2506=747%:%
%:%2507=747%:%
%:%2508=748%:%
%:%2509=748%:%
%:%2510=749%:%
%:%2511=749%:%
%:%2512=749%:%
%:%2513=749%:%
%:%2514=749%:%
%:%2515=750%:%
%:%2516=750%:%
%:%2517=750%:%
%:%2518=750%:%
%:%2519=751%:%
%:%2520=752%:%
%:%2521=752%:%
%:%2522=753%:%
%:%2523=753%:%
%:%2524=754%:%
%:%2525=754%:%
%:%2526=754%:%
%:%2527=754%:%
%:%2528=755%:%
%:%2529=755%:%
%:%2530=755%:%
%:%2531=755%:%
%:%2532=756%:%
%:%2533=756%:%
%:%2534=756%:%
%:%2535=756%:%
%:%2536=756%:%
%:%2537=756%:%
%:%2538=757%:%
%:%2539=757%:%
%:%2540=757%:%
%:%2541=757%:%
%:%2542=757%:%
%:%2543=757%:%
%:%2544=758%:%
%:%2545=758%:%
%:%2546=758%:%
%:%2547=758%:%
%:%2548=758%:%
%:%2549=758%:%
%:%2550=759%:%
%:%2551=759%:%
%:%2552=759%:%
%:%2553=759%:%
%:%2554=760%:%
%:%2555=760%:%
%:%2556=761%:%
%:%2557=761%:%
%:%2558=761%:%
%:%2559=761%:%
%:%2560=761%:%
%:%2561=762%:%
%:%2562=762%:%
%:%2563=763%:%
%:%2564=764%:%
%:%2565=765%:%
%:%2566=765%:%
%:%2567=766%:%
%:%2568=766%:%
%:%2569=766%:%
%:%2570=766%:%
%:%2571=766%:%
%:%2572=767%:%
%:%2573=767%:%
%:%2574=768%:%
%:%2575=768%:%
%:%2576=768%:%
%:%2577=768%:%
%:%2578=768%:%
%:%2579=768%:%
%:%2580=769%:%
%:%2590=772%:%
%:%2592=773%:%
%:%2593=773%:%
%:%2594=774%:%
%:%2596=780%:%
%:%2598=781%:%
%:%2599=781%:%
%:%2600=782%:%
%:%2601=783%:%
%:%2602=784%:%
%:%2603=785%:%
%:%2604=786%:%
%:%2611=787%:%
%:%2612=787%:%
%:%2613=788%:%
%:%2614=788%:%
%:%2615=788%:%
%:%2616=789%:%
%:%2617=789%:%
%:%2618=789%:%
%:%2619=789%:%
%:%2620=789%:%
%:%2621=790%:%
%:%2622=790%:%
%:%2623=791%:%
%:%2624=791%:%
%:%2625=791%:%
%:%2626=791%:%
%:%2627=791%:%
%:%2628=792%:%
%:%2629=792%:%
%:%2630=793%:%
%:%2631=793%:%
%:%2632=793%:%
%:%2633=793%:%
%:%2634=794%:%
%:%2635=794%:%
%:%2636=795%:%
%:%2637=795%:%
%:%2638=795%:%
%:%2639=795%:%
%:%2640=796%:%
%:%2650=798%:%
%:%2652=799%:%
%:%2653=799%:%
%:%2654=800%:%
%:%2655=801%:%
%:%2656=802%:%
%:%2657=803%:%
%:%2664=804%:%
%:%2665=804%:%
%:%2666=805%:%
%:%2667=805%:%
%:%2668=805%:%
%:%2669=805%:%
%:%2670=806%:%
%:%2671=806%:%
%:%2672=807%:%
%:%2673=807%:%
%:%2674=808%:%
%:%2675=808%:%
%:%2676=809%:%
%:%2677=809%:%
%:%2678=810%:%
%:%2679=810%:%
%:%2680=810%:%
%:%2681=811%:%
%:%2682=811%:%
%:%2683=811%:%
%:%2684=812%:%
%:%2685=812%:%
%:%2686=812%:%
%:%2687=812%:%
%:%2688=812%:%
%:%2689=813%:%
%:%2690=813%:%
%:%2691=814%:%
%:%2692=815%:%
%:%2693=815%:%
%:%2694=815%:%
%:%2695=815%:%
%:%2696=816%:%
%:%2697=816%:%
%:%2698=816%:%
%:%2699=816%:%
%:%2700=817%:%
%:%2701=817%:%
%:%2702=817%:%
%:%2703=817%:%
%:%2704=818%:%
%:%2710=818%:%
%:%2713=819%:%
%:%2714=820%:%
%:%2715=820%:%
%:%2716=821%:%
%:%2717=822%:%
%:%2718=823%:%
%:%2725=824%:%
%:%2726=824%:%
%:%2727=825%:%
%:%2728=825%:%
%:%2729=826%:%
%:%2730=826%:%
%:%2731=826%:%
%:%2732=827%:%
%:%2733=827%:%
%:%2734=827%:%
%:%2735=827%:%
%:%2736=827%:%
%:%2737=828%:%
%:%2738=828%:%
%:%2739=828%:%
%:%2740=828%:%
%:%2741=828%:%
%:%2742=829%:%
%:%2743=830%:%
%:%2744=830%:%
%:%2745=831%:%
%:%2746=831%:%
%:%2747=832%:%
%:%2748=832%:%
%:%2749=832%:%
%:%2750=833%:%
%:%2751=833%:%
%:%2752=833%:%
%:%2753=833%:%
%:%2754=833%:%
%:%2755=834%:%
%:%2756=834%:%
%:%2757=834%:%
%:%2758=834%:%
%:%2759=834%:%
%:%2760=835%:%
%:%2761=835%:%
%:%2762=835%:%
%:%2763=836%:%
%:%2764=836%:%
%:%2765=836%:%
%:%2766=837%:%
%:%2767=837%:%
%:%2768=837%:%
%:%2769=837%:%
%:%2770=838%:%
%:%2771=838%:%
%:%2772=839%:%
%:%2773=839%:%
%:%2774=839%:%
%:%2775=840%:%
%:%2776=840%:%
%:%2777=840%:%
%:%2778=840%:%
%:%2779=840%:%
%:%2780=841%:%
%:%2781=841%:%
%:%2782=842%:%
%:%2783=842%:%
%:%2784=842%:%
%:%2785=843%:%
%:%2786=843%:%
%:%2787=844%:%
%:%2788=844%:%
%:%2789=844%:%
%:%2790=844%:%
%:%2791=845%:%
%:%2792=845%:%
%:%2793=845%:%
%:%2794=846%:%
%:%2795=846%:%
%:%2796=847%:%
%:%2797=847%:%
%:%2798=848%:%
%:%2799=848%:%
%:%2800=849%:%
%:%2801=849%:%
%:%2802=850%:%
%:%2803=850%:%
%:%2804=850%:%
%:%2805=850%:%
%:%2806=851%:%
%:%2807=851%:%
%:%2808=851%:%
%:%2809=851%:%
%:%2810=851%:%
%:%2811=851%:%
%:%2812=852%:%
%:%2813=852%:%
%:%2814=853%:%
%:%2815=853%:%
%:%2816=853%:%
%:%2817=853%:%
%:%2818=854%:%
%:%2819=854%:%
%:%2820=855%:%
%:%2821=855%:%
%:%2822=856%:%
%:%2823=856%:%
%:%2824=857%:%
%:%2825=857%:%
%:%2826=858%:%
%:%2827=858%:%
%:%2828=858%:%
%:%2829=858%:%
%:%2830=859%:%
%:%2831=859%:%
%:%2832=859%:%
%:%2833=859%:%
%:%2834=859%:%
%:%2835=859%:%
%:%2836=860%:%
%:%2837=860%:%
%:%2838=860%:%
%:%2839=860%:%
%:%2840=860%:%
%:%2841=860%:%
%:%2842=860%:%
%:%2843=861%:%
%:%2844=861%:%
%:%2845=861%:%
%:%2846=861%:%
%:%2847=862%:%
%:%2848=862%:%
%:%2849=863%:%
%:%2850=863%:%
%:%2851=863%:%
%:%2852=863%:%
%:%2853=864%:%
%:%2854=864%:%
%:%2855=865%:%
%:%2856=865%:%
%:%2857=866%:%
%:%2858=866%:%
%:%2859=866%:%
%:%2860=867%:%
%:%2861=867%:%
%:%2862=868%:%
%:%2863=868%:%
%:%2864=868%:%
%:%2865=868%:%
%:%2866=869%:%
%:%2867=869%:%
%:%2868=869%:%
%:%2869=869%:%
%:%2870=869%:%
%:%2871=870%:%
%:%2872=870%:%
%:%2873=870%:%
%:%2874=870%:%
%:%2875=870%:%
%:%2876=870%:%
%:%2877=871%:%
%:%2878=871%:%
%:%2879=871%:%
%:%2880=871%:%
%:%2881=871%:%
%:%2882=872%:%
%:%2883=872%:%
%:%2884=873%:%
%:%2885=873%:%
%:%2886=873%:%
%:%2887=873%:%
%:%2888=873%:%
%:%2889=874%:%
%:%2895=874%:%
%:%2898=875%:%
%:%2899=876%:%
%:%2900=876%:%
%:%2903=877%:%
%:%2907=877%:%
%:%2908=877%:%
%:%2909=877%:%
%:%2914=877%:%
%:%2917=878%:%
%:%2918=879%:%
%:%2919=879%:%
%:%2920=880%:%
%:%2921=881%:%
%:%2922=882%:%
%:%2925=883%:%
%:%2929=883%:%
%:%2930=883%:%
%:%2931=883%:%
%:%2932=884%:%
%:%2933=884%:%
%:%2934=884%:%
%:%2939=884%:%
%:%2942=885%:%
%:%2943=886%:%
%:%2944=887%:%
%:%2945=888%:%
%:%2946=889%:%
%:%2947=889%:%
%:%2950=890%:%
%:%2954=890%:%
%:%2955=890%:%
%:%2956=890%:%
%:%2957=890%:%
%:%2962=890%:%
%:%2965=891%:%
%:%2966=892%:%
%:%2967=892%:%
%:%2974=893%:%
%:%2975=893%:%
%:%2976=894%:%
%:%2977=894%:%
%:%2978=894%:%
%:%2979=894%:%
%:%2980=895%:%
%:%2981=895%:%
%:%2982=895%:%
%:%2983=895%:%
%:%2984=895%:%
%:%2985=896%:%
%:%2986=896%:%
%:%2987=897%:%
%:%2988=897%:%
%:%2989=898%:%
%:%2990=898%:%
%:%2991=898%:%
%:%2992=898%:%
%:%2993=899%:%
%:%2994=899%:%
%:%2995=899%:%
%:%2996=899%:%
%:%2997=899%:%
%:%2998=899%:%
%:%2999=900%:%
%:%3000=900%:%
%:%3001=901%:%
%:%3002=901%:%
%:%3003=901%:%
%:%3004=901%:%
%:%3005=901%:%
%:%3006=901%:%
%:%3007=902%:%
%:%3008=902%:%
%:%3009=902%:%
%:%3010=902%:%
%:%3011=902%:%
%:%3012=903%:%
%:%3018=903%:%
%:%3021=904%:%
%:%3022=905%:%
%:%3023=905%:%
%:%3024=906%:%
%:%3025=907%:%
%:%3026=908%:%
%:%3033=909%:%
%:%3034=909%:%
%:%3035=910%:%
%:%3036=910%:%
%:%3037=911%:%
%:%3038=911%:%
%:%3039=912%:%
%:%3040=912%:%
%:%3041=912%:%
%:%3042=912%:%
%:%3043=912%:%
%:%3044=913%:%
%:%3045=913%:%
%:%3046=913%:%
%:%3047=913%:%
%:%3048=913%:%
%:%3049=913%:%
%:%3050=914%:%
%:%3051=914%:%
%:%3052=914%:%
%:%3053=914%:%
%:%3054=915%:%
%:%3055=915%:%
%:%3056=915%:%
%:%3057=915%:%
%:%3058=916%:%
%:%3059=916%:%
%:%3060=916%:%
%:%3061=916%:%
%:%3062=917%:%
%:%3063=917%:%
%:%3064=918%:%
%:%3065=918%:%
%:%3066=918%:%
%:%3067=918%:%
%:%3068=918%:%
%:%3069=919%:%
%:%3070=919%:%
%:%3071=919%:%
%:%3072=919%:%
%:%3073=919%:%
%:%3074=919%:%
%:%3075=920%:%
%:%3076=920%:%
%:%3077=920%:%
%:%3078=920%:%
%:%3079=921%:%
%:%3080=922%:%
%:%3081=922%:%
%:%3082=922%:%
%:%3083=922%:%
%:%3084=922%:%
%:%3085=923%:%
%:%3086=923%:%
%:%3087=923%:%
%:%3088=923%:%
%:%3089=923%:%
%:%3090=924%:%
%:%3091=924%:%
%:%3092=924%:%
%:%3093=924%:%
%:%3094=925%:%
%:%3095=925%:%
%:%3096=926%:%
%:%3097=926%:%
%:%3098=926%:%
%:%3099=926%:%
%:%3100=926%:%
%:%3101=927%:%
%:%3102=928%:%
%:%3103=928%:%
%:%3104=928%:%
%:%3105=928%:%
%:%3106=928%:%
%:%3107=929%:%
%:%3108=930%:%
%:%3109=930%:%
%:%3110=930%:%
%:%3111=930%:%
%:%3112=930%:%
%:%3113=931%:%
%:%3114=931%:%
%:%3115=931%:%
%:%3116=931%:%
%:%3117=931%:%
%:%3118=932%:%
%:%3119=932%:%
%:%3120=932%:%
%:%3121=933%:%
%:%3122=933%:%
%:%3123=933%:%
%:%3124=933%:%
%:%3125=933%:%
%:%3126=933%:%
%:%3127=934%:%
%:%3128=934%:%
%:%3129=934%:%
%:%3130=934%:%
%:%3131=934%:%
%:%3132=934%:%
%:%3133=935%:%
%:%3134=936%:%
%:%3135=936%:%
%:%3136=936%:%
%:%3137=936%:%
%:%3138=936%:%
%:%3139=937%:%
%:%3140=937%:%
%:%3141=937%:%
%:%3142=937%:%
%:%3143=937%:%
%:%3144=938%:%
%:%3145=939%:%
%:%3146=939%:%
%:%3147=939%:%
%:%3148=940%:%
%:%3149=940%:%
%:%3150=940%:%
%:%3151=941%:%
%:%3157=941%:%
%:%3160=942%:%
%:%3161=943%:%
%:%3162=943%:%
%:%3163=944%:%
%:%3166=945%:%
%:%3170=945%:%
%:%3171=945%:%
%:%3172=945%:%
%:%3177=945%:%
%:%3180=946%:%
%:%3181=947%:%
%:%3182=947%:%
%:%3183=948%:%
%:%3184=949%:%
%:%3185=950%:%
%:%3186=951%:%
%:%3187=951%:%
%:%3194=952%:%
%:%3195=952%:%
%:%3196=953%:%
%:%3197=953%:%
%:%3198=954%:%
%:%3199=954%:%
%:%3200=955%:%
%:%3201=955%:%
%:%3202=955%:%
%:%3203=955%:%
%:%3204=956%:%
%:%3205=956%:%
%:%3206=956%:%
%:%3207=957%:%
%:%3208=957%:%
%:%3209=958%:%
%:%3210=958%:%
%:%3211=959%:%
%:%3212=959%:%
%:%3213=959%:%
%:%3214=959%:%
%:%3215=959%:%
%:%3216=960%:%
%:%3217=960%:%
%:%3218=960%:%
%:%3219=960%:%
%:%3220=960%:%
%:%3221=960%:%
%:%3222=961%:%
%:%3223=961%:%
%:%3224=961%:%
%:%3225=961%:%
%:%3226=962%:%
%:%3227=962%:%
%:%3228=963%:%
%:%3229=963%:%
%:%3230=964%:%
%:%3231=964%:%
%:%3232=964%:%
%:%3233=964%:%
%:%3234=964%:%
%:%3235=965%:%
%:%3236=965%:%
%:%3237=965%:%
%:%3238=965%:%
%:%3239=965%:%
%:%3240=966%:%
%:%3241=966%:%
%:%3242=966%:%
%:%3243=966%:%
%:%3244=966%:%
%:%3245=966%:%
%:%3246=967%:%
%:%3247=967%:%
%:%3248=967%:%
%:%3249=967%:%
%:%3250=968%:%
%:%3251=968%:%
%:%3252=969%:%
%:%3253=969%:%
%:%3254=970%:%
%:%3255=970%:%
%:%3256=971%:%
%:%3257=971%:%
%:%3258=972%:%
%:%3259=972%:%
%:%3260=972%:%
%:%3261=972%:%
%:%3262=972%:%
%:%3263=973%:%
%:%3269=973%:%
%:%3272=974%:%
%:%3273=975%:%
%:%3274=975%:%
%:%3275=976%:%
%:%3278=977%:%
%:%3282=977%:%
%:%3283=977%:%
%:%3284=977%:%
%:%3285=977%:%
%:%3294=980%:%
%:%3298=982%:%
%:%3299=983%:%
%:%3300=984%:%
%:%3301=985%:%
%:%3302=986%:%
%:%3303=987%:%
%:%3304=988%:%
%:%3305=989%:%
%:%3306=990%:%
%:%3307=991%:%
%:%3308=992%:%
%:%3309=993%:%
%:%3310=994%:%
%:%3311=995%:%
%:%3313=1005%:%
%:%3314=1005%:%
%:%3315=1006%:%
%:%3316=1007%:%
%:%3317=1008%:%
%:%3318=1009%:%
%:%3319=1010%:%
%:%3320=1011%:%
%:%3321=1012%:%
%:%3328=1013%:%
%:%3329=1013%:%
%:%3330=1014%:%
%:%3331=1014%:%
%:%3332=1014%:%
%:%3333=1014%:%
%:%3334=1014%:%
%:%3335=1015%:%
%:%3336=1015%:%
%:%3337=1016%:%
%:%3338=1016%:%
%:%3339=1016%:%
%:%3340=1016%:%
%:%3341=1016%:%
%:%3342=1017%:%
%:%3343=1018%:%
%:%3344=1018%:%
%:%3345=1019%:%
%:%3346=1019%:%
%:%3347=1020%:%
%:%3348=1020%:%
%:%3349=1021%:%
%:%3350=1021%:%
%:%3351=1022%:%
%:%3352=1022%:%
%:%3353=1022%:%
%:%3354=1022%:%
%:%3355=1023%:%
%:%3356=1023%:%
%:%3357=1023%:%
%:%3358=1023%:%
%:%3359=1023%:%
%:%3360=1024%:%
%:%3361=1024%:%
%:%3362=1024%:%
%:%3363=1024%:%
%:%3364=1025%:%
%:%3365=1025%:%
%:%3366=1026%:%
%:%3367=1026%:%
%:%3368=1027%:%
%:%3369=1027%:%
%:%3370=1028%:%
%:%3371=1028%:%
%:%3372=1029%:%
%:%3373=1029%:%
%:%3374=1029%:%
%:%3375=1030%:%
%:%3376=1030%:%
%:%3377=1030%:%
%:%3378=1031%:%
%:%3379=1031%:%
%:%3380=1031%:%
%:%3381=1031%:%
%:%3382=1032%:%
%:%3383=1032%:%
%:%3384=1032%:%
%:%3385=1032%:%
%:%3386=1032%:%
%:%3387=1033%:%
%:%3388=1033%:%
%:%3389=1033%:%
%:%3390=1033%:%
%:%3391=1033%:%
%:%3392=1034%:%
%:%3393=1034%:%
%:%3394=1034%:%
%:%3395=1034%:%
%:%3396=1034%:%
%:%3397=1035%:%
%:%3398=1035%:%
%:%3399=1036%:%
%:%3400=1037%:%
%:%3401=1038%:%
%:%3402=1039%:%
%:%3403=1040%:%
%:%3404=1041%:%
%:%3405=1041%:%
%:%3406=1042%:%
%:%3407=1042%:%
%:%3408=1043%:%
%:%3409=1043%:%
%:%3410=1044%:%
%:%3411=1044%:%
%:%3412=1045%:%
%:%3413=1045%:%
%:%3414=1046%:%
%:%3415=1046%:%
%:%3416=1046%:%
%:%3417=1046%:%
%:%3418=1046%:%
%:%3419=1047%:%
%:%3420=1047%:%
%:%3421=1047%:%
%:%3422=1047%:%
%:%3423=1047%:%
%:%3424=1048%:%
%:%3425=1048%:%
%:%3426=1049%:%
%:%3427=1049%:%
%:%3428=1049%:%
%:%3429=1049%:%
%:%3430=1050%:%
%:%3431=1051%:%
%:%3432=1051%:%
%:%3433=1051%:%
%:%3434=1051%:%
%:%3435=1051%:%
%:%3436=1052%:%
%:%3437=1052%:%
%:%3438=1052%:%
%:%3439=1053%:%
%:%3440=1053%:%
%:%3441=1053%:%
%:%3442=1053%:%
%:%3443=1054%:%
%:%3444=1054%:%
%:%3445=1054%:%
%:%3446=1054%:%
%:%3447=1054%:%
%:%3448=1054%:%
%:%3449=1055%:%
%:%3450=1055%:%
%:%3451=1055%:%
%:%3452=1055%:%
%:%3453=1056%:%
%:%3454=1056%:%
%:%3455=1056%:%
%:%3456=1056%:%
%:%3457=1056%:%
%:%3458=1057%:%
%:%3459=1057%:%
%:%3460=1057%:%
%:%3461=1057%:%
%:%3462=1057%:%
%:%3463=1058%:%
%:%3464=1058%:%
%:%3465=1058%:%
%:%3466=1058%:%
%:%3467=1058%:%
%:%3468=1059%:%
%:%3469=1059%:%
%:%3470=1060%:%
%:%3471=1060%:%
%:%3472=1061%:%
%:%3473=1061%:%
%:%3474=1062%:%
%:%3475=1062%:%
%:%3476=1062%:%
%:%3477=1063%:%
%:%3478=1063%:%
%:%3479=1063%:%
%:%3480=1063%:%
%:%3481=1063%:%
%:%3482=1064%:%
%:%3483=1064%:%
%:%3484=1064%:%
%:%3485=1064%:%
%:%3486=1064%:%
%:%3487=1064%:%
%:%3488=1065%:%
%:%3489=1065%:%
%:%3490=1065%:%
%:%3491=1065%:%
%:%3492=1065%:%
%:%3493=1065%:%
%:%3494=1066%:%
%:%3495=1066%:%
%:%3496=1066%:%
%:%3497=1066%:%
%:%3498=1067%:%
%:%3499=1067%:%
%:%3500=1068%:%
%:%3501=1069%:%
%:%3502=1069%:%
%:%3503=1069%:%
%:%3504=1069%:%
%:%3505=1069%:%
%:%3506=1070%:%
%:%3507=1070%:%
%:%3508=1071%:%
%:%3509=1071%:%
%:%3510=1071%:%
%:%3511=1071%:%
%:%3512=1071%:%
%:%3513=1072%:%
%:%3514=1073%:%
%:%3515=1074%:%
%:%3516=1074%:%
%:%3517=1075%:%
%:%3518=1075%:%
%:%3519=1076%:%
%:%3520=1076%:%
%:%3521=1077%:%
%:%3522=1077%:%
%:%3523=1077%:%
%:%3524=1078%:%
%:%3525=1078%:%
%:%3526=1078%:%
%:%3527=1078%:%
%:%3528=1078%:%
%:%3529=1079%:%
%:%3530=1079%:%
%:%3531=1079%:%
%:%3532=1079%:%
%:%3533=1079%:%
%:%3534=1079%:%
%:%3535=1080%:%
%:%3536=1080%:%
%:%3537=1080%:%
%:%3538=1080%:%
%:%3539=1080%:%
%:%3540=1080%:%
%:%3541=1081%:%
%:%3542=1081%:%
%:%3543=1081%:%
%:%3544=1081%:%
%:%3545=1081%:%
%:%3546=1081%:%
%:%3547=1082%:%
%:%3548=1082%:%
%:%3549=1082%:%
%:%3550=1082%:%
%:%3551=1082%:%
%:%3552=1082%:%
%:%3553=1083%:%
%:%3554=1083%:%
%:%3555=1083%:%
%:%3556=1083%:%
%:%3557=1083%:%
%:%3558=1084%:%
%:%3559=1084%:%
%:%3560=1084%:%
%:%3561=1084%:%
%:%3562=1084%:%
%:%3563=1085%:%
%:%3564=1085%:%
%:%3565=1086%:%
%:%3566=1087%:%
%:%3567=1087%:%
%:%3568=1087%:%
%:%3569=1087%:%
%:%3570=1087%:%
%:%3572=1097%:%
%:%3574=1098%:%
%:%3575=1098%:%
%:%3576=1099%:%
%:%3577=1099%:%
%:%3578=1100%:%
%:%3579=1100%:%
%:%3580=1101%:%
%:%3581=1101%:%
%:%3582=1102%:%
%:%3583=1102%:%
%:%3584=1102%:%
%:%3585=1103%:%
%:%3586=1103%:%
%:%3587=1104%:%
%:%3588=1104%:%
%:%3589=1105%:%
%:%3590=1105%:%
%:%3591=1105%:%
%:%3592=1106%:%
%:%3593=1106%:%
%:%3594=1106%:%
%:%3595=1106%:%
%:%3596=1106%:%
%:%3597=1106%:%
%:%3598=1107%:%
%:%3599=1107%:%
%:%3600=1107%:%
%:%3601=1107%:%
%:%3602=1107%:%
%:%3603=1107%:%
%:%3604=1108%:%
%:%3605=1108%:%
%:%3606=1108%:%
%:%3607=1108%:%
%:%3608=1108%:%
%:%3609=1108%:%
%:%3610=1109%:%
%:%3611=1109%:%
%:%3612=1109%:%
%:%3613=1109%:%
%:%3614=1109%:%
%:%3615=1110%:%
%:%3616=1110%:%
%:%3617=1111%:%
%:%3618=1111%:%
%:%3619=1112%:%
%:%3620=1112%:%
%:%3621=1113%:%
%:%3622=1114%:%
%:%3623=1114%:%
%:%3624=1115%:%
%:%3625=1115%:%
%:%3626=1116%:%
%:%3627=1116%:%
%:%3628=1117%:%
%:%3629=1117%:%
%:%3630=1117%:%
%:%3631=1118%:%
%:%3632=1118%:%
%:%3633=1118%:%
%:%3634=1118%:%
%:%3635=1118%:%
%:%3636=1119%:%
%:%3637=1119%:%
%:%3638=1119%:%
%:%3639=1119%:%
%:%3640=1119%:%
%:%3641=1119%:%
%:%3642=1120%:%
%:%3643=1120%:%
%:%3644=1120%:%
%:%3645=1120%:%
%:%3646=1120%:%
%:%3647=1120%:%
%:%3648=1121%:%
%:%3649=1121%:%
%:%3650=1121%:%
%:%3651=1121%:%
%:%3652=1121%:%
%:%3653=1121%:%
%:%3654=1122%:%
%:%3655=1122%:%
%:%3656=1122%:%
%:%3657=1122%:%
%:%3658=1123%:%
%:%3659=1123%:%
%:%3660=1124%:%
%:%3661=1125%:%
%:%3662=1125%:%
%:%3663=1126%:%
%:%3664=1126%:%
%:%3665=1127%:%
%:%3666=1127%:%
%:%3667=1128%:%
%:%3668=1128%:%
%:%3669=1129%:%
%:%3670=1129%:%
%:%3671=1129%:%
%:%3672=1130%:%
%:%3673=1130%:%
%:%3674=1130%:%
%:%3675=1130%:%
%:%3676=1131%:%
%:%3677=1131%:%
%:%3678=1131%:%
%:%3679=1131%:%
%:%3680=1131%:%
%:%3681=1132%:%
%:%3682=1132%:%
%:%3683=1132%:%
%:%3684=1132%:%
%:%3685=1132%:%
%:%3686=1132%:%
%:%3687=1133%:%
%:%3688=1133%:%
%:%3689=1133%:%
%:%3690=1133%:%
%:%3691=1133%:%
%:%3692=1133%:%
%:%3693=1134%:%
%:%3694=1134%:%
%:%3695=1134%:%
%:%3696=1134%:%
%:%3697=1134%:%
%:%3698=1134%:%
%:%3699=1135%:%
%:%3700=1135%:%
%:%3701=1135%:%
%:%3702=1135%:%
%:%3703=1136%:%
%:%3704=1137%:%
%:%3705=1137%:%
%:%3706=1137%:%
%:%3707=1137%:%
%:%3708=1137%:%
%:%3709=1138%:%
%:%3710=1138%:%
%:%3711=1138%:%
%:%3712=1139%:%
%:%3713=1139%:%
%:%3714=1139%:%
%:%3715=1140%:%
%:%3716=1140%:%
%:%3717=1140%:%
%:%3718=1140%:%
%:%3719=1140%:%
%:%3720=1140%:%
%:%3721=1141%:%
%:%3722=1141%:%
%:%3723=1141%:%
%:%3724=1141%:%
%:%3725=1141%:%
%:%3726=1142%:%
%:%3727=1142%:%
%:%3728=1143%:%
%:%3729=1144%:%
%:%3730=1144%:%
%:%3731=1145%:%
%:%3732=1145%:%
%:%3733=1146%:%
%:%3734=1146%:%
%:%3735=1146%:%
%:%3736=1147%:%
%:%3737=1147%:%
%:%3738=1147%:%
%:%3739=1147%:%
%:%3740=1147%:%
%:%3741=1148%:%
%:%3742=1148%:%
%:%3743=1149%:%
%:%3744=1149%:%
%:%3745=1150%:%
%:%3746=1150%:%
%:%3747=1150%:%
%:%3748=1150%:%
%:%3749=1150%:%
%:%3750=1151%:%
%:%3751=1151%:%
%:%3752=1151%:%
%:%3753=1151%:%
%:%3754=1151%:%
%:%3755=1152%:%
%:%3756=1152%:%
%:%3757=1152%:%
%:%3758=1152%:%
%:%3759=1153%:%
%:%3760=1153%:%
%:%3761=1153%:%
%:%3762=1153%:%
%:%3763=1153%:%
%:%3764=1154%:%
%:%3765=1155%:%
%:%3766=1155%:%
%:%3767=1156%:%
%:%3768=1156%:%
%:%3769=1157%:%
%:%3770=1157%:%
%:%3771=1158%:%
%:%3772=1158%:%
%:%3773=1158%:%
%:%3774=1159%:%
%:%3775=1159%:%
%:%3776=1159%:%
%:%3777=1159%:%
%:%3778=1160%:%
%:%3779=1160%:%
%:%3780=1160%:%
%:%3781=1161%:%
%:%3782=1161%:%
%:%3783=1162%:%
%:%3784=1162%:%
%:%3785=1163%:%
%:%3786=1163%:%
%:%3787=1163%:%
%:%3788=1163%:%
%:%3789=1163%:%
%:%3790=1164%:%
%:%3791=1164%:%
%:%3792=1164%:%
%:%3793=1164%:%
%:%3794=1164%:%
%:%3795=1165%:%
%:%3796=1165%:%
%:%3797=1166%:%
%:%3798=1166%:%
%:%3799=1167%:%
%:%3800=1167%:%
%:%3801=1167%:%
%:%3802=1167%:%
%:%3803=1168%:%
%:%3804=1168%:%
%:%3805=1168%:%
%:%3806=1168%:%
%:%3807=1168%:%
%:%3808=1168%:%
%:%3809=1169%:%
%:%3810=1169%:%
%:%3811=1169%:%
%:%3812=1169%:%
%:%3813=1169%:%
%:%3814=1169%:%
%:%3815=1170%:%
%:%3816=1170%:%
%:%3817=1170%:%
%:%3818=1170%:%
%:%3819=1171%:%
%:%3820=1171%:%
%:%3821=1172%:%
%:%3822=1172%:%
%:%3823=1173%:%
%:%3824=1173%:%
%:%3825=1173%:%
%:%3826=1173%:%
%:%3827=1173%:%
%:%3828=1174%:%
%:%3829=1174%:%
%:%3830=1174%:%
%:%3831=1174%:%
%:%3832=1174%:%
%:%3833=1174%:%
%:%3834=1175%:%
%:%3835=1175%:%
%:%3836=1175%:%
%:%3837=1175%:%
%:%3838=1175%:%
%:%3839=1175%:%
%:%3840=1176%:%
%:%3841=1176%:%
%:%3842=1176%:%
%:%3843=1176%:%
%:%3844=1177%:%
%:%3845=1177%:%
%:%3846=1177%:%
%:%3847=1178%:%
%:%3848=1178%:%
%:%3849=1179%:%
%:%3850=1180%:%
%:%3851=1181%:%
%:%3852=1181%:%
%:%3853=1182%:%
%:%3854=1182%:%
%:%3855=1183%:%
%:%3856=1183%:%
%:%3857=1184%:%
%:%3858=1184%:%
%:%3859=1184%:%
%:%3860=1185%:%
%:%3861=1185%:%
%:%3862=1185%:%
%:%3863=1185%:%
%:%3864=1185%:%
%:%3865=1186%:%
%:%3866=1186%:%
%:%3867=1186%:%
%:%3868=1186%:%
%:%3869=1186%:%
%:%3870=1187%:%
%:%3871=1187%:%
%:%3872=1187%:%
%:%3873=1187%:%
%:%3874=1187%:%
%:%3875=1187%:%
%:%3876=1188%:%
%:%3877=1188%:%
%:%3878=1188%:%
%:%3879=1188%:%
%:%3880=1188%:%
%:%3881=1189%:%
%:%3882=1189%:%
%:%3883=1190%:%
%:%3884=1191%:%
%:%3885=1192%:%
%:%3886=1193%:%
%:%3887=1194%:%
%:%3888=1195%:%
%:%3889=1195%:%
%:%3890=1195%:%
%:%3891=1195%:%
%:%3892=1195%:%
%:%3893=1196%:%
%:%3894=1197%:%
%:%3895=1197%:%
%:%3896=1197%:%
%:%3897=1197%:%
%:%3898=1197%:%
%:%3899=1198%:%
%:%3900=1199%:%
%:%3901=1199%:%
%:%3902=1200%:%
%:%3903=1200%:%
%:%3904=1201%:%
%:%3905=1201%:%
%:%3906=1202%:%
%:%3907=1202%:%
%:%3908=1203%:%
%:%3909=1204%:%
%:%3910=1205%:%
%:%3911=1206%:%
%:%3912=1206%:%
%:%3913=1206%:%
%:%3914=1206%:%
%:%3915=1206%:%
%:%3916=1207%:%
%:%3917=1207%:%
%:%3918=1207%:%
%:%3919=1207%:%
%:%3920=1207%:%
%:%3921=1208%:%
%:%3922=1208%:%
%:%3923=1208%:%
%:%3924=1208%:%
%:%3925=1208%:%
%:%3926=1209%:%
%:%3927=1209%:%
%:%3928=1209%:%
%:%3929=1209%:%
%:%3930=1209%:%
%:%3931=1210%:%
%:%3932=1210%:%
%:%3933=1211%:%
%:%3934=1211%:%
%:%3935=1212%:%
%:%3936=1212%:%
%:%3937=1212%:%
%:%3938=1213%:%
%:%3939=1213%:%
%:%3940=1214%:%
%:%3941=1214%:%
%:%3942=1215%:%
%:%3943=1215%:%
%:%3944=1216%:%
%:%3945=1216%:%
%:%3946=1216%:%
%:%3947=1216%:%
%:%3948=1216%:%
%:%3949=1217%:%
%:%3950=1217%:%
%:%3951=1217%:%
%:%3952=1217%:%
%:%3953=1217%:%
%:%3954=1217%:%
%:%3955=1218%:%
%:%3956=1218%:%
%:%3957=1218%:%
%:%3958=1218%:%
%:%3959=1218%:%
%:%3960=1218%:%
%:%3961=1219%:%
%:%3962=1219%:%
%:%3963=1219%:%
%:%3964=1219%:%
%:%3965=1219%:%
%:%3966=1220%:%
%:%3967=1220%:%
%:%3968=1221%:%
%:%3969=1221%:%
%:%3970=1222%:%
%:%3971=1222%:%
%:%3972=1222%:%
%:%3973=1222%:%
%:%3974=1222%:%
%:%3975=1223%:%
%:%3976=1223%:%
%:%3977=1223%:%
%:%3978=1223%:%
%:%3979=1223%:%
%:%3980=1223%:%
%:%3981=1224%:%
%:%3982=1224%:%
%:%3983=1224%:%
%:%3984=1224%:%
%:%3985=1224%:%
%:%3986=1225%:%
%:%3987=1225%:%
%:%3988=1225%:%
%:%3989=1225%:%
%:%3990=1225%:%
%:%3991=1226%:%
%:%3992=1226%:%
%:%3993=1227%:%
%:%3994=1227%:%
%:%3995=1228%:%
%:%3996=1229%:%
%:%3997=1229%:%
%:%3998=1229%:%
%:%3999=1229%:%
%:%4000=1230%:%
%:%4001=1231%:%
%:%4002=1232%:%
%:%4003=1232%:%
%:%4004=1232%:%
%:%4005=1232%:%
%:%4006=1233%:%
%:%4007=1233%:%
%:%4008=1233%:%
%:%4009=1233%:%
%:%4010=1233%:%
%:%4011=1234%:%
%:%4012=1234%:%
%:%4013=1234%:%
%:%4014=1234%:%
%:%4015=1234%:%
%:%4016=1235%:%
%:%4017=1235%:%
%:%4018=1235%:%
%:%4019=1235%:%
%:%4020=1235%:%
%:%4021=1236%:%
%:%4022=1236%:%
%:%4023=1236%:%
%:%4024=1236%:%
%:%4025=1237%:%
%:%4026=1237%:%
%:%4027=1237%:%
%:%4028=1237%:%
%:%4029=1237%:%
%:%4030=1238%:%
%:%4031=1239%:%
%:%4032=1239%:%
%:%4033=1240%:%
%:%4034=1240%:%
%:%4035=1241%:%
%:%4036=1241%:%
%:%4037=1241%:%
%:%4038=1242%:%
%:%4039=1242%:%
%:%4040=1243%:%
%:%4041=1243%:%
%:%4042=1244%:%
%:%4043=1244%:%
%:%4044=1245%:%
%:%4045=1245%:%
%:%4046=1245%:%
%:%4047=1245%:%
%:%4048=1245%:%
%:%4049=1245%:%
%:%4050=1246%:%
%:%4051=1246%:%
%:%4052=1247%:%
%:%4053=1247%:%
%:%4054=1248%:%
%:%4055=1248%:%
%:%4056=1248%:%
%:%4057=1248%:%
%:%4058=1249%:%
%:%4059=1249%:%
%:%4060=1249%:%
%:%4061=1250%:%
%:%4062=1250%:%
%:%4063=1251%:%
%:%4064=1251%:%
%:%4065=1252%:%
%:%4066=1252%:%
%:%4067=1252%:%
%:%4068=1252%:%
%:%4069=1252%:%
%:%4070=1252%:%
%:%4071=1253%:%
%:%4072=1253%:%
%:%4073=1254%:%
%:%4074=1254%:%
%:%4075=1255%:%
%:%4076=1255%:%
%:%4077=1255%:%
%:%4078=1255%:%
%:%4079=1255%:%
%:%4080=1255%:%
%:%4081=1256%:%
%:%4082=1256%:%
%:%4083=1257%:%
%:%4084=1257%:%
%:%4085=1258%:%
%:%4086=1258%:%
%:%4087=1259%:%
%:%4088=1260%:%
%:%4089=1260%:%
%:%4090=1261%:%
%:%4091=1261%:%
%:%4092=1262%:%
%:%4093=1262%:%
%:%4094=1262%:%
%:%4095=1263%:%
%:%4096=1263%:%
%:%4097=1264%:%
%:%4098=1264%:%
%:%4099=1265%:%
%:%4100=1265%:%
%:%4101=1265%:%
%:%4102=1266%:%
%:%4103=1266%:%
%:%4104=1266%:%
%:%4105=1266%:%
%:%4106=1267%:%
%:%4107=1267%:%
%:%4108=1267%:%
%:%4109=1268%:%
%:%4110=1268%:%
%:%4111=1269%:%
%:%4112=1269%:%
%:%4113=1270%:%
%:%4114=1270%:%
%:%4115=1270%:%
%:%4116=1270%:%
%:%4117=1270%:%
%:%4118=1271%:%
%:%4119=1271%:%
%:%4120=1271%:%
%:%4121=1271%:%
%:%4122=1271%:%
%:%4123=1271%:%
%:%4124=1272%:%
%:%4125=1272%:%
%:%4126=1273%:%
%:%4127=1273%:%
%:%4128=1274%:%
%:%4129=1274%:%
%:%4130=1274%:%
%:%4131=1274%:%
%:%4132=1275%:%
%:%4133=1275%:%
%:%4134=1276%:%
%:%4135=1277%:%
%:%4136=1278%:%
%:%4137=1278%:%
%:%4138=1279%:%
%:%4139=1279%:%
%:%4140=1280%:%
%:%4141=1280%:%
%:%4142=1281%:%
%:%4143=1281%:%
%:%4144=1282%:%
%:%4145=1282%:%
%:%4146=1282%:%
%:%4147=1283%:%
%:%4148=1283%:%
%:%4149=1283%:%
%:%4150=1283%:%
%:%4151=1284%:%
%:%4152=1284%:%
%:%4153=1284%:%
%:%4154=1284%:%
%:%4155=1285%:%
%:%4156=1285%:%
%:%4157=1285%:%
%:%4158=1286%:%
%:%4159=1286%:%
%:%4160=1287%:%
%:%4161=1287%:%
%:%4162=1288%:%
%:%4163=1288%:%
%:%4164=1288%:%
%:%4165=1288%:%
%:%4166=1288%:%
%:%4167=1288%:%
%:%4168=1289%:%
%:%4169=1289%:%
%:%4170=1289%:%
%:%4171=1289%:%
%:%4172=1289%:%
%:%4173=1290%:%
%:%4174=1290%:%
%:%4175=1290%:%
%:%4176=1290%:%
%:%4177=1290%:%
%:%4178=1291%:%
%:%4179=1291%:%
%:%4180=1291%:%
%:%4181=1291%:%
%:%4182=1292%:%
%:%4183=1292%:%
%:%4184=1293%:%
%:%4185=1293%:%
%:%4186=1294%:%
%:%4187=1294%:%
%:%4188=1294%:%
%:%4189=1294%:%
%:%4190=1294%:%
%:%4191=1294%:%
%:%4192=1295%:%
%:%4193=1295%:%
%:%4194=1295%:%
%:%4195=1295%:%
%:%4196=1295%:%
%:%4197=1296%:%
%:%4198=1296%:%
%:%4199=1296%:%
%:%4200=1297%:%
%:%4201=1297%:%
%:%4202=1298%:%
%:%4203=1298%:%
%:%4204=1299%:%
%:%4205=1299%:%
%:%4206=1299%:%
%:%4207=1299%:%
%:%4208=1299%:%
%:%4209=1300%:%
%:%4210=1300%:%
%:%4211=1300%:%
%:%4212=1300%:%
%:%4213=1301%:%
%:%4214=1301%:%
%:%4215=1302%:%
%:%4216=1302%:%
%:%4217=1303%:%
%:%4218=1303%:%
%:%4219=1303%:%
%:%4220=1304%:%
%:%4221=1304%:%
%:%4222=1305%:%
%:%4223=1305%:%
%:%4224=1305%:%
%:%4225=1305%:%
%:%4226=1305%:%
%:%4227=1306%:%
%:%4228=1306%:%
%:%4229=1306%:%
%:%4230=1306%:%
%:%4231=1306%:%
%:%4232=1307%:%
%:%4233=1307%:%
%:%4234=1308%:%
%:%4235=1308%:%
%:%4236=1309%:%
%:%4237=1309%:%
%:%4238=1310%:%
%:%4239=1311%:%
%:%4240=1311%:%
%:%4241=1312%:%
%:%4242=1312%:%
%:%4243=1313%:%
%:%4244=1313%:%
%:%4245=1313%:%
%:%4246=1314%:%
%:%4247=1314%:%
%:%4248=1314%:%
%:%4249=1314%:%
%:%4250=1315%:%
%:%4251=1315%:%
%:%4252=1315%:%
%:%4253=1316%:%
%:%4254=1316%:%
%:%4255=1317%:%
%:%4256=1317%:%
%:%4257=1318%:%
%:%4258=1318%:%
%:%4259=1318%:%
%:%4260=1319%:%
%:%4261=1319%:%
%:%4262=1319%:%
%:%4263=1319%:%
%:%4264=1319%:%
%:%4265=1320%:%
%:%4266=1320%:%
%:%4267=1320%:%
%:%4268=1320%:%
%:%4269=1320%:%
%:%4270=1321%:%
%:%4271=1321%:%
%:%4272=1321%:%
%:%4273=1322%:%
%:%4274=1322%:%
%:%4275=1323%:%
%:%4276=1323%:%
%:%4277=1324%:%
%:%4278=1324%:%
%:%4279=1324%:%
%:%4280=1324%:%
%:%4281=1324%:%
%:%4282=1325%:%
%:%4283=1325%:%
%:%4284=1325%:%
%:%4285=1325%:%
%:%4286=1325%:%
%:%4287=1326%:%
%:%4288=1326%:%
%:%4289=1326%:%
%:%4290=1326%:%
%:%4291=1327%:%
%:%4292=1327%:%
%:%4293=1328%:%
%:%4294=1328%:%
%:%4295=1329%:%
%:%4296=1329%:%
%:%4297=1329%:%
%:%4298=1329%:%
%:%4299=1329%:%
%:%4300=1330%:%
%:%4301=1330%:%
%:%4302=1330%:%
%:%4303=1330%:%
%:%4304=1330%:%
%:%4305=1331%:%
%:%4306=1331%:%
%:%4307=1331%:%
%:%4308=1331%:%
%:%4309=1332%:%
%:%4310=1332%:%
%:%4311=1333%:%
%:%4312=1333%:%
%:%4313=1334%:%
%:%4314=1334%:%
%:%4315=1335%:%
%:%4316=1335%:%
%:%4317=1335%:%
%:%4318=1335%:%
%:%4319=1335%:%
%:%4320=1336%:%
%:%4321=1336%:%
%:%4322=1336%:%
%:%4323=1336%:%
%:%4324=1337%:%
%:%4325=1337%:%
%:%4326=1337%:%
%:%4327=1337%:%
%:%4328=1338%:%
%:%4329=1338%:%
%:%4330=1338%:%
%:%4331=1339%:%
%:%4332=1339%:%
%:%4333=1340%:%
%:%4334=1340%:%
%:%4335=1341%:%
%:%4336=1341%:%
%:%4337=1341%:%
%:%4338=1341%:%
%:%4339=1341%:%
%:%4340=1341%:%
%:%4341=1342%:%
%:%4342=1342%:%
%:%4343=1342%:%
%:%4344=1342%:%
%:%4345=1342%:%
%:%4346=1343%:%
%:%4347=1343%:%
%:%4348=1343%:%
%:%4349=1343%:%
%:%4350=1344%:%
%:%4351=1344%:%
%:%4352=1345%:%
%:%4353=1345%:%
%:%4354=1346%:%
%:%4355=1346%:%
%:%4356=1346%:%
%:%4357=1346%:%
%:%4358=1346%:%
%:%4359=1346%:%
%:%4360=1347%:%
%:%4361=1347%:%
%:%4362=1347%:%
%:%4363=1347%:%
%:%4364=1347%:%
%:%4365=1347%:%
%:%4366=1348%:%
%:%4367=1348%:%
%:%4368=1348%:%
%:%4369=1348%:%
%:%4370=1348%:%
%:%4371=1349%:%
%:%4372=1349%:%
%:%4373=1350%:%
%:%4374=1350%:%
%:%4375=1351%:%
%:%4376=1351%:%
%:%4377=1352%:%
%:%4378=1352%:%
%:%4379=1353%:%
%:%4380=1354%:%
%:%4381=1355%:%
%:%4382=1355%:%
%:%4383=1356%:%
%:%4384=1356%:%
%:%4385=1357%:%
%:%4386=1357%:%
%:%4387=1357%:%
%:%4388=1358%:%
%:%4389=1358%:%
%:%4390=1358%:%
%:%4391=1358%:%
%:%4392=1358%:%
%:%4393=1359%:%
%:%4394=1359%:%
%:%4395=1359%:%
%:%4396=1359%:%
%:%4397=1359%:%
%:%4398=1360%:%
%:%4399=1360%:%
%:%4400=1360%:%
%:%4401=1361%:%
%:%4402=1361%:%
%:%4403=1362%:%
%:%4404=1362%:%
%:%4405=1363%:%
%:%4406=1363%:%
%:%4407=1363%:%
%:%4408=1363%:%
%:%4409=1363%:%
%:%4410=1364%:%
%:%4411=1364%:%
%:%4412=1364%:%
%:%4413=1364%:%
%:%4414=1364%:%
%:%4415=1365%:%
%:%4416=1365%:%
%:%4417=1366%:%
%:%4418=1366%:%
%:%4419=1367%:%
%:%4420=1367%:%
%:%4421=1367%:%
%:%4422=1367%:%
%:%4423=1367%:%
%:%4424=1367%:%
%:%4425=1368%:%
%:%4426=1368%:%
%:%4427=1368%:%
%:%4428=1368%:%
%:%4429=1368%:%
%:%4430=1368%:%
%:%4431=1369%:%
%:%4432=1369%:%
%:%4433=1369%:%
%:%4434=1369%:%
%:%4435=1369%:%
%:%4436=1370%:%
%:%4437=1370%:%
%:%4438=1371%:%
%:%4439=1371%:%
%:%4440=1372%:%
%:%4441=1373%:%
%:%4442=1374%:%
%:%4443=1374%:%
%:%4444=1375%:%
%:%4445=1375%:%
%:%4446=1376%:%
%:%4447=1376%:%
%:%4448=1376%:%
%:%4449=1377%:%
%:%4450=1377%:%
%:%4451=1377%:%
%:%4452=1377%:%
%:%4453=1377%:%
%:%4454=1378%:%
%:%4455=1378%:%
%:%4456=1378%:%
%:%4457=1378%:%
%:%4458=1378%:%
%:%4459=1379%:%
%:%4460=1379%:%
%:%4461=1379%:%
%:%4462=1380%:%
%:%4463=1380%:%
%:%4464=1381%:%
%:%4465=1381%:%
%:%4466=1382%:%
%:%4467=1382%:%
%:%4468=1382%:%
%:%4469=1382%:%
%:%4470=1382%:%
%:%4471=1382%:%
%:%4472=1383%:%
%:%4473=1383%:%
%:%4474=1383%:%
%:%4475=1383%:%
%:%4476=1384%:%
%:%4477=1384%:%
%:%4478=1384%:%
%:%4479=1384%:%
%:%4480=1385%:%
%:%4481=1385%:%
%:%4482=1385%:%
%:%4483=1385%:%
%:%4484=1386%:%
%:%4485=1386%:%
%:%4486=1386%:%
%:%4487=1386%:%
%:%4488=1386%:%
%:%4489=1386%:%
%:%4490=1387%:%
%:%4491=1387%:%
%:%4492=1387%:%
%:%4493=1387%:%
%:%4494=1387%:%
%:%4495=1388%:%
%:%4496=1389%:%
%:%4497=1389%:%
%:%4498=1389%:%
%:%4499=1389%:%
%:%4500=1390%:%
%:%4501=1390%:%
%:%4502=1390%:%
%:%4503=1390%:%
%:%4504=1390%:%
%:%4505=1391%:%
%:%4506=1392%:%
%:%4507=1392%:%
%:%4508=1392%:%
%:%4509=1392%:%
%:%4510=1392%:%
%:%4511=1393%:%
%:%4512=1393%:%
%:%4513=1393%:%
%:%4514=1393%:%
%:%4515=1393%:%
%:%4516=1393%:%
%:%4517=1394%:%
%:%4518=1394%:%
%:%4519=1394%:%
%:%4520=1394%:%
%:%4521=1395%:%
%:%4522=1395%:%
%:%4523=1395%:%
%:%4524=1395%:%
%:%4525=1395%:%
%:%4526=1395%:%
%:%4527=1396%:%
%:%4528=1396%:%
%:%4529=1396%:%
%:%4530=1396%:%
%:%4531=1396%:%
%:%4532=1397%:%
%:%4533=1397%:%
%:%4534=1397%:%
%:%4535=1397%:%
%:%4536=1397%:%
%:%4537=1398%:%
%:%4538=1398%:%
%:%4539=1399%:%
%:%4540=1399%:%
%:%4541=1400%:%
%:%4542=1400%:%
%:%4543=1400%:%
%:%4544=1400%:%
%:%4545=1400%:%
%:%4546=1401%:%
%:%4547=1401%:%
%:%4548=1401%:%
%:%4549=1401%:%
%:%4550=1402%:%
%:%4551=1402%:%
%:%4552=1402%:%
%:%4553=1402%:%
%:%4554=1403%:%
%:%4555=1404%:%
%:%4556=1404%:%
%:%4557=1405%:%
%:%4558=1405%:%
%:%4559=1406%:%
%:%4560=1406%:%
%:%4561=1406%:%
%:%4562=1406%:%
%:%4563=1407%:%
%:%4564=1407%:%
%:%4565=1408%:%
%:%4566=1408%:%
%:%4567=1409%:%
%:%4568=1409%:%
%:%4569=1410%:%
%:%4570=1410%:%
%:%4571=1410%:%
%:%4572=1410%:%
%:%4573=1410%:%
%:%4574=1410%:%
%:%4575=1410%:%
%:%4576=1411%:%
%:%4577=1411%:%
%:%4578=1411%:%
%:%4579=1411%:%
%:%4580=1412%:%
%:%4581=1412%:%
%:%4582=1413%:%
%:%4583=1414%:%
%:%4584=1414%:%
%:%4585=1414%:%
%:%4586=1414%:%
%:%4587=1414%:%
%:%4588=1415%:%
%:%4589=1415%:%
%:%4590=1415%:%
%:%4591=1415%:%
%:%4592=1415%:%
%:%4593=1415%:%
%:%4594=1416%:%
%:%4595=1416%:%
%:%4596=1416%:%
%:%4597=1416%:%
%:%4598=1416%:%
%:%4599=1416%:%
%:%4600=1417%:%
%:%4601=1417%:%
%:%4602=1417%:%
%:%4603=1417%:%
%:%4604=1417%:%
%:%4605=1417%:%
%:%4606=1418%:%
%:%4607=1418%:%
%:%4608=1418%:%
%:%4609=1418%:%
%:%4610=1418%:%
%:%4611=1419%:%
%:%4612=1419%:%
%:%4613=1419%:%
%:%4614=1419%:%
%:%4615=1419%:%
%:%4616=1420%:%
%:%4617=1420%:%
%:%4618=1420%:%
%:%4619=1420%:%
%:%4620=1420%:%
%:%4621=1420%:%
%:%4622=1421%:%
%:%4623=1421%:%
%:%4624=1421%:%
%:%4625=1421%:%
%:%4626=1422%:%
%:%4627=1422%:%
%:%4628=1423%:%
%:%4629=1423%:%
%:%4630=1424%:%
%:%4631=1424%:%
%:%4632=1424%:%
%:%4633=1425%:%
%:%4634=1425%:%
%:%4635=1425%:%
%:%4636=1425%:%
%:%4637=1425%:%
%:%4638=1426%:%
%:%4639=1426%:%
%:%4640=1426%:%
%:%4641=1427%:%
%:%4642=1427%:%
%:%4643=1427%:%
%:%4644=1428%:%
%:%4645=1428%:%
%:%4646=1429%:%
%:%4647=1429%:%
%:%4648=1430%:%
%:%4649=1430%:%
%:%4650=1430%:%
%:%4651=1430%:%
%:%4652=1430%:%
%:%4653=1430%:%
%:%4654=1431%:%
%:%4655=1431%:%
%:%4656=1431%:%
%:%4657=1431%:%
%:%4658=1431%:%
%:%4659=1432%:%
%:%4660=1432%:%
%:%4661=1432%:%
%:%4662=1432%:%
%:%4663=1433%:%
%:%4664=1433%:%
%:%4665=1433%:%
%:%4666=1433%:%
%:%4667=1433%:%
%:%4668=1434%:%
%:%4669=1435%:%
%:%4670=1435%:%
%:%4671=1436%:%
%:%4672=1436%:%
%:%4673=1437%:%
%:%4674=1437%:%
%:%4675=1437%:%
%:%4676=1437%:%
%:%4677=1438%:%
%:%4678=1438%:%
%:%4679=1439%:%
%:%4680=1439%:%
%:%4681=1440%:%
%:%4682=1440%:%
%:%4683=1441%:%
%:%4684=1441%:%
%:%4685=1441%:%
%:%4686=1441%:%
%:%4687=1441%:%
%:%4688=1441%:%
%:%4689=1441%:%
%:%4690=1442%:%
%:%4691=1442%:%
%:%4692=1442%:%
%:%4693=1442%:%
%:%4694=1443%:%
%:%4695=1443%:%
%:%4696=1444%:%
%:%4697=1445%:%
%:%4698=1445%:%
%:%4699=1445%:%
%:%4700=1445%:%
%:%4701=1445%:%
%:%4702=1446%:%
%:%4703=1446%:%
%:%4704=1446%:%
%:%4705=1446%:%
%:%4706=1446%:%
%:%4707=1446%:%
%:%4708=1447%:%
%:%4709=1447%:%
%:%4710=1447%:%
%:%4711=1447%:%
%:%4712=1447%:%
%:%4713=1447%:%
%:%4714=1448%:%
%:%4715=1448%:%
%:%4716=1448%:%
%:%4717=1448%:%
%:%4718=1448%:%
%:%4719=1448%:%
%:%4720=1449%:%
%:%4721=1449%:%
%:%4722=1449%:%
%:%4723=1449%:%
%:%4724=1449%:%
%:%4725=1450%:%
%:%4726=1450%:%
%:%4727=1450%:%
%:%4728=1450%:%
%:%4729=1450%:%
%:%4730=1451%:%
%:%4731=1451%:%
%:%4732=1451%:%
%:%4733=1451%:%
%:%4734=1451%:%
%:%4735=1452%:%
%:%4736=1452%:%
%:%4737=1452%:%
%:%4738=1452%:%
%:%4739=1453%:%
%:%4740=1453%:%
%:%4741=1453%:%
%:%4742=1453%:%
%:%4743=1453%:%
%:%4744=1454%:%
%:%4745=1454%:%
%:%4746=1454%:%
%:%4747=1454%:%
%:%4748=1455%:%
%:%4749=1455%:%
%:%4750=1456%:%
%:%4751=1456%:%
%:%4752=1457%:%
%:%4753=1457%:%
%:%4754=1457%:%
%:%4755=1457%:%
%:%4756=1457%:%
%:%4757=1457%:%
%:%4758=1458%:%
%:%4759=1458%:%
%:%4760=1458%:%
%:%4761=1458%:%
%:%4762=1458%:%
%:%4763=1458%:%
%:%4764=1459%:%
%:%4765=1459%:%
%:%4766=1459%:%
%:%4767=1459%:%
%:%4768=1460%:%
%:%4769=1460%:%
%:%4770=1460%:%
%:%4771=1460%:%
%:%4772=1461%:%
%:%4773=1462%:%
%:%4774=1462%:%
%:%4775=1462%:%
%:%4776=1462%:%
%:%4777=1463%:%
%:%4778=1463%:%
%:%4779=1463%:%
%:%4780=1463%:%
%:%4781=1463%:%
%:%4782=1464%:%
%:%4783=1465%:%
%:%4784=1465%:%
%:%4785=1465%:%
%:%4786=1465%:%
%:%4787=1465%:%
%:%4788=1466%:%
%:%4789=1466%:%
%:%4790=1466%:%
%:%4791=1466%:%
%:%4792=1466%:%
%:%4793=1466%:%
%:%4794=1467%:%
%:%4795=1467%:%
%:%4796=1467%:%
%:%4797=1467%:%
%:%4798=1468%:%
%:%4799=1468%:%
%:%4800=1468%:%
%:%4801=1468%:%
%:%4802=1468%:%
%:%4803=1468%:%
%:%4804=1469%:%
%:%4805=1469%:%
%:%4806=1469%:%
%:%4807=1469%:%
%:%4808=1469%:%
%:%4809=1470%:%
%:%4810=1470%:%
%:%4811=1470%:%
%:%4812=1470%:%
%:%4813=1470%:%
%:%4814=1471%:%
%:%4815=1471%:%
%:%4816=1471%:%
%:%4817=1471%:%
%:%4818=1472%:%
%:%4819=1472%:%
%:%4820=1473%:%
%:%4821=1473%:%
%:%4822=1474%:%
%:%4823=1475%:%
%:%4824=1476%:%
%:%4825=1477%:%
%:%4826=1477%:%
%:%4827=1477%:%
%:%4828=1477%:%
%:%4829=1477%:%
%:%4830=1477%:%
%:%4831=1478%:%
%:%4832=1479%:%
%:%4833=1480%:%
%:%4834=1481%:%
%:%4835=1482%:%
%:%4836=1483%:%
%:%4837=1484%:%
%:%4838=1485%:%
%:%4839=1486%:%
%:%4840=1486%:%
%:%4841=1487%:%
%:%4842=1488%:%
%:%4843=1488%:%
%:%4844=1489%:%
%:%4845=1489%:%
%:%4846=1490%:%
%:%4847=1490%:%
%:%4848=1491%:%
%:%4849=1491%:%
%:%4850=1492%:%
%:%4851=1492%:%
%:%4852=1492%:%
%:%4853=1493%:%
%:%4854=1493%:%
%:%4855=1493%:%
%:%4856=1493%:%
%:%4857=1493%:%
%:%4858=1494%:%
%:%4859=1494%:%
%:%4860=1494%:%
%:%4861=1494%:%
%:%4862=1494%:%
%:%4863=1494%:%
%:%4864=1495%:%
%:%4865=1495%:%
%:%4866=1495%:%
%:%4867=1495%:%
%:%4868=1495%:%
%:%4869=1496%:%
%:%4870=1496%:%
%:%4871=1496%:%
%:%4872=1496%:%
%:%4873=1496%:%
%:%4874=1496%:%
%:%4875=1497%:%
%:%4876=1497%:%
%:%4877=1497%:%
%:%4878=1497%:%
%:%4879=1498%:%
%:%4880=1499%:%
%:%4881=1499%:%
%:%4882=1500%:%
%:%4883=1500%:%
%:%4884=1500%:%
%:%4885=1500%:%
%:%4886=1500%:%
%:%4887=1501%:%
%:%4888=1501%:%
%:%4889=1501%:%
%:%4890=1501%:%
%:%4891=1501%:%
%:%4892=1501%:%
%:%4893=1502%:%
%:%4894=1502%:%
%:%4895=1502%:%
%:%4896=1502%:%
%:%4897=1502%:%
%:%4898=1503%:%
%:%4899=1504%:%
%:%4900=1504%:%
%:%4901=1504%:%
%:%4902=1504%:%
%:%4903=1504%:%
%:%4904=1505%:%
%:%4905=1505%:%
%:%4906=1505%:%
%:%4907=1505%:%
%:%4908=1506%:%
%:%4909=1506%:%
%:%4910=1506%:%
%:%4911=1506%:%
%:%4912=1506%:%
%:%4913=1507%:%
%:%4914=1507%:%
%:%4915=1507%:%
%:%4916=1507%:%
%:%4917=1507%:%
%:%4918=1508%:%
%:%4919=1508%:%
%:%4920=1508%:%
%:%4921=1508%:%
%:%4922=1508%:%
%:%4923=1509%:%
%:%4924=1509%:%
%:%4925=1509%:%
%:%4926=1509%:%
%:%4927=1509%:%
%:%4928=1510%:%
%:%4929=1510%:%
%:%4930=1510%:%
%:%4931=1510%:%
%:%4932=1511%:%
%:%4933=1512%:%
%:%4934=1513%:%
%:%4935=1513%:%
%:%4936=1514%:%
%:%4937=1514%:%
%:%4938=1515%:%
%:%4939=1515%:%
%:%4940=1515%:%
%:%4941=1515%:%
%:%4942=1516%:%
%:%4943=1516%:%
%:%4944=1516%:%
%:%4945=1516%:%
%:%4946=1516%:%
%:%4947=1516%:%
%:%4948=1517%:%
%:%4949=1517%:%
%:%4950=1517%:%
%:%4951=1517%:%
%:%4952=1517%:%
%:%4953=1518%:%
%:%4954=1518%:%
%:%4955=1519%:%
%:%4956=1519%:%
%:%4957=1519%:%
%:%4958=1519%:%
%:%4959=1519%:%
%:%4960=1520%:%
%:%4961=1521%:%
%:%4962=1521%:%
%:%4963=1521%:%
%:%4964=1521%:%
%:%4965=1521%:%
%:%4966=1522%:%
%:%4967=1523%:%
%:%4968=1523%:%
%:%4969=1523%:%
%:%4970=1524%:%
%:%4971=1524%:%
%:%4972=1525%:%
%:%4973=1525%:%
%:%4974=1525%:%
%:%4975=1525%:%
%:%4976=1526%:%
%:%4977=1526%:%
%:%4978=1526%:%
%:%4979=1526%:%
%:%4980=1526%:%
%:%4981=1527%:%
%:%4982=1528%:%
%:%4983=1528%:%
%:%4984=1528%:%
%:%4985=1528%:%
%:%4986=1528%:%
%:%4987=1529%:%
%:%4988=1529%:%
%:%4989=1529%:%
%:%4990=1529%:%
%:%4991=1530%:%
%:%4992=1530%:%
%:%4993=1530%:%
%:%4994=1530%:%
%:%4995=1531%:%
%:%4996=1531%:%
%:%4997=1531%:%
%:%4998=1531%:%
%:%4999=1531%:%
%:%5000=1532%:%
%:%5001=1532%:%
%:%5002=1532%:%
%:%5003=1532%:%
%:%5004=1532%:%
%:%5005=1532%:%
%:%5006=1533%:%
%:%5007=1533%:%
%:%5008=1533%:%
%:%5009=1533%:%
%:%5010=1533%:%
%:%5011=1534%:%
%:%5012=1534%:%
%:%5013=1534%:%
%:%5014=1535%:%
%:%5015=1535%:%
%:%5016=1535%:%
%:%5017=1536%:%
%:%5018=1536%:%
%:%5019=1537%:%
%:%5020=1537%:%
%:%5021=1537%:%
%:%5022=1537%:%
%:%5023=1538%:%
%:%5024=1538%:%
%:%5025=1539%:%
%:%5026=1539%:%
%:%5027=1540%:%
%:%5028=1540%:%
%:%5029=1541%:%
%:%5030=1541%:%
%:%5031=1542%:%
%:%5032=1542%:%
%:%5033=1542%:%
%:%5034=1543%:%
%:%5035=1543%:%
%:%5036=1543%:%
%:%5037=1543%:%
%:%5038=1544%:%
%:%5039=1544%:%
%:%5040=1544%:%
%:%5041=1544%:%
%:%5042=1544%:%
%:%5043=1545%:%
%:%5044=1546%:%
%:%5045=1546%:%
%:%5046=1547%:%
%:%5047=1547%:%
%:%5048=1547%:%
%:%5049=1547%:%
%:%5050=1547%:%
%:%5051=1548%:%
%:%5052=1548%:%
%:%5053=1549%:%
%:%5054=1549%:%
%:%5055=1550%:%
%:%5056=1550%:%
%:%5057=1550%:%
%:%5058=1550%:%
%:%5059=1551%:%
%:%5060=1551%:%
%:%5061=1551%:%
%:%5062=1551%:%
%:%5063=1552%:%
%:%5064=1552%:%
%:%5065=1552%:%
%:%5066=1552%:%
%:%5067=1552%:%
%:%5068=1553%:%
%:%5069=1553%:%
%:%5070=1553%:%
%:%5071=1553%:%
%:%5072=1554%:%
%:%5073=1554%:%
%:%5074=1554%:%
%:%5075=1554%:%
%:%5076=1554%:%
%:%5077=1555%:%
%:%5078=1556%:%
%:%5079=1556%:%
%:%5080=1557%:%
%:%5081=1557%:%
%:%5082=1557%:%
%:%5083=1557%:%
%:%5084=1557%:%
%:%5085=1557%:%
%:%5086=1558%:%
%:%5087=1558%:%
%:%5088=1558%:%
%:%5089=1558%:%
%:%5090=1558%:%
%:%5091=1559%:%
%:%5092=1560%:%
%:%5093=1560%:%
%:%5094=1561%:%
%:%5095=1561%:%
%:%5096=1562%:%
%:%5097=1562%:%
%:%5098=1562%:%
%:%5099=1562%:%
%:%5100=1563%:%
%:%5101=1563%:%
%:%5102=1563%:%
%:%5103=1563%:%
%:%5104=1563%:%
%:%5105=1564%:%
%:%5106=1564%:%
%:%5107=1565%:%
%:%5108=1565%:%
%:%5109=1565%:%
%:%5110=1565%:%
%:%5111=1565%:%
%:%5112=1565%:%
%:%5113=1566%:%
%:%5114=1567%:%
%:%5115=1567%:%
%:%5116=1567%:%
%:%5117=1567%:%
%:%5118=1568%:%
%:%5119=1568%:%
%:%5120=1568%:%
%:%5121=1568%:%
%:%5122=1568%:%
%:%5123=1568%:%
%:%5124=1569%:%
%:%5125=1569%:%
%:%5126=1569%:%
%:%5127=1569%:%
%:%5128=1569%:%
%:%5129=1569%:%
%:%5130=1570%:%
%:%5131=1570%:%
%:%5132=1570%:%
%:%5133=1570%:%
%:%5134=1571%:%
%:%5135=1571%:%
%:%5136=1571%:%
%:%5137=1571%:%
%:%5138=1571%:%
%:%5139=1571%:%
%:%5140=1572%:%
%:%5141=1572%:%
%:%5142=1572%:%
%:%5143=1572%:%
%:%5144=1572%:%
%:%5145=1573%:%
%:%5146=1573%:%
%:%5147=1574%:%
%:%5148=1574%:%
%:%5149=1575%:%
%:%5150=1576%:%
%:%5151=1576%:%
%:%5152=1577%:%
%:%5153=1577%:%
%:%5154=1578%:%
%:%5155=1578%:%
%:%5156=1578%:%
%:%5157=1579%:%
%:%5158=1579%:%
%:%5159=1579%:%
%:%5160=1579%:%
%:%5161=1580%:%
%:%5162=1580%:%
%:%5163=1580%:%
%:%5164=1580%:%
%:%5165=1580%:%
%:%5166=1581%:%
%:%5167=1581%:%
%:%5168=1581%:%
%:%5169=1581%:%
%:%5170=1581%:%
%:%5171=1581%:%
%:%5172=1582%:%
%:%5173=1582%:%
%:%5174=1582%:%
%:%5175=1582%:%
%:%5176=1583%:%
%:%5177=1584%:%
%:%5178=1584%:%
%:%5179=1584%:%
%:%5180=1584%:%
%:%5181=1584%:%
%:%5182=1585%:%
%:%5183=1586%:%
%:%5184=1586%:%
%:%5185=1586%:%
%:%5186=1586%:%
%:%5187=1587%:%
%:%5188=1587%:%
%:%5189=1587%:%
%:%5190=1587%:%
%:%5191=1587%:%
%:%5192=1587%:%
%:%5193=1588%:%
%:%5194=1588%:%
%:%5195=1588%:%
%:%5196=1588%:%
%:%5197=1588%:%
%:%5198=1588%:%
%:%5199=1589%:%
%:%5200=1590%:%
%:%5201=1590%:%
%:%5202=1590%:%
%:%5203=1590%:%
%:%5204=1591%:%
%:%5205=1591%:%
%:%5206=1591%:%
%:%5207=1591%:%
%:%5208=1591%:%
%:%5209=1591%:%
%:%5210=1592%:%
%:%5211=1592%:%
%:%5212=1592%:%
%:%5213=1592%:%
%:%5214=1592%:%
%:%5215=1592%:%
%:%5216=1593%:%
%:%5217=1594%:%
%:%5218=1594%:%
%:%5219=1594%:%
%:%5220=1594%:%
%:%5221=1595%:%
%:%5222=1595%:%
%:%5223=1595%:%
%:%5224=1595%:%
%:%5225=1595%:%
%:%5226=1596%:%
%:%5227=1596%:%
%:%5228=1597%:%
%:%5229=1597%:%
%:%5230=1598%:%
%:%5231=1598%:%
%:%5232=1599%:%
%:%5233=1599%:%
%:%5234=1599%:%
%:%5235=1600%:%
%:%5236=1600%:%
%:%5237=1600%:%
%:%5238=1600%:%
%:%5239=1600%:%
%:%5240=1601%:%
%:%5241=1601%:%
%:%5242=1601%:%
%:%5243=1601%:%
%:%5244=1601%:%
%:%5245=1602%:%
%:%5246=1602%:%
%:%5247=1602%:%
%:%5248=1602%:%
%:%5249=1602%:%
%:%5250=1603%:%
%:%5251=1603%:%
%:%5252=1603%:%
%:%5253=1603%:%
%:%5254=1603%:%
%:%5255=1604%:%
%:%5256=1604%:%
%:%5257=1604%:%
%:%5258=1604%:%
%:%5259=1604%:%
%:%5260=1604%:%
%:%5261=1605%:%
%:%5262=1605%:%
%:%5263=1605%:%
%:%5264=1605%:%
%:%5265=1606%:%
%:%5266=1606%:%
%:%5267=1606%:%
%:%5268=1606%:%
%:%5269=1606%:%
%:%5270=1606%:%
%:%5271=1607%:%
%:%5272=1607%:%
%:%5273=1608%:%
%:%5274=1608%:%
%:%5275=1608%:%
%:%5276=1608%:%
%:%5277=1608%:%
%:%5278=1609%:%
%:%5279=1609%:%
%:%5280=1609%:%
%:%5281=1609%:%
%:%5282=1609%:%
%:%5283=1609%:%
%:%5284=1610%:%
%:%5285=1610%:%
%:%5286=1610%:%
%:%5287=1610%:%
%:%5288=1610%:%
%:%5289=1611%:%
%:%5290=1611%:%
%:%5291=1611%:%
%:%5292=1611%:%
%:%5293=1611%:%
%:%5294=1611%:%
%:%5295=1612%:%
%:%5296=1612%:%
%:%5297=1612%:%
%:%5298=1612%:%
%:%5299=1612%:%
%:%5300=1613%:%
%:%5301=1613%:%
%:%5302=1613%:%
%:%5303=1613%:%
%:%5304=1614%:%
%:%5305=1615%:%
%:%5306=1616%:%
%:%5307=1616%:%
%:%5308=1616%:%
%:%5309=1616%:%
%:%5310=1616%:%
%:%5311=1617%:%
%:%5312=1617%:%
%:%5313=1617%:%
%:%5314=1617%:%
%:%5315=1617%:%
%:%5316=1618%:%
%:%5317=1618%:%
%:%5318=1619%:%
%:%5319=1619%:%
%:%5320=1619%:%
%:%5321=1620%:%
%:%5322=1620%:%
%:%5323=1621%:%
%:%5324=1622%:%
%:%5325=1622%:%
%:%5326=1623%:%
%:%5327=1623%:%
%:%5328=1624%:%
%:%5329=1624%:%
%:%5330=1624%:%
%:%5331=1624%:%
%:%5332=1625%:%
%:%5333=1625%:%
%:%5334=1625%:%
%:%5335=1626%:%
%:%5336=1626%:%
%:%5337=1626%:%
%:%5338=1626%:%
%:%5339=1626%:%
%:%5340=1626%:%
%:%5341=1627%:%
%:%5342=1627%:%
%:%5343=1627%:%
%:%5344=1627%:%
%:%5345=1627%:%
%:%5346=1627%:%
%:%5347=1628%:%
%:%5348=1628%:%
%:%5349=1628%:%
%:%5350=1628%:%
%:%5351=1628%:%
%:%5352=1629%:%
%:%5353=1629%:%
%:%5354=1630%:%
%:%5355=1630%:%
%:%5356=1631%:%
%:%5357=1631%:%
%:%5358=1631%:%
%:%5359=1631%:%
%:%5360=1632%:%
%:%5361=1632%:%
%:%5362=1632%:%
%:%5363=1632%:%
%:%5364=1633%:%
%:%5365=1633%:%
%:%5366=1633%:%
%:%5367=1633%:%
%:%5368=1634%:%
%:%5369=1634%:%
%:%5370=1634%:%
%:%5371=1634%:%
%:%5372=1634%:%
%:%5373=1635%:%
%:%5374=1635%:%
%:%5375=1636%:%
%:%5376=1636%:%
%:%5377=1636%:%
%:%5378=1636%:%
%:%5379=1637%:%
%:%5380=1637%:%
%:%5381=1638%:%
%:%5382=1638%:%
%:%5383=1638%:%
%:%5384=1638%:%
%:%5385=1638%:%
%:%5386=1639%:%
%:%5387=1639%:%
%:%5388=1639%:%
%:%5389=1639%:%
%:%5390=1640%:%
%:%5391=1640%:%
%:%5392=1640%:%
%:%5393=1640%:%
%:%5394=1640%:%
%:%5395=1640%:%
%:%5396=1641%:%
%:%5397=1641%:%
%:%5398=1641%:%
%:%5399=1641%:%
%:%5400=1641%:%
%:%5401=1642%:%
%:%5402=1642%:%
%:%5403=1643%:%
%:%5404=1643%:%
%:%5405=1643%:%
%:%5406=1643%:%
%:%5407=1643%:%
%:%5408=1644%:%
%:%5409=1644%:%
%:%5410=1645%:%
%:%5416=1645%:%
%:%5419=1646%:%
%:%5420=1647%:%
%:%5421=1648%:%
%:%5422=1649%:%
%:%5423=1649%:%
%:%5430=1650%:%
%:%5431=1650%:%
%:%5432=1651%:%
%:%5433=1651%:%
%:%5434=1652%:%
%:%5435=1652%:%
%:%5436=1652%:%
%:%5437=1653%:%
%:%5438=1653%:%
%:%5439=1653%:%
%:%5440=1654%:%
%:%5441=1654%:%
%:%5442=1655%:%
%:%5443=1655%:%
%:%5444=1656%:%
%:%5445=1656%:%
%:%5446=1656%:%
%:%5447=1656%:%
%:%5448=1656%:%
%:%5449=1656%:%
%:%5450=1657%:%
%:%5451=1657%:%
%:%5452=1658%:%
%:%5453=1658%:%
%:%5454=1659%:%
%:%5455=1659%:%
%:%5456=1659%:%
%:%5457=1660%:%
%:%5458=1660%:%
%:%5459=1661%:%
%:%5460=1661%:%
%:%5461=1662%:%
%:%5462=1662%:%
%:%5463=1662%:%
%:%5464=1662%:%
%:%5465=1662%:%
%:%5466=1663%:%
%:%5467=1663%:%
%:%5468=1664%:%
%:%5469=1664%:%
%:%5470=1665%:%
%:%5471=1665%:%
%:%5472=1665%:%
%:%5473=1665%:%
%:%5474=1665%:%
%:%5475=1665%:%
%:%5476=1666%:%
%:%5477=1666%:%
%:%5478=1667%:%
%:%5479=1667%:%
%:%5480=1668%:%
%:%5481=1668%:%
%:%5482=1669%:%
%:%5483=1669%:%
%:%5484=1670%:%
%:%5485=1670%:%
%:%5486=1670%:%
%:%5487=1671%:%
%:%5488=1671%:%
%:%5489=1672%:%
%:%5490=1672%:%
%:%5491=1673%:%
%:%5492=1673%:%
%:%5493=1673%:%
%:%5494=1673%:%
%:%5495=1674%:%
%:%5496=1674%:%
%:%5497=1674%:%
%:%5498=1675%:%
%:%5499=1675%:%
%:%5500=1676%:%
%:%5501=1676%:%
%:%5502=1677%:%
%:%5503=1677%:%
%:%5504=1677%:%
%:%5505=1677%:%
%:%5506=1678%:%
%:%5507=1678%:%
%:%5508=1678%:%
%:%5509=1679%:%
%:%5510=1679%:%
%:%5511=1680%:%
%:%5512=1680%:%
%:%5513=1681%:%
%:%5514=1681%:%
%:%5515=1681%:%
%:%5516=1681%:%
%:%5517=1681%:%
%:%5518=1681%:%
%:%5519=1682%:%
%:%5520=1682%:%
%:%5521=1683%:%
%:%5522=1683%:%
%:%5523=1684%:%
%:%5524=1684%:%
%:%5525=1684%:%
%:%5526=1684%:%
%:%5527=1684%:%
%:%5528=1684%:%
%:%5529=1685%:%
%:%5530=1685%:%
%:%5531=1686%:%
%:%5532=1686%:%
%:%5533=1687%:%
%:%5534=1687%:%
%:%5535=1687%:%
%:%5536=1687%:%
%:%5537=1688%:%
%:%5538=1688%:%
%:%5539=1688%:%
%:%5540=1688%:%
%:%5541=1688%:%
%:%5542=1689%:%
%:%5543=1689%:%
%:%5544=1690%:%
%:%5545=1690%:%
%:%5546=1690%:%
%:%5547=1690%:%
%:%5548=1690%:%
%:%5549=1691%:%
%:%5550=1691%:%
%:%5551=1692%:%
%:%5552=1692%:%
%:%5553=1693%:%
%:%5554=1693%:%
%:%5555=1694%:%
%:%5556=1694%:%
%:%5557=1694%:%
%:%5558=1694%:%
%:%5559=1695%:%
%:%5560=1695%:%
%:%5561=1695%:%
%:%5562=1695%:%
%:%5563=1695%:%
%:%5564=1695%:%
%:%5565=1696%:%
%:%5566=1696%:%
%:%5567=1696%:%
%:%5568=1696%:%
%:%5569=1697%:%
%:%5570=1697%:%
%:%5571=1698%:%
%:%5572=1699%:%
%:%5573=1699%:%
%:%5574=1700%:%
%:%5575=1700%:%
%:%5576=1701%:%
%:%5586=1706%:%
%:%5587=1707%:%
%:%5589=1710%:%
%:%5590=1710%:%
%:%5597=1711%:%
%:%5598=1711%:%
%:%5599=1712%:%
%:%5600=1712%:%
%:%5601=1713%:%
%:%5602=1714%:%
%:%5603=1714%:%
%:%5604=1714%:%
%:%5605=1714%:%
%:%5606=1714%:%
%:%5607=1715%:%
%:%5608=1715%:%
%:%5609=1715%:%
%:%5610=1715%:%
%:%5611=1715%:%
%:%5612=1715%:%
%:%5613=1716%:%
%:%5614=1717%:%
%:%5615=1717%:%
%:%5616=1717%:%
%:%5617=1717%:%
%:%5618=1717%:%
%:%5619=1718%:%
%:%5620=1718%:%
%:%5621=1718%:%
%:%5622=1718%:%
%:%5623=1719%:%
%:%5624=1719%:%
%:%5625=1719%:%
%:%5626=1720%:%
%:%5627=1720%:%
%:%5628=1721%:%
%:%5629=1721%:%
%:%5630=1721%:%
%:%5631=1721%:%
%:%5632=1722%:%
%:%5633=1722%:%
%:%5634=1722%:%
%:%5635=1722%:%
%:%5636=1722%:%
%:%5637=1723%:%
%:%5638=1723%:%
%:%5639=1724%:%
%:%5640=1724%:%
%:%5641=1725%:%
%:%5642=1725%:%
%:%5643=1725%:%
%:%5644=1725%:%
%:%5645=1725%:%
%:%5646=1726%:%
%:%5647=1726%:%
%:%5648=1727%:%
%:%5649=1727%:%
%:%5650=1728%:%
%:%5651=1728%:%
%:%5652=1729%:%
%:%5653=1729%:%
%:%5654=1729%:%
%:%5655=1730%:%
%:%5656=1730%:%
%:%5657=1730%:%
%:%5658=1731%:%
%:%5659=1731%:%
%:%5660=1731%:%
%:%5661=1731%:%
%:%5662=1732%:%
%:%5663=1732%:%
%:%5664=1733%:%
%:%5665=1733%:%
%:%5666=1734%:%
%:%5667=1734%:%
%:%5668=1735%:%
%:%5669=1735%:%
%:%5670=1735%:%
%:%5671=1735%:%
%:%5672=1736%:%
%:%5673=1736%:%
%:%5674=1736%:%
%:%5675=1736%:%
%:%5676=1737%:%
%:%5677=1737%:%
%:%5678=1738%:%
%:%5679=1738%:%
%:%5680=1739%:%
%:%5681=1739%:%
%:%5682=1739%:%
%:%5683=1739%:%
%:%5684=1739%:%
%:%5685=1740%:%
%:%5686=1740%:%
%:%5687=1740%:%
%:%5688=1740%:%
%:%5689=1740%:%
%:%5690=1741%:%
%:%5691=1741%:%
%:%5692=1742%:%
%:%5693=1742%:%
%:%5694=1742%:%
%:%5695=1742%:%
%:%5696=1742%:%
%:%5697=1743%:%
%:%5698=1743%:%
%:%5699=1743%:%
%:%5700=1743%:%
%:%5701=1744%:%
%:%5702=1744%:%
%:%5703=1744%:%
%:%5704=1744%:%
%:%5705=1745%:%
%:%5706=1745%:%
%:%5707=1745%:%
%:%5708=1745%:%
%:%5709=1746%:%
%:%5710=1746%:%
%:%5711=1746%:%
%:%5712=1747%:%
%:%5713=1747%:%
%:%5714=1748%:%
%:%5715=1748%:%
%:%5716=1748%:%
%:%5717=1748%:%
%:%5718=1749%:%
%:%5719=1749%:%
%:%5720=1749%:%
%:%5721=1749%:%
%:%5722=1750%:%
%:%5723=1750%:%
%:%5724=1750%:%
%:%5725=1751%:%
%:%5726=1751%:%
%:%5727=1752%:%
%:%5728=1752%:%
%:%5729=1752%:%
%:%5730=1752%:%
%:%5731=1752%:%
%:%5732=1753%:%
%:%5733=1753%:%
%:%5734=1753%:%
%:%5735=1753%:%
%:%5736=1754%:%
%:%5737=1754%:%
%:%5738=1754%:%
%:%5739=1754%:%
%:%5740=1755%:%
%:%5741=1755%:%
%:%5742=1755%:%
%:%5743=1755%:%
%:%5744=1755%:%
%:%5745=1756%:%
%:%5746=1756%:%
%:%5747=1756%:%
%:%5748=1756%:%
%:%5749=1756%:%
%:%5750=1756%:%
%:%5751=1757%:%
%:%5752=1757%:%
%:%5753=1757%:%
%:%5754=1757%:%
%:%5755=1758%:%
%:%5756=1758%:%
%:%5757=1758%:%
%:%5758=1758%:%
%:%5759=1758%:%
%:%5760=1759%:%
%:%5761=1759%:%
%:%5762=1759%:%
%:%5763=1759%:%
%:%5764=1759%:%
%:%5765=1760%:%
%:%5766=1760%:%
%:%5767=1760%:%
%:%5768=1760%:%
%:%5769=1761%:%
%:%5770=1761%:%
%:%5771=1761%:%
%:%5772=1761%:%
%:%5773=1762%:%
%:%5774=1762%:%
%:%5775=1762%:%
%:%5776=1762%:%
%:%5777=1763%:%
%:%5778=1764%:%
%:%5779=1764%:%
%:%5780=1765%:%
%:%5781=1765%:%
%:%5782=1766%:%
%:%5783=1767%:%
%:%5784=1767%:%
%:%5785=1767%:%
%:%5786=1767%:%
%:%5787=1767%:%
%:%5788=1768%:%
%:%5789=1769%:%
%:%5790=1769%:%
%:%5791=1769%:%
%:%5792=1769%:%
%:%5793=1770%:%
%:%5794=1770%:%
%:%5795=1771%:%
%:%5796=1771%:%
%:%5797=1771%:%
%:%5798=1771%:%
%:%5799=1772%:%
%:%5800=1772%:%
%:%5801=1773%:%
%:%5802=1773%:%
%:%5803=1773%:%
%:%5804=1773%:%
%:%5805=1774%:%
%:%5806=1775%:%
%:%5807=1775%:%
%:%5808=1775%:%
%:%5809=1775%:%
%:%5810=1775%:%
%:%5811=1776%:%
%:%5812=1777%:%
%:%5813=1777%:%
%:%5814=1777%:%
%:%5815=1777%:%
%:%5816=1777%:%
%:%5817=1778%:%
%:%5818=1778%:%
%:%5819=1778%:%
%:%5820=1779%:%
%:%5821=1779%:%
%:%5822=1780%:%
%:%5823=1780%:%
%:%5824=1781%:%
%:%5825=1781%:%
%:%5826=1782%:%
%:%5827=1782%:%
%:%5828=1782%:%
%:%5829=1783%:%
%:%5830=1783%:%
%:%5831=1784%:%
%:%5832=1784%:%
%:%5833=1784%:%
%:%5834=1785%:%
%:%5835=1785%:%
%:%5836=1786%:%
%:%5837=1786%:%
%:%5838=1787%:%
%:%5839=1787%:%
%:%5840=1787%:%
%:%5841=1787%:%
%:%5842=1787%:%
%:%5843=1787%:%
%:%5844=1788%:%
%:%5845=1788%:%
%:%5846=1788%:%
%:%5847=1788%:%
%:%5848=1788%:%
%:%5849=1789%:%
%:%5850=1789%:%
%:%5851=1789%:%
%:%5852=1789%:%
%:%5853=1789%:%
%:%5854=1790%:%
%:%5855=1790%:%
%:%5856=1790%:%
%:%5857=1790%:%
%:%5858=1790%:%
%:%5859=1790%:%
%:%5860=1791%:%
%:%5861=1791%:%
%:%5862=1791%:%
%:%5863=1791%:%
%:%5864=1792%:%
%:%5865=1792%:%
%:%5866=1793%:%
%:%5867=1793%:%
%:%5868=1794%:%
%:%5869=1794%:%
%:%5870=1794%:%
%:%5871=1794%:%
%:%5872=1795%:%
%:%5873=1795%:%
%:%5874=1795%:%
%:%5875=1795%:%
%:%5876=1795%:%
%:%5877=1796%:%
%:%5878=1796%:%
%:%5879=1796%:%
%:%5880=1797%:%
%:%5881=1797%:%
%:%5882=1798%:%
%:%5883=1798%:%
%:%5884=1799%:%
%:%5885=1799%:%
%:%5886=1799%:%
%:%5887=1799%:%
%:%5888=1799%:%
%:%5889=1799%:%
%:%5890=1800%:%
%:%5891=1800%:%
%:%5892=1800%:%
%:%5893=1800%:%
%:%5894=1800%:%
%:%5895=1801%:%
%:%5896=1801%:%
%:%5897=1801%:%
%:%5898=1801%:%
%:%5899=1801%:%
%:%5900=1802%:%
%:%5901=1802%:%
%:%5902=1802%:%
%:%5903=1802%:%
%:%5904=1802%:%
%:%5905=1803%:%
%:%5906=1803%:%
%:%5907=1803%:%
%:%5908=1803%:%
%:%5909=1803%:%
%:%5910=1804%:%
%:%5911=1804%:%
%:%5912=1804%:%
%:%5913=1804%:%
%:%5914=1804%:%
%:%5915=1804%:%
%:%5916=1805%:%
%:%5917=1805%:%
%:%5918=1805%:%
%:%5919=1805%:%
%:%5920=1806%:%
%:%5921=1806%:%
%:%5922=1807%:%
%:%5923=1807%:%
%:%5924=1808%:%
%:%5925=1808%:%
%:%5926=1808%:%
%:%5927=1808%:%
%:%5928=1808%:%
%:%5929=1808%:%
%:%5930=1809%:%
%:%5931=1809%:%
%:%5932=1809%:%
%:%5933=1809%:%
%:%5934=1809%:%
%:%5935=1810%:%
%:%5936=1810%:%
%:%5937=1810%:%
%:%5938=1810%:%
%:%5939=1810%:%
%:%5940=1811%:%
%:%5941=1811%:%
%:%5942=1811%:%
%:%5943=1811%:%
%:%5944=1812%:%
%:%5945=1812%:%
%:%5946=1813%:%
%:%5947=1813%:%
%:%5948=1814%:%
%:%5949=1814%:%
%:%5950=1814%:%
%:%5951=1814%:%
%:%5952=1814%:%
%:%5953=1814%:%
%:%5954=1815%:%
%:%5955=1815%:%
%:%5956=1815%:%
%:%5957=1815%:%
%:%5958=1815%:%
%:%5959=1816%:%
%:%5960=1816%:%
%:%5961=1816%:%
%:%5962=1816%:%
%:%5963=1816%:%
%:%5964=1817%:%
%:%5965=1817%:%
%:%5966=1817%:%
%:%5967=1817%:%
%:%5968=1817%:%
%:%5969=1818%:%
%:%5970=1818%:%
%:%5971=1818%:%
%:%5972=1818%:%
%:%5973=1818%:%
%:%5974=1819%:%
%:%5975=1819%:%
%:%5976=1819%:%
%:%5977=1819%:%
%:%5978=1819%:%
%:%5979=1819%:%
%:%5980=1820%:%
%:%5981=1820%:%
%:%5982=1820%:%
%:%5983=1820%:%
%:%5984=1821%:%
%:%5985=1821%:%
%:%5986=1822%:%
%:%5987=1822%:%
%:%5988=1823%:%
%:%5989=1823%:%
%:%5990=1823%:%
%:%5991=1823%:%
%:%5992=1824%:%
%:%5993=1824%:%
%:%5994=1825%:%
%:%5995=1825%:%
%:%5996=1826%:%
%:%5997=1826%:%
%:%5998=1826%:%
%:%5999=1827%:%
%:%6000=1827%:%
%:%6001=1828%:%
%:%6002=1828%:%
%:%6003=1828%:%
%:%6004=1828%:%
%:%6005=1829%:%
%:%6006=1829%:%
%:%6007=1829%:%
%:%6008=1829%:%
%:%6009=1830%:%
%:%6010=1830%:%
%:%6011=1830%:%
%:%6012=1830%:%
%:%6013=1831%:%
%:%6014=1831%:%
%:%6015=1831%:%
%:%6016=1831%:%
%:%6017=1831%:%
%:%6018=1832%:%
%:%6019=1832%:%
%:%6020=1832%:%
%:%6021=1832%:%
%:%6022=1832%:%
%:%6023=1833%:%
%:%6024=1833%:%
%:%6025=1833%:%
%:%6026=1834%:%
%:%6027=1834%:%
%:%6028=1835%:%
%:%6029=1835%:%
%:%6030=1835%:%
%:%6031=1835%:%
%:%6032=1835%:%
%:%6033=1836%:%
%:%6034=1836%:%
%:%6035=1836%:%
%:%6036=1836%:%
%:%6037=1836%:%
%:%6038=1837%:%
%:%6039=1837%:%
%:%6040=1837%:%
%:%6041=1837%:%
%:%6042=1837%:%
%:%6043=1838%:%
%:%6044=1838%:%
%:%6045=1838%:%
%:%6046=1838%:%
%:%6047=1839%:%
%:%6048=1839%:%
%:%6049=1840%:%
%:%6050=1840%:%
%:%6051=1840%:%
%:%6052=1840%:%
%:%6053=1840%:%
%:%6054=1841%:%
%:%6055=1841%:%
%:%6056=1842%:%
%:%6057=1842%:%
%:%6058=1842%:%
%:%6059=1842%:%
%:%6060=1842%:%
%:%6061=1843%:%
%:%6062=1844%:%
%:%6063=1844%:%
%:%6064=1844%:%
%:%6065=1844%:%
%:%6066=1845%:%
%:%6067=1845%:%
%:%6068=1846%:%
%:%6069=1846%:%
%:%6070=1847%:%
%:%6071=1847%:%
%:%6072=1847%:%
%:%6073=1847%:%
%:%6074=1848%:%
%:%6075=1848%:%
%:%6076=1848%:%
%:%6077=1849%:%
%:%6078=1849%:%
%:%6079=1850%:%
%:%6080=1850%:%
%:%6081=1851%:%
%:%6082=1851%:%
%:%6083=1851%:%
%:%6084=1852%:%
%:%6085=1852%:%
%:%6086=1853%:%
%:%6087=1853%:%
%:%6088=1854%:%
%:%6089=1854%:%
%:%6090=1854%:%
%:%6091=1854%:%
%:%6092=1855%:%
%:%6093=1855%:%
%:%6094=1856%:%
%:%6095=1856%:%
%:%6096=1857%:%
%:%6097=1857%:%
%:%6098=1857%:%
%:%6099=1857%:%
%:%6100=1857%:%
%:%6101=1858%:%
%:%6102=1858%:%
%:%6103=1858%:%
%:%6104=1858%:%
%:%6105=1858%:%
%:%6106=1859%:%
%:%6107=1859%:%
%:%6108=1859%:%
%:%6109=1860%:%
%:%6110=1860%:%
%:%6111=1861%:%
%:%6112=1861%:%
%:%6113=1862%:%
%:%6114=1862%:%
%:%6115=1862%:%
%:%6116=1863%:%
%:%6117=1863%:%
%:%6118=1864%:%
%:%6119=1864%:%
%:%6120=1864%:%
%:%6121=1864%:%
%:%6122=1864%:%
%:%6123=1865%:%
%:%6124=1865%:%
%:%6125=1865%:%
%:%6126=1865%:%
%:%6127=1865%:%
%:%6128=1866%:%
%:%6129=1866%:%
%:%6130=1866%:%
%:%6131=1866%:%
%:%6132=1867%:%
%:%6133=1867%:%
%:%6134=1868%:%
%:%6135=1868%:%
%:%6136=1868%:%
%:%6137=1869%:%
%:%6138=1869%:%
%:%6139=1870%:%
%:%6140=1870%:%
%:%6141=1870%:%
%:%6142=1870%:%
%:%6143=1870%:%
%:%6144=1871%:%
%:%6145=1871%:%
%:%6146=1872%:%
%:%6147=1872%:%
%:%6148=1873%:%
%:%6149=1873%:%
%:%6150=1873%:%
%:%6151=1873%:%
%:%6152=1873%:%
%:%6153=1874%:%
%:%6154=1874%:%
%:%6155=1874%:%
%:%6156=1874%:%
%:%6157=1875%:%
%:%6158=1875%:%
%:%6159=1876%:%
%:%6160=1876%:%
%:%6161=1876%:%
%:%6162=1876%:%
%:%6163=1877%:%
%:%6164=1877%:%
%:%6165=1877%:%
%:%6166=1878%:%
%:%6167=1878%:%
%:%6168=1879%:%
%:%6169=1879%:%
%:%6170=1879%:%
%:%6171=1880%:%
%:%6172=1880%:%
%:%6173=1881%:%
%:%6174=1881%:%
%:%6175=1881%:%
%:%6176=1881%:%
%:%6177=1881%:%
%:%6178=1882%:%
%:%6179=1882%:%
%:%6180=1883%:%
%:%6181=1883%:%
%:%6182=1884%:%
%:%6183=1884%:%
%:%6184=1884%:%
%:%6185=1884%:%
%:%6186=1884%:%
%:%6187=1885%:%
%:%6188=1885%:%
%:%6189=1885%:%
%:%6190=1885%:%
%:%6191=1886%:%
%:%6192=1886%:%
%:%6193=1887%:%
%:%6194=1887%:%
%:%6195=1887%:%
%:%6196=1887%:%
%:%6197=1888%:%
%:%6198=1888%:%
%:%6199=1888%:%
%:%6200=1889%:%
%:%6201=1889%:%
%:%6202=1890%:%
%:%6203=1890%:%
%:%6204=1890%:%
%:%6205=1890%:%
%:%6206=1890%:%
%:%6207=1891%:%
%:%6208=1891%:%
%:%6209=1892%:%
%:%6210=1892%:%
%:%6211=1892%:%
%:%6212=1893%:%
%:%6213=1893%:%
%:%6214=1894%:%
%:%6215=1894%:%
%:%6216=1894%:%
%:%6217=1894%:%
%:%6218=1894%:%
%:%6219=1895%:%
%:%6220=1895%:%
%:%6221=1895%:%
%:%6222=1895%:%
%:%6223=1895%:%
%:%6224=1896%:%
%:%6225=1896%:%
%:%6226=1896%:%
%:%6227=1896%:%
%:%6228=1897%:%
%:%6229=1897%:%
%:%6230=1898%:%
%:%6231=1898%:%
%:%6232=1898%:%
%:%6233=1898%:%
%:%6234=1898%:%
%:%6235=1899%:%
%:%6236=1899%:%
%:%6237=1900%:%
%:%6238=1900%:%
%:%6239=1900%:%
%:%6240=1900%:%
%:%6241=1900%:%
%:%6242=1901%:%
%:%6243=1901%:%
%:%6244=1901%:%
%:%6245=1901%:%
%:%6246=1902%:%
%:%6247=1902%:%
%:%6248=1903%:%
%:%6249=1903%:%
%:%6250=1904%:%
%:%6251=1904%:%
%:%6252=1904%:%
%:%6253=1904%:%
%:%6254=1904%:%
%:%6255=1905%:%
%:%6256=1905%:%
%:%6257=1905%:%
%:%6258=1906%:%
%:%6259=1906%:%
%:%6260=1907%:%
%:%6261=1907%:%
%:%6262=1907%:%
%:%6263=1907%:%
%:%6264=1907%:%
%:%6265=1908%:%
%:%6266=1908%:%
%:%6267=1908%:%
%:%6268=1908%:%
%:%6269=1908%:%
%:%6270=1909%:%
%:%6271=1909%:%
%:%6272=1909%:%
%:%6273=1909%:%
%:%6274=1910%:%
%:%6275=1910%:%
%:%6276=1911%:%
%:%6277=1911%:%
%:%6278=1911%:%
%:%6279=1912%:%
%:%6280=1912%:%
%:%6281=1913%:%
%:%6282=1913%:%
%:%6283=1913%:%
%:%6284=1913%:%
%:%6285=1913%:%
%:%6286=1914%:%
%:%6287=1914%:%
%:%6288=1914%:%
%:%6289=1914%:%
%:%6290=1914%:%
%:%6291=1915%:%
%:%6292=1915%:%
%:%6293=1915%:%
%:%6294=1915%:%
%:%6295=1916%:%
%:%6296=1916%:%
%:%6297=1917%:%
%:%6298=1917%:%
%:%6299=1917%:%
%:%6300=1917%:%
%:%6301=1917%:%
%:%6302=1917%:%
%:%6303=1918%:%
%:%6304=1918%:%
%:%6305=1918%:%
%:%6306=1918%:%
%:%6307=1919%:%
%:%6308=1919%:%
%:%6309=1920%:%
%:%6310=1920%:%
%:%6311=1921%:%
%:%6312=1921%:%
%:%6313=1922%:%
%:%6314=1922%:%
%:%6315=1923%:%
%:%6316=1923%:%
%:%6317=1923%:%
%:%6318=1923%:%
%:%6319=1924%:%
%:%6320=1924%:%
%:%6321=1925%:%
%:%6322=1925%:%
%:%6323=1925%:%
%:%6324=1925%:%
%:%6325=1925%:%
%:%6326=1926%:%
%:%6327=1926%:%
%:%6328=1927%:%
%:%6329=1927%:%
%:%6330=1927%:%
%:%6331=1928%:%
%:%6332=1928%:%
%:%6333=1929%:%
%:%6334=1930%:%
%:%6335=1930%:%
%:%6336=1930%:%
%:%6337=1930%:%
%:%6338=1931%:%
%:%6339=1932%:%
%:%6345=1932%:%
%:%6348=1933%:%
%:%6349=1934%:%
%:%6350=1934%:%
%:%6353=1935%:%
%:%6357=1935%:%
%:%6358=1935%:%
%:%6359=1935%:%
%:%6360=1935%:%
%:%6361=1935%:%
%:%6366=1935%:%
%:%6369=1936%:%
%:%6370=1937%:%
%:%6371=1938%:%
%:%6372=1939%:%
%:%6373=1939%:%
%:%6380=1940%:%
%:%6381=1940%:%
%:%6382=1941%:%
%:%6383=1941%:%
%:%6384=1942%:%
%:%6385=1942%:%
%:%6386=1942%:%
%:%6387=1943%:%
%:%6388=1943%:%
%:%6389=1943%:%
%:%6390=1943%:%
%:%6391=1943%:%
%:%6392=1944%:%
%:%6398=1944%:%
%:%6401=1945%:%
%:%6402=1946%:%
%:%6403=1946%:%
%:%6406=1947%:%
%:%6410=1947%:%
%:%6411=1947%:%
%:%6412=1947%:%
%:%6413=1947%:%
%:%6418=1947%:%
%:%6421=1948%:%
%:%6422=1949%:%
%:%6423=1949%:%
%:%6430=1950%:%
%:%6431=1950%:%
%:%6432=1951%:%
%:%6433=1951%:%
%:%6434=1951%:%
%:%6435=1951%:%
%:%6436=1952%:%
%:%6437=1952%:%
%:%6438=1952%:%
%:%6439=1952%:%
%:%6440=1952%:%
%:%6441=1952%:%
%:%6442=1953%:%
%:%6443=1953%:%
%:%6444=1953%:%
%:%6445=1953%:%
%:%6446=1954%:%
%:%6452=1954%:%
%:%6455=1955%:%
%:%6456=1956%:%
%:%6457=1956%:%
%:%6460=1957%:%
%:%6464=1957%:%
%:%6465=1957%:%
%:%6466=1957%:%
%:%6467=1958%:%
%:%6468=1958%:%
%:%6473=1958%:%
%:%6476=1959%:%
%:%6477=1960%:%
%:%6478=1960%:%
%:%6485=1961%:%
%:%6486=1961%:%
%:%6487=1962%:%
%:%6488=1962%:%
%:%6489=1963%:%
%:%6490=1963%:%
%:%6491=1963%:%
%:%6492=1963%:%
%:%6493=1963%:%
%:%6494=1964%:%
%:%6495=1964%:%
%:%6496=1965%:%
%:%6497=1965%:%
%:%6498=1966%:%
%:%6499=1966%:%
%:%6500=1966%:%
%:%6501=1966%:%
%:%6502=1966%:%
%:%6503=1967%:%
%:%6509=1967%:%
%:%6512=1968%:%
%:%6513=1969%:%
%:%6514=1969%:%
%:%6517=1970%:%
%:%6522=1971%:%


% optional bibliography
\nocite{vdw}
\bibliographystyle{abbrv}
\bibliography{root}

\end{document}

%%% Local Variables:
%%% mode: latex
%%% TeX-master: t
%%% End:
